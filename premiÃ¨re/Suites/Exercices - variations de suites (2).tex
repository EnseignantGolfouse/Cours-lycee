\documentclass[
	classe=$1^{ere}STI2D$,
	landscape,
	twocolumn
]{exercice}

\usepackage{tikz-repère}
\usepackage{tcolorbox}

\setlength{\columnsep}{1cm}

\title{Exercices : variations de suites}

\begin{document}

\newcommand{\Exercice}{
	\maketitle

	\begin{exercice}
		Une entreprise cherche à vendre des savons. Elle souhaite savoir quel type de réduction elle peut proposer en cas de large achat.

		Elle décide donc qu'en cas de vente de $n$ savons, le prix individuel sera de $uₙ = 5\dfrac{n}{n + a}$€, où $a$ est un paramètre qui reste à déterminer. \medskip

		\begin{enumerate}
			\item Si $a = 0$, que remarque-t'on à propos de $uₙ$ ?
			\item Compléter : Le fait d'offrir une réduction signifie que la suite $u$ est \correctionDots{décroissante}
			\item Si $a = 2$, calculer le prix individuel lors de la vente de $1$, $2$ et $3$ savons. \correction{$u₁ = \frac{5}{3}, u₂ = \frac{5}{2}, u₃ = 3$}

			      Pourquoi le choix $a = 2$ ne peut alors pas convenir ? \correction{car $u$ est alors croissante} \bigskip

			      On suppose à présent que, pour avoir une réduction, il est nécessaire de choisir $a < 0$.
			\item Si $a = -1,5$, calculer $u₁$. Que remarque-t'on ? \correction{$u₁ = -10$€ : c'est impossible !}

			      On doit donc choisir $a > \correctionDots{-1}$.
			\item L'entreprise souhaite de plus que la vente d'un seul savon revienne à un prix individuel $50\%$ plus élevé qu'en cas d'achat de $2$ savon.

			      Exprimer cette contrainte par une équation, et la résoudre pour trouver la valeur de $a$.

			      \ifdefined\makeCorrection
				      {\color{red}
					      L'équation est

					      \begin{align*}
						      5\dfrac{1}{1 + a} & = 1,5 × 5 \dfrac{2}{2 + a}1 \\
						      2 + a             & = 1,5 × 2 × (1 + a)         \\
						      2 + a             & = 3 + 3a                    \\
						      -1                & = 2a                        \\
						      a                 & = -0,5
					      \end{align*}
				      }
			      \fi
		\end{enumerate}
	\end{exercice}
}

\setcounter{exercice}{1}
\Exercice

\newpage

\setcounter{exercice}{1}
\Exercice

\end{document}