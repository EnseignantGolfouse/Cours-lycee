\documentclass[
	classe=$2^{de}$
]{informatique}

\usepackage{tcolorbox}


\begin{document}

\title{Activité : le paradoxe de Saint-Petersbourg}
\maketitle

On considère le jeu de « Pile ou face » suivant :
\begin{tcolorbox}
	Le joueur gagne le double de sa mise en cas de victoire (« Pile »), et perd sa mise en cas de défaite (« Face »).

	Le joueur peut rejouer tant que sa réserve d'argent le lui permet.
\end{tcolorbox}

Pierre, qui dispose d'une réserve de $1000$€, décide de jouer de la manière suivante :
\begin{itemize}
	\item Il mise $1$€ au départ du jeu ;
	\item Tant qu'il perd, il rejoue une partie en doublant sa mise ;
	\item Il arrête si il gagne, ou si il n'a plus assez d'argent.
\end{itemize}

\begin{enumerate}
	\item Que se passe-t'il si Pierre joue et fait Face-Face-Face-Pile ?
	\item Afin de simuler une partie, on propose la fonction (incomplète) ci-dessous :
	      % TODO: demander à traduire cette fonction en français ? OU, écrire du pseudo-code... Je préfère qu'ils apprennent à lire un programme.
	      \begin{lstlisting}
def jeu():
	from random import randint
	victoire = False
	mise = 1
	reserve = 1000
	while REMPLACE_MOI_DANS_LA_QUESTION_3:
		reserve = reserve - mise
		# La variable tirage représente un lancé de la pièce : elle vaut .... si
		# le résultat est Pile, et .... sinon.
		tirage = randint(0,1)
		# Si le résultat du lancé est .........
		if tirage == 0:
			# Pierre gagne deux fois sa mise
			reserve = reserve + 2 * mise
			victoire = True
		else:
			# ................
			mise = mise * 2
	return reserve - 1000
\end{lstlisting}

	\tipbox{Rappel : les lignes commençant par \texttt{\#} sont des \uline{commentaires} : elle seront ignorées par le programme. Elle servent à décrire le comportement du programme pour le lecteur.}

	      Écrire dans les pointillés des commentaires une description en français de ce que le programme modélise.

	\item Retrouver, dans l'énoncé, la condition à vérifier dans la boucle \texttt{while} de la fonction.

	      Recopier et compléter la fonction dans spyder.
	\item Que représente la valeur renvoyée par la fonction ?
	
	\correctionDots{Il s'agit du gain de Pierre.}
	\item Écrire un programme qui simule l'exécution de $n$ parties, $n$ étant un nombre choisi par l’utilisateur, et qui renvoie la fréquence de parties avec un gain positif d’argent.
	\item Effectuer plusieurs simulations afin de conjecturer une valeur approchée de la probabilité de gagner de l'argent.
\end{enumerate}

\end{document}