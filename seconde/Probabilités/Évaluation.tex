\documentclass[
	classe=$2^{de}$
]{évaluation}

\renewcommand{\arraystretch}{1.4}

\usepackage{diagbox}

\date{26 mai 2023}

\begin{document}

\title{Évaluation probabilités (sujet A)}
\maketitle

\begin{exercice}[2,5]
	\begin{enumerate}
		\item Compléter les phrases suivantes :
		\begin{itemize}
			\item $\overline{A}$ est l'évènement \correctionDots{contraire} de $A$.
			\item $A ∩ B$ est l'évènement $A$ \correctionDots{ET} $B$.
			\item $A ∪ B$ est l'évènement $A$ \correctionDots{ou} $B$.
		\end{itemize}
		\item Si on sait que $P(A) = 0,4$, $P(B) = 0,3$ et $P(A ∩ B) = 0,1$, quelle est la probabilité de $A ∪ B$ ?
	\end{enumerate}
\end{exercice}

\begin{exercice}[3]
	On lance deux dés cubiques équilibrés, qui sont chacun numérotés $1$, $1$, $2$, $3$, $3$, $3$.
	\begin{enumerate}
		\item Faire un arbre qui représente cette situation.
		\item Quelle est la probabilité d'obtenir un $1$ au premier lancé et un $3$ au deuxième lancé ?
		\item Quelle est la probabilité d'obtenir \textbf{au moins} un $2$ sur les deux lancés ?
	\end{enumerate}
\end{exercice}

\begin{exercice}[4,5]
	On veut tester l'efficacité de trois traitement, notés $A$, $B$ et $C$.

	On teste le traitement sur $1000$ personnes :

	\begin{itemize}
		\item Le traitement $A$ a réussi sur $130$ personnes, et échoué sur $10$.
		\item Le traitement $B$ a échoué sur $20$ personnes.
		\item $860$ personnes ayant reçu un traitement ($A$, $B$ ou $C$) ont guéri avec succès.
		\item $450$ personnes ont reçu le traitement $C$.
	\end{itemize}

	\begin{enumerate}
		\item Compléter le tableau suivant :
		      \begin{center}
			      \begin{tabular}{|l|*{3}{>{\centering}p{2cm}|}}
				      \hline
				      \diagbox{Traitement}{Succès} & Réussi             & Échoue             & TOTAL \tabularnewline \hline
				      Traitement $A$               & \correction{$130$} & \correction{$10$}  & \correction{$140$} \tabularnewline \hline
				      Traitement $B$               & \correction{$390$} & \correction{$20$}  & \correction{$410$} \tabularnewline \hline
				      Traitement $C$               & \correction{$340$} & \correction{$110$} & \correction{$450$} \tabularnewline \hline
				      TOTAL                        & \correction{$860$} & \correction{$140$} & \correction{$1000$} \tabularnewline \hline
			      \end{tabular}
		      \end{center} \bigskip

		      On définit les évènements suivants :
		      \begin{multicols}{2}
			      \begin{itemize}
				      \item $T_A$ : «La personne a reçu le traitement $A$»
				      \item $T_B$ : «La personne a reçu le traitement $B$»
				      \item $T_C$ : «La personne a reçu le traitement $C$»
				      \item $R$ : «le traitement a réussi»
				      \item $E$ : «le traitement a échoué»
			      \end{itemize}
		      \end{multicols}
		\item Calculer $P(C ∩ E)$ et $P(A ∩ R)$.
		\item Quel semble être le meilleur traitement ? Justifier.
	\end{enumerate}
\end{exercice}

%===============================================
%================== SUJET B ====================
%===============================================

\newpage\setcounter{exercice}{1}
\title{Évaluation probabilités (sujet B)}
\maketitle

\begin{exercice}[2,5]
	\begin{enumerate}
		\item Compléter les phrases suivantes :
		\begin{itemize}
			\item $\overline{A}$ est l'évènement \correctionDots{contraire} de $A$.
			\item $A ∩ B$ est l'évènement $A$ \correctionDots{ET} $B$.
			\item $A ∪ B$ est l'évènement $A$ \correctionDots{ou} $B$.
		\end{itemize}
		\item Si on sait que $P(A) = 0,5$, $P(B) = 0,4$ et $P(A ∩ B) = 0,2$, quelle est la probabilité de $A ∪ B$ ?
	\end{enumerate}
\end{exercice}

\begin{exercice}[3]
	On lance deux dés cubiques équilibrés, qui sont chacun numérotés $1$, $2$, $2$, $3$, $3$, $3$.
	\begin{enumerate}
		\item Faire un arbre qui représente cette situation.
		\item Quelle est la probabilité d'obtenir un $1$ au premier lancé et un $3$ au deuxième lancé ?
		\item Quelle est la probabilité d'obtenir \textbf{au moins} un $2$ sur les deux lancés ?
	\end{enumerate}
\end{exercice}

\begin{exercice}[4,5]
	On veut tester l'efficacité de trois traitement, notés $A$, $B$ et $C$.

	On teste le traitement sur $1000$ personnes :

	\begin{itemize}
		\item Le traitement $A$ a réussi sur $120$ personnes, et échoué sur $10$.
		\item Le traitement $B$ a échoué sur $30$ personnes.
		\item $850$ personnes ayant reçu un traitement ($A$, $B$ ou $C$) ont guéri avec succès.
		\item $440$ personnes ont reçu le traitement $C$.
	\end{itemize}

	\begin{enumerate}
		\item Compléter le tableau suivant :
		      \begin{center}
			      \begin{tabular}{|l|*{3}{>{\centering}p{2cm}|}}
				      \hline
				      \diagbox{Traitement}{Succès} & Réussi             & Échoue             & TOTAL \tabularnewline \hline
				      Traitement $A$               & \correction{$120$} & \correction{$10$}  & \correction{$130$} \tabularnewline \hline
				      Traitement $B$               & \correction{$400$} & \correction{$30$}  & \correction{$430$} \tabularnewline \hline
				      Traitement $C$               & \correction{$330$} & \correction{$110$} & \correction{$440$} \tabularnewline \hline
				      TOTAL                        & \correction{$850$} & \correction{$150$} & \correction{$1000$} \tabularnewline \hline
			      \end{tabular}
		      \end{center} \bigskip

		      On définit les évènements suivants :
		      \begin{multicols}{2}
			      \begin{itemize}
				      \item $T_A$ : «La personne a reçu le traitement $A$»
				      \item $T_B$ : «La personne a reçu le traitement $B$»
				      \item $T_C$ : «La personne a reçu le traitement $C$»
				      \item $R$ : «le traitement a réussi»
				      \item $E$ : «le traitement a échoué»
			      \end{itemize}
		      \end{multicols}
		\item Calculer $P(A ∩ E)$ et $P(C ∩ R)$.
		\item Quel semble être le meilleur traitement ? Justifier.
	\end{enumerate}
\end{exercice}

\end{document}