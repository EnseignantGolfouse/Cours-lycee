\documentclass[
	classe=$2^{de}$,
]{exercice}

\usepackage{tcolorbox}
\usepackage{diagbox}
\usepackage{setspace}

\setstretch{1.2}
\renewcommand{\arraystretch}{1.25}
\newgeometry{a4paper,top=2cm,bottom=1cm,left=1.4cm,right=1.4cm}

\title{Activité : paradoxe de Simpson}

\begin{document}

\maketitle

On va étudier un paradoxe de statistiques : le paradoxe de Simpson.

\begin{tcolorbox}
	Un patient atteint d'une maladie grave cherche à savoir quel traitement serait le meilleur pour sa maladie. Le médecin lui indique alors que deux traitements existent : le A, et le B.
\end{tcolorbox}

\begin{enumerate}
	\item Le médecin lui fournit des données tirées d'une étude : d'après celle-ci :
	      \begin{itemize}
		      \item Le traitement A a fonctionné sur $162$ sujets, et échoué sur $38$.
		      \item Le traitement B a fonctionné sur $110$ sujets, et échoué sur $90$.
	      \end{itemize}

	      \begin{enumerate}
		      \item Combien de sujets ont testé le traitement A ? \correctionDots{$200$}
		      \item Combien de sujets ont testé le traitement B ? \correctionDots{$200$}
		      \item Quel traitement semble le plus efficace, et pourquoi ?

		            \correctionDots{Le traitement A semble plus efficace, car il a $81\%$ de taux de réussite, par rapport à $55\%$ pour le traitement B\hspace{9.2em}}
	      \end{enumerate}
	\item Le patient, pas encore sûr de lui, décide d'aller voir un autre médecin. Celui ci lui fournit \textbf{la même étude} scientifique, mais lui révèle qu'elle a été menée dans deux établissements différents : on les appellera les hôpitaux 1 et 2. On a regroupé les données dans les tableaux suivants :

	      \vspace{1em}
	      \begin{minipage}{0.25\linewidth}
		      Hôpital 1 :
	      \end{minipage}
	      \begin{minipage}{0.7\linewidth}
		      \begin{tabular}{|l|*{2}{>{\centering}p{2cm}|}}
			      \hline
			      \diagbox{$X$ = Traitement}{$Y$ = Succès} & Réussi & Échoue \tabularnewline
			      \hline
			      Traitement A                             & $151$  & $18$     \tabularnewline
			      \hline
			      Traitement B                             & $9$    & $1$    \tabularnewline
			      \hline
		      \end{tabular}
	      \end{minipage}
	      \vspace{1em}

	      \begin{minipage}{0.25\linewidth}
		      Hôpital 2 :
	      \end{minipage}
	      \begin{minipage}{0.7\linewidth}
		      \begin{tabular}{|l|*{2}{>{\centering}p{2cm}|}}
			      \hline
			      \diagbox{$X$ = Traitement}{$Y$ = Succès} & Réussi & Échoue \tabularnewline
			      \hline
			      Traitement A                             & $11$   & $20$     \tabularnewline
			      \hline
			      Traitement B                             & $101$  & $89$    \tabularnewline
			      \hline
		      \end{tabular}
	      \end{minipage}

	      \begin{enumerate}
		      \item Combiner les données des hôpitaux 1 et 2 dans le tableau suivant, et vérifier qu'on obtient bien les données de la question $1$ : \medskip

		            \begin{minipage}{0.25\linewidth}
			            Cumulé :
		            \end{minipage}
		            \begin{minipage}{0.7\linewidth}
			            \begin{tabular}{|l|*{2}{>{\centering}p{2cm}|}}
				            \hline
				            \diagbox{$X$ = Traitement}{$Y$ = Succès} & Réussi             & Échoue \tabularnewline
				            \hline
				            Traitement A                             & \correction{$162$} & \correction{$38$}     \tabularnewline
				            \hline
				            Traitement B                             & \correction{$110$} & \correction{$90$}    \tabularnewline
				            \hline
			            \end{tabular}
		            \end{minipage}

		      \item Quel semble être le traitement le plus efficace dans l'hôpital 1 ?

		            \correctionDots{Le traitement B semble plus efficace, car il a $90\%$ de taux de réussite, par rapport à $89,3\%$ pour le traitement A\hspace{8.3em}}
		      \item Quel semble être le traitement le plus efficace dans l'hôpital 2 ?

		            \correctionDots{Le traitement B semble plus efficace, car il a $53,1\%$ de taux de réussite, par rapport à $35,5\%$ pour le traitement A\hspace{7.4em}}
	      \end{enumerate}
\end{enumerate}

\begin{tcolorbox}
	\begin{multicols}{2}
		D'après la question $1$, le traitement \correctionDots{A} est plus efficace.
		
		\columnbreak

		D'après la question $2$, le traitement \correctionDots{B} est plus efficace.
	\end{multicols}
\end{tcolorbox}

\end{document}