\documentclass[
	classe=$2^{de}$
]{exercice}

\usepackage{tcolorbox}

\renewcommand{\arraystretch}{1.3}

\title{Activité : les jeunes et le portable}

\begin{document}

\maketitle

\begin{tcolorbox}
	Un centre de loisir accueille $250$ jeunes de $7$ à $18$ ans. On a effectué une étude statistiques sur la possession d'un téléphone portable en $2018$ :
	\begin{itemize}
		\item Sur $40$ lycéens, $5$\% d'entre eux ne possède pas de téléphone portable.
		\item $76$ collégiens possèdent un téléphone portable.
		\item $87$ écoliers ne possèdent pas de téléphone portable.
		\item Au total, $147$ des jeunes possèdent un téléphone portable.
	\end{itemize}
\end{tcolorbox}

\begin{enumerate}
	\item Remplir le tableau suivant :
	      \begin{center}
		      \begin{tabular}{|l|c|c|c|c|}
			      \hline
			                                           & Écoliers           & Collégiens        & Lycéens           & TOTAL              \\ \hline
			      Possède un téléphone portable        & \correction{$33$}  & \correction{$76$} & \correction{$38$} & \correction{$147$} \\ \hline
			      Ne possède pas de téléphone portable & \correction{$87$}  & \correction{$14$} & \correction{$2$}  & \correction{$103$} \\ \hline
			      TOTAL                                & \correction{$120$} & \correction{$90$} & \correction{$40$} & $250$              \\ \hline
		      \end{tabular}
	      \end{center}
	\item On s'intéresse aux évènements :
	      \begin{itemize}
		      \item $T$ «Le jeune choisi possède un téléphone portable»
		      \item $E$ «Le jeune choisi est un écolier»
		      \item $C$ «Le jeune choisi est un collégien»
		      \item $L$ «Le jeune choisi est un lycéen»
	      \end{itemize}
	      \begin{enumerate}
		      \item Décrire par une phrase l'évènement $\overline{T}$ (l'évènement contraire de $T$). \medskip

		            \correctionOr{}{................................................................................................................................................}
		      \item Calculer $P(T)$ et $P(\overline{T})$. \medskip

		            \correctionOr{}{................................................................................................................................................}
	      \end{enumerate}
	\item On choisit un jeune au hasard parmi cette population. Décrire la situation avec un arbre de probabilités.
	\item \begin{enumerate}
		      \item Décrire par une phrase l'évènement $L ∩ T$. \medskip

		            \correctionOr{}{................................................................................................................................................}
		      \item Calculer la probabilité de cet évènement en utilisant le tableau. \correctionDots{$\dfrac{38}{250} = 15,2\%$}
		      \item Calculer la probabilité de cet évènement en utilisant l'arbre, et vérifier qu'on obtient bien le même résultat. \correctionDots{$0,16 × 0,95 = 15,2\%$}
	      \end{enumerate}
	\item En utilisant l'arbre ou le tableau, calculer la probabilité de $\overline{E} ∩ T$. \correctionDots{blablabla $45,6\%$}
	\item Pour calculer la probabilité de $\overline{E} ∪ T$, on peut utiliser la formule de l'union :

	      \begin{align*}
		      P(\overline{E} ∪ T) & = \correctionDots{P(\overline{E})} + \correctionDots{P(T)} - \correctionDots{P(\overline{E} ∩ T)} \\
		                          & = \correctionDots{65,2\%}
	      \end{align*}
\end{enumerate}

\end{document}