\documentclass[
	classe=$2^{de}$,
]{exercice}

\usepackage{tikz-repère}
\usepackage{tcolorbox}
\usepackage{makecell}
\usepackage{colortbl}
\usetikzlibrary{calc}

\title{Activité : jeu de Franc Carreau}

\begin{document}

\maketitle

\section{Expérience}

\begin{tcolorbox}
	On dispose d'un quadrillage de 8cm de côté, et d'un jeton.

	On dit que le jeton est à « franc-carreau » si il ne touche pas les lignes du quadrillage.

	Il faut néanmoins que le jeton atterrisse dans le quadrillage. Si ce n'est pas le cas, le lancé n'est pas valide, et il faut le refaire.
\end{tcolorbox}

\begin{enumerate}
	\item Effectuer $16$ lancés du jeton. Quelle est la fréquence de francs-carreaux obtenus ? \correctionDots{$0,2$}
	\item On va mettre en commun les lancés de tous les élèves, afin d'obtenir un plus grand échantillon.

	      On va donc remplir ensemble le tableur suivant :
	      \begin{center}
		      \definecolor{mytabularcolor}{gray}{0.9}
		      % \renewcommand{\arraystretch}{1.6}
		      \hspace*{-1cm}\begin{tabular}{|l|c|*{16}{>{\centering}p{0.4cm}|}}
			      \hline
			      \rowcolor{mytabularcolor} \makecell{\          \\[0.5em] }   & $A$                            & $B$ & $C$ & $D$ & $E$ & $F$ & $G$ & $H$ & $I$ & $J$ & $K$  & $L$  & $M$  & $N$  & $O$  & $P$  & $Q$ \tabularnewline \hline
			      \cellcolor{mytabularcolor}$1$ & \makecell[l]{\ \\[-0.75em] Élève numéro : \\[0.25em] }   & $1$ & $2$ & $3$ & $4$ & $5$ & $6$ & $7$ & $8$ & $9$ & $10$ & $11$ & $12$ & $13$ & $14$ & $15$ & $16$ \tabularnewline \hline
			      \cellcolor{mytabularcolor}$2$ & \makecell[l]{\ \\[-0.75em] Cumul du nombre                                                                                                                    \\ de lancés \\[0.25em] } & $16$ & \phantom{$10$} & \phantom{$10$} & \phantom{$10$} & \phantom{$10$} & \phantom{$10$} & \phantom{$10$} & \phantom{$10$} & \phantom{$10$} & \phantom{$10$} & \phantom{$10$} & \phantom{$10$} & \phantom{$10$} & \phantom{$10$} & \phantom{$10$} & \phantom{$10$} \tabularnewline \hline
			      \cellcolor{mytabularcolor}$3$ & \makecell[l]{\ \\[-0.75em] Cumul du nombre                                                                                                                    \\ de francs-carreaux \\[0.25em] } &  &  &  &  &  &  &  &  &  &  &  &  &  &  &  &  \tabularnewline \hline
			      \cellcolor{mytabularcolor}$4$ & \makecell[l]{\ \\[-0.75em] Fréquence cumulée                                                                                                                               \\ du nombre de \\ francs-carreaux \\[0.25em] } &  &  &  &  &  &  &  &  &  &  &  &  &  &  &  &  \tabularnewline \hline
		      \end{tabular}
	      \end{center}

	      Dans un tableur, quelle formule doit-on rentrer dans la case $B4$ pour calculer la fréquence ?

	      \correctionDots{$=B3 / B2$}
	\item Dans le repère ci-dessous, placer les fréquences cumulées obtenues, en fonction du nombre de lancé :

	      \begin{center}
		      \begin{tikzpicture}[scale=0.7]
			      \tikzRepere{0.5}{16}{0.5}{8}[][]

			      \foreach \x in {0,20,...,160} {
					      \draw[thick] (\x / 10,0) -- ++(0,-0.2) node[below] {$\x$};
				      }
			      \foreach \y in {0,0.25,0.5,0.75,1} {
					      \draw[thick] (0,\y*8) -- ++(-0.2,0) node[left] {$\y$};
				      }
			      \node[above] at (0,8.7) {fréquence cumulée};
			      \node[left] at (21.7,0) {nombre de lancés};
		      \end{tikzpicture}
	      \end{center}
	\item Quelle semble être la probabilité d'obtenir un franc-carreau ? \correctionDots{}
\end{enumerate}

\newpage

\section{Calcul de la probabilité}

\begin{enumerate}
	\item A votre avis, comment varie la probabilité de franc-carreau si on change la taille du jeton ?

	      \dotfill

	      \dotfill

	      \dotfill
	\item Et si on change la dimension du quadrillage ?

	      \dotfill

	      \dotfill

	      \dotfill
\end{enumerate}

On a représenté ci-dessous un carreau de la grille, de $8$cm de côté. Le jeton fait $\correctionDots{2\phantom{1}}$cm de rayon.

\begin{center}
	\pgfmathsetmacro{\x}{8}
	\pgfmathsetmacro{\y}{2}
	\begin{tikzpicture}
		\coordinate (A) at (0,0);
		\coordinate (B) at (\x,0);
		\coordinate (C) at (\x,\x);
		\coordinate (D) at (0,\x);

		\node[below left] at (A) {$A$};
		\node[below right] at (B) {$B$};
		\node[above right] at (C) {$C$};
		\node[above left] at (D) {$D$};

		\draw (A) -- (B) -- (C) -- (D) -- cycle;

		\coordinate (AInner) at (\y,\y);
		\coordinate (BInner) at (\x-\y,\y);
		\coordinate (CInner) at (\x-\y,\x-\y);
		\coordinate (DInner) at (\y,\x-\y);

		\draw[dashed,fill=white] (AInner) -- (BInner) -- (CInner) -- (DInner) -- cycle;

		\draw[dashed] (CInner) -- ++(\y+1,0)
		(C) -- ++(1,0);
		\draw[dashed] (BInner) -- ++(\y+1,0)
		(B) -- ++(1,0);
		\draw[dashed] (BInner) -- ++(0,-\y-1)
		(B) -- ++(0,-1);
		\draw[dashed] (AInner) -- ++(0,-\y-1)
		(A) -- ++(0,-1);

		\draw[thick,<->] ($(CInner) + (\y+1,0)$) -- node[right] {\correction{$\y$cm}} ($(C) + (1,0)$);
		\draw[thick,<->] ($(BInner) + (0,-\y-1)$) -- node[below] {\correction{$\y$cm}} ($(B) + (0,-1)$);
		\draw[thick,<->] ($(BInner) + (\y+1,0)$) -- node[right] {\correction{$\y$cm}} ($(B) + (1,0)$);
		\draw[thick,<->] ($(AInner) + (0,-\y-1)$) -- node[below] {\correction{$\y$cm}} ($(A) + (0,-1)$);
	\end{tikzpicture}
\end{center}

\begin{enumerate}
	\item[3.] Dans quelle zone doit être le centre du jeton pour qu'il y ai un franc-carreau ? Pourquoi ?

		\dotfill

		\dotfill

		\dotfill
	\item[4.] En déduire la probabilité d'avoir un franc-carreau. Retrouve-t'on le résultat de la partie 1 ?

		\correctionOr{$4^2 / 8^2 = 0,25$}{\dotfill}

		\dotfill

		\dotfill
	\item[5.] Quelle serait cette probabilité si la grille faisait $6$cm de côté ?

		\correctionOr{$4^2 / 6^2 = 0,4444$}{\dotfill}

		\dotfill

		\dotfill
\end{enumerate}

\end{document}