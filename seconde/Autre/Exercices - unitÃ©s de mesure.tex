\documentclass[10pt]{automatisme}

\renewcommand{\arraystretch}{1.3}

\begin{document}

\begin{frame}
	Un employé doit envoyer un rapport par courrier. Ce rapport est composé de $40$ feuilles, qu'il mettra dans une seule grande enveloppe.

	Il cherche donc à savoir quel prix il va devoir payer pour l'envoi de ce rapport. \bigskip

	\begin{minipage}{0.45\textwidth}
		Grille tarifaire donnée par la poste :
		\begin{center}
			\begin{tabular}{|c|c|}
				\hline
				Poids jusqu'à ... & Tarif   \\ \hline
				$20$g             & $1,42$€ \\ \hline
				$100$g            & $2,39$€ \\ \hline
				$250$g            & $4,33$€ \\ \hline
				$0,5$kg           & $6,27$€ \\ \hline
				$2$kg et plus     & $8,21$€ \\ \hline
			\end{tabular}
		\end{center}
	\end{minipage}\hspace{0.015\textwidth}\vrule\hspace{0.015\textwidth}
	\begin{minipage}{0.45\textwidth}
		Informations réunies par l'employé
		\begin{center}
			\begin{itemize}
				\item Masse d'un paquet de $20$ enveloppes : $2,6$hg
				\item Dimension d'une feuille A4 : $21$cm de largeur, $29,7$cm de hauteur
				\item Grammage d’une feuille A4 : $80$g/m²
			\end{itemize}
		\end{center}
	\end{minipage}
\end{frame}

\end{document}