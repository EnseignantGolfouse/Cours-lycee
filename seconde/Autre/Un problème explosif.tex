\documentclass[
	classe=$2^{de}$,
	headerTitle=Problème,
	landscape,
	twocolumn
]{exercice}

\usepackage{préambule}
\usetikzlibrary{positioning}
\usepackage{clipboard}

\setlength{\columnsep}{1cm}

\begin{document}

\Copy{Explosion}{
	\begin{center}
		\LARGE

		Un problème explosif
	\end{center}

	\notebox{\small
		Ce problème est tiré du TFJM², une compétition mathématique pour lycéen⋅nes.
	}

	\vspace{1em}

	L'agent secret 1234 doit désamorcer une bombe. Mais il y a un souci : la bombe est protégé par un cadenas à chiffres !

	\begin{itemize}
		\item Le cadenas possède 4 roues, et une seule combinaison peut l'ouvrir.
		\item Chaque roue a des chiffres de 0 à 9.
		\item On ne peut tourner une roue que dans un sens : 0 → 1 → 2 → ⋯ → 9 → 0 → ⋯.
		\item ATTENTION ! Si on rentre une combinaison déjà essayée, la bombe explose automatiquement !
	\end{itemize}

	Lorsque l'agent arrive, le cadenas est sur la position 0-0-0-0.

	Son but est donc de tester le plus de combinaisons possibles, en espérant tomber sur la bonne.

	\begin{enumerate}
		\item On suppose dans cette question que les trois derniers chiffres de la combinaison sont 0.

		      Comment l'agent doit-il procéder pour désamorcer la bombe ?
		\item Comment doit-il procéder si il sait que les deux derniers chiffres de la combinaison sont 0 ?
		\item Peut-il désamorcer la bombe à coup sûr si il ne connait aucun chiffre ?
		\item On suppose maintenant que l'agent n'a pas le droit de faire tourner deux fois de suite la même roue : il doit changer de roue à chaque fois. Peut-il encore désamorcer la bombe ? Comment ?
	\end{enumerate}
}

\newpage

\Paste{Explosion}

\end{document}