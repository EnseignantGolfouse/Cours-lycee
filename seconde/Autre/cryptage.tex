\documentclass{beamer}

\renewcommand{\familydefault}{\sfdefault}
\renewcommand{\arraystretch}{1.3}

\title{Cryptage RSA}

\begin{document}

\begin{frame}{Chiffrement de César}
	On ajoute le même nombre à chaque lettre du message.

	Par exemple, si le nombre choisit est $5$, la lettre $B$ devient $G$.
\end{frame}

\begin{frame}{Chiffre de Vigenère}
	On choisit un mot qui nous servira de \textbf{clé}.

	Puis, on code chaque lettre du message ainsi :

	\begin{center}
		\begin{tabular}{|l|c|c|c|c|c|c|c|}
			\hline
			Clé      & C         & L         & E & C & L & E & C \\ \hline
			Message  & M         & E         & S & S & A & G & E \\ \hline
			Résultat & M + C = P & E + L = Q & X & V & M & L & H \\ \hline
		\end{tabular}
	\end{center}
\end{frame}

\begin{frame}{Chiffrage RSA}
	% Inventé en 1977 pour l'échange d'information sur internet.
	On pose :
	\begin{itemize}
		\item $p = 5$, $q = 7$.
		\item $n = p \times q$, $φ = (p - 1) × (q - 1)$.
		      % n = 35, φ = 24
		\item $e = 7$.
	\end{itemize}

	\begin{enumerate}
		\item Trouver un nombre $d$ tel que le reste de la division euclidienne de $d \times e$ par $φ$ soit $1$ (Utiliser la calculatrice). % d = 7
		\item Pour \textbf{Chiffrer} un message :

		      la lettre $l$ devient le reste de la division euclidienne de $l^e$ par $n$.
		\item Pour \textbf{Déchiffrer} un message :

		      la lettre $l$ devient le reste de la division euclidienne de $l^d$ par $n$.
	\end{enumerate}
\end{frame}

\end{document}