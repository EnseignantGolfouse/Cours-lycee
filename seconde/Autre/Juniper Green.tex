\documentclass{beamer}

\usepackage{préambule}

\begin{document}

\begin{frame}
	\begin{minipage}{0.45\textwidth}
		\begin{LARGE}
			\begin{center}
				\begin{TAB}(r,0cm,0cm)[0pt,5cm,5cm]{|c|c|c|c|c|}{|c|c|c|c|}
					1 & 2 & 3 & 4 & 5 \\
					6 & 7 & 8 & 9 & 10 \\
					11 & 12 & 13 & 14 & 15 \\
					16 & 17 & 18 & 19 & 20 \\
				\end{TAB}
			\end{center}
		\end{LARGE}
	\end{minipage}
	\begin{minipage}{0.45\textwidth}
		\uline{\textbf{Règles :}}
		\begin{itemize}
			\item À chaque tour, on sélectionne un nombre qui n'a pas encore été sélectionné.
			\item Le nombre doit être un \textbf{multiple} ou \textbf{diviseur} du nombre précédent. On commence par le nombre de son choix.
			\item La partie s'arrête lorsqu'on ne peut plus jouer de coup.
			\item Le but est d'avoir sélectionné un maximum de nombres à la fin.
		\end{itemize}
	\end{minipage}
\end{frame}
% Exemple

\end{document}