\documentclass[
	classe=$2^{de}$,
	landscape,
	twocolumn,
	headerTitle={Interrogation}
]{évaluation}

\setlength{\columnsep}{1cm}
\renewcommand{\arraystretch}{1.3}

\date{9 décembre 2022}

\begin{document}

%%%%%%%%%%%%%%%%%%%%%%%%%%%%%%%%%%%%%%%%%%%%%%%%%
% SUJET A
%%%%%%%%%%%%%%%%%%%%%%%%%%%%%%%%%%%%%%%%%%%%%%%%%

\title{Interrogation : proportions et pourcentages (sujet A)}
\maketitle

\begin{exercice}
	\begin{center}
		\begin{tikzpicture}
			\draw (0,0) ellipse (4 and 1.5);
			\node at (-1,0.5) {×};
			\node at (-1.6,0.7) {×};
			\node at (-0.1,-0.75) {×};
			\node at (2,-0.2) {×};
			\node at (-1.5,-0.7) {×};
			\node at (-2.5,0.5) {×};
			\node at (1.68,0.3) {×};
			\node at (0.8,-1.25) {×};

			\node at (1,0.2) {{\Large ∘}};
			\node at (-1,-0.5) {{\Large ∘}};
			\node at (0.9,-0.6) {{\Large ∘}};
			\node at (-2,0.2) {{\Large ∘}};
			\node at (0.2,0.75) {{\Large ∘}};
			\node at (0,0.1) {{\Large ∘}};
			\node at (-2.3,-0.5) {{\Large ∘}};
			\node at (2.3,0.35) {{\Large ∘}};
			\node at (2.7,-0.2) {{\Large ∘}};
		\end{tikzpicture}
	\end{center}

	\begin{enumerate}
		\item Donner la proportion de croix dans la population ci-dessus : \correctionDots{$\dfrac{8}{17}$}
		\item Si l'image était en couleur, on verrait que la proportion de cercles bleus \textbf{parmi les cercles} est de $\dfrac{2}{3}$. Quelle est alors la proportion de cercles bleus dans la population globale ? \correctionDots{$\dfrac{2}{3} × \dfrac{9}{17} = \dfrac{6}{17}$}
	\end{enumerate}
\end{exercice}

\begin{exercice}
	Calculer les évolutions suivantes :
	\begin{enumerate}
		\item $100$ augmenté de $80\%$ : \correctionDots{$180$}
		\item $70$ augmenté de $30\%$ : \correctionDots{$91$}
		\item $120$ diminué de $25\%$ : \correctionDots{$90$}
		\item $12$ augmenté de $150\%$, puis diminué de $60\%$ : \correctionDots{$12$}
	\end{enumerate}
\end{exercice}

\begin{exercice}
	Compléter le tableau ci-dessous :
	\begin{center}
		\begin{tabular}{|c|c|c|c|}
			\hline
			départ & arrivée & variation absolue   & variation relative   \\ \hline
			$40$   & $60$    & \correction{$20$}   & \correction{$0,5$}   \\ \hline
			$200$  & $150$   & \correction{$-50$}  & \correction{$-0,25$} \\ \hline
			$75$   & $300$   & \correction{$225$}  & \correction{$3$}     \\ \hline
			$300$  & $30$    & \correction{$-270$} & \correction{$-0,9$}  \\ \hline
		\end{tabular}
	\end{center}
\end{exercice}

\newpage
\setcounter{exercice}{1}
%%%%%%%%%%%%%%%%%%%%%%%%%%%%%%%%%%%%%%%%%%%%%%%%%
% SUJET B
%%%%%%%%%%%%%%%%%%%%%%%%%%%%%%%%%%%%%%%%%%%%%%%%%

\title{Interrogation : proportions et pourcentages (sujet B)}
\maketitle

\begin{exercice}
	\begin{center}
		\begin{tikzpicture}
			\draw (0,0) ellipse (4 and 1.5);
			\node at (-1,0.5) {×};
			\node at (-1.6,0.7) {×};
			\node at (-0.1,-0.75) {×};
			\node at (2,-0.2) {×};
			\node at (-1.5,-0.7) {×};

			\node at (-2.5,0.5) {{\Large ∘}};
			\node at (1.68,0.3) {{\Large ∘}};
			\node at (0.8,-1.25) {{\Large ∘}};
			\node at (1,0.2) {{\Large ∘}};
			\node at (-1,-0.5) {{\Large ∘}};
			\node at (0.9,-0.6) {{\Large ∘}};
			\node at (-2,0.2) {{\Large ∘}};
			\node at (0.2,0.75) {{\Large ∘}};
			\node at (0,0.1) {{\Large ∘}};
			\node at (-2.3,-0.5) {{\Large ∘}};
			\node at (2.3,0.35) {{\Large ∘}};
			\node at (2.7,-0.2) {{\Large ∘}};
		\end{tikzpicture}
	\end{center}

	\begin{enumerate}
		\item Donner la proportion de croix dans la population ci-dessus : \correctionDots{$\dfrac{5}{17}$}
		\item Si l'image était en couleur, on verrait que la proportion de cercles bleus \textbf{parmi les cercles} est de $\dfrac{2}{3}$. Quelle est alors la proportion de cercles bleus dans la population globale ? \correctionDots{$\dfrac{2}{3} × \dfrac{12}{17} = \dfrac{8}{17}$}
	\end{enumerate}
\end{exercice}

\begin{exercice}
	Calculer les évolutions suivantes :
	\begin{enumerate}
		\item $100$ augmenté de $70\%$ : \correctionDots{$170$}
		\item $70$ augmenté de $40\%$ : \correctionDots{$108$}
		\item $120$ diminué de $35\%$ : \correctionDots{$78$}
		\item $15$ augmenté de $150\%$, puis diminué de $60\%$ : \correctionDots{$15$}
	\end{enumerate}
\end{exercice}

\begin{exercice}
	Compléter le tableau ci-dessous :
	\begin{center}
		\begin{tabular}{|c|c|c|c|}
			\hline
			départ & arrivée & variation absolue   & variation relative   \\ \hline
			$60$   & $90$    & \correction{$30$}   & \correction{$0,5$}   \\ \hline
			$400$  & $300$   & \correction{$-200$} & \correction{$-0,25$} \\ \hline
			$125$  & $500$   & \correction{$375$}  & \correction{$3$}     \\ \hline
			$800$  & $80$    & \correction{$-720$} & \correction{$-0,9$}  \\ \hline
		\end{tabular}
	\end{center}
\end{exercice}

\end{document}