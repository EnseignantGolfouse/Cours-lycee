\documentclass[
	classe=$2^{de}$,
	landscape,
	twocolumn
]{coursclass}

\setlength{\columnsep}{1cm}

\begin{document}

\newcommand{\Cours}{
	\begin{definition}[Moyenne, moyenne pondérée]
		Si on dispose d'une série de valeurs $x₁, ⋯, xₙ$,
		\begin{itemize}
			\item on peut calculer leur \textbf{moyenne} :
			      $$ M = \dfrac{x₁ + ⋯ + xₙ}{n} $$
			\item La moyenne peut être pondérée, c'est-à-dire que chaque valeur est multipliée par un coefficient $cᵢ$ :
			      $$ M = \dfrac{c₁ × x₁ + ⋯ + cₙ × xₙ}{c₁ + ⋯ + cₙ} $$
		\end{itemize}
	\end{definition}

	\begin{exemple}
		La moyenne de la série de notes $8 ; 11 ; 12 ; 17$ est
		$$ M = \dfrac{\correction{8 + 11 + 12 + 17}}{\correction{4}} = \correction{\dfrac{48}{4} = 12} $$

		Si la quatrième note (le $17$) était coefficient $2$, et que toutes les autres notes sont coefficient $1$, la moyenne devient
		$$ M = \dfrac{\correction{8 + 11 + 12 + 2 × 17}}{\correction{1 + 2 + 1 + 1}} = \correction{\dfrac{65}{5} = 13} $$
	\end{exemple}

	\begin{propriete}[Linéarité de la moyenne]
		\begin{itemize}
			\item Si on ajoute le même nombre $a$ à chaque valeur, la moyenne

			      augmente de \correctionDots{$a$}.
			\item Si on multiplie chaque valeur par un nombre $b$, la moyenne est

			      multipliée par \correctionDots{$b$}.
		\end{itemize}
	\end{propriete}
}

\Cours

\newpage
\Cours

\end{document}