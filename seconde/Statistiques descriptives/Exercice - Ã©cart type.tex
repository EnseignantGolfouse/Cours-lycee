\documentclass{beamer}

\usepackage{préambule}

\begin{document}

\begin{frame}
	\textbf{Recopier l'énoncé (y compris le tableau), et répondre aux questions dans son cahier :} \vspace{1em}

	\onslide<2>{
		On a demandé aux employés s'une entreprise leur temps de trajet depuis leur domicile jusqu'à leur lieu de travail. Leur réponses ont étés placées dans le tableau suivant :
		\begin{center}
			\begin{tabular}{|l|*{5}{>{\centering}p{0.7cm}|}}
				\hline
				Temps (en minutes) & 5  & 10 & 20 & 30 & 40 \tabularnewline \hline
				Effectif           & 10 & 15 & 5  & 35 & 50 \tabularnewline \hline
			\end{tabular}
		\end{center}

		\begin{enumerate}
			\item Calculer la moyenne $\overline{x}$ du temps de trajet des employés, au dixième près. \correctionOr{$29,1$ minutes}{}
			\item Déterminer l'écart-type $σ$ de cette série, au dixième près. \correctionOr{$12,4$ minutes}{}
			\item Calculer le pourcentage d'employés dont le temps de trajet appartient à l'intervalle $[\overline{x} - σ ; \overline{x} + σ]$.

			      \correctionOr{$[\overline{x} - σ ; \overline{x} + σ] ≈ [16,7 ; 41,5]$, ce qui comprend les employés à $20$, $30$ et $40$ minutes, soit $80$ employés sur un total de $115$, soit $69,6\%$}{}
		\end{enumerate}
	}
\end{frame}

\end{document}