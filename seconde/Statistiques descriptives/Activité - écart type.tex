\documentclass[
	classe=$2^{de}$,
	headerTitle=Activité,
	landscape,
	twocolumn
]{exercice}

\renewcommand{\arraystretch}{1.3}
\setlength{\columnsep}{1cm}

\title{Activité : écart-type}

\begin{document}

\newcommand{\Activite}{
	\maketitle

	On imagine un groupe de $4$ personnes, dans lequel l'âge moyen est $70$ ans.

	\begin{enumerate}
		\item Une répartition possible des tailles dans ce groupe est :
		      \begin{itemize}
			      \item[a.] $70$ ans ; $70$ ans ; $70$ ans ; $70$ ans (tout le monde a le même âge)
		      \end{itemize}

		      Donner une autre répartition possible des âges du groupe :
		      \begin{itemize}
			      \item[b.] Si seulement deux personnes ont le même âge : \correction{$70$ ; $70$ ; $71$ ; $69$}
			      \item[c.] Si toutes les personnes ont des âges différents : \correction{$68$ ; $69$ ; $71$ ; $72$}
			      \item[d.] Si une personne a $2$ ans, et une autre $78$ : \correction{$2$ ; $78$ ; $100$ ; $100$}
		      \end{itemize}
		\item On cherche maintenant à savoir si le groupe est plus ou moins homogène (c'est-à-dire si les personnes ont des âges similaires).

		      Pour cela, on va utiliser la \textbf{variance} :

		      Choisir une des répartitions obtenues dans la question 1 : \correctionDots{c.}
		      \begin{itemize}
			      \item Pour chaque personne du groupe, calculer l'écart entre son âge et l'âge moyen :

			            \correction{$2$ans, $1$an, $2$ans et $1$an}
			      \item Mettre chacun des résultats obtenus au carré :

			            \correction{$4$, $1$, $4$ et $1$}
			      \item Faire la somme des résultats obtenus, et la diviser par le nombre de personnes (ici 4) :

			            \correction{$\dfrac{4 + 1 + 4 + 1}{4} = 2,5$ans²}
		      \end{itemize}
		\item Quelle est \uline{l'unité} du nombre ainsi obtenu ? \correction{des années au carré}
		\item Quelle opération doit-on faire pour obtenir des années ? \correction{une racine carré.}

		      On obtient alors l'\textbf{écart-type}.
		\item En reprenant ces étapes, calculer l'écart-type de chaque répartition possible au dixième d'année près :
		      \begin{multicols}{2}
			      \begin{itemize}
				      \item[a.] \correction{$0$}
				      \item[b.] \correction{$\sqrt{0,5} ≈ 0,7$ ans}
				      \item[c.] \correction{$\sqrt{2,5} ≈ 1,6$ ans}
				      \item[d.] \correction{$\sqrt{1622} ≈ 40,3$ ans}
			      \end{itemize}
		      \end{multicols}
	\end{enumerate}
}

\Activite

\ifdefined\makeCorrection
\else
	\newpage
	\Activite
\fi

\end{document}