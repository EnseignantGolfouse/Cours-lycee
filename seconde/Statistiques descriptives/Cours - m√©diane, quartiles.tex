\documentclass[
	classe=$2^{de}$,
	twocolumn,
	landscape
]{coursclass}

\setlength{\columnsep}{1.2cm}

\begin{document}

\newcommand{\Cours}{
	\begin{definition}[Médiane, quartiles]
		Si on a une série de valeurs $x₁,⋯,xₙ$ rangées dans l'ordre croissant, alors
		\begin{itemize}
			\item La \correctionDots{médiane} est la valeur telle que $50\%$ des valeurs sont supérieures ou égales, et $50\%$ des valeurs sont inférieures ou égales.
			\item Le \correctionDots{1\textsuperscript{er} quartile $Q₁$} est la plus petite valeur de la série telle qu'au moins $25\%$ des valeurs lui sont inférieures ou égales.
			\item Le \correctionDots{3\textsuperscript{e} quartile $Q₃$} est la plus petite valeur de la série telle qu'au moins $75\%$ des valeurs lui sont inférieures ou égales.
		\end{itemize}
	\end{definition}

	\begin{definition}[Écart interquartile]
		L'écart interquartile est la valeur $Q₃ - Q₁$.
	\end{definition}

	\begin{exemple}
		Si on a la série $5$ ; $12$ ; $12$ ; $14$ ; $16$ ; $21$ ; $22$ ; $23$, alors
		\begin{itemize}
			\item La médiane est \correctionOr{entre $14$ et $16$ : traditionnellement, on dira qu'elle vaut $15$.}{.....}
			\item Le premier quartile est \correctionDots{$12$}, le troisième quartile est \correctionDots{$21$}.
			\item L'écart interquartile est \correctionDots{$21 - 12 = 9$}.
		\end{itemize}
	\end{exemple}
}

\Cours

\newpage

\Cours

\end{document}