\documentclass[
	classe=$2^{de}$,
	headerTitle=Évaluation\space Chapitre\space 4
]{évaluation}

\usepackage{tcolorbox}

\renewcommand{\arraystretch}{1.4}

\title{Évaluation : statistiques descriptives (Sujet A)}
\date{20 janvier 2023}

\newcommand{\makeCorrection}{}
\begin{document}

\maketitle

\begin{exercice}\

	\begin{enumerate}
		\item La proportion de moustiques porteurs de maladie et sensibles à l'antimoustique est \correction{$0,12×0,8=0,096=9,6\%$}.
		\item Le nombre total de moustiques étudiés est alors \correction{$2400 ÷ 0,12 = 20\ 000$}.
	\end{enumerate}
\end{exercice}

\begin{exercice}\

	\begin{center}
		\begin{tabular}{|c|c|c|c|}
			\hline
			Valeur de départ & Valeur d'arrivée    & Variation absolue   & Variation relative   \\ \hline
			$100$            & $200$               & \correction{$100$}  & \correction{1}       \\ \hline
			$240$            & \correction{$798$}  & $558$               & \correction{$2,325$} \\ \hline
			$1000$           & \correction{$8100$} & \correction{$7100$} & $7,1$                \\ \hline
			$50$             & \correction{$10$}   & $-40$               & \correction{$-0,8$}  \\ \hline
		\end{tabular}
	\end{center}
\end{exercice}

\begin{exercice}\

	\begin{enumerate}
		\item La moyenne de cette série est

		      \correction{$\dfrac{2 + 3 + 4 + 4 + 5 + 7 + 12 + 13 + 13 + 15 + 16 + 17 + 17 + 18 + 19 + 20}{16} ≈ 11,56$}.
		\item La médiane est \correction{médiane: $13$}.

		      Le premier quartile est \correction{$Q₁ = 4$}.

		      Le troisième quartile est \correction{$Q₃ = 17$}.
		\item L'écart-type de cette série est

		      \correction{$\sqrt{\dfrac{(2 - 11,56)² + (3 - 11,56)² + (4 - 11,56)² + (4 - 11,56)² + ⋯ + (20 - 11,56)²}{16}} ≈ 6,15$}.
		\item Si toutes les notes ont baissée d'un point, la moyenne et la médiane ont également baissé d'un point. La moyenne de cet autre élève est donc de \correction{$10,56$}, et sa médiane est de \correction{$12$}.
	\end{enumerate}
\end{exercice}

\begin{exercice}\

	\begin{enumerate}
		\item Pour trouver le meilleur joueur, on peut calculer leur moyenne respectives. La moyenne de points du premier joueur est de \correction{$16,04$}, tandis que celle du second est de \correction{$15,52$}. Le premier joueur semble donc meilleur.
		\item Le joueur 1 a fait au moins 19 points lors de \correction{22} matchs sur 80 : il est donc accepté.

		      Le joueur 2 a fait au moins 19 points lors de \correction{12} matchs sur 80 : il est donc refusé.
		\item On va calculer l'écart-type de chaque joueur :

		      Joueur 1 : \correction{$\sqrt{\dfrac{11×(12 - 16,04)² + 12×(13 - 16,04)² + ⋯ + 12×(20-16,04)²}{80}} ≈ 2,82$}

		      Joueur 2 : \correction{$\sqrt{\dfrac{1×(12 - 16,04)² + 4×(13 - 16,04)² + ⋯ + 5×(20-16,04)²}{80}} ≈ 1,95$}

		      Il semble donc que le joueur 2 fasse des scores plus homogènes.
	\end{enumerate}
\end{exercice}

\begin{exercice}\

	\begin{enumerate}
		\item Avant le nouvel employé, la somme de tous les salaires est de : $2400×12 = 28800$

		      Après son arrivée, cette somme est de : $2500×13 = 32500€$.

		      Le nouveau salaire est donc de $32500-28800 = 3700€$.
		\item Avant le nouvel employé, la somme de tous les salaires est de : $2400×12 = 28800$

		      Après son arrivée, cette somme est de : $(2400×1,02)×13 = 31824$.

		      Le nouveau salaire est donc de $31824-28800 = 3024€$.
	\end{enumerate}
\end{exercice}

%============================================
%=============== SUJET B ====================
%============================================
\newpage
\title{Évaluation : statistiques descriptives (Sujet B)}
\setcounter{exercice}{1}

\maketitle

\begin{exercice}\

	\begin{enumerate}
		\item La proportion de moustiques porteurs de maladie et sensibles à l'antimoustique est \correction{$0,14×0,85=0,119=11,9\%$}.
		\item Le nombre total de moustiques étudiés est alors \correction{$2100 ÷ 0,14 = 15\ 000$}.
	\end{enumerate}
\end{exercice}

\begin{exercice}\

	\begin{center}
		\begin{tabular}{|c|c|c|c|}
			\hline
			Valeur de départ & Valeur d'arrivée     & Variation absolue    & Variation relative   \\ \hline
			$150$            & $300$                & \correction{$150$}   & \correction{1}       \\ \hline
			$320$            & \correction{$848$}   & $528$                & \correction{$1,65$}  \\ \hline
			$2000$           & \correction{$12600$} & \correction{$10600$} & $5,3$                \\ \hline
			$80$             & \correction{$20$}    & $-60$                & \correction{$-0,75$} \\ \hline
		\end{tabular}
	\end{center}
\end{exercice}

\begin{exercice}\

	\begin{enumerate}
		\item La moyenne de cette série est

		      \correction{$\dfrac{2 + 4 + 5 + 5 + 7 + 7 + 12 + 13 + 13 + 15 + 16 + 16 + 17 + 18 + 19 + 20}{16} ≈ 11,81$}.
		\item La médiane est \correction{médiane: $13$}.

		      Le premier quartile est \correction{$Q₁ = 5$}.

		      Le troisième quartile est \correction{$Q₃ = 16$}.
		\item L'écart-type de cette série est

		      \correction{$\sqrt{\dfrac{(2 - 11,81)² + (4 - 11,81)² + (4 - 11,81)² + ⋯ + (20 - 11,81)²}{16}} ≈ 6,15$}.
		\item Si toutes les notes ont baissée d'un point, la moyenne et la médiane ont également baissé d'un point. La moyenne de cet autre élève est donc de \correction{$10,81$}, et sa médiane est de \correction{$12$}.
	\end{enumerate}
\end{exercice}

\begin{exercice}\

	\begin{enumerate}
		\item Pour trouver le meilleur joueur, on peut calculer leur moyenne respectives. La moyenne de points du premier joueur est de \correction{$16,04$}, tandis que celle du second est de \correction{$15,52$}. Le premier joueur semble donc meilleur.
		\item Le joueur 1 a fait au moins 19 points lors de \correction{22} matchs sur 80 : il est donc accepté.

		      Le joueur 2 a fait au moins 19 points lors de \correction{12} matchs sur 80 : il est donc refusé.
		\item On va calculer l'écart-type de chaque joueur :

		      Joueur 1 : \correction{$\sqrt{\dfrac{11×(12 - 16,04)² + 12×(13 - 16,04)² + ⋯ + 12×(20-16,04)²}{80}} ≈ 2,82$}

		      Joueur 2 : \correction{$\sqrt{\dfrac{1×(12 - 16,04)² + 4×(13 - 16,04)² + ⋯ + 5×(20-16,04)²}{80}} ≈ 1,95$}

		      Il semble donc que le joueur 2 fasse des scores plus homogènes.
	\end{enumerate}
\end{exercice}

\begin{exercice}\

	\begin{enumerate}
		\item Avant le nouvel employé, la somme de tous les salaires est de : $2600×12 = 31200$

		      Après son arrivée, cette somme est de : $2700×13 = 35100€$.

		      Le nouveau salaire est donc de $35100-31200 = 3900€$.
		\item Avant le nouvel employé, la somme de tous les salaires est de : $2600×12 = 31200$

		      Après son arrivée, cette somme est de : $(2600×1,02)×13 = 34476$.

		      Le nouveau salaire est donc de $34476-31200 = 3276€$.
	\end{enumerate}
\end{exercice}

\end{document}