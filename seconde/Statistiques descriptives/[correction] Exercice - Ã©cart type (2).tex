\documentclass{exercice}

\newcommand{\makeCorrection}{}
\begin{document}

Trois personnes gagnent respectivement $1500$, $2000$ et $3400$ euros par mois.
\begin{itemize}
	\item Le revenu moyen est : \correction{$(1500 + 2000 + 3200)/3 = 2300$}
	\item Pour calculer la variance, on doit écrire :

	      \correction{$((1500 - 2300)² + (2000 - 2300)² + (3200 - 2300)²) / 3$}

	      \correction{$ ≈ 646 666,66$}
	\item Pour calculer l'écart-type (au centime près), on doit écrire :

	      \correction{$√646 666 ≈ 804,15€$}
\end{itemize}



Dans un magasin, on trouve 5 pommes, dont les poids sont :

$110$g ; $110$g ; $123$g ; $123$g ; $130$g
\begin{itemize}
	\item Le poids moyen est : \correction{$(110 + 110 + 123 + 123 + 130)/5 = 119,2$g}
	\item Pour calculer la variance, on doit écrire :

	      \correction{$((110 - 126,5)² + (110 - 126,5)² + (123 - 126,5)² + (123 - 126,5)² + (130 - 126,5)²) / 5$}

	      \correction{$ ≈ 62,96$g²}
	\item Pour calculer l'écart-type (au dixième de gramme près), on doit écrire :

	      \correction{$√62,96 ≈ 7,9$g}
\end{itemize}



Dans un groupe de 10 personnes, 6 mesurent 170 cm, et 4 mesurent 175 cm.
\begin{itemize}
	\item La taille moyenne est : \correction{$(6×170 + 4×175)/10 = 1720/10 = 172$}
	\item Pour calculer la variance, on doit écrire :

	      \correction{$(6 × (170 - 172)² + 4 × (175 - 172)²) / 10$}

	      \correction{$ = 6$}
	\item Pour calculer l'écart-type (au millimètre près), on doit écrire :

	      \correction{$√6 ≈ 2,4$cm}
\end{itemize}

\end{document}