\documentclass{beamer}

\usepackage{préambule-cours}

\setbeamertemplate{itemize item}{\bullet}
\setbeamersize{
	text margin left=0.5cm,
	text margin right=0.7cm
}
\setlength{\leftmargini}{1.5em}

\begin{document}

\begin{frame}
	\begin{definition}[Population]
		\begin{itemize}
			\item Une \textbf{population} est un ensemble d'éléments, appelés les \textbf{individus}.
			\item Une \textbf{sous-population} est une partie de la population.
			\item Le nombre total d'individus dans la population est appelé l'\textbf{effectif total}.
		\end{itemize}
	\end{definition}

	\begin{exemple}
		Dans la population des élèves de la classe, l'effectif total est ...... .

		Les élèves présent dans ce groupe forment une sous-population, d'effectif ...... .
	\end{exemple}
\end{frame}

\begin{frame}
	\begin{remarque}
		Les individus d'une population ne sont pas toujours des personnes.

		Par exemple, on peut parler de la \textit{population} d'une trousse, dont les \textit{individus} sont les stylos, et une \textit{sous-population} est formée par les stylos rouges.
	\end{remarque}
\end{frame}

\begin{frame}
	\begin{definition}[Proportion]
		On considère une population dont l'effectif total est $N$, et une sous-population dont l'effectif est $n$.
		\begin{itemize}
			\item La \textbf{proportion} d'individus dans la sous-population est $p = \dfrac{n}{N}$.
			\item On peut exprimer cette proportion en pourcentage, en la multipliant par $100$ :

			      $\left(\dfrac{n}{N}×100\right) \%$ des individus sont dans la sous-population.
		\end{itemize}
	\end{definition}

	\begin{exemple}
		La proportion des élèves dans ce groupe est $\dfrac{...}{...} ≈ ......$,

		ou ...... \%.
	\end{exemple}
\end{frame}

\begin{frame}
	\begin{remarque}
		Prendre $x\%$ d'une valeur revient à la multiplier par $\dfrac{x}{100}$.
	\end{remarque}
\end{frame}

\end{document}