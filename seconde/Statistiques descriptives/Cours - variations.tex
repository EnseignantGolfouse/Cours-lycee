\documentclass{beamer}

\usepackage{préambule-cours}

\begin{document}

\begin{frame}
	\begin{definition}[Variations]
		Lorsqu'on passe d'une valeur $V₁$ à une valeur $V₂$, on dit qu'il s'agit d'une \textbf{évolution}. On a alors :
		\begin{itemize}
			\item $V₂ - V₁$ est la \textbf{variation absolue}.
			\item $\dfrac{V₂ - V₁}{V₁}$ est la \textbf{variation relative} \onslide<2>{, aussi appelée le \textbf{taux d'évolution}.}
		\end{itemize}
	\end{definition}

	\begin{exemple}
		Une personne ayant $1\ 000\ 000$ d'euros gagne $1\ 000\ 000$ €.

		\begin{itemize}
			\item la variation absolue est de $1\ 000\ 000$ €.
			\item la variation relative est de $\dfrac{1\ 000\ 000}{100\ 000\ 000} = 0,01$, ou $1\%$.
		\end{itemize}
	\end{exemple}
\end{frame}

\end{document}