\documentclass[
	classe=$2^{de}$,
]{exercice}

\renewcommand{\arraystretch}{1.3}

\title{Exercices : Variations}

\begin{document}

\maketitle

Voici le prix d'un médicament (en €) entre 2016 et 2019 :
\begin{center}
	\begin{tabular}{|l|c|}
		\hline
		     & Prix du médicament \\ \hline
		2016 & $13$ €             \\ \hline
		2017 & $15$ €             \\ \hline
		2018 & $18$ €             \\ \hline
		2019 & $23$ €             \\ \hline
	\end{tabular}
\end{center}

\begin{exercice}
	\begin{enumerate}
		\item Calculer les taux d'évolution (en \%) du prix du médicament (arrondi au dixième) :
		      \begin{enumerate}
			      \item De 2016 à 2017 ; \correction{$\dfrac{2}{13} ≈ 15,4$ \%}
			      \item De 2017 à 2018 ; \correction{$\dfrac{3}{15} = 20$ \%}
			      \item De 2018 à 2019 \correction{$\dfrac{5}{18} ≈ 27,8$ \%}
		      \end{enumerate}
		\item Quel est le pourcentage d'évolution de 2016 à 2019 ? \correction{$\dfrac{10}{23} ≈ 76,9$ \%}
	\end{enumerate}
\end{exercice}

\begin{minipage}{0.47\textwidth}
	\begin{exercice}
		{\large\uline{Étude d'une augmentation}\vspace{0.5em}}

		Un salarié rémunéré 1500 € par mois va être augmenté de $8\%$.
		Quel sera son nouveau salaire ?

		\vspace{1em}\textbf{Méthode vue au collège} : tableau de proportionnalité\vspace{0.5em}

		\begin{tabular}{|l|l|l|}
			\hline
			Salaire      & \hspace{2em} 100 & \hspace{1em} \correction{1 500} \\ \hline
			Augmentation & \correction{8}   & \correction{120}                \\ \hline
		\end{tabular}
		\vspace{0.7em}

		Augmentation = \correction{120 €}
		\begin{align*}
			\text{Nouveau salaire} & = \text{Ancien salaire} + \text{Augmentation} \\
			                       & = \correction{1620 €}
		\end{align*}

		\vspace{1em}\textbf{Méthode plus rapide (lycée)} :
		\begin{align*}
			\text{Nouveau salaire} & = \text{Ancien salaire} + \text{Augmentation} \\
			                       & =  \correction{1500 + 0,08 × 1500}            \\
			                       & = \correction{1620 €}
		\end{align*}

		\begin{tabularx}{\linewidth}{|X|}
			\hline
			\textbf{Bilan} : pour augmenter un nombre de $8\%$, \\ il suffit de \\
			\correctionDots{le multiplier par $1,08$.}
			\\ \hline
		\end{tabularx}
	\end{exercice}
\end{minipage}
\hfill\vline\hfill
\begin{minipage}{0.45\textwidth}
	\begin{exercice}{\large\uline{Étude d'une diminution}\vspace{0.5em}}

		Une veste au prix initial de 180 € va être soldée de $15\%$. Quel sera le prix soldé ?

		\vspace{1em}\textbf{Méthode vue au collège} : tableau de proportionnalité\vspace{0.5em}

		\begin{tabular}{|l|l|l|}
			\hline
			Prix initial           & \hspace{2em} 100 & \hspace{2em} \correction{180} \\ \hline
			Diminution (Réduction) & \correction{15}  & \correction{27}               \\ \hline
		\end{tabular}
		\vspace{0.7em}

		Réduction = \correction{27 €}
		\begin{align*}
			\text{Prix soldé} & = \text{Prix initial} - \text{Réduction} \\
			                  & = \correction{153 €}
		\end{align*}

		\vspace{1em}\textbf{Méthode plus rapide (lycée)} :
		\begin{align*}
			\text{Prix soldé} & = \text{Prix initial} - \text{Réduction} \\
			                  & =  \correction{180 - 0,15 × 180}         \\
			                  & = \correction{153 €}
		\end{align*}

		\begin{tabularx}{\linewidth}{|X|}
			\hline
			\textbf{Bilan} : pour diminuer un nombre de $15\%$, \\ il suffit de \\
			\correctionDots{le multiplier par $0,85$.\hfill}
			\\ \hline
		\end{tabularx}
	\end{exercice}
\end{minipage}

\vspace{1em}

\begin{exercice}
	{\large\uline{Succession d'augmentations}\vspace{0.5em}}

	Un compte épargne propose un taux d'augmentation annuel de $2\%$. On décide d'y déposer 22 000 €. Quel sera la somme disponible sur le compte dans 5 ans ?

	\begin{tabular}{|l|c|c|c|c|c|c|}
		\hline
		Année                   & 2022   & 2023                & 2024                  & 2025                  & 2026                  & 2027                  \\ \hline
		Somme disponible (en €) & 22 000 & \correction{22 440} & \correction{22 888,8} & \correction{23 346,5} & \correction{23 813,5} & \correction{24 289,7} \\ \hline
	\end{tabular}

	\vspace{1em}
	\begin{tabularx}{\linewidth}{|X|}
		\hline
		\vspace{0.1em}
		\textbf{Bilan} : pour augmenter 5 fois de suite un nombre de $2\%$,  il suffit de  \correctionDots{multiplier par $1.02^5$}
		\vspace{0.3em}
		\\ \hline
	\end{tabularx}
\end{exercice}

\end{document}