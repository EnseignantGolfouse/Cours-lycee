\documentclass{beamer}
\usepackage{préambule}

\renewcommand{\arraystretch}{1.4}

\begin{document}

\begin{frame}
	Deux bibliothèques vendent des livres. Elles ont compilé les prix de leur livres dans le tableau suivant :


	\begin{center}
		\begin{tabular}{|l|c|c|c|c|}
			\hline
			Prix                             & $5$€  & $7$€  & $10$€ & $16$€ \\ \hline
			Effectifs dans la bibliothèque 1 & $58$  & $213$ & $29$  & $165$ \\ \hline
			Effectifs dans la bibliothèque 2 & $197$ & $76$  & $110$ & $42$  \\ \hline
		\end{tabular}
	\end{center}

	\begin{enumerate}
		\item Reproduire le tableau.
		\item Combien de livres à $7$€ la bibliothèque $1$ vend-elle ? \correction{$213$}
		\item Combien de livres au total sont vendu dans chaque bibliothèque ?
		      \correctionOr{$465$ et $425$}{}
		\item Dans quelle bibliothèque le prix moyen est-il le moins élevé ?
		      \correctionOr{$10,1$ et $7,7$}{}
		\item Déterminer l'écart-type des prix pour chaque bibliothèque, au centime près.
		      \correctionOr{$4,47$€ et $3,41$€}{}
	\end{enumerate}
\end{frame}

\end{document}