\documentclass[
	classe=$2^{de}$
]{coursclass}

\title{Chapitre 4 : Statistiques descriptives}
\author{}
\date{}

\begin{document}

\maketitle

\begin{definition}[pourcentage]
	\begin{itemize}
		\item Prendre $x\%$ d'une valeur revient à la multiplier par $\dfrac{x}{100}$.
		\item Augmenter une valeur par $x\%$ revient à la multiplier par $1 + \dfrac{x}{100}$.
		\item Diminuer une valeur par $x\%$ revient à la multiplier par $1 - \dfrac{x}{100}$.
	\end{itemize}
\end{definition}

\begin{exemple}
	Augmenter 5 000 de 20\% revient à calculer
	$$ 5\ 000 × \left(1 + \frac{20}{100}\right) = 5\ 000 × 1,20 = 6\ 000 $$
\end{exemple}

\begin{definition}[vocabulaire]
	\begin{itemize}
		\item Une \textbf{évolution} est une augmentation ou une diminution.
		\item Si l'évolution est exprimée en pourcentage, le pourcentage est appelé le \textbf{taux d'évolution}.
		\item Lorsqu'on multiplie une valeur $v$ par un nombre $c$ pour obtenir une nouvelle valeur $v'$ ($v × c = v'$), on dit que $c$ est le \textbf{coefficient multiplicateur}.
	\end{itemize}
\end{definition}

\begin{exemple}
	Si on a une augmentation de 15\%, le coefficient multiplicateur est $1 + \frac{15}{100} = 1,15$.

	Si on a une diminution de 6\%, le coefficient multiplicateur est $1 - \frac{6}{100} = 0,94$.
\end{exemple}

\begin{greybox}[frametitle={Remarque}]
	Si le coefficient d'une évolution est \textit{supérieur à $1$}, c'est une augmentation. Sinon, c'est une diminution.
\end{greybox}

\begin{propriete}[Évolutions successives et coefficient global]
	Lorsqu'on applique plusieurs évolutions successives, on obtient le \textbf{coefficient global} en multipliant les coefficients.
\end{propriete}

\begin{exemple}
	Si on applique une augmentation de 20\%, suivie d'une diminution de 20\%, l'évolution a pour coefficient global
	$$ \left(1 + \frac{20}{100}\right) × \left(1 - \frac{20}{100}\right) = 1,2 × 0,8 = 0,96 $$
	On a donc globalement une diminution.
\end{exemple}

\begin{propriete}[Évolution réciproque]
	Pour revenir à la valeur initiale avant une évolution de coefficient $c$, on doit \textit{diviser} par $c$.

	Cette nouvelle évolution est appelée \textbf{l'évolution réciproque}, et son coefficient est le \textbf{coefficient réciproque} $cᵣ = \frac{1}{c}$.
\end{propriete}

\end{document}