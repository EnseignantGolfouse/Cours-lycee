\documentclass[
	classe=$2^{de}$
]{évaluation}

\usepackage{tcolorbox}

\renewcommand{\arraystretch}{1.4}

\title{Évaluation : Statistiques descriptives (rattrapage)}
\date{$1^{er}$ février 2023}

\begin{document}

\maketitle

\begin{tcolorbox}
	La calculatrice est autorisée.

	Les deux exercices sont à faire sur le sujet.

	Le barème est donné à titre \textit{indicatif}.
\end{tcolorbox}

\begin{exercice}[4]
	Remplir le tableau ci-dessous :

	\begin{center}
		\begin{tabular}{|c|c|c|c|}
			\hline
			Valeur de départ & Valeur d'arrivée    & Variation absolue   & Variation relative    \\ \hline
			$100$            & $200$               & \correction{$100$}  & \correction{$1$}      \\ \hline
			$240$            & \correction{$798$}  & $558$               & \correction{$2,325$}  \\ \hline
			$560$            & $350$               & \correction{$-210$} & \correction{$-0,375$} \\ \hline
			$1000$           & \correction{$8100$} & \correction{$7100$} & $7,1$                 \\ \hline
			$50$             & \correction{$10$}   & $-40$               & \correction{$-0,8$}   \\ \hline
		\end{tabular}
	\end{center}
\end{exercice}

\begin{exercice}[6]

	\begin{tcolorbox}
		Avant de donner un résultat, écrire le calcul qui a mené à ce résultat (même si il est fait à la calculatrice).
	\end{tcolorbox}

	Un footballer veut analyser ses performances. Il a donc reporté dans le tableau ci-dessous le nombre de points qu'il a fait par match.

	\begin{center}
		\begin{tabular}{|l|c|c|c|c|c|}
			\hline
			Points           & 0  & 1  & 2  & 3  & 4 \\ \hline
			Nombre de matchs & 22 & 23 & 17 & 11 & 7 \\ \hline
		\end{tabular}
	\end{center}

	\begin{enumerate}
		\item Combien de matchs a-t'il joué ?

		      \vspace{2em}\correction{$80$}
		\item Déterminer la moyenne de cette série, au millième près.

		      \vspace{2em}\correction{$1,475$}
		\item Déterminer la médiane de cette série, ainsi que les premier et troisième quartiles.

		      \vspace{2em}\correction{$1$, $0$ et $2$}
		\item Calculer l'écart-type de cette série, au centième près.

		      \vspace{2em}\correction{$1,26$}
	\end{enumerate}
\end{exercice}

\end{document}