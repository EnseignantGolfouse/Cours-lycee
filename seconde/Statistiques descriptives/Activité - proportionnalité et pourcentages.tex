\documentclass[
	classe=$2^{de}$,
	headerTitle=Activité,
	twocolumn,
	landscape
]{exercice}

\newcounter{MySection}
\setcounter{MySection}{1}

\newcommand{\mysection}[1]{{\large\uline{\arabic{MySection}) #1}}\vspace{1em}\stepcounter{MySection}}

\renewcommand{\arraystretch}{1.5}
\setlength{\columnsep}{1cm}

\title{Proportionnalité et pourcentages}

\begin{document}

\newcommand{\Exercices}{
	\maketitle

	\begin{exercice}
		Au 1ᵉʳ janvier 2017, la France, hors Mayotte,
		compte 35,7 millions de logements. Les rési-
		dences principales représentent 82,1 \% du parc, les résidences secondaires et logements occasionnels 9,5 \% et les logements vacants 8,4 \% \textit{(Insee, Tableau de l’économie française, 2018)}.
		\begin{enumerate}
			\item Quelle est la population étudiée ici ? \correction{Les logements}
			\item Citer des sous-populations de cette population. \correction{résidences principales, résidences secondaires, logements occasionnels, logements vacants.}
			\item Déterminer le nombre de logements vacants en France au 1ᵉʳ janvier 2017. \correction{2 998 800}
		\end{enumerate}
	\end{exercice}

	\begin{exercice}\

		Une peintre veut faire de gigantesque toiles. Pour cela, elle fabrique elle-même ses couleurs, à partir de peinture rouge, verte, bleue et noire.

		La peintre a fourni les proportions de chaque couleur, ainsi que la quantité de noir :

		\begin{center}
			\begin{tabular}{|c|c|c|c|c||c|}
				\hline
				                & Rouge   & Vert    & Bleu    & Noir    & Quantité de noir \\
				                & (en \%) & (en \%) & (en \%) & (en \%) & (en litres)      \\ \hline
				Fuschia         & 32      & 8       & 20      & 40      & 20               \\ \hline
				Or              & 33      & 22      & 0       & 45      & 15               \\ \hline
				Azur            & 4       & 17      & 26      & 53      & 26.5             \\ \hline
				Jaune           & 33      & 33      & 0       & 34      & 10               \\ \hline
				Argile          & 31      & 31      & 32      & 6       & 12               \\ \hline
				Indigo          & 16      & 4       & 32      & 48      & 12               \\ \hline
				Gris anthracite & 6       & 6       & 6       & 82      & 41               \\ \hline
			\end{tabular}
		\end{center}

		Pour chacune des couleurs demandées, calculer combien de litres de rouge, de vert, de bleu et de noir sont nécessaires, et organiser ces données dans un tableau.
	\end{exercice}
}

\Exercices

\newpage

\Exercices

\end{document}