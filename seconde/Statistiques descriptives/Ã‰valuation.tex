\documentclass[
	classe=$2^{de}$,
	headerTitle=Évaluation\space Chapitre\space 4
]{évaluation}

\usepackage{tcolorbox}

\renewcommand{\arraystretch}{1.4}

\title{Évaluation : statistiques descriptives (Sujet A)}
\date{20 janvier 2023}

\begin{document}

\maketitle

\begin{tcolorbox}
	La calculatrice est autorisée.

	L'exercice 2 est à faire sur le sujet : les autres exercices sont à faire sur une copie à part.

	Le barème est donné à titre \textit{indicatif}.
\end{tcolorbox}

\begin{exercice}[2]

	Une étude veut déterminer l'utilité d'un produit antimoustique. Elle étudie donc une population de moustiques, et détermine que $12\%$ sont porteurs de maladie dangereuses pour l'humain.

	Parmi les moustiques porteurs, $80\%$ sont sensibles à l'antimoustique.

	\begin{enumerate}
		\item Quelle est la proportion de moustiques porteurs de maladie et sensibles à l'antimoustique ? \correction{$0,12×0,8=0,096=9,6\%$}
		\item Le nombre de moustiques porteurs de maladie dans cette étude était de $2400$.

		      Quelle est alors le nombre total de moustiques étudiés ? \correction{$20000$}
	\end{enumerate}
\end{exercice}

\begin{exercice}[4]
	Remplir le tableau ci-dessous :

	\begin{center}
		\begin{tabular}{|c|c|c|c|}
			\hline
			Valeur de départ & Valeur d'arrivée    & Variation absolue   & Variation relative   \\ \hline
			$100$            & $200$               & \correction{$100$}  & \correction{1}       \\ \hline
			$240$            & \correction{$798$}  & $558$               & \correction{$2,325$} \\ \hline
			$1000$           & \correction{$8100$} & \correction{$7100$} & $7,1$                \\ \hline
			$50$             & \correction{$10$}   & $-40$               & \correction{$-0,8$}  \\ \hline
		\end{tabular}
	\end{center}
\end{exercice}

\begin{exercice}[6]

	On considère la série de notes d'un élève ci-dessous :

	$$ 2; 3; 4; 4; 5; 7; 12; 13; 13; 15; 16; 17; 17; 18; 19; 20 $$

	\begin{enumerate}
		\item Quelle est la moyenne de cette série (arrondie au centième près) ? \correction{$11,56$}
		\item Déterminer la médiane de cette série, ainsi que les premier et troisième quartiles. \correction{médiane: $13$} \correction{$Q₁: 4$} \correction{$Q₃: 17$}
		\item Calculer l'écart-type de cette série, au centième près. \correction{$6,15$}
		\item On considère un autre élève, qui a obtenu les mêmes notes, mais diminuées de 1. Quel est alors la moyenne de cet élève ? Et la médiane de ses notes ?
	\end{enumerate}
\end{exercice}

\begin{exercice}[5]
	Deux joueurs de basketball cherchent à évaluer leur performance. Ils notent donc le nombre de points qu'ils on fait par match, et compilent leur résultats dans le tableau ci-dessous :

	\begin{center}
		\begin{tabular}{|l|c|c|c|c|c|c|c|c|c|}
			\hline
			Nombre de points      & 12 & 13 & 14 & 15 & 16 & 17 & 18 & 19 & 20 \tabularnewline \hline
			Effectifs du joueur 1 & 11 & 12 & 7  & 5  & 6  & 10 & 7  & 10 & 12 \tabularnewline \hline
			Effectifs du joueur 2 & 1  & 4  & 22 & 21 & 11 & 8  & 1  & 7  & 5 \tabularnewline \hline
		\end{tabular}
	\end{center}

	\begin{enumerate}
		\item Entre les deux, lequel semble être le meilleur joueur ? Argumenter en utilisant des indicateurs statistiques.
		\item Afin d'être admis dans le club qu'ils visent, il est requis qu'ils marquent $19$ points ou plus lors d'au moins un quart de leur matchs. Sont-ils admis dans ce club ?
		\item Un autre club requiert que les joueurs puissent faire des scores homogènes, c'est à dire que leur score varie peu d'un match à l'autre. Lequel des deux joueurs convient mieux à ce critère ?
	\end{enumerate}
\end{exercice}

\begin{exercice}[3]
	Le salaire moyen d'une entreprise de douze salariés est de $2400€$.
	\begin{enumerate}
		\item Calculer le salaire d’un employé supplémentaire sachant que le salaire moyen a augmenté de $100€$. \correctionOr{Somme des salaires avant : $2400×12 = 28800$ / après : $2500×13 = 32500€$. Le nouveau salaire est donc de $32500-28800 = 3700€$.}{}
		\item Calculer le salaire d’un employé supplémentaire sachant que le salaire moyen a augmenté de $2\%$. \correctionOr{Somme des salaires avant : $2400×12 = 28800$ / après : $2400×1,02×13 = 31824$. Le nouveau salaire est donc de $31824-28800 = 3024€$.}{}
	\end{enumerate}
\end{exercice}

%============================================
%=============== SUJET B ====================
%============================================
\newpage
\title{Évaluation : statistiques descriptives (Sujet B)}
\setcounter{exercice}{1}

\maketitle

\begin{tcolorbox}
	La calculatrice est autorisée.

	L'exercice 2 est à faire sur le sujet : les autres exercices sont à faire sur une copie à part.
\end{tcolorbox}

\begin{exercice}[2]

	Une étude veut déterminer l'utilité d'un produit antimoustique. Elle étudie donc une population de moustiques, et détermine que $14\%$ sont porteurs de maladie dangereuses pour l'humain.

	Parmi les moustiques porteurs, $85\%$ sont sensibles à l'antimoustique.

	\begin{enumerate}
		\item Quelle est la proportion de moustiques porteurs de maladie et sensibles à l'antimoustique ? \correction{$0,14×0,85=0,119=11,6\%$}
		\item Le nombre de moustiques porteurs de maladie dans cette étude était de $2100$.

		      Quelle est alors le nombre total de moustiques étudiés ? \correction{$15000$}
	\end{enumerate}
\end{exercice}

\begin{exercice}[4]
	Remplir le tableau ci-dessous :

	\begin{center}
		\begin{tabular}{|c|c|c|c|}
			\hline
			Valeur de départ & Valeur d'arrivée     & Variation absolue    & Variation relative   \\ \hline
			$150$            & $300$                & \correction{$150$}   & \correction{1}       \\ \hline
			$320$            & \correction{$848$}   & $528$                & \correction{$1,65$}  \\ \hline
			$2000$           & \correction{$12600$} & \correction{$10600$} & $5,3$                \\ \hline
			$80$             & \correction{$20$}    & $-60$                & \correction{$-0,75$} \\ \hline
		\end{tabular}
	\end{center}
\end{exercice}

\begin{exercice}[6]

	On considère la série de notes d'un élève ci-dessous :

	$$ 2; 4; 5; 5; 7; 7; 12; 13; 13; 15; 16; 16; 17; 18; 19; 20 $$

	\begin{enumerate}
		\item Quelle est la moyenne de cette série (arrondie au centième près) ? \correction{$11,81$}
		\item Déterminer la médiane de cette série, ainsi que les premier et troisième quartiles. \correction{médiane: $13$} \correction{$Q₁: 5$} \correction{$Q₃: 16$}
		\item Calculer l'écart-type de cette série, au centième près. \correction{$5,75$}
		\item On considère un autre élève, qui a obtenu les mêmes notes, mais diminuées de 1. Quel est alors la moyenne de cet élève ? Et la médiane de ses notes ?
	\end{enumerate}
\end{exercice}

\begin{exercice}[5]
	Deux joueurs de basketball cherchent à évaluer leur performance. Ils notent donc le nombre de points qu'ils on fait par match, et compilent leur résultats dans le tableau ci-dessous :

	\begin{center}
		\begin{tabular}{|l|c|c|c|c|c|c|c|c|c|}
			\hline
			Nombre de points      & 12 & 13 & 14 & 15 & 16 & 17 & 18 & 19 & 20 \tabularnewline \hline
			Effectifs du joueur 1 & 11 & 12 & 7  & 5  & 6  & 10 & 7  & 10 & 12 \tabularnewline \hline
			Effectifs du joueur 2 & 1  & 4  & 22 & 21 & 11 & 8  & 1  & 7  & 5 \tabularnewline \hline
		\end{tabular}
	\end{center}

	\begin{enumerate}
		\item Entre les deux, lequel semble être le meilleur joueur ? Argumenter en utilisant des indicateurs statistiques.
		\item Afin d'être admis dans le club qu'ils visent, il est requis qu'ils marquent $19$ points ou plus lors d'au moins un quart de leur matchs. Sont-ils admis dans ce club ?
		\item Un autre club requiert que les joueurs puissent faire des scores homogènes, c'est à dire que leur score varie peu d'un match à l'autre. Lequel des deux joueurs convient mieux à ce critère ?
	\end{enumerate}
\end{exercice}

\begin{exercice}[3]
	Le salaire moyen d'une entreprise de douze salariés est de $2600€$.
	\begin{enumerate}
		\item Calculer le salaire d’un employé supplémentaire sachant que le salaire moyen a augmenté de $100€$. \correctionOr{Somme des salaires avant : $2600×12 = 31200$ / après : $2700×13 = 35100$. Le nouveau salaire est donc de $35100-31200 = 3900€$.}{}
		\item Calculer le salaire d’un employé supplémentaire sachant que le salaire moyen a augmenté de $2\%$. \correctionOr{Somme des salaires avant : $2600×12 = 31200$ / après : $2600×1,02×13 = 34476$. Le nouveau salaire est donc de $34476-31200 = 3276€$.}{}
	\end{enumerate}
\end{exercice}

\end{document}