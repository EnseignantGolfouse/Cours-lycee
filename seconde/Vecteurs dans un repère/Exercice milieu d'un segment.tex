\documentclass[
	classe=$2^{de}$,
	landscape,
	twocolumn,
]{exercice}

\usepackage{tikz-repère}
\usetikzlibrary{calc}

\setlength{\columnsep}{1cm}

\begin{document}

\newcommand{\Exercice}{
\begin{center}
	\begin{tikzpicture}[scale=0.7]
		\tikzRepere{-5.5}{5.5}{-5.5}{5.5}[][]
		\node[below left] at (0,0) {$O$};
		\coordinate (A) at (-2,0.5);
		\coordinate (B) at (4,1);
		\coordinate (C) at (-4.5,-3);
		\coordinate (D) at (-3,3);

		\foreach \p in {A,B,C,D} {
				\node at (\p) {×};
				\node[below left] at (\p) {$\p$};
			}

		\ifdefined\makeCorrection
			\coordinate (E) at ($(A) + (-1, -4.5)$);
			\coordinate (F) at ($(B) + (-1, -4.5)$);
			\coordinate (M1) at ($(A)!0.5!(B)$);
			\coordinate (M2) at ($(C)!0.5!(D)$);
			\coordinate (M3) at ($(A)!0.5!(F)$);

			\foreach \p in {E,F,M1,M2,M3} {
					\node[red] at (\p) {×};
					\node[red,below left] at (\p) {$\p$};
				}
		\fi
	\end{tikzpicture}
\end{center}

\begin{itemize}
	\item Lire les coordonnées des points $A(\correction{-2};\correction{0,5})$, $B(\correction{\phantom{0,}4};\correction{\phantom{0,}1})$, 
	
	$C(\correction{-4,5};\correction{-3})$ et $D(\correction{-3};\correction{\phantom{0,}3})$.
	\item Par le calcul, déterminer le milieu du segment $[AB]$ :

	      \correction{$\left(\dfrac{-2 + 4}{2};\dfrac{0,5 + 1}{2}\right) = (1 ; 0,75)$}
	\item Par le calcul, déterminer le milieu du segment $[CD]$ :

	      \correction{$\left(\dfrac{-4,5 + (-3)}{2};\dfrac{-3 + 3}{2}\right) = (-3,75 ; 0)$}
	\item Placer les points $E$ et $F$, tels que $\vec{AE} = \vec{BF} = \begin{pmatrix}-1 \\ -4,5\end{pmatrix}$.
	\item Déterminer les coordonnées du centre du parallélogramme $AEFB$.
\end{itemize}
}

\Exercice

\newpage
\Exercice

\end{document}