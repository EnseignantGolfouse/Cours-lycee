\documentclass[
	classe=$2^{de}$
]{informatique}

\usepackage{hyperref}
\usepackage{tcolorbox}
\hypersetup{
    colorlinks=true,
    urlcolor=blue,
    pdftitle={Activité : Carré bordé dans Geogebra},
}

\newcommand{\Point}[3]{#1(\correctionOr{{\color{red}#2}}{\phantom{3333}};\correctionOr{{\color{red}#3}}{\phantom{3333}})}

\title{Activité : Carré bordé dans Geogebra}

\begin{document}

\maketitle

\section*{Un carré bordé}

À partir d'un carré $ABCD$ dont le côté mesure $1$, on construit un quadrilatère $EFGH$ de la façon suivante :

On choisit un nombre réel $a$ positif, puis on place les points $E$, $F$, $G$ et $H$ définis par les relations :

\begin{align*}
	\vec{BE} & = a\vec{AB} & \vec{CF} & = a\vec{BC} & \vec{DG} & = a\vec{CD} & \vec{AH} & = a\vec{DA}
\end{align*}

On s'intéresse à la nature du quadrilatère $EFGH$.

\begin{enumerate}
	\item \begin{enumerate}
		      \item Dans Geogebra, tracer le carré $ABCD$ puis, à l'aide d'un curseur $a$, les points $E$, $F$, $G$ et $H$.

		            \begin{tcolorbox}
			            \uline{AIDE} : Voir la vidéo suivante : \url{https://lycee.hachette-education.com/Barbazo/2de/#chapitre_7_p210_TP3_tutoriel_logiciel_de_geometriemp4}
		            \end{tcolorbox}
		      \item Faire varier le curseur et conjecturer la nature du quadrilatère $EFGH$.

		            \correction{Il semble que $EFGH$ soit un carré.}
	      \end{enumerate}
	\item On considère le repère $(A ; \vec{AB}; \vec{AD})$.

	      \begin{enumerate}
		      \item Justifier que ce repère est un repère orthonormé.

		            \correction{$ABCD$ est un carré, donc on a $AB = AD$ et l'angle $ABD$ est droit : donc ce repère est orthonormé.}
		      \item À l'aide des relations vectorielles définissant les points $E$, $F$, $G$ et $H$, déterminer, dans ce repère, les coordonnées de chacun de ces quatres points.

		            \begin{center}
			            \begin{multicols}{2}
				            $\Point{E}{1+a}{0}$

				            $\Point{G}{-a}{1}$

				            \columnbreak

				            $\Point{F}{1}{1+a}$

				            $\Point{H}{0}{-a}$
			            \end{multicols}
		            \end{center}
		      \item Calculer les longueurs $EF$, $FG$, $GH$ et $HE$.

		            $EF = \correction{\sqrt{a² + (1 + a)²}}$ \medskip

		            $FG = \correction{\sqrt{(1 + a)² + a²} }$ \medskip

		            $GH = \correction{\sqrt{a² + (1 + a)²}}$ \medskip

		            $HE = \correction{\sqrt{(1 + a)² + a²}}$ \medskip
		      \item Valider ou invalider la conjecture faite à la question $1$.

		            \correction{Toutes ces longueurs sont égales : il suffit donc de montrer qu'un des angles est droit.}

		            \correction{Pour ce faire, on va utiliser la réciproque du théorème de Pythagore : ici on a}

		            \correction{$EF² = a² + (1 + a²)$, $HE² = (1+a)² + a²$ et $HF² = 1² + (1 + 2a)²$}

		            \correction{Ainsi $EF² + HE² = 2a² + 2(1 + 2a + a²) = 4a² + 4a + 2$, et $HF² = 1 + 1 + 4a + 4a² = EF² + HE²$.}

		            \correction{Donc l'angle $FEH$ est droit, et donc $EFGH$ est un carré.}
	      \end{enumerate}
\end{enumerate}

\end{document}