\documentclass[
	classe=$2^{de}$
]{exercice}

\usetikzlibrary{calc}

\title{Mobile en équilibre}

\begin{document}

\maketitle

On construit un mobile en suspendant deux masses $m_A = 20$ g et $m_B = 30$ g aux extrémités d'une tige $[AB]$.

\begin{center}
	\begin{tikzpicture}
		\newcommand{\masseARayon}{1}
		\newcommand{\masseBRayon}{1.5}
		\coordinate (A) at (0,0);
		\coordinate (B) at (10,0);
		\coordinate (M) at (6,0);

		\draw ($(A) + (0,-3)$) -- (A) -- (B) -- ++(0,-2);
		\draw (M) -- ++(0,2);

		\foreach \p/\pos in {A/above,B/above,M/below} {
				\node[thick] at (\p) {×};
				\node[\pos] at (\p) {$\p$};
			}
		\draw ($(A) + (0,-3) + (0,-\masseARayon)$) circle (\masseARayon);
		\draw ($(B) + (0,-2) + (0,-\masseBRayon)$) circle (\masseBRayon);
	\end{tikzpicture}
\end{center}

Le poids de la tige est négligeable. Les lois de la physique indiquent que le mobile est en équilibre lorsque $20\vec{MA} + 30\vec{MB} = \vec{0}$. On cherche à déterminer la position du point $M$ sur la tige $[AB]$.

\begin{enumerate}
	\item En utilisant l'égalité $\vec{MB} = \vec{MA} + \vec{AB}$, démontrer que $\vec{AM} = \dfrac{3}{5}\vec{AB}$.

	      \correctionOr{{\color{red}On a $20\vec{MA} + 30\vec{MB} = \vec{0}$, donc $\vec{MB} = -\dfrac{2}{3}\vec{MA}$.

				      Ainsi on a $-\dfrac{2}{3}\vec{MA} = \vec{MA} + \vec{AB}$, donc $-\dfrac{5}{3}\vec{MA} = \vec{AB}$, soit $\vec{AM} = \dfrac{3}{5}\vec{AB}$.}}{}
	\item Comment interpréter cette relation dans le contexte de l'exercice ?

	      \correctionOr{{\color{red}Ainsi, il faut placer le point $M$ au trois cinquièmes de la tige.}}{}
\end{enumerate}

\end{document}