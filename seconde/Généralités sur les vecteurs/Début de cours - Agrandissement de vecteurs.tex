\documentclass{beamer}

\usepackage{préambule}

\renewcommand{\myArrow}{-{Latex[length=2mm, width=1mm]}}

\begin{document}

\newcommand{\Presentation}
{\begin{frame}
	\begin{center}
		\begin{tikzpicture}[scale=0.8]
			\foreach \x in {-3,...,3} {
					\foreach \y in {-3,...,2} {
							\node at (\x, \y) {\tiny ×};
						}
				}

			\draw[\myArrow] (-1,1) -- node[right] {$\vec{a}$} ++(0,1);
			\draw[\myArrow] (-2,-1) -- node[above] {$\vec{b}$} ++(1,0);
			\draw[\myArrow] (-3,0) -- node[left] {$\vec{c}$} ++(0,-3);
			\draw[\myArrow] (0,0) -- node[above] {$\vec{d}$} ++(1,1);
			\draw[\myArrow] (-3,2) -- node[above] {$\vec{e}$} ++(2,-2);
			\draw[\myArrow] (-1,-2) -- node[above] {$\vec{f}$} ++(-1,-1);
			\draw[\myArrow] (1,0) -- node[above] {$\vec{g}$} ++(2,2);
			\draw[\myArrow] (2,-2) -- node[above] {$\vec{h}$} ++(1,-1);
			\draw[\myArrow] (-1,-3) -- node[above left] {$\vec{i}$} ++(4,3);
		\end{tikzpicture}
	\end{center}

	Sur la figure ci-dessus, donner :
	\begin{itemize}
		\item Un vecteur égal à $2\vec{d}$ : \correction{$\vec{g}$}
		\item Un vecteur égal à $\frac{1}{2}\vec{e}$ : \correction{$\vec{h}$}
		\item Un vecteur égal à $-\vec{f}$ : \correction{$\vec{d}$}
		\item Un vecteur égal à $-\frac{1}{3}\vec{c}$ : \correction{$\vec{a}$}
	\end{itemize}

	Exprimer le vecteur $\vec{g}$ en fonction de $\vec{a}$ et de $\vec{b}$ : $\vec{g} = \correctionDots{2}×\vec{a} + \correctionDots{2}×\vec{b}$
\end{frame}}

\Presentation

\newcommand{\makeCorrection}{}
\Presentation

\end{document}