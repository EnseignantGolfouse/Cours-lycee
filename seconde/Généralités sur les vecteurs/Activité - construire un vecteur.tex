\documentclass[
	classe=$2^{de}$,
	headerTitle=Activité,
	landscape,twocolumn
]{exercice}

\usetikzlibrary{calc}

\title{Activité : \\ construction à la règle et au compas}

\begin{document}

\newcommand{\Activite}{
	\maketitle

	En utilisant la règle et le compas, construire :
	\begin{itemize}
		\item Le représentant de $\vec{u}$ d'origine $A$.
		\item Le représentant de $\vec{v}$ d'extrémité $B$.
		\item Le représentant de $\vec{u} + \vec{v}$ d'extrémité $C$.
		\item Le représentant de $\vec{u} + \vec{u} + \vec{v}$ d'origine $D$.
	\end{itemize}

	\begin{center}
		\begin{tikzpicture}
			\coordinate (uStart) at (0,0);
			\coordinate (uVec) at (3,1);
			\coordinate (vStart) at (5,1);
			\coordinate (vVec) at (-1,-2.5);
			\coordinate (uvVec) at ($(uVec) + (vVec)$);
			\coordinate (uuvVec) at ($(uVec) + (uVec) + (vVec)$);
			\coordinate (A) at (1,-3);
			\coordinate (B) at (6,-3);
			\coordinate (C) at (1,-7);
			\coordinate (D) at (6,-7);

			\draw[->] (uStart) -- node[above] {$\vec{u}$} ++(uVec);
			\draw[->] (vStart) -- node[above left] {$\vec{v}$} ++(vVec);

			\foreach \p in {A,B,C,D} {
					\node at (\p) {×};
					\node[above left] at (\p) {$\p$};
				}
			\ifdefined\makeCorrection
				\draw[red,->] (A) -- ++(uVec);
				\draw[red,->] (B) ++(vVec) -- (B);
				\draw[red,->] (C) ++(uvVec) -- (C);
				\draw[red,->] (D) -- ++(uuvVec);
			\fi
		\end{tikzpicture}
	\end{center}
}

\Activite

\newpage

\Activite

\end{document}