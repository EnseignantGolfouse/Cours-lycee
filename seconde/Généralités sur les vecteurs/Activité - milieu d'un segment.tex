\documentclass[
	classe=$2^{de}$,
	headerTitle=Activité,
	twocolumn,
	landscape
]{exercice}

\usepackage{tcolorbox}

\setlength{\columnsep}{1cm}

\title{Activité : milieu d'un segment}

\begin{document}

\newcommand{\Activite}{
\maketitle

\begin{center}
	\vspace{0.5cm}
	\begin{tikzpicture}
		\coordinate (A) at (0,0);
		\coordinate (M) at (2,-1);
		\coordinate (B) at (4,-2);
		\coordinate (C) at (-2,-3);

		\node[above] at (A) {A};
		\node[above] at (B) {B};
		\node[left] at (C) {C};
		\node[above right] at (M) {M};
		\draw (M) ++(0.1,0.2) -- ++(-0.2,-0.4);
		\node at (C) {×};
		\draw (A) -- (B);
	\end{tikzpicture}
	\vspace{0.5cm}

	On veut montrer que \squared{$\widevec{CA} + \widevec{CB} = 2\widevec{CM}$}.
\end{center}

\begin{tcolorbox}
	\uline{Remarque} : on pourra se servir de la figure ci-dessus pour faire un schéma : en revanche, il n'est pas nécéssaire de faire une figure exacte, ou de mesurer des longueurs pour compléter cette activité.
\end{tcolorbox}


\begin{enumerate}
	\item Soit $D$ l'image du point $A$ par le vecteur $\widevec{CB}$.

	      Montrer alors que $ADBC$ est un parallélogramme. \vspace{3em}
	\item Quelle propriété ont les diagonales d'un parallélogramme ? \vspace{1em}

	      En déduire que $\widevec{CD} = 2\widevec{CM}$. \vspace{1em}
	\item Montrer alors que $\widevec{CA} + \widevec{CB} = 2\widevec{CM}$.
\end{enumerate}
}

\Activite

\newpage

\Activite

\end{document}