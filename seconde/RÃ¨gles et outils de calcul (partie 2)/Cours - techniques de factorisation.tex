\documentclass[noheader]{coursclass}

\newgeometry{margin=0.7cm}

\begin{document}

\newcommand{\Technique}{
	\begin{greybox}[frametitle={Techniques de factorisation}]
		\begin{itemize}
			\item Si il n'y a pas de terme constant, on peut factoriser par $x$.
			\item Sinon, l'expression à factoriser est sous la forme $ax² + bx + c$. On s'intéresse alors aux termes $ax²$ et $c$ :
			      \begin{itemize}
				      \item On écrit $ax² = (\sqrt{a}x)$, et $c = \sqrt{c}$.
				      \item Si $b > 0$, on essaie de factoriser en $(\sqrt{a}x + c)²$.
				      \item Si $b < 0$, on essaie de factoriser en $(\sqrt{a}x - c)²$.
				      \item Si $b = 0$, on essaie de factoriser en $(\sqrt{a}x + c)(\sqrt{a}x - c)$.
			      \end{itemize}
		\end{itemize}
	\end{greybox}

	\begin{exemple}
		\begin{itemize}
			\item On veut factoriser $2x² - 3x$. On remarque qu'il n'y a pas de terme constant, donc
			      $$2x² - 3x = \correctionDots{x(2x - 3)}$$
			\item On veut factoriser $9x² + 30x + 25$. On pose $9x² = (\correctionDots{3x})²$ et $25 = \correctionDots{5}²$. De plus $b \correctionDots{>} 0$, donc
			      $$ 9x² + 30x + 25 = \correctionDots{(3x + 5)²} $$
			      On peut vérifier que c'est vrai en développant la partie de droite.
			\item On veut factoriser $100x² - 4$. On pose $100x² = (\correctionDots{10x})²$ et $4 = \correctionDots{2}²$. De plus $b \correctionDots{=} 0$, donc
			      $$ 100x² - 4 = \correctionDots{(10x + 2)(10x - 2)} $$
			      On peut vérifier que c'est vrai en développant la partie de droite.
		\end{itemize}
	\end{exemple}
}

\Technique

\vfill

\Technique




\end{document}