\documentclass[
	classe=$2^{de}$
]{coursclass}

\title{Chapitre 5 : Règles de calcul (partie 2)}
\author{}
\date{}

\begin{document}

\maketitle

\begin{propriete}[Double distributivité]
	Si $a, b, c$ et $d$ sont quatres nombres réels, on a

	$$ (a + b)(c + d) = ac + ad + bc + bd $$

	On dit que l'expression de gauche est \textbf{factorisée}, et que l'expression de droite est \textbf{développée}.
\end{propriete}

\begin{propriete}[Identités remarquables]
	Si $a$ et $b$ sont des nombres réels, on a
	\begin{itemize}
		\item $(a + b)² = a² + 2ab + b²$
		\item $(a - b)² = a² - 2ab + b²$
		\item $(a + b)(a - b) = a² - b²$
	\end{itemize}

	Ces égalités sont à connaître par cœur !
\end{propriete}
\begin{proof}
	On va prouver l'égalité 1, en utilisant la double distributivité :

	Soient $a$ et $b$ deux nombres réels. Alors :
	\begin{align*}
		(a + b)² & = (a + b)(a + b)        \\
		         & = a×a + a×b + b×a + b×b \\
		         & = a² + a×b + a×b + b²   \\
		         & = a² + 2ab + b²
	\end{align*}
\end{proof}

\begin{propriete}
	Si $a$ et $b$ sont des nombres réels, on a :

	$$ (ab)² = a²b² \text{\hspace{1em} et \hspace{1em}} \left(\dfrac{a}{b}\right)² = \dfrac{a²}{b²} $$
\end{propriete}

\begin{propriete}
	Si $a$ et $b$ sont des nombres réels, on a :
	\begin{itemize}
		\item $\sqrt{ab} = \sqrt{a}\sqrt{b} \text{\hspace{1em} et \hspace{1em}} \sqrt{\dfrac{a}{b}} = \dfrac{\sqrt{a}}{\sqrt{b}}$
		\item $\sqrt{a + b} ≤ \sqrt{a} + \sqrt{b}$
	\end{itemize}
\end{propriete}

\begin{greybox}
	Montrer que les deux égalités et l'inégalité ci-dessus sont vérifiées pour $a = 9$ et $b = 16$ :

	\begin{itemize}
		\item $\sqrt{9×16} = \sqrt{144} = 12$, et $\sqrt{9}×\sqrt{16}=3×4=12$
		\item $\sqrt{\dfrac{9}{16}} = \sqrt{0,5625} = 0,75$, et $\dfrac{\sqrt{9}}{\sqrt{16}} = \dfrac{3}{4} = 0,75$
		\item $\sqrt{16 + 9} = \sqrt{25} = 5$, et $\sqrt{16} + \sqrt{9}=4 + 3=7$
	\end{itemize}
\end{greybox}

\begin{propriete}[Produit nul]
	Si un produit est nul, alors au moins un des facteurs est nul.

	\begin{center}
		Si $A × B = 0$, alors $A = 0$ ou $B = 0$.
	\end{center}
\end{propriete}

\begin{exemple}
	Résoudre l'équation $(x - 3)(2x + 4) = 0$ :

	\begin{itemize}
		\item On sait que $(x - 3)(2x + 4) = 0$. Donc on a $x - 3 = 0$ ou $2x + 4 = 0$.
		\item Si $x - 3 = 0$, alors $x = 3$.
		\item Si $2x + 4 = 0$, alors $x = -2$.
		\item L'ensemble de solutions de cette équation est donc $\{-2 ; 3\}$.
	\end{itemize}
\end{exemple}

\end{document}