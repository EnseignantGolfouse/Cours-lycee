\documentclass[
	classe=$2^{de}$
]{coursclass}

\title{Chapitre 5 : Règles de calcul (partie 2)}
\author{}
\date{}

\begin{document}

\maketitle

\begin{propriete}[Double distributivité]
	Si $a, b, c$ et $d$ sont quatres nombres réels, on a

	$$ (a + b)(c + d) = ac + ad + bc + bd $$

	On dit que l'expression de gauche est \textbf{factorisée}, et que l'expression de droite est \textbf{développée}.
\end{propriete}

\begin{propriete}[Identités remarquables]
	Si $a$ et $b$ sont des nombres réels, on a
	\begin{itemize}
		\item $(a + b)² = a² + 2ab + b²$
		\item $(a - b)² = a² - 2ab + b²$
		\item $(a + b)(a - b) = a² - b²$
	\end{itemize}

	Ces égalités sont à connaître par cœur !
\end{propriete}

\end{document}