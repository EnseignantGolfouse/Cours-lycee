\documentclass[
	classe=$2^{de}$,
	exercices=Exercices\space chapitre\space 1
]{exercice}

\usepackage{multirow}
\usepackage{makecell}

\newcounter{exercicesCounter}
\setcounter{exercicesCounter}{1}
\newcommand{\makeExercice}{
	\uline{\textbf{Exercice 2.\arabic{exercicesCounter}}}\stepcounter{exercicesCounter}
}

\renewcommand{\arraystretch}{2.5}
\newcommand{\tableSpacing}{0.6em}

\title{Règles de calcul (Avancé)}

\begin{document}


\maketitle

\makeExercice : \textbf{Démontrer} les égalités suivantes :

\vspace{\tableSpacing}
\begin{tabularx}{\linewidth}{|rX|rX|rX|}
	\hline
	$7 − 8 × 3 + 9$           & $= 12 × \dfrac{26}{13 × (-3)}$  & $\dfrac{\frac{2}{5}}{9} + \dfrac{34}{45}$ & $= 4 × \dfrac{11 - (-9)}{100}$  & $\dfrac{-4 × (1 - 6)}{\frac{1}{9} - \frac{2}{3}}$ & $= -2² × 3²$            \\
	$\correction{7 - 24 + 9}$ & $\correction{= -\dfrac{24}{3}}$ & $\correction{\dfrac{36}{45}}$             & $\correction{= \dfrac{20}{25}}$ & $\correction{\dfrac{20}{-\frac{5}{9}}}$           & $\correction{= -4 × 9}$ \\
	$\correction{-8}$         & $\correction{= -8}$             & $\correction{\dfrac{4}{5}}$               & $\correction{= \dfrac{4}{5}}$   & $\correction{-36}$                                & $\correction{= -36}$    \\ \hline
\end{tabularx}
\vspace{\tableSpacing}

\makeExercice : Recopier les expressions suivantes puis factoriser :

\vspace{\tableSpacing}
\begin{tabularx}{\linewidth}{|l|X|l|}
	\hline
	$A = (2x + 5) × 8 + (2x + 5) × 7$ & $B = (4x−1)(x−6)+(4x−1)(2x+8)$         & $C = (5x + 2)(3x − 4) − 5x − 2$     \\
	$\correction{A = (2x + 5) × 15}$  & $\correction{B = (4x - 1) × (3x + 2)}$ & $\correction{C = (5x + 2)(3x - 5)}$ \\ \hline
\end{tabularx}
\vspace{\tableSpacing}

\makeExercice : Utiliser la liste de fraction données, ainsi que les quatres opérations $+, -, ×$ et $÷$ pour obtenir le nombre cible.

\begin{greybox}[frametitle={Exemple}]
	\uline{liste} = $\dfrac{1}{2}, \dfrac{8}{2}, \dfrac{3}{4}$ \hspace{3em} \uline{cible} = $2$

	Une solution possible est alors $\dfrac{1}{2} × \dfrac{8}{2} = \dfrac{1 × 8}{2 × 2} = 2$.
\end{greybox}

\begin{minipage}{0.45\textwidth}   %left column
	\begin{itemize}
		\item[] \uline{liste} = $\dfrac{4}{6}, \dfrac{1}{6}, \dfrac{10}{5}$ \hspace{3em} \uline{cible} = $1$ \vspace{0.7em}

		      \correction{$\dfrac{4}{6} + \left(\dfrac{1}{6} × \dfrac{10}{5}\right) = \dfrac{4}{6} + \dfrac{10}{30} = \dfrac{4}{6} + \dfrac{2}{6} = 1$} \vspace{0.7em}
		      \hrule
		\item[] \uline{liste} = $\dfrac{1}{8}, \dfrac{3}{8}, \dfrac{6}{5}, \dfrac{18}{10}$ \hspace{3em} \uline{cible} = $2$ \vspace{0.7em}

		      \correction{$\dfrac{1}{8} + \dfrac{3}{8} + \left(\dfrac{18}{10} ÷ \dfrac{6}{5}\right) = \dfrac{1}{2} + \dfrac{90}{60} = \dfrac{1}{2} + \dfrac{3}{2} = 2$} \vspace{0.7em}
	\end{itemize}
\end{minipage}
\hfill\vline\hfill
\begin{minipage}{0.45\textwidth} %right column
	\begin{itemize}
		\item[] \uline{liste} = $\dfrac{7}{12}, \dfrac{15}{8}, \dfrac{1}{4}, \dfrac{1}{2}, \dfrac{10}{24}$ \hspace{3em} \uline{cible} = $3$ \vspace{0.7em}

		      \correction{$\dfrac{15}{8} + \left(\dfrac{1}{4} × \dfrac{1}{2}\right) + \dfrac{7}{12} + \dfrac{10}{24}$}

		      \correction{$ = \dfrac{15}{8} + \dfrac{1}{8} + \dfrac{7}{12} + \dfrac{5}{12} = 2 + 1 = 3$}
		      \vspace{0.7em}
		      \hrule
		\item[] \uline{liste} = $\dfrac{8}{7}, \dfrac{2}{7}, \dfrac{14}{6}, \dfrac{3}{24}, \dfrac{25}{8}$ \hspace{3em} \uline{cible} = $6$ \vspace{0.7em}

		      \correction{$\left(\dfrac{8}{7}+\dfrac{2}{7}÷\dfrac{14}{6}\right) × \left(\dfrac{25}{8}-\dfrac{3}{24}\right)$}

		      \correction{$= \left(\dfrac{8}{7} + \dfrac{6}{7}\right) × \dfrac{24}{8} = 2 × 3 = 6$}
		      \vspace{0.7em}
	\end{itemize}
\end{minipage}

\end{document}