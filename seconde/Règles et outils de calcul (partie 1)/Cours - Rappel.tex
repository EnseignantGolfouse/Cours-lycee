\documentclass[a4paper,10pt]{article}

\usepackage{préambule}

\title{Cours chapitre 1}
\author{Rappel sur les règles de calcul}
\date{}

\begin{document}

\maketitle

\section{Calculs avec des fractions}

\begin{definition}[Inverse]
	L’\textbf{inverse} d’un nombre non nul $a$ est $\dfrac{1}{a}$.

	Si $a$ et $b$ sont non nuls, l'\textbf{inverse} de la fraction $\dfrac{a}{b}$ est $\dfrac{b}{a}$.
\end{definition}

\begin{definition}[Fraction irréductible]
	La fraction $\dfrac{a}{b}$ est \textbf{irréductible} si on ne peut pas la simplifier.

	Autrement dit, $a$ et $b$ n’ont pas de facteur premier en commun dans leur décomposition en produit de facteurs premiers.
\end{definition}

\begin{itemize}
	\item \uline{Différentes écritures pour un rationnel} : Lorsque $b$ est non nul, pour $k ≠ 0$, \squareFrame{$\dfrac{a × k}{b × k} = \dfrac{a}{b} = \dfrac{a ÷ k}{b ÷ k}$}

	      On utilise cette égalité pour \textbf{simplifier} une fraction ou \textbf{réduire au même dénominateur} deux fractions.
	\item \uline{Position du signe « moins »} : lorsque $b$ est non nul, \squareFrame{$\dfrac{- a}{b} = \dfrac{a}{- b} = - \dfrac{a}{b}$}
	\item \uline{Égalité de fractions} : lorsque $b$ et $d$ sont non nuls, \squareFrame{$\dfrac{a}{b} = \dfrac{c}{d}$ si et seulement si $a × d = b × c$}
	\item \uline{Additionner ou soustraire deux fractions de même dénominateur} : lorsque $b$ est non nul, \squareFrame{$\dfrac{a}{b} + \dfrac{c}{b} = \dfrac{a + c}{b}$}

	      Si les deux fractions ont des dénominateurs différents, il faut les réduire au même dénominateur pour pouvoir les
	      additionner.
	\item \uline{Multiplier un nombre par une fraction} : lorsque $b$ est non nul, \squareFrame{$c × \dfrac{a}{b} = \dfrac{c × a}{b}$}
	\item \uline{Multiplier deux fractions} : lorsque $b$ et $d$ sont non nuls, \squareFrame{$\dfrac{a}{b} × \dfrac{c}{d} = \dfrac{a × b}{c × d}$}
	\item \uline{Diviser par une fraction} : Diviser par une fraction revient à multiplier par son inverse : lorsque $a$ et $b$ sont non nuls, \squareFrame{$\cfrac{x}{\frac{a}{b}} = x ÷ \dfrac{a}{b} = x × \dfrac{b}{a}$}
\end{itemize}

\pagebreak
\section{Calculs algébriques}

\begin{vocabulaire}
	\begin{tabularx}{\linewidth}{XXX}
		L'\textbf{opposé} de $x$ est $-x$. & L'\textbf{inverse} de $x$ est $\frac{1}{x}$. & Le \textbf{carré} de $x$ est $x²$.
	\end{tabularx}
\end{vocabulaire}

\begin{greybox}[frametitle={Notation}]
	\newcommand{\localSpacing}{1.8em}
	\begin{tabular}{llll}
		$3 × x = 3x$ \hspace{\localSpacing} & $x × y = xy$ \hspace{\localSpacing} & $6 × (x + 2) = 6(x + 2)$ \hspace{\localSpacing} & $(x - 1) × (x + 7) = (x - 1)(x + 7)$
	\end{tabular}
\end{greybox}

\begin{greybox}[frametitle={Règle des signes}]
	\begin{tabularx}{\linewidth}{XXXX}
		$x × y = xy$ & $- x × y = -xy$ & $x × (-y) = -xy$ & $(-x) × (-y) = xy$
	\end{tabularx}
	\vspace{1em}

	Devant les parenthèses :
	\begin{itemize}
		\item S’il y a un signe « $+$ » devant des parenthèses, supprimer les parenthèses et \textbf{garder} les signes.
		\item S’il y a un signe « $-$ » devant des parenthèses, supprimer les parenthèses et \textbf{changer} les signes.
	\end{itemize}
\end{greybox}

\begin{definition}[]
	\begin{itemize}
		\item \textbf{Développer} un produit signifie le transformer en une somme algébrique.
		\item \textbf{Réduire} une expression développée signifie l’écrire sous la forme d’une somme algébrique contenant le moins de termes possible.
		\item \textbf{Factoriser} une somme algébrique signifie la transformer en produit.
	\end{itemize}
\end{definition}

\begin{propriete}[Distributivité simple]
	Pour tous nombres $a$, $b$ et $k$ :
	\begin{align*}
		k × (a + b) & = ka + kb \hspace{5em}\text{(développement)}     \\
		ka + kb     & = k × (a + b) \hspace{5em}\text{(factorisation)}
	\end{align*}
\end{propriete}

\end{document}