\documentclass[noheader]{évaluation}

\usetikzlibrary{calc}

\begin{document}

\begin{center}
	\LARGE Évaluation : Règles et outils de calcul (sujet A)
\end{center}

\begin{exercice}
	\begin{multicols}{2}
		\begin{enumerate}
			\item $x = 1$
			\item $x = \frac{4}{3}$
			\item $x = 7$ ou $x = -3$
			\item $x = -5$ ou $x = -1$
		\end{enumerate}
	\end{multicols}
\end{exercice}

\begin{exercice}
	Pour chaque phrase ci-dessous, répondre par vrai ou faux, en justifiant :
	\begin{enumerate}
		\item VRAI, car $5 × 6 + 2 = 30 + 2 = 32$
		\item FAUX, car $7 × (-2) - 9 = -14 - 9 = -23 ≠ -14$
		\item VRAI car $-2 × (-1) = 2$ et $6 × (-1) + 5 = -6 + 5 = -1$, et $2 > -1$.
		\item VRAI car $|3 × 6 - 20| + 7 × 3 = |18 - 20| + 21 = |{-}2| + 21 = 2 + 21 = 23$.
		\item FAUX car $|2 × (-10) + 3 × 6| = |{-}20 + 18| = |{-}2| = 2$, et $19 - 10 × 2 = 19 - 20 = -1$.
	\end{enumerate}
\end{exercice}

\begin{exercice}
	\begin{enumerate}
		\item Si le périmètre du carré $ABCD$ est égal au périmètre du triangle $CDE$, $x$ vérifie l'équation $4 × (x + 1) = x + 1 + x + x$.
		\item Donner la solution de cette équation.
		      \begin{align*}
			      4 × (x + 1) & = x + 1 + x + x \\
			      4x + 4      & = 3x + 1        \\
			      x + 4       & = 1             \\
			      x           & = -3
		      \end{align*}
		      Ainsi on ne peut pas construire cette figure, car une longueur ne peut pas être égale à $-3$.
	\end{enumerate}
\end{exercice}

\begin{exercice}
	\begin{enumerate}
		\item  \begin{multicols}{3}
			      Si $x = 2$,
			      \columnbreak
			      \begin{align*}
				      A & = \dfrac{3 × 2}{5 × (2 - 3)} \\
				        & = \dfrac{6}{5 × (-1)}        \\
				        & = -\dfrac{6}{5}              \\
			      \end{align*}
			      \columnbreak
			      \begin{align*}
				      B & = \dfrac{-4 × 2 + 8}{-3 × 5 - 10} \\
				        & = \dfrac{-8 + 8}{-15 - 10}        \\
				        & = \dfrac{0}{-25}                  \\
				        & = 0                               \\
			      \end{align*}
		      \end{multicols}


		      \begin{multicols}{3}
			      Si $x = 8$,
			      \columnbreak
			      \begin{align*}
				      A & = \dfrac{3 × 8}{5 × (8 - 3)} \\
				        & = \dfrac{24}{5 × 5}          \\
				        & = \dfrac{24}{25}             \\
			      \end{align*}
			      \columnbreak
			      \begin{align*}
				      B & = \dfrac{-4 × 8 + 8}{-3 × 5 - 10} \\
				        & = \dfrac{-32 + 8}{-15 - 10}       \\
				        & = \dfrac{-24}{-25}                \\
				        & = \dfrac{24}{25}                  \\
			      \end{align*}
		      \end{multicols}
		\item $8$ est une solution de $\dfrac{3x}{5 × (x - 3)} = \dfrac{-4x + 8}{-3 × 5 - 10}$.
	\end{enumerate}
\end{exercice}

\begin{exercice}
	Soit $x$ l'abscisse du point $X$. Les instructions reviennent à effectuer le calcul: $ (x - 6) ÷ 3 - 5 $

	Ainsi, si on résoud l'équation, on obtient :
	\begin{align*}
		(x - 6) ÷ 3 - 5 & = 7  \\
		(x - 6) ÷ 3     & = 12 \\
		x - 6           & = 36 \\
		x               & = 42
	\end{align*}
\end{exercice}

%================================================
\newpage
%================================================

\setcounter{exercice}{1}

\begin{center}
	\LARGE Évaluation : Règles et outils de calcul (sujet B)
\end{center}

\begin{exercice}
	\begin{multicols}{2}
		\begin{enumerate}
			\item $x = 1$
			\item $x = \frac{8}{9}$
			\item $x = -3$ ou $x = 9$
			\item $x = -7$ ou $x = 3$
		\end{enumerate}
	\end{multicols}
\end{exercice}

\begin{exercice}
	Pour chaque phrase ci-dessous, répondre par vrai ou faux, en justifiant :
	\begin{enumerate}
		\item VRAI, car $5 × 4 + 2 = 20 + 2 = 22$
		\item FAUX, car $7 × (-3) - 9 = -21 - 9 = -30 ≠ -21$
		\item VRAI car $-3 × (-2) = 6$ et $6 × (-2) + 5 = -12 + 5 = -7$, et $6 > -7$.
		\item VRAI car $|2 × 5 - 18| + 6 × 4 = |10 - 10| + 24 = |{-}8| + 24 = 8 + 24 = 32$.
		\item FAUX car $|2 × (-5) + 3 × 6| = |{-}10 + 18| = |{-}8| = 8$, et $19 - 5 × 4 = 19 - 20 = -1$.
	\end{enumerate}
\end{exercice}

\begin{exercice}
	\begin{enumerate}
		\item Si le périmètre du carré $ABCD$ est égal au périmètre du triangle $CDE$, $x$ vérifie l'équation $4 × (x + 2) = x + 2 + x + x$.
		\item Donner la solution de cette équation.
		      \begin{align*}
			      4 × (x + 2) & = x + 2 + x + x \\
			      4x + 8      & = 3x + 2        \\
			      x + 8       & = 2             \\
			      x           & = -6
		      \end{align*}
		      Ainsi on ne peut pas construire cette figure, car une longueur ne peut pas être égale à $-6$.
	\end{enumerate}
\end{exercice}

\begin{exercice}
	\begin{enumerate}
		\item  \begin{multicols}{3}
			      Si $x = 2$,
			      \columnbreak
			      \begin{align*}
				      A & = \dfrac{3 × 2}{5 × (2 - 3)} \\
				        & = \dfrac{6}{5 × (-1)}        \\
				        & = -\dfrac{6}{5}              \\
			      \end{align*}
			      \columnbreak
			      \begin{align*}
				      B & = \dfrac{-4 × 2 + 8}{-3 × 5 - 10} \\
				        & = \dfrac{-8 + 8}{-15 - 10}        \\
				        & = \dfrac{0}{-25}                  \\
				        & = 0                               \\
			      \end{align*}
		      \end{multicols}


		      \begin{multicols}{3}
			      Si $x = 8$,
			      \columnbreak
			      \begin{align*}
				      A & = \dfrac{3 × 8}{5 × (8 - 3)} \\
				        & = \dfrac{24}{5 × 5}          \\
				        & = \dfrac{24}{25}             \\
			      \end{align*}
			      \columnbreak
			      \begin{align*}
				      B & = \dfrac{-4 × 8 + 8}{-3 × 5 - 10} \\
				        & = \dfrac{-32 + 8}{-15 - 10}       \\
				        & = \dfrac{-24}{-25}                \\
				        & = \dfrac{24}{25}                  \\
			      \end{align*}
		      \end{multicols}
		\item $8$ est une solution de $\dfrac{3x}{5 × (x - 3)} = \dfrac{-4x + 8}{-3 × 5 - 10}$.
	\end{enumerate}
\end{exercice}

\begin{exercice}
	Soit $x$ l'abscisse du point $X$. Les instructions reviennent à effectuer le calcul: $ (x - 5) ÷ 3 - 6 $

	Ainsi, si on résoud l'équation, on obtient :
	\begin{align*}
		(x - 5) ÷ 3 - 6 & = 8  \\
		(x - 5) ÷ 3     & = 14 \\
		x - 5           & = 42 \\
		x               & = 47
	\end{align*}
\end{exercice}

\end{document}