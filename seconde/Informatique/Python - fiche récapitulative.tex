\documentclass[
	classe=$2^{de}$,
	headerTitle=Fiche\space récapitulative
]
{informatique}

\usepackage{tcolorbox}

\title{Python - fiche récapitulative}

\begin{document}

\maketitle

\begin{tcolorbox}
	Python est un language de programmation : il permet d'écrire des instructions qui seront ensuite exécutées par l'ordinateur.

	Cette fiche ne décrit que les bases de Python : on ne parle pas de \texttt{class}, de dictionnaires, ou encore de libraries.
\end{tcolorbox}

\subsection*{Affichage}

Pour afficher du texte, des nombres ou autre, on utilise la fonction \texttt{print} :
\begin{lstlisting}
print(5) # affiche «5»
print(3.8, 6)  # affiche «3.8 6»
print("bonjour !") # affiche «bonjour !»
\end{lstlisting}

\subsection*{Variables}

En python, le symbol \texttt{=} n'a pas le même sens qu'en mathématiques : il sert à \textit{assigner une variable}.

\begin{lstlisting}
x = 3
print(x) # affiche «3»
\end{lstlisting}

Ici la valeur de \texttt{x} peut changer au cours du programme :

\begin{lstlisting}
x = 3
print(x) # affiche «3»
x = 5
print(x) # affiche «5»
x = x + 6 # la nouvelle valeur de x est : l'ancienne valeur de x, additionnée à 6
print(x) # affiche «11»
\end{lstlisting}

\subsection*{Types}

Chaque valeur en python a un \textbf{type} :
\begin{itemize}
	\item \textbf{entier} (\texttt{int}) : ce sont les nombres entiers, comme \texttt{0}, \texttt{15} ou \texttt{-6}.
	\item \textbf{flottant} (\texttt{float}) : ce sont les nombres à virgule, comme \texttt{2.3}, \texttt{-0.75} ou \texttt{500.2}.

	      À noter : en python, on utilise un point plutôt qu'une virgule.
	\item \textbf{booléen} (\texttt{bool}) : ce sont les valeur \texttt{True} (Vrai) et \texttt{False} (Faux).
	\item \textbf{chaîne de caractères} (\texttt{string}) : ce sont les morceaux de texte : par exemple, \texttt{"mon texte"} est une chaîne de caractères.
\end{itemize}

\subsection*{Calculs}

On peut utiliser les quatres opérations de bases sur les nombres (entiers et flottants) :

\begin{multicols}{2}
	\begin{itemize}
		\item addition avec \texttt{+}
		\item soustraction avec \texttt{-}
		\item multiplication avec \texttt{*}
		\item division avec \texttt{/}
	\end{itemize}
\end{multicols}

On peut également :
\begin{itemize}
	\item Concaténer deux chaînes de caractères avec \texttt{+} : \squareFrame[noshadow]{\texttt{"bon" + "jour !"}}
	\item Tester si deux valeurs sont égales : \squareFrame[noshadow]{\texttt{3 == 3}}
	\item Tester si deux valeurs sont différentes : \squareFrame[noshadow]{\texttt{5 != "bonjour"}}
	\item Tester si une valeur est plus grande ou plus petite qu'une autre :
	      \squareFrame[noshadow]{\texttt{-1 < 0}}, \squareFrame[noshadow]{\texttt{-1 <= 0}}, \squareFrame[noshadow]{\texttt{0 >= 0}}, \squareFrame[noshadow]{\texttt{1 > 0}}
	\item Faire une division euclidienne : \texttt{//} donne le quotient, et \texttt{\%} donne le reste. Par exemple,

	      \squareFrame[noshadow]{\texttt{(13 // 3) == 4}} et \squareFrame[noshadow]{\texttt{(13 \% 3) == 1}}.
\end{itemize}

\subsection*{Boucles}

Il y a deux manière de faire une boucle en Python :
\begin{itemize}
	\item La boucle \texttt{for} :
	      \begin{lstlisting}
for i in range(6):
    print(i)
# affiche «0 1 2 3 4 5»
	      \end{lstlisting}
	\item La boucle \texttt{while} :
	      \begin{lstlisting}
i = 0
while i < 6:
    print(i)
	i += 1
# affiche «0 1 2 3 4 5»
		  \end{lstlisting}
\end{itemize}

\subsection*{Conditions}

On peut dire à Python d'éxécuter du code seulement si un condition est vraie avec les mots-clé \texttt{if}, \texttt{else} et \texttt{elif} :
\begin{lstlisting}
x = 3

if x < 2:      # Si x est plus petit que 2
	print("a")
elif x > 5:    # Sinon, si x est plus grand que 5
	print("b")
else:          # Sinon
	print("c")
# affiche «c», mais pas «a» ni «b».
\end{lstlisting}

\subsection*{Fonctions}

Pour pouvoir lancer le même code avec des valeurs différentes, Python permet de définir des fonctions avec le mot-clé \texttt{def} :

\begin{lstlisting}
def ma_fonction(x, y):
    print(x, y)

ma_fonction(2, 3) # affiche «2 3»
ma_fonction(5, -1) # affiche «5 -1»
\end{lstlisting}

\subsection*{Listes}

On peut regrouper les objets dans un autres type : les \textbf{listes} (\texttt{list}) :
\begin{lstlisting}
ma_liste = [1, 2, 3] # une liste de 3 éléments
print(ma_liste[0]) # affiche le premier élément (on compte à partir de 0 en Python)
ma_liste[1] = 6
print(ma_liste) # affiche «[1, 6, 3]»
\end{lstlisting}

\end{document}