\documentclass[
	classe=$2^{de}$,
	headerTitle=Python\space :\space turtle
]
{informatique}

\usepackage{tcolorbox}

\newcommand{\codebox}[1]{\squared{\texttt{#1}}}

\title{Python - Tortues et avions}

\begin{document}

\maketitle

On va aujourd'hui utiliser le module \texttt{turtle} de python.

\subsection*{Premiers dessins}

Pour commencer à utiliser \texttt{turtle}, on doit lancer les commandes suivantes dans l'interpréteur (la fenêtre en bas à droite de Spyder) :
\begin{lstlisting}
>>> import turtle
>>> t = turtle.Turtle()
\end{lstlisting}

On pourra maintenant utiliser \texttt{t} pour dessiner sur la fenêtre apparue :
\begin{itemize}
	\item \codebox{t.up()} permet de \textit{lever le stylo} : la tortue de dessinera plus dans cet état. Pour qu'elle puisse à nouveau dessiner, on utilise \codebox{t.down()}.
	\item La tortue peut \textit{avancer} avec \codebox{t.forward(...)} (mettre un nombre à la place des \texttt{...}).

	      Elle peut aussi \textit{reculer} avec \codebox{t.backward(...)}.
	\item Pour la faire tourner, on utilisera \codebox{t.right(...)} ou \codebox{t.left(...)}.
	\item Pour revenir au centre, on utilise \codebox{t.home()}. Pour effacer l'écran, on utilise \codebox{t.clear()}.
\end{itemize}

\begin{enumerate}
	\item À quel angle correspond \codebox{t.right(90)} ?
	\item Dessiner : \begin{itemize}
		      \item Un carré
		      \item Un pentagone
		      \item Un hexagone
	      \end{itemize}

	      Essayer de tracer ces figures en utilisant une boucle \texttt{for}.
\end{enumerate}

\subsection*{Avion}

On va reprendre le tracé de la trajectoire d'un avion avec des vecteurs, fait la semaine dernière. Pour rappel :
\begin{itemize}
	\item La gravité est un vecteur constant $\vec{g}$, dirigé vers le bas.
	\item À chaque unité de temps, la vitesse de l'avion $\vec{v}$ \textbf{devient} $\vec{v} + \vec{g}$.
	\item À chaque unité de temps, l'avion est déplacé de $\vec{v}$.
\end{itemize}

Ici l'avion sera représenté par notre tortue.
\begin{enumerate}
	\item Que fait la fonction \codebox{t.position()} ?

	      Si on écrit \codebox{x, y = t.position()}, à quoi sont alors égal \texttt{x} et \texttt{y} ?
	\item Que fait la fonction \codebox{t.setpos(..., ...)} ?
	\item Écrire ainsi les variables \texttt{vitessex} et \texttt{vitessey} permettant de stocker la vitesse de l'avion, ainsi que \texttt{gravitéx} et \texttt{gravitéy} permettant de stocker la gravité.
	\item Compléter le code suivant afin de faire progresser l'avion :
	      \begin{lstlisting}
vitessex = ...
vitessey = ...
gravitéx = ...
gravitéy = ...
for i in range(100):
	vitessex = vitessex + ...
	vitessey = vitessey + ...
	x, y = t.position()
	x = x + ...
	y = y + ...
	t.setpos(..., ...)
		  \end{lstlisting}
	\item (BONUS) Faire en sorte que l'avion rebondisse en touchant le bas de l'écran.
\end{enumerate}

\end{document}