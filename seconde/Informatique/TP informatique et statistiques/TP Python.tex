\documentclass[
	classe=$2^{de}$,
	headerTitle=Python
]{informatique}

\usepackage{awesomebox}

\title{Python : listes et statistiques}

\begin{document}

\maketitle

\warningbox{Répondre aux questions dans son cahier !}

\tipbox{Lorsqu'il est demandé de compléter un morceau de code, il est souvent pratique d'utiliser le panneau de \textit{gauche} de Spyder. Tout ce qui y est écrit peut alors être lancé dans l'interpréteur (fenêtre de droite) en appuyant sur la touche \texttt{F5}.}

\section{Listes}

Nous allons utiliser des \uline{listes} en Python.

En Python, une liste est définie en ouvrant des crochets \texttt{[]}, et en plaçant des éléments à l'intérieur, séparés par des virgules.

\begin{lstlisting}
maliste = [6, 9, 10, 5 + 2]
\end{lstlisting}

on peut accéder au éléments d'une liste en \uline{indexant}:

\begin{lstlisting}
print(maliste[0]) # Affiche le premier élément de la liste
\end{lstlisting}

Rentrer le code ci-dessus dans l'interpréteur.

\begin{enumerate}
	\item Afficher le 4ème élément.
	\item Que se passe-t'il si on essaie d'afficher le 5ème élément de cette liste ?

	      Il est possible d'obtenir la longueur d'une liste en utilisant la fonction \texttt{len}. On peut également obtenir tous les éléments de la liste l'un après l'autre, en utilisant une boucle \texttt{for ... in liste}.

	      \begin{lstlisting}
autreliste = [2, 5, 7, 11]
for x in autreliste:
    print(x)
print("la longueur de autreliste est ", len(autreliste))
\end{lstlisting}

	\item Qu'affiche la fonction ci-dessus ?

	      Enfin, il est possible de générer les listes d'une autre manière, dite 'impérative'

	      \begin{lstlisting}
liste1 = [i for i in range(50)]
liste2 = [2 * i for i in range(50)]
\end{lstlisting}

	\item Que contient la liste \texttt{liste1} ?
	\item Que contient la liste \texttt{liste2} ?
	\item Compléter le code ci-dessous pour obtenir la somme de tous les éléments de la liste \texttt{liste1}:

	      \begin{lstlisting}
total = 0
for x in ...:
    ...
print(total)
\end{lstlisting}
\end{enumerate}

\newpage

\section{Lancers de dés}

\notebox{Cette activité est une reproduction du TP 3 page 297 du manuel.}

\begin{enumerate}
	\item \begin{enumerate}
		      \item Reproduire la fonction ci-dessous puis la compléter pour qu’elle renvoie l’effectif d’une valeur d’une liste donnée.

		            \begin{lstlisting}
from random import randint
import matplotlib.pyplot as plt


def nombre_de(valeur, liste):
	nb=0
	for x in liste:
		if x==...:
			...
	return nb
\end{lstlisting}
		      \item Que renvoie \texttt{nombre\_de(1, [2,1,4,3,1,3,6,1,7,12])} ?
	      \end{enumerate}
	\item On considère l'expérience aléatoire suivante : on lance deux dés équilibrés à six faces et on fait la somme des deux faces obtenues.

	      Le programme suivant écrit à la suite de la fonction \texttt{nombre\_de}, permet de représenter par un nuage de points le nombre d’obtentions de chaque issue lorsqu’on répète 1 000 fois  l’expérience précédente.

	      \begin{lstlisting}
L = [randint(1,6)+randint(1,6) for i in range(1000)]
for k in range(...,...):
	plt.plot(k,nombre_de(k,L),'*')
plt.show()
\end{lstlisting}

	      \begin{enumerate}
		      \item Compléter ce programme et commenter le graphique obtenu.
		      \item Modifier ce programme de façon à répéter 10 000 fois l'expérience aléatoire.
		      \item Modifier ce programme pour qu’il calcule la moyenne des sommes obtenues après 10 000 lancers des deux dés.
	      \end{enumerate}
\end{enumerate}

\end{document}