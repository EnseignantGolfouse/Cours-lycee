\documentclass[
	classe=$2^{de}$
]{exercice}

\usepackage{tikz-repère}

\newcommand{\FonctionF}[1]{
	(#1*#1 - 16*#1 + 140) / 10
}
\newcommand{\CalculF}[1]{
	\pgfmathparse{(#1*#1 - 16*#1 + 140) / 10}
	\pgfmathresult
}
\renewcommand{\arraystretch}{1.4}

\title{Activité : représentation d'une réduction}

\begin{document}

\maketitle

Un magasin propose une réduction si on achète un lot de savons.

Si on achète $n$ savons, le prix individuel d'un savon est donné par la fonction $f$ telle que

$$f(n) = \dfrac{1}{10}(x² - 16x + 140)$$

\begin{enumerate}
	\item Combien coûte l'achat d'un savon seul ? \correction{$f(1) = \CalculF{1}$}
	\item Calculer le prix individuel lorsqu'on achète deux savons, et vérifier qu'on bénéficie bien d'une réduction. \correction{$f(2) = \CalculF{2}$}
	\item Un employé du magasin affirme qu'il y a un problème avec la réduction : on va donc essayer de le vérifier en calculant le prix individuel pour un achat allant de $1$ à $7$ savons.

	      Remplir alors le tableau suivant :
	      \begin{center}
		      \begin{tabular}{|l|*{7}{>{\centering}p{1cm}|}}
			      \hline
			      Nombre de savons & 1                        & 2                        & 3                 & 4                        & 5                        & 6                        & 7 \tabularnewline \hline
			      Prix individuel  & \correction{\CalculF{1}} & \correction{\CalculF{2}} & \correction{10.1} & \correction{\CalculF{4}} & \correction{\CalculF{5}} & \correction{\CalculF{6}} & \correction{\CalculF{7}} \tabularnewline \hline
		      \end{tabular}
	      \end{center}
	      À-t'on toujours une réduction ?
	\item Face à son insistance, on décide de visualiser la situation avec un graphe :

	      \begin{center}
		      \begin{tikzpicture}
			      \tikzRepere{0}{11}{0}{8}[][][$n$][prix (en €)]
			      \foreach \n in {1,...,10} {
					      \draw (\n,0) -- ++(0,-0.2) node[below] {$\n$};
				      }
			      \foreach \y in {1,...,8} {
					      \draw (0,\y) -- ++(-0.2,0) node[left] {$\pgfmathparse{int(\y+6)}\pgfmathresult$};
				      }

			      \ifdefined\makeCorrection
				      \foreach \x in {0,...,10} {
						      \pgfmathsetmacro\y{(\x*\x - 16*\x + 140) / 10 - 6}
						      \node[red] at (\x,\y) {×};
					      }
				      \draw[very thick,red,domain=0:10] plot({\x},{\FonctionF{\x} - 6});
			      \fi
		      \end{tikzpicture}
	      \end{center}

	      Dans le repère suivant, placer les points $(n ; f(n))$ pour $n$ allant de $1$ à $10$. Relier ensuite ces points pour former le graphe de la fonction $f$.

	      Que remarque-t'on ?
	\item Calculer le prix individuel d'un savon si on achète un lot de $40$ savons. \correction{$f(40) = \CalculF{40}$}

	      La formule proposée dans l'énoncé parait-elle raisonnable ?
\end{enumerate}

\end{document}