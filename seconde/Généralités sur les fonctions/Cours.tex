\documentclass[
	classe=$2^{de}$,
	headerTitle=Cours\space Chapitre\space 7
]{coursclass}

\usetikzlibrary{automata,calc,positioning}

\title{Chapitre 7 : Fonctions}
\date{}
\author{}

\begin{document}

\maketitle

\begin{definition}[Fonction, image, antécédent]
	Une \textbf{fonction} est un procédé qui à un nombre réel $x$ associe un unique nombre réel $f(x)$.

	\begin{minipage}{0.65\textwidth}
		\begin{itemize}
			\item $f(x)$ est \textbf{L'image} de $x$ par la fonction $f$. On représente une image par la lettre $y$, et on écrit alors

			      $$ f(x) = y $$
			\item $x$ est \textbf{UN antécédent} de $y$.
		\end{itemize}
	\end{minipage}
	\begin{minipage}{0.3\textwidth}
		\hspace{2em}
		\begin{tikzpicture}
			\node (X) {$x$};
			\node (F) [right=of X] {};
			\node (Y) [right=of F] {$f(x)$};
			\path[\myArrow] (X) edge node[above] {$f$} (Y);
		\end{tikzpicture}
	\end{minipage}
\end{definition}

\begin{remarque}
	\begin{itemize}
		\item Il n'y a \uline{qu'une seule image} pour un nombre donné.
		\item Il peut y avoir \uline{plusieurs antécédents} pour un nombre donné.
	\end{itemize}
\end{remarque}

\begin{definition}[Calcul d'image]
	Si on a une expression \textbf{algébrique} de la fonction $f$, on peut calculer l'image d'un nombre en remplaçant $x$ par ce nombre dans l'expression de la fonction.
\end{definition}

\begin{exemple}
	Si $f$ est la fonction qui à $x$ associe $3x + 2$ :
	\begin{itemize}
		\item $f(2) = 3×2 + 2 = 8$
		\item Attention : si on remplace $x$ par une expression complexe, il faut ajouter des parenthèses. Par exemple,

		      $f(1 + 3) = 3×(1 + 3) + 2 = 3×4 + 2 = 14$
	\end{itemize}
\end{exemple}

\begin{definition}[Domaine de définition]
	L'ensemble des nombres ayant une image par la fonction $f$ est appelé le \textbf{domaine de définition} de $f$. On le note $𝒟_f$.
\end{definition}

\begin{exemple}
	\begin{itemize}
		\item Si $f$ est la fonction qui à $x$ associe $\dfrac{1}{x}$, alors $x$ ne peut pas être $0$.

		      Son domaine de définition est $𝒟_f = \intervalle{]}{-∞}{0}{[} ∪ \intervalle{]}{0}{+∞}{[}$.
		\item Si $f$ représente une longueur, $x$ ne peut pas être négatif.

		      Son domaine de définition est $𝒟_f = \intervalle{[}{0}{+∞}{[}$.
	\end{itemize}
\end{exemple}

\newpage

\begin{definition}[Courbe représentative]
	La \textbf{courbe représentative} d'une fonction $f$ est l'ensemble des points $(x ; y)$ tels que $y = f(x)$.
\end{definition}

\begin{definition}[fonction affine]
	Une \textbf{fonction affine} est une fonction définie par
	$$ f(x) = mx + p $$
	où $m$ et $p$ sont des nombres réels fixés.
\end{definition}

\begin{remarque}
	\begin{itemize}
		\item Si $p = 0$, on a alors $f(x) = mx$, donc la fonction est \textbf{linéaire}.
		\item Si $m = 0$, on a alors $f(x) = p$, donc la fonction est \textbf{constante}.
	\end{itemize}
\end{remarque}

\begin{propriete}[Représentation d'une fonction affine]
	La représentation graphique d'une fonction affine est une droite.
\end{propriete}

\end{document}
