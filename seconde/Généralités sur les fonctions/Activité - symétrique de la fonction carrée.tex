\documentclass[
	classe=$2^{de}$
]{exercice}

\usepackage{tikz-repère}

\renewcommand{\arraystretch}{1.3}

\title{Activité : symétrique de la fonction carrée}

\begin{document}

\maketitle

\begin{center}
	\begin{tikzpicture}[scale=0.8]
		\tikzRepere{0}{16}{0}{16}[1][1]
		% \foreach \x in {2,...,4} {
		% \draw[thick] (\x,0) -- ++(0,-0.2) node[below] {$\x$};
		% }
		\ifdefined\makeCorrection
			\foreach \x in {0,...,4} {
					\node[blue] at (\x,\x*\x) {×};
					\node[red] at (\x*\x,\x) {×};
				}
			\draw[thick,domain=0:16] plot({\x},{\x}) node[above left] {$g$};
		\fi
	\end{tikzpicture}
\end{center}

\begin{enumerate}
	\item Soit $f$ la fonction $f(x) = x^2$.

	      Dans le repère ci-dessus, placer les points $(x, f(x))$ pour $x ∈ \{0, 1, 2, 3, 4\}$.
	\item Tracer le graphe de la fonction $g(x) = x$ pour $x$ entre $0$ et $16$.
	\item Pour chaque point placé à la question 1, placer son symétrique par rapport à la courbe de $g$.

	      Donner alors les coordonnées de ces points dans le tableau suivant :
	      \begin{center}
		      \begin{tabular}{|l|*{5}{>{\centering}p{1cm}|}}
			      \hline
			      Abscisse du point : & \correction{0} & \correction{1} & \correction{4} & \correction{9} & \correction{16} \tabularnewline \hline
			      Ordonnée du point : & \correction{0} & \correction{1} & \correction{2} & \correction{3} & \correction{4} \tabularnewline \hline
		      \end{tabular}
	      \end{center}
	\item Si on appelle $h$ la fonction dont la courbe correspond à ces nouveaux points, quelle semble être l'expression de $h$ ?
	      $$ h(x) = \correctionDots{\sqrt{x}} $$
\end{enumerate}

\end{document}