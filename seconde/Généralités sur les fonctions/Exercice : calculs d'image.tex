\documentclass[
	classe=$2^{de}$,
	landscape,
	twocolumn
]{exercice}

\usepackage{tikz-repère}

\setlength{\columnsep}{1cm}

\title{Exercices : Calculs d'images}

\begin{document}

\newcommand{\Exercice}{
	\maketitle

	\begin{enumerate}
		\item Soit $f$ la fonction telle que $f(x) = 3x + 1$.
		      \begin{enumerate}
			      \item Calculer l'image par $f$ de $2$, $-5$ et $0$.
			      \item  On a alors :
			            \begin{itemize}
				            \item $2$ est un antécédent de $\correctionDots{\phantom{-}7}$.
				            \item $-5$ est un antécédent de $\correctionDots{-14}$.
				            \item $0$ est un antécédent de $\correctionDots{\phantom{-}1}$.
			            \end{itemize}
		      \end{enumerate}
		\item Soit $f$ la fonction telle que $f(x) = (x + 3)(2x - 4)$.

		      \begin{enumerate}
			      \item Calculer l'image par $f$ de $-3$, $0$ et $2$.
			      \item  On a alors :
			            \begin{itemize}
				            \item $\correctionDots{-3}$ et $\correctionDots{\phantom{-}2}$ sont des antécédents de $\correctionDots{\phantom{-}0}$.
				            \item $\correctionDots{\phantom{-}0}$ est un antécédent de $\correctionDots{-12}$.
			            \end{itemize}
		      \end{enumerate}
		\item Soit $f$ la fonction telle que $f(x) = \dfrac{x + 1}{x - 7}$

		      \begin{enumerate}
			      \item Calculer l'image par $f$ de $4$ et $-1$.
			      \item Que peut-on dire de l'image par $f$ de $7$ ?
		      \end{enumerate}
	\end{enumerate}
}

\Exercice

\newpage

\Exercice

\end{document}