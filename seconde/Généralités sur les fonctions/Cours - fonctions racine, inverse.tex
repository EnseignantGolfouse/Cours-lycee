\documentclass[noheader]{coursclass}

\usepackage{tikz-repère}
\usepackage{tkz-tab}
\usepackage{diagbox}

\renewcommand{\arraystretch}{1.3}

\begin{document}

\begin{greybox}[frametitle={Fonction racine carrée}]
	La \textbf{fonction racine carrée} est $f(x) = \sqrt{x}$, définie sur $\intervalle{[}{0}{+∞}{[}$.

	\begin{center}
		\begin{tabular}{|l|*{6}{>{\centering}p{1.2cm}|}}
			\hline
			$x$        & $0$ & $1$ & $2$     & $3$     & $4$ & $5$ \tabularnewline \hline
			$\sqrt{x}$ & $0$ & $1$ & $≈1,41$ & $≈1,73$ & $2$ & $≈2,24$ \tabularnewline \hline
		\end{tabular} \medskip

		\begin{tikzpicture}[scale=0.7]
			\tikzRepere{-0.5}{8.5}{-0.5}{4.5}
			\draw[very thick,blue,domain=0:1] plot({\x},{sqrt(\x)});
			\draw[very thick,blue,domain=1:9] plot({\x},{sqrt(\x)}) node[right] {$f$};
		\end{tikzpicture} \medskip


		\begin{tikzpicture}[scale=0.7]
			\tkzTabInit{$x$ / 1 , $\sqrt{x}$ / 2}{$0$, $+∞$}
			\tkzTabVar{-/ $0$, +/ }
		\end{tikzpicture}
	\end{center}
\end{greybox}

\vfill

\begin{greybox}[frametitle={Fonction inverse}]
	La \textbf{fonction inverse} est $f(x) = \dfrac{1}{x}$, définie sur $\intervalle{]}{-∞}{0}{[}∪\intervalle{]}{0}{+∞}{[}$.

	\begin{center}
		\renewcommand{\arraystretch}{1.2}
		\begin{tabular}{|l|*{9}{c|}}
			\hline
			$x$            & $-4$    & $-3$     & $-2$   & $-1$ & $0$ & $1$ & $2$   & $3$     & $4$  \tabularnewline \hline
			               &         &          &        &      &     &     &       &         & \tabularnewline
			$\dfrac{1}{x}$ & $-0,25$ & $≈-0,33$ & $-0,5$ & $-1$ & NON & $1$ & $0,5$ & $≈0,33$ & $0,25$ \tabularnewline
			               &         &          &        &      &     &     &       &         & \tabularnewline \hline
		\end{tabular} \medskip

		\begin{tikzpicture}[scale=0.7]
			\tikzRepere{-3.5}{3.5}{-3.5}{3.5}
			\draw[very thick,blue,domain=-4:-0.25] plot({\x},{1/\x});
			\draw[very thick,blue,domain=0.25:4] plot({\x},{1/\x}) node[right] {$f$};
		\end{tikzpicture} \medskip

		\begin{tikzpicture}[scale=0.7]
			\tkzTabInit{$x$ / 1 , $\dfrac{1}{x}$ / 2}{$-∞$, $0$, $+∞$}
			\tkzTabVar{+/ , -D+/{}/{}, -/ }
		\end{tikzpicture}
	\end{center}
\end{greybox}

\end{document}