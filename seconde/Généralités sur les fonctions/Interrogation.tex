\documentclass[
	classe=$2^{de}$,
	landscape,
	twocolumn
]{évaluation}

\usepackage{tikz-repère}

\setlength{\columnsep}{1cm}
\newcommand{\computeF}{}
\newcommand{\reponseDots}{......}
\newcommand{\MyCorrectionDots}[1]{
	\correctionOr{{\color{red}#1}}{\reponseDots}
}

\date{31 mars 2023}

\begin{document}

\title{Interrogation : fonctions (sujet A)}
\maketitle

\begin{exercice}
	Soit $f$ la fonction telle que $f(x) = \dfrac{1}{2}x × (x - 1)$.
	\renewcommand{\computeF}[1]{
		\pgfmathparse{int(0.5*#1*(#1 - 1))}\pgfmathresult
	}

	\begin{enumerate}
		\item Calculer $f(1)$.
		\item Remplir alors la phrase suivante :

		      $\MyCorrectionDots{\computeF{1}}$ est l'image de $\MyCorrectionDots{1}$ par $f$.
		\item Calculer $f(-2)$ et $f(3)$.
		\item Remplir alors la phrase suivante :

		      $\MyCorrectionDots{-2}$ et $\MyCorrectionDots{3}$ sont des antécédents de $\MyCorrectionDots{\computeF{3}}$ par $f$.
	\end{enumerate}
\end{exercice}

\begin{exercice}
	\begin{center}
		\begin{tikzpicture}[scale=0.8]
			\tikzRepere{-5}{5}{-3}{3}
			\draw[very thick,blue,domain=-5:5] plot({\x}, {2*cos((3.1415*(\x+1)/4) r)}) node[right] {$f$};
		\end{tikzpicture}
	\end{center}
	Sur le graphe ci-dessus, lire :
	\begin{itemize}
		\item $f(-3) = \MyCorrectionDots{0}$
		\item $f(-1) = \MyCorrectionDots{2}$
		\item $f(1) = \MyCorrectionDots{0}$
	\end{itemize}
\end{exercice}

%================================================
%================== SUJET B =====================
%================================================
\newpage
\setcounter{exercice}{1}
\title{Interrogation : fonctions (sujet B)}
\maketitle

\begin{exercice}
	Soit $f$ la fonction telle que $f(x) = \dfrac{1}{2}x × (x - 3)$.
	\renewcommand{\computeF}[1]{
		\pgfmathparse{int(0.5*#1*(#1 - 3))}\pgfmathresult
	}

	\begin{enumerate}
		\item Calculer $f(2)$.
		\item Remplir alors la phrase suivante :

		      $\MyCorrectionDots{\computeF{2}}$ est l'image de $\MyCorrectionDots{2}$ par $f$.
		\item Calculer $f(-1)$ et $f(4)$.
		\item Remplir alors la phrase suivante :

		      $\MyCorrectionDots{-1}$ et $\MyCorrectionDots{4}$ sont des antécédents de $\MyCorrectionDots{\computeF{4}}$ par $f$.
	\end{enumerate}
\end{exercice}

\begin{exercice}
	\begin{center}
		\begin{tikzpicture}[scale=0.8]
			\tikzRepere{-5}{5}{-3}{3}
			\draw[very thick,blue,domain=-5:5] plot({\x}, {2*sin((3.1415*(\x+1)/4) r)}) node[right] {$f$};
		\end{tikzpicture}
	\end{center}
	Sur le graphe ci-dessus, lire :
	\begin{itemize}
		\item $f(-3) = \MyCorrectionDots{-2}$
		\item $f(-1) = \MyCorrectionDots{0}$
		\item $f(1) = \MyCorrectionDots{2}$
	\end{itemize}
\end{exercice}

\end{document}