\documentclass[
	classe=$2^{de}$
]{exercice}

\usepackage{tikz-repère}

\title{Activité : fonction carrée}

\begin{document}

\maketitle

On lance une balle dans les aires, et on veut suivre sa position.

On suppose que la balle est lancée depuis l'abscisse $x = -30$cm, et que sa hauteur $h$ en fonction de son abscisse $x$ est donnée par : $h(x) = -\dfrac{x^2}{10} + 90$ (en cm).

\begin{enumerate}
	\item D'après l'énoncé, quelle doit être la hauteur de la balle lorsque $x = -30$cm ?

	      Vérifier que $h(-30)$ donne le même résultat.
	\item Calculer $h(0)$, $h(10)$, $h(20)$ et $h(30)$ :

	      $$ h(0) = \correctionDots{90}\ \ h(10) = \correctionDots{80}\ \ h(20) = \correctionDots{50}\ \ h(30) = \correctionDots{\phantom{1}0} $$

	      Que remarque-t'on à propos de $h(30)$ ?

	      \correction{On a $h(30) = h(-30) = 0$}
	\item Calculer $h(-20)$. \correctionDots{$50$}

	      Que peut-on remarquer par rapport à la question précédente ?

	      \correction{On a $h(20) = h(-20) = 50$cm}
	\item Donner alors sans calcul la valeur de $h(-10)$. \correctionDots{$80$cm}
	\item Tracer dans le repère ci-dessous la courbe de la fonction $h$:

	      \begin{center}
		      \begin{tikzpicture}[scale=0.8]
			      \pgfmathsetmacro\xScaling{5}
			      \pgfmathsetmacro\yScaling{10}
			      \pgfmathsetmacro\scaling{10}
			      \tikzRepere{-30/\xScaling}{30/\xScaling}{0}{10}[\xScaling][\yScaling]

			      \ifdefined\makeCorrection1
				      \draw[thick,blue,domain=-30:30] plot({\x / \xScaling},{(-(\x)*(\x)/10 + 90) / \yScaling});
			      \fi
		      \end{tikzpicture}
	      \end{center}
	\item Quelle est alors la hauteur maximale atteinte par la balle ? \correctionDots{$90$cm}

	      En quelle abscisse cette hauteur est-elle atteinte ? \correctionDots{$0$\phantom{9cm}}

	      En quelle abscisse la balle est-elle retombée sur le sol ? \correctionDots{$30$\phantom{cm}}
\end{enumerate}

\end{document}