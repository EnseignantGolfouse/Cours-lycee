\documentclass[
	classe=$2^{de}$,
	headerTitle=Activité,
	landscape,
	twocolumn
]{exercice}

\usepackage{tcolorbox}

\setlength{\columnsep}{1cm}
\renewcommand{\arraystretch}{1.3}
\newcommand{\CalculF}[1]{\directlua{tex.print(100*#1+70)}}
\newcommand{\CalculG}[1]{\directlua{tex.print(118*#1+25)}}

\title{Activité : sur la route}

\begin{document}

\newcommand{\Activite}{
	\maketitle

	\begin{tcolorbox}
		On mesure les positions de deux voitures sur une route :
		\begin{itemize}
			\item La première part au kilomètre $70$, et roule à $100$km/h.
			\item La deuxième part au kilomètre $25$, et roule à $118$km/h.
		\end{itemize}
	\end{tcolorbox}

	\begin{enumerate}
		\item Quelle est la position de chaque voiture :
		      \begin{itemize}
			      \item Au bout d'une heure ? \correction{$170$km et $143$km}
			      \item Au bout de deux heures ? \correction{$270$km et $261$km}
		      \end{itemize}
		\item Au bout d'un nombre variable $t$ d'heures, quelle sera la position de la première voiture ?

		      Écrire alors l'expression d'une fonction $f$, qui donne la position de la voiture en fonction du temps $t$ écoulé. \correction{$f(t) = 70 + 100t$}
		\item Trouver de même une fonction $g$ qui donne la position de la deuxième voiture en fonction du temps $t$ écoulé. \correction{$f(t) = 25 + 118t$}
		\item Remplir alors le tableau suivant avec les valeurs de $f(t)$ et $g(t)$, pour les valeurs de $t$ demandées :
		      \begin{center}
			      \begin{tabular}{|l|*{5}{>{\centering}p{1cm}|}}
				      \hline
				      $t$ (en heures)        & 0                        & 1                        & 2                        & 3                        & 4 \tabularnewline \hline
				      $f(t)$ (en kilomètres) & \correction{\CalculF{0}} & \correction{\CalculF{1}} & \correction{\CalculF{2}} & \correction{\CalculF{3}} & \correction{\CalculF{4}} \tabularnewline \hline
				      $g(t)$ (en kilomètres) & \correction{\CalculG{0}} & \correction{\CalculG{1}} & \correction{\CalculG{2}} & \correction{\CalculG{3}} & \correction{\CalculG{4}} \tabularnewline \hline
			      \end{tabular}
		      \end{center}
		\item À l'aide de ce tableau, établir une fourchette du moment où les deux voitures sont au même niveau. \correction{Entre l'heure $2$ et $3$.}
		\item On suppose que les voitures s'arrêtent après $5$ heures de route.

		      Quel est alors de domaine de définition de ces fonctions ?
	\end{enumerate}
}

\Activite

\newpage

\Activite

\end{document}