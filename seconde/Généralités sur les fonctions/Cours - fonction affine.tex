\documentclass[noheader]{coursclass}

\usetikzlibrary{automata,calc,positioning}

\title{Chapitre 7 : Fonctions}
\date{}
\author{}

\begin{document}

\newcommand{\Cours}{
\begin{definition}[fonction affine]
	Une \textbf{fonction affine} est une fonction définie par
	$$ f(x) = mx + p $$
	où $m$ et $p$ sont des nombres réels fixés.
\end{definition}

\begin{remarque}
	\begin{itemize}
		\item Si $p = 0$, on a alors $f(x) = mx$, donc la fonction est \textbf{linéaire}.
		\item Si $m = 0$, on a alors $f(x) = p$, donc la fonction est \textbf{constante}.
	\end{itemize}
\end{remarque}
}

\Cours

\vfill

\Cours

\vfill

\Cours

\vfill

\Cours

\end{document}
