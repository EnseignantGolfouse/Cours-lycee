\documentclass[
	classe=$2^{de}$
]{exercice}

\usepackage{tikz-repère}

\title{Activité : intersections de droites}

\begin{document}

\maketitle

\begin{center}
	\begin{tikzpicture}[scale=0.8]
		\tikzRepere{-4.5}{4.5}{-4.5}{4.5}

		% \ifdefined\makeCorrection
		\draw[very thick,red,domain=-3:2] plot({\x},{2*\x + 1}) node[right] {$(d_1)$};
		\draw[very thick,blue,domain=-1/3:3] plot({\x},{-3*\x + 4}) node[right] {$(d_2)$};
		\draw[very thick,green,domain=-5:5] plot({\x},{0.5*\x - 1}) node[right] {$(d_3)$};
		% \fi
	\end{tikzpicture}
\end{center}

\begin{enumerate}
	\item Tracer dans le repère ci-dessus les droites, et donner un vecteur directeur :
	      \begin{itemize}
		      \item $(d_1)$ d'équation cartésienne $-2x + y - 1 = 0$.\hspace{0.2em} Vecteur directeur : $\begin{pmatrix}\correction{1}\\\correction{\phantom{-}2}\end{pmatrix}$
		      \item $(d_2)$ d'équation cartésienne $3x + y - 4 = 0$.\hspace{1em} Vecteur directeur : $\begin{pmatrix}\correction{1}\\\correction{-3}\end{pmatrix}$
		      \item $(d_3)$ d'équation cartésienne $x - 2y - 2 = 0$.\hspace{1em} Vecteur directeur : $\begin{pmatrix}\correction{2}\\\correction{\phantom{-}1}\end{pmatrix}$
	      \end{itemize}
	\item Pour chaque droite, manipuler l'équation cartésienne afin d'obtenir un $y$ isolé : \medskip
	      \begin{itemize}
		      \setlength{\itemsep}{1em}
		      \item $(d_1)$ : $y = \correctionOr{2x + 1}{................}$
		      \item $(d_2)$ : $y = \correctionOr{4 - 3x}{................}$
		      \item $(d_3)$ : $y = \correctionOr{0,5x - 1}{................}$
	      \end{itemize}
	\item On cherche maintenant à trouver le point à l'intersection de $(d_1)$ et $(d_2)$. D'après la question ci dessus, les coordonnées $(x ; y)$ de ce point vérifient deux équations : lesquelles ?

	      \newcommand{\MyDots}{.......................}
	      \begin{align*}
		      \correctionOr{y = 2x + 1}{\MyDots} &  & \text{et} &  &  & = \correctionOr{y = 4 - 3x}{\MyDots}
	      \end{align*}
	\item Combiner ces deux équations pour trouver la valeur de $x$.

	      En déduire la valeur de $y$.

	      Ainsi le point à l'intersection de $(d_1)$ et $(d_2)$ a pour coordonnées $(\ \correctionDots{ab}\ ;\ \correctionDots{ab}\ )$
	\item Déterminer de même les coordonnées des points à l'intersection de
	      \begin{itemize}
		      \item $(d_1)$ et $(d_3)$ : $(\ \correctionDots{ab}\ ;\ \correctionDots{ab}\ )$
		      \item $(d_2)$ et $(d_3)$ : $(\ \correctionDots{ab}\ ;\ \correctionDots{ab}\ )$
	      \end{itemize}
\end{enumerate}

\end{document}