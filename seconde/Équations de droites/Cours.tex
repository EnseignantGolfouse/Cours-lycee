\documentclass[
	classe=$2^{de}$
]{coursclass}

\usepackage{tikz-repère}
\usetikzlibrary{calc}

\title{Chapitre 8 : Équations de droites}
\date{}

\begin{document}

\maketitle

\begin{definition}[Vecteur directeur]
	\begin{minipage}{0.55\textwidth}
		Si on dispose d'une droite $(d)$ et de deux points $A$ et $B$ sur cette droite, alors $\vec{AB}$ est \textbf{un vecteur directeur} de $(d)$.
	\end{minipage}\hspace{0.02\textwidth}
	\begin{minipage}{0.4\textwidth}
		\begin{tikzpicture}[scale=0.8]
			\draw (-2,-1) -- (4,2) node[right] {$(d)$};
			\coordinate (A) at (0,0);
			\coordinate (B) at (3,1.5);
			\foreach \p in {A,B} {
					\node[above left] at (\p) {$\p$};
					\node at (\p) {×};
				}
			\draw[thick,\myArrow] (A) -- node[above left] {$\vec{AB}$} (B);
		\end{tikzpicture}
	\end{minipage}
\end{definition}

\begin{propriete}
	Si on dispose d'un point $A$ et d'un vecteur $\vec{u}$,

	La droite $(d)$ passant par $A$ de vecteur directeur $u$ est constituée de tous les points $M$ vérifiant :
	$$ \vec{u} \text{ et } \vec{AM} \text{ sont colinéaires} $$

	Autrement dit, une droite peut être définie par un point et un vecteur.
\end{propriete}

\begin{exemple}
	On donne $A(-2 ; -3)$ et $\vec{u}\begin{pmatrix}1 \\ 2\end{pmatrix}$. Comme vecteurs colinéaires à $\vec{u}$, on a par exemple $\begin{pmatrix}2 \\ 4\end{pmatrix}$, $\begin{pmatrix}-1 \\ -2\end{pmatrix}$, $\begin{pmatrix}0,5 \\ 1\end{pmatrix}$ ou encore $\begin{pmatrix}3 \\ 6\end{pmatrix}$ : On peut donc placer

	\begin{multicols}{2}
		\begin{itemize}
			\item $M₁$ tel que $\vec{AM₁} = \begin{pmatrix}2 \\ 4\end{pmatrix}$
			\item $M₂$ tel que $\vec{AM₂} = \begin{pmatrix}-1 \\ -2\end{pmatrix}$
			\item $M₃$ tel que $\vec{AM₃} = \begin{pmatrix}0,5 \\ 1\end{pmatrix}$
			\item $M₄$ tel que $\vec{AM₄} = \begin{pmatrix}3 \\ 6\end{pmatrix}$
		\end{itemize}
	\end{multicols}

	\begin{center}
		\begin{tikzpicture}[scale=0.7]
			\tikzRepere{-5}{5}{-5}{5}

			\coordinate (A) at (-2,-3);
			\coordinate (vecU) at (1,2);
			\coordinate (M1) at ($(A) + (2,4)$);
			\coordinate (M2) at ($(A) + (-1,-2)$);
			\coordinate (M3) at ($(A) + (0.5,1)$);
			\coordinate (M4) at ($(A) + (3,6)$);

			\foreach \p in {A,M1,M2,M3,M4} {
					\node at (\p) {×};
					\node[below right] at (\p) {$\p$};
				}
		\end{tikzpicture}
	\end{center}

	On voit que ces points s'alignent pour former une droite.
\end{exemple}

\begin{definition}[Équation cartésienne d'une droite]
	Soient $a$, $b$ et $c$ trois nombres réels, tels qu'au moins l'un des nombres $a$ et $b$ est non nul.

	Alors l'ensemble des solutions de l'équation

	$$ ax + by + c = 0 $$

	forme une droite, dont le vecteur directeur est $\begin{pmatrix}-b \\ a\end{pmatrix}$.

	L'équation $ax + by + c = 0$ est appelée \textbf{l'équation cartésienne} de la droite.
\end{definition}

\begin{exemple}
	\begin{minipage}{0.55\textwidth}
		$3x - 2y + 1 = 0$ est l'équation cartésienne de la droite $(d)$ représentée ci-contre.

		En effet, le point de coordonnées $(1 ; 2)$ appartient à la droite, car $3 × 1 - 2 × 2 + 1 = 0$.
	\end{minipage}\hspace{0.03\textwidth}
	\begin{minipage}{0.4\textwidth}
		\begin{tikzpicture}[scale=0.7]
			\tikzRepere{-2.5}{2.5}{-2.5}{2.5}
			\draw[very thick,red,domain=-7/3:5/3] plot({\x},{1.5*\x + 0.5});
			\draw[thick,dashed] (1,0) -- (1,2) -- (0,2);
		\end{tikzpicture}
	\end{minipage}
\end{exemple}

\end{document}