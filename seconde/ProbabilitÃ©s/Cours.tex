\documentclass[
	classe=$2^{de}$
]{coursclass}

\renewcommand{\arraystretch}{1.5}

\title{Chapitre 8 : Probabilités}
\author{}
\date{}

\begin{document}

\maketitle

\begin{definition}[Univers, issues, évènements]
	\begin{itemize}
		\item Une \textbf{expérience aléatoire} est une expérience dont on ne peut pas prédire le résultat à l'avance.
		\item Lorsqu'on fait une expérience aléatoire, les résultats que l'on peut obtenir sont appelés les \textbf{issues}.
		\item L'ensemble complet des issues est appelé \textbf{l'univers}. On le note généralement $Ω$.
		\item Un ensemble d'issues est un \textbf{évènement}. On le note généralement avec une lettre majuscule, comme $A$, $B$, ...
		\item Si on a un évènement $A$, on appelle $\overline{A}$ l'\textbf{évènement contraire} de $A$ : c'est l'évènement qui contient les issues qui ne sont pas dans $A$.
	\end{itemize}
\end{definition}

\begin{exemple}
	\begin{itemize}
		\item Si on lance un dé et qu'on regarde le résultat, il s'agit d'une expérience aléatoire.
		\item Les issues sont alors $1$, $2$, $3$, $4$, $5$ et $6$. L'univers est $Ω = \{1, 2, 3, 4, 5, 6\}$.
		\item On peut noter $A$ l'évènement « Le résultat obtenu est pair ». $A$ contient alors les issues $2$, $4$ et $6$.
		\item $\overline{A}$ contient les issues $1$, $3$ et $5$.
	\end{itemize}
\end{exemple}

\begin{definition}[Loi de probabilité]
	Si on a un univers, une \textbf{loi de probabilité} consiste à :
	\begin{itemize}
		\item Associer un probabilité entre $0$ et $1$ à chaque issue ;
		\item Tel que la somme des probabilités des issues soit $1$.
	\end{itemize}
\end{definition}

\begin{exemple}
	\begin{multicols}{2}
		La loi de probabilité d'un dé est :
		\begin{center}
			\begin{tabular}{|l|c|c|c|c|c|c|}
				\hline
				Issue       & $1$           & $2$           & $3$           & $4$           & $5$           & $6$           \\ \hline
				Probabilité & $\frac{1}{6}$ & $\frac{1}{6}$ & $\frac{1}{6}$ & $\frac{1}{6}$ & $\frac{1}{6}$ & $\frac{1}{6}$ \\ \hline
			\end{tabular}
		\end{center}

		\columnbreak

		La loi de probabilité d'un dé truqué peut être :
		\begin{center}
			\begin{tabular}{|l|c|c|c|c|c|c|}
				\hline
				Issue       & $1$           & $2$           & $3$           & $4$           & $5$           & $6$           \\ \hline
				Probabilité & $\frac{1}{12}$ & $\frac{1}{3}$ & $\frac{1}{12}$ & $\frac{1}{4}$ & $\frac{1}{6}$ & $\frac{1}{12}$ \\ \hline
			\end{tabular}
		\end{center}
	\end{multicols}
\end{exemple}

\end{document}