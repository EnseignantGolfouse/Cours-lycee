\documentclass[
	classe=$2^{de}$,
	exercices=Exercices\space chapitre\space 1
]{exercice}

\usepackage{préambule}

\newcounter{exercicesCounter}
\setcounter{exercicesCounter}{1}
\newcommand{\makeExercice}{
	\uline{\textbf{Exercice 1.\arabic{exercicesCounter}}}\stepcounter{exercicesCounter}
}

\renewcommand{\arraystretch}{1.5}
\newcommand{\tableSpacing}{0.6em}

\title{Règles de calcul}

\begin{document}

\maketitle

\section{Priorités opératoires}

\makeExercice : Compléter chaque calcul en deux étapes, en respectant les priorités opératoires.

\vspace{\tableSpacing}
\begin{tabularx}{\linewidth}{|X|X|X|X|X|X|}
	\hline
	$A = 5 + 6 × 3$           & $B = 8 - 10 ÷ 2$         & $C = 6 × 8 + 3 × 5$        & $D = 7 × (3 + 8)$         & $E = (3 + 10) × 2$        & $F = 15 × 3 - 8 × 2$       \\
	$A = \correction{5 + 18}$ & $B = \correction{8 - 5}$ & $C = \correction{48 + 15}$ & $D = \correction{7 × 11}$ & $E = \correction{13 × 2}$ & $F = \correction{45 - 16}$ \\
	$A = \correction{23}$     & $B = \correction{3}$     & $C = \correction{63}$      & $D = \correction{77}$     & $E = \correction{26}$     & $F = \correction{29}$      \\ \hline
\end{tabularx}
\vspace{\tableSpacing}

\makeExercice : Compléter directement, en effectuant le calcul de tête.

\vspace{\tableSpacing}
\begin{tabularx}{\linewidth}{|X|X|X|X|X|X|}
	\hline
	$A = 10 - 13$         & $B = -8 + 1$          & $C = -7 - 3$           & $D = -8 + 2$          & $E = -1 - 1$          & $F = -7 + 11$        \\
	$A = \correction{-3}$ & $B = \correction{-7}$ & $C = \correction{-10}$ & $D = \correction{-6}$ & $E = \correction{-2}$ & $F = \correction{4}$ \\ \hline
\end{tabularx}
\vspace{\tableSpacing}

\makeExercice : Compléter directement, en effectuant le calcul de tête. Déterminer d'abord le signe du résultat !

\vspace{\tableSpacing}
\begin{tabularx}{\linewidth}{|X|X|X|X|X|X|}
	\hline
	$A = -6 × 3$           & $B = -2 × (-10)$      & $C = 12 ÷ (-3)$       & $D = -3 × 13$          & $E = 8 × (-8)$         & $F = -9 × (-9)$       \\
	$A = \correction{-18}$ & $B = \correction{20}$ & $C = \correction{-4}$ & $D = \correction{-39}$ & $E = \correction{-64}$ & $F = \correction{81}$ \\ \hline
\end{tabularx}
\vspace{\tableSpacing}

\makeExercice : Compléter chaque calcul en deux étapes, en respectant les priorités opératoires.

\vspace{\tableSpacing}
\begin{tabularx}{\linewidth}{|l|l|l|l|l|l|}
	\hline
	$A = -6 × 3 + 2$           & $B = 6 - 8 × 2$           & $C = 1 + 3 × (-7)$           & $D = 20 ÷ (-3 - 1)$          & $E = (3 - 5) × 6$         & $F = 13-6×(-2)$               \\
	$A = \correction{-18 + 2}$ & $B = \correction{6 - 16}$ & $C = \correction{1 + (-21)}$ & $D = \correction{20 ÷ (-4)}$ & $E = \correction{-2 × 6}$ & $F = \correction{13 - (-12)}$ \\
	$A = \correction{-16}$     & $B = \correction{-10}$    & $C = \correction{-20}$       & $D = \correction{-5}$        & $E = \correction{-12}$    & $F = \correction{25}$         \\ \hline
\end{tabularx}
\vspace{\tableSpacing}

\section{Fractions}

\makeExercice : Simplifier au maximum les fractions suivantes :

\renewcommand{\arraystretch}{3.5}

\vspace{\tableSpacing}
\begin{tabularx}{\linewidth}{|X|l|X|X|l|X|}
	\hline
	$A = \dfrac{7 × 3}{7 × 8}$                      & $B = \dfrac{6 × 2}{7 × 6}$                      & $C = \dfrac{3 × 3 × 11}{11 × 5 × 3}$            & $D = \dfrac{10 × 4 × 5}{5 × 11 × 3}$              & $E = \dfrac{7}{2} × \dfrac{2}{10}$               & $F = \dfrac{14}{2} × \dfrac{2}{10} × \dfrac{13}{14}$ \\
	$A = \correction{\dfrac{3}{8}}$                 & $B = \correction{\dfrac{2}{7}}$                 & $C = \correction{\dfrac{3}{5}}$                 & $D = \correction{\dfrac{40}{33}}$                 & $E = \correction{\dfrac{7}{10}}$                 & $F = \correction{\dfrac{13}{10}}$                    \\ \hline
	$G = \dfrac{10}{5} = \correction{\dfrac{2}{1}}$ & $H = \dfrac{21}{9} = \correction{\dfrac{7}{3}}$ & $I = \dfrac{49}{7} = \correction{\dfrac{7}{1}}$ & $J = \dfrac{80}{100} = \correction{\dfrac{4}{5}}$ & $K = \dfrac{11}{66} = \correction{\dfrac{1}{6}}$ & $L = \dfrac{42}{36} = \correction{\dfrac{7}{6}}$     \\ \hline
\end{tabularx}
\vspace{\tableSpacing}

\makeExercice : Compléter les calculs puis simplifier les fractions si possible :

\renewcommand{\arraystretch}{2.5}
\vspace{\tableSpacing}
\begin{tabularx}{\linewidth}{|X|X|X|X|X|X|}
	\hline
	$A = \frac{7+2}{7-3} = \correction{\frac{9}{4}}$ & $B = \frac{5+1}{3+5} = \correction{\frac{3}{4}}$ & $C = \frac{3 × 3}{3 + 8} = \correction{\frac{9}{11}}$ & $D = \frac{2-7}{2+1} = \correction{-\frac{5}{3}}$ & $E = \frac{8-8}{4} = \correction{0}$ & $F = \frac{15-2}{9+4} = \correction{1}$ \\ \hline
	$G = \dfrac{8-11}{-1-7}$                         & $H = \dfrac{10 × 17}{2}$                         & $I = \dfrac{4 - 3 × 6}{-1 + 15}$                      & $J = \dfrac{-8 × (-7)}{-2}$                       & $K = \dfrac{-10 + 2}{-10 - 2}$       & $L = \dfrac{-6 × (-3)}{-6-3}$           \\
	$G = \correction{\dfrac{-3}{-8}}$                & $H = \correction{\dfrac{170}{2}}$                & $I = \correction{\dfrac{-14}{14}}$                    & $J = \correction{\dfrac{56}{-2}}$                 & $K = \correction{\dfrac{-8}{-12}}$   & $L = \correction{\dfrac{18}{-9}}$       \\
	$G = \correction{\dfrac{3}{8}}$                  & $H = \correction{85}$                            & $I = \correction{-1}$                                 & $J = \correction{-28}$                            & $K = \correction{\dfrac{2}{3}}$      & $L = \correction{-2}$                   \\ \hline
\end{tabularx}
\vspace{\tableSpacing}

\end{document}