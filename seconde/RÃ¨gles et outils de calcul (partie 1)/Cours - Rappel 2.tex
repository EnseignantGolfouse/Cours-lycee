\documentclass[
	10pt,
	classe=$2^{de}$,
]{coursclass}

\usepackage{préambule}

\title{Cours chapitre 1}
\author{Règles de calcul}
\date{}

\begin{document}

\maketitle

\setcounter{section}{1}
\section{Calculs algébriques}

\begin{vocabulaire}
	\begin{tabularx}{\linewidth}{XXX}
		L'\textbf{opposé} de $x$ est $-x$. & L'\textbf{inverse} de $x$ est $\frac{1}{x}$. & Le \textbf{carré} de $x$ est $x²$.
	\end{tabularx}
\end{vocabulaire}

\begin{greybox}[frametitle={Notation}]
	\newcommand{\localSpacing}{1.8em}
	\begin{tabular}{llll}
		$3 × x = 3x$ \hspace{\localSpacing} & $x × y = xy$ \hspace{\localSpacing} & $6 × (x + 2) = 6(x + 2)$ \hspace{\localSpacing} & $(x - 1) × (x + 7) = (x - 1)(x + 7)$
	\end{tabular}
\end{greybox}

\begin{greybox}[frametitle={Règle des signes}]
	\begin{tabularx}{\linewidth}{XXXX}
		$x × y = xy$ & $- x × y = -xy$ & $x × (-y) = -xy$ & $(-x) × (-y) = xy$
	\end{tabularx}
	\vspace{1em}

	Devant les parenthèses :
	\begin{itemize}
		\item S’il y a un signe « $+$ » devant des parenthèses, supprimer les parenthèses et \textbf{garder} les signes.
		\item S’il y a un signe « $-$ » devant des parenthèses, supprimer les parenthèses et \textbf{changer} les signes.
	\end{itemize}
\end{greybox}

\begin{definition}[]
	\begin{itemize}
		\item \textbf{Développer} un produit signifie le transformer en une somme algébrique.
		\item \textbf{Réduire} une expression développée signifie l’écrire sous la forme d’une somme algébrique contenant le moins de termes possible.
		\item \textbf{Factoriser} une somme algébrique signifie la transformer en produit.
	\end{itemize}
\end{definition}

\begin{propriete}[Distributivité simple]
	Pour tous nombres $a$, $b$ et $k$ :
	\begin{align*}
		k × (a + b) & = ka + kb \hspace{5em}\text{(développement)}     \\
		ka + kb     & = k × (a + b) \hspace{5em}\text{(factorisation)}
	\end{align*}
\end{propriete}

\end{document}