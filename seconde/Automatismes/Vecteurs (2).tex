\documentclass{automatisme}

\usepackage{tikz-repère}

\begin{document}

\begin{frame}
	\begin{center}
		\begin{tikzpicture}[scale=0.6]
			\tikzRepere{-5}{5}{-5}{5}[][]
			\coordinate (A) at (2,1);
			\coordinate (B) at (1,4);
			\coordinate (C) at (0,-3);
			\coordinate (D) at (-4,3);

			\foreach \p in {A,...,D} {
					\node at (\p) {×};
					\node[above right] at (\p) {$\p$};
				}
		\end{tikzpicture}
	\end{center}

	Donner les coordonnées des vecteurs :
	\begin{multicols}{3}
		\begin{itemize}
			\item[] $\vec{AB}\begin{pmatrix}\correction{-1} \\ \correction{3}\end{pmatrix}$
			\item[] $\vec{DB} + \vec{CA}\begin{pmatrix}\correction{7} \\ \correction{5}\end{pmatrix}$
			\item[] $\dfrac{1}{2}\vec{AD} - \vec{BD}\begin{pmatrix}\correction{2} \\ \correction{2}\end{pmatrix}$
		\end{itemize}
	\end{multicols}
\end{frame}

\end{document}