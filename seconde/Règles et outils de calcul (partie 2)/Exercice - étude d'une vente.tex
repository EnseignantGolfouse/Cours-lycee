\documentclass[
	classe=$2^{de}$,
	headerTitle=Exercices,
	landscape,
	twocolumn
]{exercice}

\setlength{\columnsep}{1cm}

\title{Exercice : étude d'une vente}


\begin{document}

\newcommand{\Vente}{
	\maketitle

	Une entreprise vend du grain à des agriculteurs. On cherche à faire une étude de son bénéfice quotidien. \medskip

	\begin{minipage}{0.45\linewidth}
		Si l'entreprise produit $x$ tonnes de grain, cela lui coûte

		$$ C(x) = 50x + 500 € $$
	\end{minipage}
	\hfill\vline\hfill
	\begin{minipage}{0.45\linewidth}
		Si l'entreprise vend $x$ tonnes de grain, cela lui rapporte

		$$ R(x) = 110x - x² € $$
	\end{minipage} \bigskip

	On suppose que tout ce qui est produit est vendu.

	\begin{enumerate}
		\item D'où peut venir le $+\ 500$ dans la formule de $C(x)$ ?

		      \correction{Il peut par exemple venir du salaire des employés, ou de la maintenance d'une machine.}
		\item Donner la formule du bénéfice $B(x)$ :

		      $$ \correction{B(x) = -x² + 60x - 500} $$
		\item On cherche à présent à savoir si l'entreprise réalise bel et bien un bénéfice.

		      Si l'entreprise ne produit et ne vend pas de grain, quel est son bénéfice ? \correctionDots{$-100€$}
		\item Si l'entreprise produit et vend $20$ tonnes de grain, quel est son bénéfice ? \correctionDots{$\phantom{-}300€$}

		      Ainsi l'entreprise commence à faire un bénéfice entre \correctionDots{$0\phantom{2}$} et \correctionDots{$20$} tonnes.
		\item Montrer que $B(x) = -(x - 10)(x - 50)$ :

		      \ifdefined\makeCorrection
			      {\color{red}
				      \begin{align*}
					      -(x - 10)(x - 50) & = -(x² - 10x - 50x + 500) \\
					                        & = -x² + 60x - 500         \\
					                        & = B(x)
				      \end{align*}
			      }
		      \else
			      \vspace{7em}
		      \fi
		\item Résoudre l'équation $x₁ - 10 = 0$ \correctionDots{$x₁ = 10$}

		      Quelle est alors la valeur de $B(x₁)$ ? \correctionDots{$B(x₁) = 0€$}
		\item À partir de combien de tonnes l'entreprise commence-t'elle à faire du bénéfice ? \correction{À partir de $10$ tonnes}
	\end{enumerate}
}

\Vente

\newpage

\Vente

\end{document}