\documentclass[
	classe=$2^{de}$
]{coursclass}

\usepackage{hyperref}

\title{Chapitre 5 : Règles de calcul (partie 2)}
\author{}
\date{}

\begin{document}

\maketitle

\begin{propriete}[Double distributivité]
	Si $a, b, c$ et $d$ sont quatres nombres réels, on a

	$$ (a + b)(c + d) = ac + ad + bc + bd $$

	On dit que l'expression de gauche est \textbf{factorisée}, et que l'expression de droite est \textbf{développée}.
\end{propriete}

\begin{propriete}[Identités remarquables]
	Si $a$ et $b$ sont des nombres réels, on a
	\begin{itemize}
		\item $(a + b)² = a² + 2ab + b²$
		\item $(a - b)² = a² - 2ab + b²$
		\item $(a + b)(a - b) = a² - b²$
	\end{itemize}

	Ces égalités sont à connaître par cœur !
\end{propriete}
\begin{proof}
	On va prouver l'égalité 1, en utilisant la double distributivité :

	Soient $a$ et $b$ deux nombres réels. Alors :
	\begin{align*}
		(a + b)² & = (a + b)(a + b)        \\
		         & = a×a + a×b + b×a + b×b \\
		         & = a² + a×b + a×b + b²   \\
		         & = a² + 2ab + b²
	\end{align*}
\end{proof}

\begin{propriete}
	Si $a$ et $b$ sont des nombres réels, on a :

	$$ (ab)² = a²b² \text{\hspace{1em} et \hspace{1em}} \left(\dfrac{a}{b}\right)² = \dfrac{a²}{b²} $$
\end{propriete}

\begin{propriete}
	Si $a$ et $b$ sont des nombres réels, on a :
	\begin{itemize}
		\item $\sqrt{ab} = \sqrt{a}\sqrt{b} \text{\hspace{1em} et \hspace{1em}} \sqrt{\dfrac{a}{b}} = \dfrac{\sqrt{a}}{\sqrt{b}}$
		\item $\sqrt{a + b} ≤ \sqrt{a} + \sqrt{b}$
	\end{itemize}
\end{propriete}

\begin{greybox}
	Montrer que les deux égalités et l'inégalité ci-dessus sont vérifiées pour $a = 9$ et $b = 16$ :

	\begin{itemize}
		\item $\sqrt{9×16} = \sqrt{144} = 12$, et $\sqrt{9}×\sqrt{16}=3×4=12$
		\item $\sqrt{\dfrac{9}{16}} = \sqrt{0,5625} = 0,75$, et $\dfrac{\sqrt{9}}{\sqrt{16}} = \dfrac{3}{4} = 0,75$
		\item $\sqrt{16 + 9} = \sqrt{25} = 5$, et $\sqrt{16} + \sqrt{9}=4 + 3=7$
	\end{itemize}
\end{greybox}

\begin{propriete}[Produit nul] \label{propriete:Produit nul}
	Si un produit est nul, alors au moins un des facteurs est nul.

	\begin{center}
		Si $A × B = 0$, alors $A = 0$ ou $B = 0$.
	\end{center}
\end{propriete}

\begin{exemple}
	Résoudre l'équation $(x - 3)(2x + 4) = 0$ :

	\begin{itemize}
		\item On sait que $(x - 3)(2x + 4) = 0$. Donc on a $x - 3 = 0$ ou $2x + 4 = 0$.
		\item Si $x - 3 = 0$, alors $x = 3$.
		\item Si $2x + 4 = 0$, alors $x = -2$.
		\item L'ensemble de solutions de cette équation est donc $\{-2 ; 3\}$.
	\end{itemize}
\end{exemple}

\begin{propriete}[Résolution par factorisation]
	Si on a une équation de la forme $ax² + bx + c = 0$, on peut \textit{parfois} la résoudre en \textbf{factorisant}, et en utilisant la propriété du \hyperref[propriete:Produit nul]{produit nul}.
\end{propriete}

\begin{greybox}[frametitle={Techniques de factorisation}]
	\begin{itemize}
		\item Si il n'y a pas de terme constant, on peut factoriser par $x$.
		\item Sinon, l'expression à factoriser est sous la forme $ax² + bx + c$. On s'intéresse alors aux termes $ax²$ et $c$ :
		      \begin{itemize}
			      \item On écrit $ax² = (\sqrt{a}x)$, et $c = \sqrt{c}$.
			      \item Si $b > 0$, on essaie de factoriser en $(\sqrt{a}x + c)²$.
			      \item Si $b < 0$, on essaie de factoriser en $(\sqrt{a}x - c)²$.
			      \item Si $b = 0$, on essaie de factoriser en $(\sqrt{a}x + c)(\sqrt{a}x - c)$.
		      \end{itemize}
	\end{itemize}
\end{greybox}

\begin{exemple}
	\begin{itemize}
		\item On veut factoriser $2x² - 3x$. On remarque qu'il n'y a pas de terme constant, donc
		      $$2x² - 3x = x(2x - 3)$$
		\item On veut factoriser $9x² + 30x + 25$. On pose $9x² = (3x)²$ et $25 = 5²$. De plus $b > 0$, donc
		      $$9x² + 30x + 25 = (3x + 5)²$$
		      On peut vérifier que c'est vrai en développant la partie de droite.
		\item On veut factoriser $100x² - 4$. On pose $100x² = (10x)²$ et $4 = 2²$. De plus $b = 0$, donc
		      $$100x² - 4 = (10x + 2)(10x - 2)$$
		      On peut vérifier que c'est vrai en développant la partie de droite.
	\end{itemize}
\end{exemple}

\end{document}