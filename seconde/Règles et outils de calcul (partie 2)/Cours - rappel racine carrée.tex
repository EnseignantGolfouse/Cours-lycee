\documentclass[noheader]{coursclass}

\begin{document}

\newcommand{\Rappel}{
	\begin{palebox}[frametitle={Rappel : racine carrée}]
		Si $a$ est un nombre positif, on note $\sqrt{a}$ et on appelle \textbf{racine carrée de $a$} le nombre \uline{positif} tel que : $(\sqrt{a})² = a$.

		La racine carrée d’un nombre négatif n’existe donc pas puisqu’un carré est toujours positif.
	\end{palebox}
}

\Rappel\vspace{1em}

\Rappel\vspace{1em}

\Rappel\vspace{1em}

\Rappel\vspace{1em}

\Rappel\vspace{1em}

\Rappel\vspace{1em}

\Rappel\vspace{1em}

\end{document}