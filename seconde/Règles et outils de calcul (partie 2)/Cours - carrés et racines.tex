\documentclass[noheader]{coursclass}

\begin{document}

\newcommand{\Cours}{
	\begin{propriete}
		Si $a$ et $b$ sont des nombres réels, on a :

		$$ (ab)² = a²b² \text{\hspace{1em} et \hspace{1em}} \left(\dfrac{a}{b}\right)² = \dfrac{a²}{b²} $$
	\end{propriete}

	\begin{propriete}
		Si $a$ et $b$ sont des nombres réels, on a :
		\begin{itemize}
			\item $\sqrt{ab} = \sqrt{a}\sqrt{b} \text{\hspace{1em} et \hspace{1em}} \sqrt{\dfrac{a}{b}} = \dfrac{\sqrt{a}}{\sqrt{b}}$
			\item $\sqrt{a + b} ≤ \sqrt{a} + \sqrt{b}$
		\end{itemize}
	\end{propriete}

	\begin{greybox}
		Montrer que les deux égalités et l'inégalité ci-dessus sont vérifiées pour $a = 9$ et $b = 16$ :

		\begin{itemize}
			\item \correction{$\sqrt{9×16} = \sqrt{144} = 12$, et $\sqrt{9}×\sqrt{16}=3×4=12$}\vspace{1em}
			\item \correction{$\sqrt{\dfrac{9}{16}} = \sqrt{0,5625} = 0,75$, et $\dfrac{\sqrt{9}}{\sqrt{16}} = \dfrac{3}{4} = 0,75$}\vspace{1em}
			\item \correction{$\sqrt{16 + 9} = \sqrt{25} = 5$, et $\sqrt{16} + \sqrt{9}=4 + 3=7$}\vspace{1em}
		\end{itemize}
	\end{greybox}
}

\Cours\vfill

\Cours

\end{document}