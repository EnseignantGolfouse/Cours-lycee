\documentclass[
	classe=$2^{de}$,
]{évaluation}

\usepackage{tcolorbox}

\date{3 février 2023}

\newcommand{\makeCorrection}{}
\begin{document}

\title{Évaluation : règles de calcul (2) (Sujet A)}
\maketitle

\begin{tcolorbox}
	La calculatrice est \textbf{interdite}.

	Tous les calculs doivent être détaillés.
\end{tcolorbox}

\begin{exercice}[2]
	Effectuer les calculs suivants, en détaillant :
	\begin{enumerate}
		\item[$A =$] $\sqrt{36} \correction{ = 6}$
		\item[$B =$] $\sqrt{36 × 64} \correction{ = \sqrt{49} × \sqrt{64} = 7 × 8 = 56}$
		\item[$C =$] $\sqrt{\dfrac{100}{81}} \correction{ = \dfrac{\sqrt{100}}{\sqrt{81}} = \dfrac{10}{9}}$
		\item[$D =$] $\sqrt{169 - 144} \correction{ = \sqrt{25} = 5}$
	\end{enumerate}
\end{exercice}

\begin{exercice}[1,5]
	Donner les identités remarquables :
	\begin{enumerate}
		\item[\circled{1}] $(a + b)² = \correction{a² + 2ab + b²}$
		\item[\circled{2}] $(a - b)² = \correction{a² - 2ab + b²}$
		\item[\circled{3}] $(a + b)(a - b) = \correction{a² - b²}$
	\end{enumerate}
\end{exercice}

\begin{exercice}[2,5]
	\begin{enumerate}
		\item Développer l'expression $(3x - 2)(4x + 3)$, en détaillant les calculs.

			      {\color{red}\begin{align*}
					      (3x - 2)(4x + 3) & = 3x × 4x - 2 × 4x + 3 × 3x - 2 × 3 \\
					                       & = 12x² - 8x + 9x - 6                \\
					                       & = 12x² + x - 6                      \\
				      \end{align*}}
		\item En déduire toutes les solutions de l'équation $12x² + x = 6$.

			      {\color{red}Cette équation est équivalente à $12x² + x - 6 = 0 = (3x - 2)(4x + 3)$.

				      Il y a donc deux solutions :
				      \begin{itemize}
					      \item Soit $3x - 2 = 0$, et alors $x = \dfrac{2}{3}$
					      \item Soit $4x + 3 = 0$, et alors $x = -\dfrac{3}{4}$
				      \end{itemize}}
	\end{enumerate}
\end{exercice}

\begin{exercice}[4]
	Résoudre les équations ci-dessous. Si une identité remarquable est utilisée, indiquer laquelle.
	\begin{enumerate}
		\item $x² + 6x + 9 = 0$

		      {\color{red}On utilise l'identité remarquable \circled{1} :

				      $x² + 6x + 9 = (x + 3)² = 0$

				      Donc la seule solution est $x + 3 = 0$, soit $x = -3$.}
		\item $x(x + 6) = 0$

		      {\color{red}Il y a deux solutions à cette équation :
				      \begin{itemize}
					      \item Soit $x = 0$
					      \item Soit $x + 6 = 0$, et donc $x = -6$
				      \end{itemize}}
		\item $x² = 81$

		      {\color{red}Il y a deux solutions à cette équation :
				      \begin{itemize}
					      \item Soit $x = \sqrt{81} = 9$
					      \item Soit $x = -\sqrt{81} = -9$
				      \end{itemize}}
		\item $25x² - 10x = -1$

		      {\color{red}Ceci est équivalent à $25x² - 10x + 1 = 0$

				      On utilise l'identité remarquable \circled{2} :

				      $25x² - 10x + 1 = (5x - 1)² = 0$

				      Donc la seule solution est $5x - 1 = 0$, soit $x = \frac{1}{5}$.}
	\end{enumerate}
\end{exercice}

%===========================================
%================= SUJET B =================
%===========================================
\newpage
\setcounter{exercice}{1}

\title{Évaluation : règles de calcul (2) (Sujet B)}
\maketitle

\begin{tcolorbox}
	La calculatrice est \textbf{interdite}.

	Tous les calculs doivent être détaillés.
\end{tcolorbox}

\begin{exercice}[2]
	Effectuer les calculs suivants, en détaillant :
	\begin{enumerate}
		\item[$A =$] $\sqrt{25} \correction{ = 5}$
		\item[$B =$] $\sqrt{36 × 64} \correction{ = \sqrt{36} × \sqrt{64} = 6 × 8 = 48}$
		\item[$C =$] $\sqrt{\dfrac{81}{100}} \correction{ = \dfrac{\sqrt{81}}{\sqrt{100}} = \dfrac{9}{10}}$
		\item[$D =$] $\sqrt{169 - 144} \correction{ = \sqrt{25} = 5}$
	\end{enumerate}
\end{exercice}

\begin{exercice}[1,5]
	Donner les identités remarquables :
	\begin{enumerate}
		\item[\circled{1}] $(a + b)² = \correction{a² + 2ab + b²}$
		\item[\circled{2}] $(a - b)² = \correction{a² - 2ab + b²}$
		\item[\circled{3}] $(a + b)(a - b) = \correction{a² - b²}$
	\end{enumerate}
\end{exercice}

\begin{exercice}[2,5]
	\begin{enumerate}
		\item Développer l'expression $(5x - 2)(7x + 3)$, en détaillant les calculs.

			      {\color{red}
				      \begin{align*}
					      (5x - 2)(7x + 3) & = 5x×7x - 2×7x + 3×5x - 2×3 \\
					                       & = 35x² - 14x + 15x - 6      \\
					                       & = 35x² + x - 6
				      \end{align*}}
		\item En déduire toutes les solutions de l'équation $35x² + x = 6$.

			      {\color{red}Cette équation est équivalente à $35x² + x - 6 = 0 = (5x - 2)(7x + 3)$.

				      Il y a donc deux solutions :
				      \begin{itemize}
					      \item Soit $5x - 2 = 0$, et alors $x = \dfrac{2}{5}$
					      \item Soit $7x + 3 = 0$, et alors $x = -\dfrac{3}{7}$
				      \end{itemize}}
	\end{enumerate}
\end{exercice}

\begin{exercice}[4]
	Résoudre les équations ci-dessous. Si une identité remarquable est utilisée, indiquer laquelle.
	\begin{enumerate}
		\item $x² + 10x + 25 = 0$

		      {\color{red} On utilise l'identité remarquable \circled{1} :

				      $x² + 10x + 25 = (x + 5)² = 0$

				      Donc la seule solution est $x + 5 = 0$, soit $x = -5$.}
		\item $x(x + 8) = 0$

		      {\color{red}Il y a deux solutions à cette équation :
				      \begin{itemize}
					      \item Soit $x = 0$
					      \item Soit $x + 8 = 0$, et donc $x = -8$
				      \end{itemize}}
		\item $x² = 64$

		      {\color{red}Il y a deux solutions à cette équation :
				      \begin{itemize}
					      \item Soit $x = \sqrt{64} = 8$
					      \item Soit $x = -\sqrt{64} = -8$
				      \end{itemize}}
		\item $9x² - 6x = -1$

		      {\color{red}Ceci est équivalent à $9x² - 6x + 1 = 0$

				      On utilise l'identité remarquable \circled{2} :

				      $9x² - 6x + 1 = (3x - 1)² = 0$

				      Donc la seule solution est $3x - 1 = 0$, soit $x = \frac{1}{3}$.}
	\end{enumerate}
\end{exercice}

\end{document}