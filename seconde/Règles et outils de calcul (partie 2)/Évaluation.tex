\documentclass[
	classe=$2^{de}$,
	landscape,
	twocolumn
]{évaluation}

\usepackage{tcolorbox}

\setlength{\columnsep}{1cm}

\date{3 février 2023}

\begin{document}

\title{Évaluation : règles de calcul (2) (Sujet A)}
\maketitle

\begin{tcolorbox}
	La calculatrice est \textbf{interdite}.

	Tous les calculs doivent être détaillés.
\end{tcolorbox}

\begin{exercice}[2]
	Effectuer les calculs suivants, en détaillant :
	\begin{enumerate}
		\item[$A =$] $\sqrt{36}$
		\item[$B =$] $\sqrt{49 × 64}$
		\item[$C =$] $\sqrt{\dfrac{100}{81}}$
		\item[$D =$] $\sqrt{169 - 144}$
	\end{enumerate}
\end{exercice}

\begin{exercice}[1,5]
	Donner les identités remarquables :
	\begin{enumerate}
		\item[\circled{1}] $(a + b)² = $
		\item[\circled{2}] $(a - b)² = $
		\item[\circled{3}] $(a + b)(a - b) = $
	\end{enumerate}
\end{exercice}

\begin{exercice}[2,5]
	\begin{enumerate}
		\item Développer l'expression $(3x - 2)(4x + 3)$, en détaillant les calculs.
		\item En déduire toutes les solutions de l'équation $12x² + x = 6$.
	\end{enumerate}
\end{exercice}

\begin{exercice}[4]
	Résoudre les équations ci-dessous. Si une identité remarquable est utilisée, indiquer laquelle.
	\begin{enumerate}
		\item $x² + 6x + 9 = 0$
		\item $x(x + 6) = 0$
		\item $x² = 81$
		\item $25x² - 10x = -1$
	\end{enumerate}
\end{exercice}

%===========================================
%================= SUJET B =================
%===========================================
\newpage
\setcounter{exercice}{1}

\title{Évaluation : règles de calcul (2) (Sujet B)}
\maketitle

\begin{tcolorbox}
	La calculatrice est \textbf{interdite}.

	Tous les calculs doivent être détaillés.
\end{tcolorbox}

\begin{exercice}[2]
	Effectuer les calculs suivants, en détaillant :
	\begin{enumerate}
		\item[$A =$] $\sqrt{25}$
		\item[$B =$] $\sqrt{36 × 64}$
		\item[$C =$] $\sqrt{\dfrac{81}{100}}$
		\item[$D =$] $\sqrt{169 - 144}$
	\end{enumerate}
\end{exercice}

\begin{exercice}[1,5]
	Donner les identités remarquables :
	\begin{enumerate}
		\item[\circled{1}] $(a + b)² = $
		\item[\circled{2}] $(a - b)² = $
		\item[\circled{3}] $(a + b)(a - b) = $
	\end{enumerate}
\end{exercice}

\begin{exercice}[2,5]
	\begin{enumerate}
		\item Développer l'expression $(5x - 2)(7x + 3)$, en détaillant les calculs.
		\item En déduire toutes les solutions de l'équation $35x² + x = 6$.
	\end{enumerate}
\end{exercice}

\begin{exercice}[4]
	Résoudre les équations ci-dessous. Si une identité remarquable est utilisée, indiquer laquelle.
	\begin{enumerate}
		\item $x² + 10x + 25 = 0$
		\item $x(x + 8) = 0$
		\item $x² = 64$
		\item $9x² - 6x = -1$
	\end{enumerate}
\end{exercice}

\end{document}