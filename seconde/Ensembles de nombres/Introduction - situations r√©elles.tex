\documentclass{beamer}

\usepackage{préambule}

\begin{document}

\begin{frame}
	Trouver une situation (un minimum réelle) dans laquelle les nombres suivants apparaissent \textbf{et sont indispensables} :
	\begin{itemize}
		\item $29$
		      \pause
		\item $-7$
		      \pause
		\item $5,1$
		      \pause
		\item $-13,2$
		      \pause
		\item $\dfrac{1}{11}$
		      \pause
		\item $π$
		      \pause
		\item $\sqrt{3}$
	\end{itemize}
\end{frame}


\begin{frame}
	Idées possibles :
	\begin{itemize}
		\item $29$ : Il y a 29 personnes.
		\item $-7$ : La température est $7$ degrés en dessous de zéro.
		\item $5,1$ :
		\item $-13,2$ : On peut repérer une position sur une droite.
		\item $\dfrac{1}{11}$ : Faire un partage à parts égales entre $11$ participants.
		\item $π$ : Si on veut déterminer le périmètre d'un cercle (architecture).
		\item $\sqrt{3}$ : Faire des mesures ! Par exemple, si on veut savoir la hauteur d'un bâtiment.
	\end{itemize}
\end{frame}

\end{document}