\documentclass[
	classe=$2^{de}$,
	headerTitle=Activité\space Chapitre\space 2
]{exercice}

\title{Activité : retour sur les puissances}

% \newcommand{\makeCorrection}{}
\begin{document}

\maketitle

\begin{exercice}
	Pour chacune des expressions ci-dessous, l'écrire sous la forme d'un seul nombre mis à une puissance :
	\begin{multicols}{2}
		\begin{enumerate}
			\item $6 × 6 × 6 = \correctionDots{6³}$
			\item $\dfrac{1}{5} × \dfrac{1}{5} × \dfrac{1}{5} × \dfrac{1}{5} = \correctionDots{5^{-4}}$
			\item $\dfrac{-12}{(-12) × (-12) × (-12)} = \correctionDots{(-12)^{-2}}$
			\item $(8 × 8 × 8 × 8) × (8 × 8) = \correctionDots{8⁶}$
			      \columnbreak
			\item $30¹⁰ × 30⁵ = \correctionDots{30¹⁵}$
			\item $7³ × 7^{-20} = \correctionDots{7^{-17}}$
			\item $(-1)¹⁹ × (-1)^{-5} = \correctionDots{(-1)¹⁴}$
			\item $\dfrac{9⁸}{9³} = \correctionDots{9⁵}$
		\end{enumerate}
	\end{multicols}
\end{exercice}

\vspace{1em}
\begin{exercice}\

	\begin{enumerate}
		\item Écrire les nombres suivants sous forme d'entier ou de nombre décimal :
		      \begin{multicols}{3}
			      \begin{enumerate}
				      \item $2 × 10^{3} = \correctionDots{2000}$
				      \item $7,9 × 10^{5} = \correctionDots{790000}$
				      \item $1,2 × 10^{-2} = \correctionDots{0,012}$
			      \end{enumerate}
		      \end{multicols}
		      L'écriture \squareFrame{$x × 10ⁿ$} est appellée \correctionDots{la notation scientifique}. \vspace{1em}
		\item Rechercher le nombre d'habitant de la métropole de Lyon : \correctionDots{$1\ 411\ 571$}

		      Écrire ce nombre sous forme scientifique (on mettra \textbf{deux} chiffres après la virgule) : \correctionDots{$1,41 × 10⁶$}
		\item Rechercher la probabilité de gagner au loto : \correctionDots{$\dfrac{1}{19\ 068\ 840}$}

		      Écrire ce nombre sous forme scientifique (on mettra \textbf{trois} chiffres après la virgule) : \correctionDots{$5,244 × 10^{-7 }$}
		\item Rechercher la vitesse de la lumière, en mètres par seconde : \correctionDots{$299\ 792\ 458$ m/s}

		      Écrire ce nombre sous forme scientifique (on mettra \textbf{quatre} chiffres après la virgule) : \correctionDots{$2,9979 × 10⁸$}
	\end{enumerate}
\end{exercice}

\vspace{1em}
\begin{exercice}\
	On imagine une variante du loto, dans laquelle les boules tirées sont remises en jeu à chaque tirage. Dans ce nouveau jeu, il faut pour gagner que tous les numéros choisis soient tirés dans le bon ordre.

	\begin{enumerate}
		\item Sachant qu'il y a $49$ numéros au loto, quelle est la probabilité de tirer le bon numéro à chacun des \vspace{0em}

		      tirages ?\correctionDots{$\dfrac{1}{49}$}
		\item Au total, écrire la probabilité d'obtenir les cinq bons numéros sous forme de puissance : \correctionDots{$49^{-5}$}
		\item Écrire cette probabilité sous forme scientifique (on mettra deux chiffres après la virgule) :

		      \correctionDots{$3,54 × 10^{-9}$}
	\end{enumerate}
\end{exercice}

\vspace{1em}
\begin{exercice}\

	Selon la légende, le jeu d'échecs a été inventé en Inde par un sage nommé Sissa. Pour le remercier, le roi lui demanda ce qu'il voulait.

	Sissa demanda alors que le roi prenne le plateau d'échecs, et le remplisse de grains de riz de la manière suivante : qu'il pose un grain sur la deuxième case, puis deux sur la deuxième, puis quatre sur la troisième, et ainsi de suite en doublant à chaque case.

	\uline{Question } : le roi aura-t'il de quoi satisfaire la demande de Sissa ?
\end{exercice}

\end{document}