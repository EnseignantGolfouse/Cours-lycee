\documentclass{beamer}

\usepackage{préambule}

\begin{document}

\begin{frame}
	Compléter avec le signe $∈$ ou $∉$ :

	\begin{multicols}{3}
		\begin{enumerate}
			\pause
			\item $2 \correctionDots{∈} \{1, 2, 3\}$
			      \pause
			\item $6 \correctionDots{∉} \{3, 4, 5, 7, 8\}$
			      \pause
			\item $-2 \correctionDots{∉} ℕ$
			      \pause
			\item $\dfrac{2}{3} \correctionDots{∈} ℝ$
			      \pause
			\item $5 \correctionDots{∈} [5 ; 8[$
			      \pause
			\item $8 \correctionDots{∉} [5 ; 8[$
		\end{enumerate}
	\end{multicols}
	\pause
	{\small 5.} Soit $(OI)$ la droite des abscisses dans un repère. A-t-on :
	\pause
	\begin{itemize}
		\item $(5 ; 0) ∈ (OI)$ ?
		\item $(0 ; 5) ∈ (OI)$ ?
	\end{itemize}
\end{frame}

\end{document}