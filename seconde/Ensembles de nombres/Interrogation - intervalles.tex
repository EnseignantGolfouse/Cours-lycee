\documentclass{beamer}

\usepackage{préambule}

\newcommand\scalemath[2]{\scalebox{#1}{\mbox{\ensuremath{\displaystyle #2}}}}

\setbeamersize{
	text margin left=0.5cm,
	text margin right=0.5cm
}

\begin{document}
\small

\begin{frame}
	\begin{multicols}{2}
		\uline{Sujet A}

		\begin{enumerate}
			\item À quels intervalles correspondent les inéquations suivantes :
			      \begin{multicols}{3}
				      {\small\color{blue}a.} $\scalemath{0.85}{x < 2}$

				      {\small\color{blue}b.} $\scalemath{0.85}{x > -6}$

				      {\small\color{blue}c.} $\scalemath{0.85}{x ≥ 5,3}$
			      \end{multicols}
			\item Donner l'intervalle constitué des solutions de l'inéquation {\small\color{blue}a.} ET de l'inéquation {\small\color{blue}b.}
			\item Donner l'intervalle constitué des solutions de l'inéquation {\small\color{blue}b.} OU de l'inéquation {\small\color{blue}c.}
			\item Représenter les intervalles $\big]{-}∞\ ;\ -1\big]$, $\big[2\ ;\ 3,5\big[$ et $\big]6\ ;\ +∞\big[$ sur une droite graduée.
		\end{enumerate}

		\setlength{\columnseprule}{0.7pt}
		\columnbreak
		\setlength{\columnseprule}{0pt}
		\uline{Sujet B}

		\begin{enumerate}
			\item À quels intervalles correspondent les inéquations suivantes :
			      \begin{multicols}{3}
				      {\small\color{blue}a.} $\scalemath{0.85}{x < 12}$

				      {\small\color{blue}b.} $\scalemath{0.85}{x > 7}$

				      {\small\color{blue}c.} $\scalemath{0.85}{x ≥ 2,8}$
			      \end{multicols}
			\item Donner l'intervalle constitué des solutions de l'inéquation {\small\color{blue}a.} ET de l'inéquation {\small\color{blue}b.}
			\item Donner l'intervalle constitué des solutions de l'inéquation {\small\color{blue}b.} OU de l'inéquation {\small\color{blue}c.}
			\item Représenter les intervalles $\big]{-}∞\ ;\ -3\big]$, $\big[{-}1\ ;\ 0,5\big[$ et $\big]3\ ;\ +∞\big[$ sur une droite graduée.
		\end{enumerate}
	\end{multicols}
\end{frame}

\end{document}