\documentclass{beamer}

\usepackage{préambule}

\begin{document}

\begin{frame}

	Reproduire le tableau suivant :

	\vspace{1em}

	\renewcommand{\arraystretch}{1.3}
	\newcommand{\myPadding}{ \hspace{1.5em} }
	\begin{tabular}{|c|c|c|c|c|}
		\hline
		\myPadding$ℕ$\myPadding & \myPadding$ℤ$\myPadding & \myPadding$𝔻$\myPadding & \myPadding$ℚ$\myPadding & \myPadding$ℝ$\myPadding \\ \hline
		                        &                         &                         &                         &                         \\
		                        &                         &                         &                         &                         \\
		                        &                         &                         &                         &                         \\
		                        &                         &                         &                         &                         \\
		                        &                         &                         &                         &                         \\ \hline
	\end{tabular}

	\vspace{1em}

	Puis, placer les nombres suivants dans le plus petit ensemble auquel ils appartiennent :

	$2$ ; $7$ ; $-2$ ; $-2,1$ ; $\dfrac{5}{3}$ ; $-\dfrac{10}{2}$ ; $\sqrt{5}$ ; $\sqrt{4}$ ; $8,9547627$
\end{frame}

\end{document}