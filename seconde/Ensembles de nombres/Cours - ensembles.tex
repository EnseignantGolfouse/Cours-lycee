\documentclass[
	classe=$2^{de}$
]{coursclass}

\title{Chapitre 2 : Ensembles de nombres}
\date{}
\author{}

\begin{document}

\section{Vocabulaire des ensembles}

\begin{definition}[Ensemble, éléments]
	Un \textbf{ensemble} contient des \textbf{éléments}.

	Si $e$ est un élément dans $E$, on note \squareFrame{$e ∈ E$}.

	Si un élément $e$ n'est \uline{pas} dans $E$, on note \squareFrame{$e ∉ E$}.
\end{definition}

\begin{exemple}
	\begin{itemize}
		\item $1 ∈ \{1, 2, 3\}$, $2 ∈ \{1, 2, 3\}$, et $3 ∈ \{1, 2, 3\}$. En revanche, $4 ∉ \{1, 2, 3\}$.
	\end{itemize}
\end{exemple}

\begin{definition}[intersection, union]
	Soient $A$ et $B$ deux ensembles. On note
	\begin{itemize}
		\item \squareFrame{$A ∩ B$} l'\textbf{intersection} de $A$ et de $B$, l'ensemble dont les éléments sont dans $A$ \uline{et} dans $B$.

		      On prononce « $A$ \textbf{inter} $B$ ».
		\item \squareFrame{$A ∪ B$} l'\textbf{union} de $A$ et de $B$, l'ensemble dont les éléments sont dans $A$ \uline{ou} dans $B$.

		      On prononce « $A$ \textbf{union} $B$ ».
	\end{itemize}
\end{definition}

\begin{exemple}
	\begin{itemize}
		\item $\{1, 2, 3\} ∩ \{2, 3, 4\} = \{2, 3\}$
		\item $\{1, 2, 3\} ∪ \{2, 3, 4\} = \{1, 2, 3, 4\}$
		\item $[-1 ; +∞[ ∩ ]-∞ ; 1] = [-1 ; 1]$
	\end{itemize}
\end{exemple}

\begin{definition}[sous-ensemble]
	Si tous les éléments de $B$ sont dans $A$, on dit que $B$ est un \textbf{sous-ensemble} de $A$, et on note \squareFrame{$ B ⊂ A $}

	Sinon, on note \squareFrame{$B \not⊂ A$}.
\end{definition}

\begin{exemple}
	\begin{itemize}
		\item $\{1, 2\} ⊂ \{1, 2, 3\}$
		\item $\{1, 2, 4\} \not⊂ \{1, 2, 3\}$, car $4 ∉ \{1, 2, 3\}$.
		\item $ℕ ⊂ ℤ$
	\end{itemize}
\end{exemple}

\end{document}