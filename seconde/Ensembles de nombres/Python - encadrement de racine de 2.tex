\documentclass[
	classe=$2^{de}$,
]
{informatique}

\title{Encadrement de $\sqrt{2}$}

\begin{document}

\maketitle

\begin{enumerate}
	\item Montrer que $\sqrt{2}$ est une solution de l'équation $x^2 - 2 = 0$ :

	      \correction{si $x = \sqrt{2}$, alors $x^2 - 2 = \sqrt{2}^2 - 2 = 2 - 2 = 0$.}
	\item On veut pouvoir, en python, calculer rapidement le résultat de $x^2 - 2$ pour n'importe quel $x$. On veut donc une \uline{fonction}, qui prend $x$ en entrée et renvoie $x^2 - 2$.

	      Recopier la fonction suivante dans l'éditeur, et la compléter :

	      \begin{lstlisting}
def f(x):
    return ... # compléter ici
	      \end{lstlisting}
	\item on va faire afficher 100 valeurs de \texttt{f} pour $x$ entre $1$ et $2$.

	      \begin{itemize}
		      \item Quel est alors l'écart entre chaque valeur de $x$ ? \correctionDots{$0.01$}

		            On appelle cet écart le \uline{pas}.
		      \item Recopier le code suivant dans l'éditeur, et le compléter :
		            \begin{lstlisting}
a = 1
b = 2
N = 100
pas = ...
for i in range(...):
	x = a + ... * pas
	print("x = ", x, "f(x) = ", f(x))
			        \end{lstlisting}
		      \item Donner alors une approximation de $\sqrt{2}$ : \correctionDots{$1,4$}.
	      \end{itemize}
	\item On veut maintenant généraliser la méthode ci-dessus : on utilise donc une \textbf{fonction}
	      \begin{lstlisting}
def affiche_valeurs(a, b, N):
    pas = ...
    for i in range(N):
        x = a + i * pas
        print("x = ", x, "f(x) = ", f(x))
	      \end{lstlisting}

	      Comment calcule-t'on alors le pas correct pour que \texttt{x} aille de \texttt{a} à \texttt{b} ? \correction{\texttt{pas = (b - a) / N}}
	\item Modifier la fonction ci-dessus afin qu'elle s'arrête dès que \texttt{f(x)} dépasse 0.

	      L'utiliser alors pour déterminer une valeur de $\sqrt{2}$ au millième près : \correctionDots{$\sqrt{2} ≈ 1,414$}
	\item On veut maintenant visualiser la fonction \texttt{f} : Recopier le code suivant dans l'éditeur :
	      \begin{lstlisting}
import matplotlib.pyplot as plt

# representation affiche N + 1 valeur de f(x) entre a et b. 
def representation(a, b, N):
    pas = ...
    for i in range(N+1):
        x = a + i * pas
        plt.plot(x, f(x), 'b.')


representation(0, 2, 100)   
plt.plot([0,2], [0,0], 'r--')
plt.show()
          \end{lstlisting}
	      \begin{itemize}
		      \item Si on veut que \texttt{x} prenne \texttt{N + 1} valeurs entre \texttt{a} et \texttt{b}, quelle doit être la valeur du pas ?

		            \correctionDots{$\texttt{pas} = \frac{b - a}{N}$}.
		      \item Compléter alors le programme ci-dessus.
		      \item Lancer le programme, et lire une approximation de $\sqrt{2}$ sur l'image obtenue.

		            Quelle est la précision de l'approximation ? \correctionDots{$1/100$}
	      \end{itemize}
\end{enumerate}

\end{document}