% !TEX root = ./première-activité.tex
\documentclass[première-activité-correction.tex]{subfiles}

% APMEP n°185, JEUX 8

\newcounter{QCMQuestionCounter}
\setcounter{QCMQuestionCounter}{1}
\newcounter{QCMReponseCounter}

\newcommand{\QCMQuestion}[1]{
	& & & & & & & \\
	\multicolumn{8}{l}{\arabic{QCMQuestionCounter}. #1}\stepcounter{QCMQuestionCounter} \\
}
\newcommand{\QCMReponse}[8]{
	\setcounter{QCMReponseCounter}{1}         \alph{QCMReponseCounter}) #1 & [#2] & \stepcounter{QCMReponseCounter}           \alph{QCMReponseCounter}) #3 & [#4] & \stepcounter{QCMReponseCounter}           \alph{QCMReponseCounter}) #5 & [#6] & \stepcounter{QCMReponseCounter}           \alph{QCMReponseCounter}) #7 & [#8] \\
}

\title{{\huge Q.C.M. et dessin}\\ Racines carrées}
\author{}
\date{}

\begin{document}

\maketitle

\begin{minipage}{0.5\linewidth}
	Pour chaque question de ce QCM, il y a une ou plusieurs bonnes réponses.

	Si tu penses que la réponse de la première question est «a», trace, dans le cadre, le segment [ag], et ainsi de suite. \\

	Ce dessin est constitué de quatre lettres, qui désignent le mot : ...................
\end{minipage}
\hspace{0.08\linewidth}
\begin{minipage}{0.4\linewidth}
	\newcommand{\placePoint}[2]{
		#1 * \linewidth / 4.8 - \linewidth / 8, \linewidth - #2 * \linewidth / 4.8 + \linewidth / 8
	}

	\begin{tikzpicture}
		\draw[color=blue,line width=0.18cm,rounded corners=5pt] (0,0) -- ++(\linewidth, 0) -- ++(0, \linewidth) -- ++(-\linewidth, 0) -- cycle;

		\foreach \x/\y/\l in {
				1/1/a, 2/1/b, 3/1/c, 4/1/d, 5/1/e,
				1/2/f, 2/2/g, 3/2/h, 4/2/i, 5/2/j,
				1/3/k, 2/3/l, 3/3/m, 4/3/n, 5/3/o,
				1/4/p, 2/4/q, 3/4/r, 4/4/s, 5/4/t,
				1/5/u, 2/5/v, 3/5/w, 4/5/y, 5/5/z} {
				\coordinate (\l) at (\placePoint{\x}{\y});
				\node at (\l) {⋅};
				\node[above left] at (\l) {{\scriptsize \l}};
			}
		\ifdefined\modeCorrection
			\foreach \a/\b in {a/g,l/v,r/t,d/m,g/c,o/z,d/o,h/r,r/w,k/l,l/m,h/j,c/h,a/k} {
					\draw[red] (\a) -- (\b);
				}
		\fi
	\end{tikzpicture}
\end{minipage}

\begin{tabular}{llllllll}
	\QCMQuestion{La racine carrée de 100 est}
	\QCMReponse{10}{ag}{10 ou -10}{ci}{50}{gk}{10 000}{lr}
	\QCMQuestion{Parmi les quatre égalités, laquelle (lesquelles) sont fausse(s) ?}
	\QCMReponse{$(\sqrt{17²}) = 17$}{lp}{$(\sqrt{13})² = 13$}{bg}{$\sqrt{9,25} = 3,5$}{lv}{$\sqrt{(-3)²} = -3$}{rt}
	\QCMQuestion{Avec $x=\sqrt{5}$, l'expression $(x-1)(x+5)$ est égale à :}
	\QCMReponse{$4\sqrt{5} - 3$}{de}{$2\sqrt{5} + 4$}{di}{$4\sqrt{5}$}{dm}{$6\sqrt{5} + 10$}{ej}
	\QCMQuestion{$(2 - \sqrt{3})² =$}
	\QCMReponse{$7+4\sqrt{3}$}{in}{$7-2\sqrt{3}$}{cg}{$7-4\sqrt{3}$}{gc}{$1$}{cn}
	\QCMQuestion{$\sqrt{3}×\sqrt{6} =$}
	\QCMReponse{18}{uw}{$\sqrt{18}$}{oz}{$3\sqrt{2}$}{do}{$9\sqrt{2}$}{pq}
	\QCMQuestion{$\sqrt{3}/\sqrt{15} =$}
	\QCMReponse{$\sqrt{3/15}$}{hr}{$\sqrt{1/5}$}{rw}{$1/\sqrt{5}$}{kl}{$\sqrt{5}/5$}{lm}
	\QCMQuestion{$\sqrt{5}+\sqrt{20} =$}
	\QCMReponse{}{hj}{}{gl}{}{jo}{}{ac}
	\QCMQuestion{$\sqrt{3² + 4²} =$}
	\QCMReponse{}{lu}{}{nx}{}{wz}{}{ch}
	\QCMQuestion{Lequel de ces quatre nombres n'est pas égal au trois autres ?}
	\QCMReponse{}{nw}{}{nz}{}{mz}{}{ak}
\end{tabular}

\end{document}