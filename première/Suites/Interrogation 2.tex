\documentclass[
	classe=$1^{ere}STI2D$,
	twocolumn,
	landscape,
]{évaluation}

\setlength{\columnsep}{1cm}
\renewcommand{\correctionDots}[1]{
	\correctionOr{{\color{red}#1}}{........................................}
}
\newcommand{\correctionDotsSmall}[1]{
	\correctionOr{{\color{red}#1}}{........}
}

\date{24 mai 2023}

\begin{document}

\title{Interrogation : suites (sujet A)}
\maketitle

\begin{exercice}
	\begin{enumerate}
		\item Soit $u$ la suite définie par $uₙ = 3n - 2$ pour $n ≥ 0$.
		      \begin{enumerate}
			      \item $u$ est-elle définie explicitement ou par récurrence ? \bigskip

			            \correctionDots{explicitement}
			      \item Donner la valeur de $u_5$ : \correctionDotsSmall{$13$}
		      \end{enumerate}
		\item Soit $w$ la suite définie par $w₀ = 3$ et $w_{n+1} = 2w_n$ pour $n ≥ 0$.
		      \begin{enumerate}
			      \item $w$ est-elle définie explicitement ou par récurrence ? \bigskip

			            \correctionDots{par récurrence}
			      \item Donner la valeur de $w_3$ : \correctionDotsSmall{$24$}
		      \end{enumerate}
	\end{enumerate}
\end{exercice}

\begin{exercice}
	Soit $u$ la suite définie par $u_0 = 0$ et $u_{n+1} = 3uₙ + 1$.
	\begin{enumerate}
		\item Calculer :

		      \begin{align*}
			      u_1 & = \correctionDotsSmall{1} & u_2 & = \correctionDotsSmall{4} & u_3 & = \correctionDotsSmall{13} & u_4 & = \correctionDotsSmall{40}
		      \end{align*}
		\item On définit la suite $w$ telle que $w_n = 2u_n + 1$.

		      Calculer :

		      \begin{align*}
			      w_0 & = \correctionDotsSmall{1} & w_1 & = \correctionDotsSmall{3} & w_2 & = \correctionDotsSmall{9} & w_3 & = \correctionDotsSmall{27}
		      \end{align*}
		\item Quelle semble être la nature de la suite $w$ ? \bigskip

		      \correctionDots{Elle semble géométrique de raison $3$}
	\end{enumerate}
\end{exercice}


\newpage\setcounter{exercice}{1}
\title{Interrogation : suites (sujet B)}
\maketitle

\begin{exercice}
	\begin{enumerate}
		\item Soit $u$ la suite définie par $uₙ = 4n + 3$ pour $n ≥ 0$.
		      \begin{enumerate}
			      \item $u$ est-elle définie explicitement ou par récurrence ? \bigskip

			            \correctionDots{explicitement}
			      \item Donner la valeur de $u_5$ : \correctionDotsSmall{$23$}
		      \end{enumerate}
		\item Soit $w$ la suite définie par $w₀ = 5$ et $w_{n+1} = 2w_n$ pour $n ≥ 0$.
		      \begin{enumerate}
			      \item $w$ est-elle définie explicitement ou par récurrence ? \bigskip

			            \correctionDots{par récurrence}
			      \item Donner la valeur de $w_3$ : \correctionDotsSmall{$40$}
		      \end{enumerate}
	\end{enumerate}
\end{exercice}

\begin{exercice}
	Soit $u$ la suite définie par $u_0 = 0$ et $u_{n+1} = 2uₙ + 3$.
	\begin{enumerate}
		\item Calculer :

		      \begin{align*}
			      u_1 & = \correctionDotsSmall{3} & u_2 & = \correctionDotsSmall{9} & u_3 & = \correctionDotsSmall{21} & u_4 & = \correctionDotsSmall{45}
		      \end{align*}
		\item On définit la suite $w$ telle que $w_n = u_n + 3$.

		      Calculer :

		      \begin{align*}
			      w_0 & = \correctionDotsSmall{3} & w_1 & = \correctionDotsSmall{6} & w_2 & = \correctionDotsSmall{12} & w_3 & = \correctionDotsSmall{24}
		      \end{align*}
		\item Quelle semble être la nature de la suite $w$ ? \bigskip

		      \correctionDots{Elle semble géométrique de raison $2$}
	\end{enumerate}
\end{exercice}

\end{document}