\documentclass[
	classe=$1^{ere}STI2D$
]{coursclass}

\title{Chapitre 6 : Suites numériques}
\author{}
\date{}

\begin{document}

\maketitle

\begin{definition}[Suite numérique]
	Une \textbf{suite numérique} est une liste ordonnée et numérotée de nombres. On la numérote généralement à partir de $0$ ou de $1$.

	Les éléments de cette liste sont appelés des \textbf{termes}.

	Le numéro de chaque élément est appelé son \textbf{indice}.
\end{definition}

\begin{remarque}
	Le $n$-ième terme de la liste $u$ peut être noté $uₙ$ ou $u(n)$.
\end{remarque}

\begin{definition}[Définition fonctionnelle]
	Une suite est définie de manière \textbf{fonctionnelle} ou \textbf{explicite} si le terme d'indice $n$ peut être calculé sans connaître les termes précédents.
\end{definition}

\begin{exemple}
	La suite $u₀ = 3$, $u₁ = 7$, $u₂ = 11$, $u₃ = 15$, ..... Peut être définie par la formule $uₙ = 3 + 4 × n$ : c'est une définition fonctionnelle.
\end{exemple}

\begin{definition}[Définition par récurrence]
	Une suite $u$ est définie \textbf{par récurrence} si on dispose :
	\begin{itemize}
		\item du terme initial $u₀$ (ou $u₁$)
		\item d'une manière de calculer $u_{n+1}$ à partir de $uₙ$
	\end{itemize}
\end{definition}

\begin{exemple}
	On peut définir la suite $v$ par récurrence :
	\begin{itemize}
		\item $v₁ = 4$
		\item $v_{n+1} = vₙ + n$
	\end{itemize}
	On a alors :
	\begin{itemize}
		\item $v₁ = 4$
		\item $v₂ = v₁ + 1 = 4 + 1 = 5$
		\item $v₃ = v₂ + 2 = 5 + 2 = 7$
		\item $v₄ = v₃ + 3 = 7 + 3 = 10$
		\item .....
	\end{itemize}
\end{exemple}

\begin{definition}[Suite arithmétique]
	Une suite est \textbf{arithmétique} si on passe au terme suivant en \uline{ajoutant} toujours la même valeur.

	Cette valeur est appelée la \textbf{raison} de la suite.
\end{definition}

\begin{definition}[Suite géométrique]
	Une suite est \textbf{géométrique} si on passe au terme suivant en \uline{multipliant} toujours par la même valeur.

	Cette valeur est appelée la \textbf{raison} de la suite.
\end{definition}

\begin{exemple}
	\begin{itemize}
		\item Soit $u$ une suite arithmétique de raison $3$, telle que $u₀ = 2$.

		      On a alors
		      \begin{itemize}
			      \item $u₀ = 2$
			      \item $u₁ = u₀ + 3 = 2 + 3 = 5$
			      \item $u₂ = u₁ + 3 = 5 + 3 = 8$
			      \item $u₃ = u₂ + 3 = 8 + 3 = 11$
		      \end{itemize}
		\item Soit $v$ une suite géométrique de raison $\frac{1}{2}$, telle que $v₀	= 1$.

		      On a alors
		      \begin{itemize}
			      \setlength{\itemsep}{0.6em}
			      \item $v₀ = 1$
			      \item $v₁ = v₀ × \frac{1}{2} = 1 × \frac{1}{2} = \frac{1}{2}$
			      \item $v₂ = v₁ × \frac{1}{2} = \frac{1}{2} × \frac{1}{2} = \frac{1}{4}$
			      \item $v₃ = v₂ × \frac{1}{2} = \frac{1}{4} × \frac{1}{2} = \frac{1}{8}$
		      \end{itemize}
	\end{itemize}
\end{exemple}

\begin{propriete}[Représentation graphique]
	\begin{itemize}
		\item Si une suite est représentée par un nuage de points alignés, elle est arithmétique.
		\item Si une suite est représentée par un nuage de points exponentiel, elle est géométrique.
	\end{itemize}
\end{propriete}

\end{document}