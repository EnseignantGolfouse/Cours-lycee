\documentclass[
	classe=$1^{ere}STI2D$
]{coursclass}

\title{Chapitre 6 : Suites numériques}
\author{}
\date{}

\begin{document}

\maketitle

\begin{definition}[Suite numérique]
	Une \textbf{suite numérique} est une liste ordonnée et numérotée de nombres. On la numérote généralement à partir de $0$ ou de $1$.

	Les éléments de cette liste sont appelés des \textbf{termes}.

	Le numéro de chaque élément est appelé son \textbf{indice}.
\end{definition}

\begin{remarque}
	Le $n$-ième terme de la liste $u$ peut être noté $uₙ$ ou $u(n)$.
\end{remarque}

\begin{definition}[Définition fonctionnelle]
	Une suite est définie de manière \textbf{fonctionnelle} ou \textbf{explicite} si le terme d'indice $n$ peut être calculé sans connaître les termes précédents.
\end{definition}

\begin{exemple}
	La suite $u₀ = 3$, $u₁ = 7$, $u₂ = 11$, $u₃ = 15$, ..... Peut être définie par la formule $uₙ = 3 + 4 × n$ : c'est une définition fonctionnelle.
\end{exemple}

\begin{definition}[Définition par récurrence]
	Une suite $u$ est définie \textbf{par récurrence} si on dispose :
	\begin{itemize}
		\item du terme initial $u₀$ (ou $u₁$)
		\item d'une manière de calculer $u_{n+1}$ à partir de $uₙ$
	\end{itemize}
\end{definition}

\begin{exemple}
	On peut définir la suite $v$ par récurrence :
	\begin{itemize}
		\item $v₁ = 4$
		\item $v_{n+1} = vₙ + n$
	\end{itemize}
	On a alors :
	\begin{itemize}
		\item $v₁ = 4$
		\item $v₂ = v₁ + 1 = 4 + 1 = 5$
		\item $v₃ = v₂ + 2 = 5 + 2 = 7$
		\item $v₄ = v₃ + 3 = 7 + 3 = 10$
		\item .....
	\end{itemize}
\end{exemple}

\end{document}