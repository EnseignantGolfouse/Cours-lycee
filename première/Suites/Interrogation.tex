\documentclass[
	classe=$1^{ere}STI2D$,
	landscape,
	twocolumn
]{évaluation}

\usepackage{tikz-repère}
\usetikzlibrary{calc}

\begin{luacode}
function suite_geometrique(init, raison, n)
	v = init
	while n ~= 0 do
		v = v * raison
		n = n - 1
	end
	return v
end
\end{luacode}

\setlength{\columnsep}{1cm}
\newdimen\mydim
\newcommand\gety[1]{
    \pgfextracty\mydim{\pgfpointanchor{#1}{center}}
}

\date{24 mars 2023}

\begin{document}

\title{Interrogation : suites (sujet A)}
\maketitle

\begin{exercice}
	Soit $u$ la suite telle que $u₀ = 0$, et $u_{n+1} = 2uₙ - 3$.

	\begin{enumerate}
		\item La suite est-elle définie explicitement ou par récurrence ?
		\item Donner les termes d'indice $1$ à $4$ de cette suite.
	\end{enumerate}
\end{exercice}

\begin{exercice}
	Soit $v$ une suite géométrique de raison $1{,}2$\ , et de premier terme $v₀ = 2$.

	\begin{enumerate}
		\item Calculer les termes d'indice $1$ à $4$ de cette suite.
		\item Représenter la suite $v$ dans le repère ci-dessous :
		      \begin{center}
			      \begin{tikzpicture}[scale=0.9]
				      \tikzRepere{0}{4}{0}{5}[]
				      \foreach \i in {0,...,4} {
						      \node[below] at (\i,-0.2) {$\i$};
						      \ifdefined\makeCorrection
							      \coordinate (V) at (\i,\directlua{tex.print(suite_geometrique(2, 1.2,\i))});
							      \node[red] at (V) {×};
							      \node[red,below left] at (V) {$v_{\i}$};
						      \fi
					      }
			      \end{tikzpicture}
		      \end{center}
	\end{enumerate}
\end{exercice}

\begin{exercice}
	Soit $w$ une suite arithmétique de raison $3$ et de premier terme $w₀ = -1$.
	\begin{enumerate}
		\item Donner l'expression de $w_{n+1}$ en fonction de $wₙ$.
		\item À l'aide de la calculatrice, donner le terme d'indice $12$ de $w$.
	\end{enumerate}
\end{exercice}

%=================================================
%=================================================
%=================================================
\newpage
\setcounter{exercice}{1}

\title{Interrogation : suites (sujet B)}
\maketitle

\begin{exercice}
	Soit $u$ la suite telle que $u₀ = 0$, et $u_{n+1} = 3uₙ - 2$.

	\begin{enumerate}
		\item La suite est-elle définie explicitement ou par récurrence ?
		\item Donner les termes d'indice $1$ à $4$ de cette suite.
	\end{enumerate}
\end{exercice}

\begin{exercice}
	Soit $v$ une suite géométrique de raison $1{,}5$\ , et de premier terme $v₀ = 1$.

	\begin{enumerate}
		\item Calculer les termes d'indice $1$ à $4$ de cette suite.
		\item Représenter la suite $v$ dans le repère ci-dessous :
		      \begin{center}
			      \begin{tikzpicture}[scale=0.9]
				      \tikzRepere{0}{4}{0}{5}[]
				      \foreach \i in {0,...,4} {
						      \node[below] at (\i,-0.2) {$\i$};
						      \ifdefined\makeCorrection
							      \coordinate (V) at (\i,\directlua{tex.print(suite_geometrique(1, 1.5, \i))});
							      \node[red] at (V) {×};
							      \node[red,below left] at (V) {$v_{\i}$};
						      \fi
					      }
			      \end{tikzpicture}
		      \end{center}
	\end{enumerate}
\end{exercice}

\begin{exercice}
	Soit $w$ une suite arithmétique de raison $4$ et de premier terme $w₀ = -2$.
	\begin{enumerate}
		\item Donner l'expression de $w_{n+1}$ en fonction de $wₙ$.
		\item À l'aide de la calculatrice, donner le terme d'indice $12$ de $w$.
	\end{enumerate}
\end{exercice}

\end{document}