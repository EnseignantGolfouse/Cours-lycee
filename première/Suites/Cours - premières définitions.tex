\documentclass[noheader]{coursclass}

\title{Chapitre 6 : Suites numériques}
\author{}
\date{}

\begin{document}

\newcommand{\Definition}{
	\begin{definition}[Suite numérique]
		Une \textbf{suite numérique} est une liste ordonnée et numérotée de nombres. On la numérote généralement à partir de $0$ ou de $1$.

		Les éléments de cette liste sont appelés des \textbf{termes}.

		Le numéro de chaque élément est appelé son \textbf{indice}.
	\end{definition}

	\begin{remarque}
		Le $n$-ième terme de la liste $u$ peut être noté $uₙ$ ou $u(n)$.
	\end{remarque}

	\begin{definition}[Définition fonctionnelle]
		Une suite est définie de manière \textbf{fonctionnelle} ou \textbf{explicite} si le terme d'indice $n$ peut être calculé sans connaître les termes précédents.
	\end{definition}

	\begin{definition}[Définition par récurrence]
		Une suite $u$ est définie \textbf{par récurrence} si on dispose :
		\begin{itemize}
			\item du terme initial $u₀$ (ou $u₁$)
			\item d'une manière de calculer $u_{n+1}$ à partir de $uₙ$
		\end{itemize}
	\end{definition}
}

\Definition

\vfill

\Definition

\end{document}