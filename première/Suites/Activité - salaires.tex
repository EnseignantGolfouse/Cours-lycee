\documentclass[
	classe=$1^{ere}STI2D$
]{exercice}

\begin{luacode}
function u(n)
	local x = 2100
	while (n ~= 1) do
		x = x + 60
		n = n - 1
	end
	return math.floor(x * 100) / 100
end

function v(n)
	local x = 1800
	while (n ~= 1) do
		x = x * 1.025
		n = n - 1
	end
	return math.floor(x * 100) / 100
end

function w(n)
	local x = 2300
	while (n ~= 1) do
		x = x + 20 + 0.01 * x
		n = n - 1
	end
	return math.floor(x * 100) / 100
end
\end{luacode}

\newcommand{\uList}[1]{
	\directlua{tex.print(u(#1))}
}
\newcommand{\vList}[1]{
	\directlua{tex.print(v(#1))}
}
\newcommand{\wList}[1]{
	\directlua{tex.print(w(#1))}
}

\title{Activité : calculs de salaires}

\begin{document}

\maketitle

En sortant d'une école de commerce, on reçoit trois offres d'emploi différentes :
\begin{itemize}
	\item L'entreprise 1 nous propose un salaire mensuel de $2100$€ la première année, avec une augmentation de $60$€ par an.
	\item L'entreprise 2 nous propose un salaire mensuel de $1800$€ la première année, avec une augmentation de $2,5$\% par an.
	\item L'entreprise 3 nous propose un salaire mensuel de $2300$€ la première année. Tous les ans, le salaire mensuel augmente de $20$€, plus une hausse de $1$\% par rapport au salaire mensuel de l'année précédente.
\end{itemize}

On note :
\begin{itemize}
	\item $uₙ$ le salaire au bout de la $n$-ième année dans l'entreprise 1. Ainsi $u₁ = 2100$€.
	\item $vₙ$ le salaire au bout de la $n$-ième année dans l'entreprise 2. Ainsi $v₁ = 1800$€.
	\item $wₙ$ le salaire au bout de la $n$-ième année dans l'entreprise 3. Ainsi $w₁ = 2300$€.
\end{itemize}

\begin{enumerate}
	\item Entreprise $1$ :
	      \begin{enumerate}
		      \item Calculer $u₂$ et $u₃$.
		      \item La suite $u$ est-elle définie explicitement ou par récurrence ?

		            Exprimer $u_{n+1}$ en fonction de $uₙ$.
	      \end{enumerate}
	\item Entreprise $2$ :
	      \begin{enumerate}
		      \item Calculer $v₂$.
		      \item La suite $v$ est-elle définie explicitement ou par récurrence ?

		            Donner une définition de la suite $v$.
	      \end{enumerate}
	\item Entreprise $3$ :
	      \begin{enumerate}
		      \item Calculer $w₃$.
		      \item La suite $w$ est-elle définie explicitement ou par récurrence ?

		            Donner une définition de la suite $w$.
	      \end{enumerate}
	\item On va maintenant comparer ces trois suites :

	      Dans la calculatrice (modèle NumWorks), aller dans la catégorie «Suites», et l'utiliser pour remplir le tableau suivant :

	      \begin{center}
		      \renewcommand{\arraystretch}{1.3}
		      \begin{tabular}{|l|c|c|c|c|c|}
			      \hline
			      Indice & $1$                    & $10$                    & $20$                    & $30$                    & $40$                    \\ \hline
			      $u$    & \correction{\uList{1}} & \correction{\uList{10}} & \correction{\uList{20}} & \correction{\uList{30}} & \correction{\uList{40}} \\ \hline
			      $v$    & \correction{\vList{1}} & \correction{\vList{10}} & \correction{\vList{20}} & \correction{\vList{30}} & \correction{\vList{40}} \\ \hline
			      $w$    & \correction{\wList{1}} & \correction{\wList{10}} & \correction{\wList{20}} & \correction{\wList{30}} & \correction{\wList{40}} \\ \hline
		      \end{tabular}
	      \end{center}
	\item Quelle semble être la meilleure offre si on reste $10$ ans dans l'entreprise ?

	      Et si on reste $20$ ans ?

	      Et si on reste $40$ ans ?
	\item Quelle offre nous fait arriver le plus vite à un salaire de $3000$€ par mois ?
\end{enumerate}

\end{document}