\documentclass[
	classe=$1^{ere}STI2D$
]{exercice}

\title{Activité : calculs de salaires}

\begin{document}

\maketitle

En sortant d'une école de commerce, on reçoit trois offres d'emploi différentes :
\begin{itemize}
	\item L'entreprise 1 nous propose un salaire mensuel de $2100$€ la première année, avec une augmentation de $60$€ par an.
	\item L'entreprise 2 nous propose un salaire mensuel de $1800$€ la première année, avec une augmentation de $2,5$\% par an.
	\item L'entreprise 3 nous propose un salaire mensuel de $2300$€ la première année. Tous les ans, le salaire mensuel augmente de $20$€, plus une hausse de $1$\% par rapport au salaire mensuel de l'année précédente.
\end{itemize}

On note :
\begin{itemize}
	\item $uₙ$ le salaire au bout de la $n$-ième année dans l'entreprise 1. Ainsi $u₁ = 2100$€.
	\item $vₙ$ le salaire au bout de la $n$-ième année dans l'entreprise 2. Ainsi $v₁ = 1800$€.
	\item $wₙ$ le salaire au bout de la $n$-ième année dans l'entreprise 3. Ainsi $w₁ = 2300$€.
\end{itemize}

\end{document}