\documentclass[
	classe=$1^{ere}STI2D$
]{exercice}

\usepackage{tikz-repère}
\usepackage{xfp}

\begin{luacode}
-- Returns a function, that can be called with a parameter n
function SUITE(init, next)
	local f = function(n)
		local x = init
		while n ~= 0 do
			x = next(x)
			n = n - 1
		end
		return x
	end
	return f
end

function PRINT_NUMBER(x, precision)
	local n = math.floor(x)
	tex.print(n)
	if n ~= x and precision ~= 0 then
		local frac = x - n
		tex.print(",")
		local puissance = 1
		while precision ~= 0 do
			puissance = puissance * 10
			precision = precision - 1
		end
		tex.print(math.floor(frac*puissance))
	end
end

function SALAIRE_A(n)
	return SUITE(21150, function(x) return x + 721 end)(n - 1)
end

function SALAIRE_B(n)
	return SUITE(20000, function(x) return x * 1.05 end)(n - 1)
end

Y_SCALE_D = 500
function SALAIRE_D(n)
	return SUITE(20400, function(x) return x + 600 end)(n - 1)
end

Y_SCALE_E = 1500
function SALAIRE_E(n)
	return SUITE(48000, function(x) return x * 0.95 end)(n - 1)
end
\end{luacode}

\renewcommand{\arraystretch}{1.4}
\newcommand{\SalaireA}[1]{
	\directlua{PRINT_NUMBER(SALAIRE_A(#1), 2)}
}
\newcommand{\SalaireB}[1]{
	\directlua{PRINT_NUMBER(SALAIRE_B(#1), 2)}
}
\newcommand{\SalaireD}[1]{
	\directlua{PRINT_NUMBER(SALAIRE_D(#1), 0)}
}
\newcommand{\SalaireE}[1]{
	\directlua{PRINT_NUMBER(SALAIRE_E(#1), 0)}
}
\newcommand{\SalaireDscaled}[1]{
	\directlua{tex.print(SALAIRE_D(#1)/Y_SCALE_D)}
}
\newcommand{\SalaireEscaled}[1]{
	\directlua{tex.print(SALAIRE_E(#1)/Y_SCALE_E)}
}

\title{Activité : calculs de salaires}

\begin{document}

\maketitle

Anne et Brahim sont employés chacun dans une entreprise différente. On a répertorié dans le tableau suivant les salaires annuels de leurs 4 premières années :

\begin{center}
	\begin{tabular}{|l|*{4}{>{\centering}p{1.7cm}|}}
		\hline
		Année  & $1$             & $2$             & $3$             & $4$ \tabularnewline \hline
		Anne   & $\SalaireA{1}$€ & $\SalaireA{2}$€ & $\SalaireA{3}$€ & $\SalaireA{4}$€ \tabularnewline \hline
		Brahim & $\SalaireB{1}$€ & $\SalaireB{2}$€ & $\SalaireB{3}$€ & $\SalaireB{4}$€ \tabularnewline \hline
	\end{tabular}
\end{center}

Pour chacun d'entre eux :
\begin{enumerate}
	\item Donner la différence et le quotient des salaires de deux années successives :

	      \begin{minipage}{0.47\textwidth}
		      \newcommand{\SalaireDiffA}[1]{
			      \directlua{PRINT_NUMBER(SALAIRE_A(#1 + 1) - SALAIRE_A(#1), 2)}
		      }
		      \newcommand{\SalaireDiffB}[1]{
			      \directlua{PRINT_NUMBER(SALAIRE_B(#1 + 1) - SALAIRE_B(#1), 2)}
		      }
		      \begin{tabular}{|l|*{3}{>{\centering}p{1.7cm}|}}
			      \hline
			             & Année $2$ - Année $1$            & Année $3$ - Année $2$            & Année $4$ - Année $3$            \tabularnewline \hline
			      Anne   & \correction{$\SalaireDiffA{1}$€} & \correction{$\SalaireDiffA{2}$€} & \correction{$\SalaireDiffA{3}$€} \tabularnewline \hline
			      Brahim & \correction{$\SalaireDiffB{1}$€} & \correction{$\SalaireDiffB{2}$€} & \correction{$\SalaireDiffB{3}$€} \tabularnewline \hline
		      \end{tabular}
	      \end{minipage}
	      \begin{minipage}{0.47\textwidth}
		      \newcommand{\SalaireQuotientA}[1]{
			      \directlua{PRINT_NUMBER(SALAIRE_A(#1 + 1) / SALAIRE_A(#1), 4)}
		      }
		      \newcommand{\SalaireQuotientB}[1]{
			      \directlua{PRINT_NUMBER(SALAIRE_B(#1 + 1) / SALAIRE_B(#1), 4)}
		      }
		      \renewcommand{\arraystretch}{1.6}
		      \begin{tabular}{|l|*{3}{>{\centering}p{1.8cm}|}}
			      \hline
			             & $\dfrac{\text{Année} 2}{\text{Année 1}}$ & $\dfrac{\text{Année} 3}{\text{Année 2}}$ & $\dfrac{\text{Année} 4}{\text{Année 3}}$            \tabularnewline \hline
			      Anne   & \correction{$\SalaireQuotientA{1}$}      & \correction{$\SalaireQuotientA{2}$}      & \correction{$\SalaireQuotientA{3}$} \tabularnewline \hline
			      Brahim & \correction{$\SalaireQuotientB{1}$}      & \correction{$\SalaireQuotientB{2}$}      & \correction{$\SalaireQuotientB{3}$} \tabularnewline \hline
		      \end{tabular}
		      \renewcommand{\arraystretch}{1.4}
	      \end{minipage} \medskip
	\item Que remarque-t'on ? \correction{La première ligne du premier tableau/deuxième ligne du second tableau est constante}
	\item À quel type de suite chacun de ces salaires semble-t'ils correspondre ? \correction{La première semble être une suite arithmétique, et la deuxième une suite géométrique.}
	\item En déduire leurs salaires lors de leur cinquième et sixième année.

	      \correctionOr{{\color{red}Le salaire d'Anne va devenir $\SalaireA{5}$€ puis $\SalaireA{6}$€.

				      Celui de Brahim va devenir $\SalaireB{5}$€ puis $\SalaireB{6}$€}}{}
	\item Dans le repère ci-dessous, on a représenté les salaires annuels de Dali et Elise.

	      \begin{minipage}{0.47\textwidth}
		      \pgfmathsetmacro\YNombre{10}
		      \pgfmathsetmacro\MaYScale{500}
		      \pgfmathsetmacro\YUn{20000}

		      \pgfmathsetmacro\YUnScaled{\fpeval{\YUn / \MaYScale}}
		      \begin{tikzpicture}[scale=0.8]
			      \tikzRepere{0}{7}{0}{\YNombre}[][]
			      \foreach \y in {0,...,\YNombre} {
					      \node[left] at (-0.2,\y) {$\inteval{\y*\MaYScale + \YUn}$€};
				      }
			      \foreach \x in {1,...,8} {
					      \node[below] at (\x-1,-0.2) {$\x$};
					      \node[thick,blue] at (\x-1,\SalaireDscaled{\x}-\YUnScaled) {×};
				      }
			      \node[below] at (3,-1) {Salaire de Dali};
		      \end{tikzpicture}
	      \end{minipage}
	      \begin{minipage}{0.47\textwidth}
		      \pgfmathsetmacro\YNombre{13}
		      \pgfmathsetmacro\MaYScale{1500}
		      \pgfmathsetmacro\YUn{30000}

		      \pgfmathsetmacro\YUnScaled{\fpeval{\YUn / \MaYScale}}
		      \begin{tikzpicture}[scale=0.8]
			      \tikzRepere{0}{7}{0}{\YNombre}[][]
			      \foreach \y in {0,...,\YNombre} {
					      \node[left] at (-0.2,\y) {$\fpeval{\y*\MaYScale + \YUn}$€};
				      }
			      \foreach \x in {1,...,8} {
					      \node[below] at (\x-1,-0.2) {$\x$};
					      \node[thick,blue] at (\x-1,\SalaireEscaled{\x}-\YUnScaled) {×};
				      }
			      \node[below] at (3,-1) {Salaire d'Elise};
		      \end{tikzpicture}
	      \end{minipage}

	      À quel type de suite semblent correspondre chacun de ces salaires ? \correction{La première ressemble à une suite arithmétique, la deuxième à une suite géométrique.}

	      Donner alors leurs salaires pour les années $9$ et $10$.

		  \correctionOr{{\color{red}Le salaire de Dali va devenir environ $\SalaireD{9}$€ puis $\SalaireD{10}$€.

		  Celui d'Elise va devenir environ $\SalaireE{9}$€ puis $\SalaireE{10}$€}}{}
\end{enumerate}


\end{document}