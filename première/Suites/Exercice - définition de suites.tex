\documentclass{automatisme}

\begin{document}

\begin{frame}
	\uline{\Large{Exercice 1 page 26}} \bigskip

	Soit $u$ la suite définie par $uₙ = n² + 3n + 5$, pour $n > 0$.
	\begin{enumerate}
		\item La suite $u$ est-elle définie explicitement ou par récurrence ?
		\item Calculer les cinq premiers termes de la suite $u$.
		\item Représenter graphiquement ces $5$ termes.
	\end{enumerate} \vfill

	\uline{\Large{Exercice 2 page 26}} \bigskip

	Soit $v$ la suite définie par $v₀ = 2$, et $v_{n+1} = 2vₙ + 3$.
	\begin{enumerate}
		\item La suite $v$ est-elle définie explicitement ou par récurrence ?
		\item Calculer les quatre premiers termes de la suite $v$.
		\item Représenter graphiquement ces $4$ termes.
	\end{enumerate}
\end{frame}

\begin{frame}
	{\Large\uline{Exercice : définition de suites}}

	\begin{enumerate}
		\item Pour chaque suite ci-dessous, dire si elle est définie explicitement ou par récurrence :
		      \begin{itemize}
			      \setlength{\itemsep}{0.8em}
			      \item $uₙ = 2n + 2$, $n > 0$
			      \item $v₀ = 0$, $v_{n+1} = 2vₙ + 2$
			      \item $w₁ = 4$, $w_{n+1} = wₙ + 2$
		      \end{itemize}
		\item Donner le terme d'indice $3$ de chaque suite.
		\item \ [OPTIONEL] Montrer que la suite $u$ et la suite $w$ sont les mêmes.
	\end{enumerate}
\end{frame}

\end{document}