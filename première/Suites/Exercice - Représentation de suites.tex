\documentclass[
	classe=$1^{ere}STI2D$,
	landscape,
	twocolumn
]{exercice}

\usepackage{tikz-repère}
\usepackage{tcolorbox}

\setlength{\columnsep}{1cm}

\title{Exercice : représentation de suites}

\begin{document}

\maketitle

\begin{exercice}
	On se donne deux suites arithmétiques :
	\begin{itemize}
		\item La suite $u$ de raison $1,5$, telle que $u₀ = -1$.
		\item La suite $v$ de raison $0,8$, telle que $v₀ = 2,5$.
	\end{itemize}

	Représenter les 7 premières valeurs de ces suites dans le repère ci-dessous :
	\begin{center}
		\begin{tikzpicture}
			\tikzRepere{0}{6}{-1}{10}[1][1]
			\ifdefined\makeCorrection
				\foreach \n in {0,...,6} {
						\node[red] at (\n,1.5*\n-1) {×};
						\node[red,above right] at (\n,1.5*\n-1) {$u_{\n}$};
						\node[red] at (\n,0.8*\n+2.5) {×};
						\node[red,above left] at (\n,0.8*\n+2.5) {$v_{\n}$};
					}
			\fi
			\foreach \n in {2,...,6} {
					\draw[thick] (\n,0) -- ++(0,-0.2) node[below] {$\n$};
				}
			\foreach \y in {2,...,10} {
					\draw[thick] (0,\y) -- ++(-0.2,0) node[left] {$\y$};
				}
		\end{tikzpicture}
	\end{center}

	Que peut-on dire des points correspondant à chaque suite ?
\end{exercice}

\newpage

\begin{tcolorbox}
	Cette représentation graphique est appelée un \textbf{nuage de points}.
\end{tcolorbox}

\begin{exercice}
	On se donne deux suites géométriques :
	\begin{itemize}
		\item La suite $u$ de raison $1,6$, telle que $u₀ = 0,5$.
		\item La suite $v$ de raison $0,7$, telle que $v₀ = 8$.
	\end{itemize}

	Représenter les 7 premières valeurs de ces suites dans le repère ci-dessous :
	\begin{center}
		\begin{tikzpicture}
			\tikzRepere{0}{6}{-1}{10}[1][1]
			\ifdefined\makeCorrection
				\foreach \n in {0,...,6} {
						\node[red] at (\n,0.5*1.6^\n) {×};
						\node[red,above left] at (\n,0.5*1.6^\n) {$u_{\n}$};
						\node[red] at (\n,8*0.7^\n) {×};
						\node[red,above left] at (\n,8*0.7^\n) {$v_{\n}$};
					}
			\fi
			\foreach \n in {2,...,6} {
					\draw[thick] (\n,0) -- ++(0,-0.2) node[below] {$\n$};
				}
			\foreach \y in {2,...,10} {
					\draw[thick] (0,\y) -- ++(-0.2,0) node[left] {$\y$};
				}
		\end{tikzpicture}
	\end{center}

	Que peut-on dire des points correspondant à chaque suite ?
\end{exercice}

\end{document}