\documentclass[
	classe=$1^{ere}STI2D$,
]{exercice}

\title{Reconnaître des suites arithmétiques et géométriques}

\begin{document}

\maketitle

\begin{exercice}
	\begin{enumerate}
		\item On donne les deux premiers termes d'une suite arithmétique $u$ ci dessous :
		      $$ u₀ = 11\ ;\ u₁ = 25\ ;\ u₂ = \correctionDots{39} $$

		      Quelle est le prochain terme de la suite ?

		      Quelle est alors la raison de cette suite ? $r = \correctionDots{14}$
		\item On donne les deux premiers termes d'une suite géométrique $v$ ci dessous :
		      $$ u₀ = 3\ ;\ u₁ = 18\ ;\ u₂ = \correctionDots{108} $$

		      Quelle est le prochain terme de la suite ?

		      Quelle est alors la raison de cette suite ? $r = \correctionDots{6\ }$
		\item On en déduit les propriétés ci-dessous :

		      \begin{itemize}
			      \item Pour trouver la raison d'une suite arithmétique, on \correctionDots{fait la différence entre} deux termes successifs.
			      \item Pour trouver la raison d'une suite géométrique, on \correctionDots{fait le rapport entre} deux termes successifs.
		      \end{itemize}
	\end{enumerate}
\end{exercice}

\begin{exercice}\

	\begin{enumerate}
		\item Dans une suite arithmétique, la \correctionOr{différence}{.........................} entre deux termes est constante.
		\item Dans une suite géométrique, le \correctionOr{rapport}{.........................} entre deux termes est constant.
	\end{enumerate}
\end{exercice}

\begin{exercice}

	Pour chacune des suites ci-dessous :
	\begin{itemize}
		\item Donner l'expression de $u_{n+1} - uₙ$ et $\dfrac{u_{n+1}}{uₙ}$ en fonction de $n$, en la simplifiant au maximum.
		\item Dire alors si la suite est arithmétique, géométrique (ainsi que sa raison), ou ni l'un ni l'autre.
	\end{itemize}

	\begin{multicols}{2}
		\begin{enumerate}[a.]
			\item $uₙ = 2n$
			\item $uₙ = 3ⁿ$
			\item $uₙ = 6n² + 3n$
			\item $uₙ = 5n + 6$
		\end{enumerate}
	\end{multicols}

	\ifdefined\makeCorrection
		\begin{enumerate}[a.]
			\color{red}
			\item $u_{n+1} - uₙ = 2(n + 1) - 2n = 2(n + 1 - n) = 2$.

			      $\dfrac{u_{n+1}}{uₙ} = \dfrac{2(n + 1)}{2n} = \dfrac{n + 1}{n}$

			      Donc $u$ est arithmétique de raison $2$.
			\item $u_{n+1} - uₙ = 3^{n+1} - 3ⁿ = 3 × 3ⁿ - 3ⁿ = 2 × 3ⁿ$.

			      $\dfrac{u_{n+1}}{uₙ} = \dfrac{3^{n+1}}{3ⁿ} = \dfrac{3 × 3ⁿ}{3ⁿ} = 3$

			      Donc $u$ est géométrique de raison $3$.
			\item $u_{n+1} - uₙ = 6(n+1)² + 3(n + 1) - (6n² + 3n) = 6(n² + 2n + 1) + 3n + 3 - 6n² - 3n = 12n + 9$.

			      $\dfrac{u_{n+1}}{uₙ} = \dfrac{6(n+1)² + 3(n+1)}{6n² + 3n} = \dfrac{6n² + 15n + 9}{6n² + 3n}$

			      Donc $u$ n'est ni arithmétique, ni géométrique.
			\item $u_{n+1} - uₙ = 5(n + 1) + 6 - (5n + 6) = 5n + 5 + 6 - 5n - 6 = 5$.

			      $\dfrac{u_{n+1}}{uₙ} = \dfrac{5(n + 1) + 6}{5n + 6} = \dfrac{5n + 11}{5n + 6}$

			      Donc $u$ est arithmétique de raison $5$.
		\end{enumerate}
	\fi
\end{exercice}

\end{document}