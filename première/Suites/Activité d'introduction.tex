\documentclass[
	classe=$1^{ere} STI2D$,
	headerTitle=Activité,
	landscape,
	twocolumn
]{exercice}

\setlength{\columnsep}{1cm}

\title{Activité : introduction à la notion de suite}

\begin{document}

\newcommand{\Activite}{
\maketitle

\begin{enumerate}
	\item Compléter les listes de nombres suivantes de manière logique, en calculant deux valeurs supplémentaires :

	      \begin{enumerate}
		      \item Liste $u$ : 3 ; 7 ; 11 ; 15 ; 19 ; \correctionOr{23}{.........}\ ; \correctionOr{27}{.........}\
		      \item Liste $v$ : 2 ; 6 ; 18 ; 54 ; 162 ; \correctionOr{486}{.........}\ ; \correctionOr{1458}{.........}\
		      \item Liste $w$ : 1 ; 1 ; 2 ; 3 ; 5 ; 8 ; 13 ; \correctionOr{21}{.........}\ ; \correctionOr{34}{.........}\
	      \end{enumerate}
	\item On décide de numéroter les éléments de chaque liste, à partir de $0$ ou de $1$. \medskip

	      Par exemple, si on numérote $u$ à partir de $0$, on a 
		  
		  $u₀ = 3$ ; $u₁ = 7$ ; $u₂ = 11$ ; .... \medskip

	      Alors que si on numérote $v$ à partir de $1$, on a 
		  
		  $v₁ = 2$ ; $v₂ = 6$ ; $v₃ = 18$ ; .... \bigskip

	      Si on numérote la liste $w$ à partir de $0$, quelle est la valeur de $w₄$ ?

	\item Les éléments $\{3 ; 7 ; 11 ; 15 ; 19 ; ... \}$ de la liste $u$ s'appellent les \textbf{termes}.

	      Leur numéro s'appelle l'\textbf{indice}. Par exemple $u₂ = 11$, donc le terme d'indice $2$ de la liste $u$ est $11$.

	      \begin{enumerate}
		      \item Déterminer le terme d'indice $2$ de la liste $w$.
		      \item Est-ce le $2ᵉ$ ou le $3ᵉ$ terme ?
	      \end{enumerate}
\end{enumerate}
}

\Activite

\newpage
\Activite

\end{document}