\documentclass[
	classe=$1^{ere}STI2D$
]{informatique}

\usepackage{tcolorbox}
\usepackage{setspace}

\renewcommand{\arraystretch}{1.4}

\title{Activité : Étude d'une colonie de bactéries}

\begin{document}

\maketitle

Un laboratoire pharmaceutique met en culture une colonie de bactéries \textit{E.coli} comptant $5$ milliers d'individus à midi. Le technicien de laboratoire estime qu'à chaque minute le nombre de bactéries augmente de $5$\%.

\begin{tcolorbox}
	\textbf{À quelle heure la colonie dépassera les $10\ 000$ individus ?} \bigskip

	On modélise l'évolution de la population de bactéries par une suite géométrique $u$ de premier terme $u₀ = 5$ et de raison $1,05$. Pour tout entier naturel $n$, $uₙ$ désigne le nombre (en milliers) de bactéries en culture $n$ minutes après midi.

	Pour répondre à la question posée, on peut donc calculer les termes successifs de la suite jusqu'à ce que l'un d'entre eux soit supérieur à $10$. On poursuit donc le calcul de $uₙ$ tant que $uₙ < 10$.
\end{tcolorbox}

\begin{enumerate}
	\item Remplir le tableau ci-dessous (en arrondissant au millième) :
	      \begin{center}
		      \begin{tabular}{|l|*{8}{>{\centering}p{1cm}|}}
			      \hline
			      Étape $n$      & 0   & 1      & 2                    & 3                    & 4                    & 5                    & 6                    & 7             \tabularnewline \hline
			      Valeur de $uₙ$ & $5$ & $5,25$ & \correction{$5,512$} & \correction{$5,788$} & \correction{$6,078$} & \correction{$6,381$} & \correction{$6,700$} & \correction{$7,036$} \tabularnewline \hline
		      \end{tabular}
	      \end{center}
	\item La démarche de la question $1$ peut être traduite par un algorithme.

	      Compléter l'algorithme ci-dessous, écrit en pseudo-language, qui donne les instructions permettant de résoudre le problème donné :

	      {\setstretch{1.3}\begin{center}
		      \ifdefined\makeCorrection
			      \begin{lstlisting}
u ← 5
n ← 0
Tant que u < 10000 faire :
	u ← u × 1,05
	n ← n + 1
Afficher n
\end{lstlisting}
		      \else
			      \begin{lstlisting}
u ← ......
n ← ......
Tant que ...... faire :
    u ← ......
	n ← ......
Afficher n
\end{lstlisting}
		      \fi
	      \end{center}}
	\item On implémente maintenant cet algorithme en Python.

	      Recopier et compléter le code proposé ci-dessous dans l'éditeur de Spyder :

	      \begin{center}
		      \ifdefined\makeCorrection
			      \begin{lstlisting}
u = 5
n = 0
while u < 10000:
	u = u * 1.05
	n = n + 1
print(n)
\end{lstlisting}
		      \else
			      \begin{lstlisting}
u = ...
n = ...
while ... :
    u = ...
    n = ...
print(...)
\end{lstlisting}
		      \fi
	      \end{center}
		  Répondre alors à la question de l'encadré. \correctionOr{{\color{red}La colonie dépassera les $10\ 000$ individus à $12$h$15$.}}{....................................................}
	\item Une autre membre du laboratoire fait une autre estimation : chaque minute, le nombre de bactérie augmente de $5$\%, avant de diminuer de $100$.

	      Modifier le code pour incorporer cette nouvelle hypothèse. Quelle est à présent le résultat renvoyé par l'algorithme ? \correctionDots{37}
	\item Écrire et exécuter le code suivant :
	      \begin{center}
		      \begin{lstlisting}
u = 5
l = [5]
for i in range(50):
    u = u * 1.05 - 200
	l.append(u)
print(l)
\end{lstlisting}
	      \end{center}

	      Ici \texttt{l} est une \textbf{liste} : on la définit en délimitant avec des crochets une série de nombres séparés par des virgules.

	      Que contient ici la liste \texttt{l} ? \correctionDots{Elle contient les termes successifs de la suite $u$.}
\end{enumerate}

\end{document}