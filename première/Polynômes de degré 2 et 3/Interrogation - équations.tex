\documentclass[
	classe=$1^{ere}STI2D$,
	headerTitle=Interrogation
]{évaluation}

\usepackage{tcolorbox}

\title{Interrogation : expressions littérale de degré 2 et équations de degré 2 et 3 (Sujet A)}
\date{13 janvier 2023}

\begin{document}

\newcommand{\spacing}{4em}

\maketitle

\begin{tcolorbox}
	La calculatrice n'est \uline{pas} autorisée. Les calculs doivent être détaillés.
\end{tcolorbox}

\begin{exercice}
	Développer les expressions suivantes :
	\begin{multicols}{2}
		\begin{itemize}
			\item $A = (3x + 2)²$\vspace{\spacing}

			      \correction{$A = (3x)² + 2×2×3x + 2²$}

			      \correction{$A = 9x² + 12x + 4$}
			\item $C = (2x - 5)(-x + 6)$\vspace{\spacing}

			      \correction{$C = 2x×(-x) + (-5)×(-x) + 2x×6 + (-5)×6$}

			      \correction{$C = -2x² + 17x - 30$}
			\item $B = (-5x + 2)(-5x - 2)$\vspace{\spacing}

			      \correction{$B = (-5x)² - 2²$}

			      \correction{$B = 25x² - 4$}
			\item $D = (x - 8)²$\vspace{\spacing}

			      \correction{$D = x² - 2×8×x + 8²$}

			      \correction{$D = x² - 16x + 64$}
		\end{itemize}
	\end{multicols}
\end{exercice}

\begin{exercice}
	Factoriser les expressions suivantes :
	\begin{multicols}{2}
		\begin{itemize}
			\item $A = 6x² + x$\vspace{\spacing}

			      \correction{$A = x(6x + 1)$}
			\item $C = x² - 25$\vspace{\spacing}

			      \correction{$C = (x + 5)(x - 5)$}
			\item $B = 3x² - 9x$\vspace{\spacing}

			      \correction{$B = 3x(x - 3)$}
			\item $D = 4x² + 12x + 9$\vspace{\spacing}

			      \correction{$D = (2x + 3)²$}
		\end{itemize}
	\end{multicols}
\end{exercice}

\begin{exercice}
	Donner toutes les solutions des équations suivantes :
	\begin{multicols}{2}
		\begin{itemize}
			\item $x² = 36$\vspace{\spacing}

			      \correction{$x = 6$ OU $x = -6$}
			\item $x³ = -125$\vspace{\spacing}

			      \correction{$x = -5$}
			\item $(x - 3)(x + 2) = 0$\vspace{\spacing}

			      \correction{$x = 3$ OU $x = -2$}
			\item $x² + 7x = 0$\vspace{\spacing}

			      \correction{$x = 0$ OU $x = -7$}
		\end{itemize}
	\end{multicols}
\end{exercice}

%===============================================
%================= SUJET B =====================
%===============================================
\newpage
\setcounter{exercice}{1}

\title{Interrogation : expressions littérale de degré 2 et équations de degré 2 et 3 (Sujet B)}

\maketitle

\begin{tcolorbox}
	La calculatrice n'est \uline{pas} autorisée. Les calculs doivent être détaillés.
\end{tcolorbox}

\begin{exercice}
	Développer les expressions suivantes :
	\begin{multicols}{2}
		\begin{itemize}
			\item $A = (2x + 3)²$\vspace{\spacing}

			      \correction{$= (2x)² + 2×3×2x + 3²$}

			      \correction{$= 4x² + 12x + 9$}
			\item $C = (2x - 7)(-x + 5)$\vspace{\spacing}

			      \correction{$= 2x×(-x) + (-7)×(-x) + 2x×5 + (-7)×5$}

			      \correction{$= -2x² + 17x - 35$}
			\item $B = (-6x + 3)(-6x - 3)$\vspace{\spacing}

			      \correction{$= (-6x)² - 3²$}

			      \correction{$= 36x² - 9$}
			\item $D = (x - 7)²$\vspace{\spacing}

			      \correction{$= x² - 2×7×x + 7²$}

			      \correction{$= x² - 14x + 49$}
		\end{itemize}
	\end{multicols}
\end{exercice}

\begin{exercice}
	Factoriser les expressions suivantes :
	\begin{multicols}{2}
		\begin{itemize}
			\item $A = 5x² + x$\vspace{\spacing}

			      \correction{$A = x(5x + 1)$}
			\item $C = x² - 36$\vspace{\spacing}

			      \correction{$C = (x + 6)(x - 6)$}
			\item $B = 4x² - 8x$\vspace{\spacing}

			      \correction{$B = 4x(x - 2)$}
			\item $D = 9x² + 12x + 4$\vspace{\spacing}

			      \correction{$D = (3x + 2)²$}
		\end{itemize}
	\end{multicols}
\end{exercice}

\begin{exercice}
	Donner toutes les solutions des équations suivantes :
	\begin{multicols}{2}
		\begin{itemize}
			\item $x² = 100$\vspace{\spacing}

			      \correction{$x = 10$ OU $x = -10$}
			\item $x³ = -125$\vspace{\spacing}

			      \correction{$x = -5$}
			\item $(x - 4)(x + 7) = 0$\vspace{\spacing}

			      \correction{$x = 4$ OU $x = -7$}
			\item $x² + 9x = 0$\vspace{\spacing}

			      \correction{$x = 0$ OU $x = -9$}
		\end{itemize}
	\end{multicols}
\end{exercice}

\end{document}