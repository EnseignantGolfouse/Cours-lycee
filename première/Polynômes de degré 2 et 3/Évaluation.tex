\documentclass[
	classe=$1^{ere}STI2D$,
	headerTitle=Évaluation\space Chapitre\space 4
]{évaluation}

\usepackage{tikz-repère}
\usepackage{tcolorbox}

\renewcommand{\arraystretch}{1.4}

\date{27 janvier 2023}

\begin{document}

\title{Évaluation (Sujet A) : polynômes de degré 2 et 3}
\maketitle

\begin{tcolorbox}
	La calculatrice est autorisée.

	L'exercice 2 est à faire sur le sujet, le reste sur une feuille à part.
\end{tcolorbox}

\begin{exercice}
	Pour chaque fonction polynôme de degré $2$ ci-dessous, donner la valeur des coefficients $a$, $b$ et $c$, ainsi que les coordonnées du sommet de la courbe de la fonction.
	\begin{enumerate}
		\item $f(x) = 2x² - 4x + 1$
		\item $g(x) = 12x² + x - 9$
		\item $h(x) = x² + 1$
	\end{enumerate}
\end{exercice}

\begin{exercice}
	Pour chaque courbe ci-dessous, donner les coordonnées du sommet, les racines si elles existent, et le signe de $a$ :
	\begin{center}
		\begin{tabular}{|*{4}{>{\centering}p{4cm}|}}
			\hline
			A & B                                                                                                     & C & D \tabularnewline \hline
			\begin{tikzpicture}[scale=0.45]
				\tikzRepere{-3.5}{3.5}{-5.5}{5.5}
				\draw[blue,very thick,domain=-2.74:4] plot({\x},{0.5*\x*\x - 1.5*\x - 1.875}) node[above left] {$𝒞_f$};
			\end{tikzpicture}
			  & \begin{tikzpicture}[scale=0.45]
				    \tikzRepere{-3.5}{3.5}{-5.5}{5.5}
				    \draw[blue,very thick,domain=-4:3.9] plot({\x},{-2/9*\x*\x - 2/3*\x}) node[above] {$𝒞_f$};
			    \end{tikzpicture}
			  & \begin{tikzpicture}[scale=0.45]
				    \tikzRepere{-3.5}{3.5}{-5.5}{5.5}
				    \draw[blue,very thick,domain=-0.45:4] plot({\x},{\x*\x - 4*\x + 4}) node[left] {$𝒞_f$};
			    \end{tikzpicture}
			  & \begin{tikzpicture}[scale=0.45]
				    \tikzRepere{-3.5}{3.5}{-5.5}{5.5}
				    \draw[blue,very thick,domain=-3.41:-0.58] plot({\x},{2*\x*\x + 8*\x + 10}) node[below right] {$𝒞_f$};
			    \end{tikzpicture}
			\tabularnewline \hline
			$S(\correctionOr{0,5}{\hspace{2em}};\correctionOr{-3}{\hspace{2em}})$
			  & $S(\correctionOr{-1,5}{\hspace{2em}};\correctionOr{0,5}{\hspace{2em}})$
			  & $S(\correctionOr{2}{\hspace{2em}};\correctionOr{0}{\hspace{2em}})$
			  & $S(\correctionOr{-2}{\hspace{2em}};\correctionOr{2}{\hspace{2em}})$
			\tabularnewline \hline
			Racines : \correction{$-1$ et $4$}
			  & Racines : \correction{$-3$ et $0$}
			  & Racines : \correction{$2$}
			  & Racines : \correction{Aucune}
			\tabularnewline \hline
			Signe de $a$ : $a$ \correctionDots{$>$} $0$
			  & Signe de $a$ : $a$ \correctionDots{$<$} $0$
			  & Signe de $a$ : $a$ \correctionDots{$>$} $0$
			  & Signe de $a$ : $a$ \correctionDots{$>$} $0$
			\tabularnewline \hline
		\end{tabular}
	\end{center}
\end{exercice}

\begin{exercice}
	Soit $f$ une fonction définie par $f(x) = x² + 3x - 4$.
	\begin{enumerate}
		\item Quels sont les coefficients $a$, $b$ et $c$ de cette fonction ?
		\item Comment sont orientés les bras de la fonction ? Justifier.
		\item Quelles sont les coordonnées du sommet de la courbe de $f$ ? Justifier.
		\item Tracer le graphe de la fonction $f$ entre $-5$ et $3$ (prendre deux unités par carreau en ordonnée).
		\item Déterminer graphiquement les racines de $f$. Écrire alors $f$ sous forme factorisée.
	\end{enumerate}
\end{exercice}

\begin{exercice}
	Résoudre les équations suivantes :
	\begin{enumerate}
		\item $(x - 6)(x + 4) = 0$
		\item $3x(2x - 8) = 0$
		\item $(4x + 2)² = 100$
		\item $2x(3x - 7) + 6(3x - 7) = 0$
	\end{enumerate}
\end{exercice}

\begin{exercice}
	Soit $g$ une fonction définie par $g(x) = 3x² + 21x - 54$.
	\begin{enumerate}
		\item Montrer que $g(x) = (3x - 6)(x + 9)$.
		\item Quelles sont les racines de cette fonction ?
		\item Comment sont orientés les bras de la fonction ? Justifier.
		\item Quelles sont les coordonnées du sommet de la courbe de $g$ ? Justifier.
		\item Dresser le tableau de variations de $g$.
	\end{enumerate}
\end{exercice}

%==============================================
%================ SUJET B =====================
%==============================================
\newpage
\setcounter{exercice}{1}

\title{Évaluation (Sujet B) : polynômes de degré 2 et 3}
\maketitle

\begin{tcolorbox}
	La calculatrice est autorisée.

	L'exercice 2 est à faire sur le sujet, le reste sur une feuille à part.
\end{tcolorbox}

\begin{exercice}
	Pour chaque fonction polynôme de degré $2$ ci-dessous, donner la valeur des coefficients $a$, $b$ et $c$, ainsi que les coordonnées du sommet de la courbe de la fonction.
	\begin{enumerate}
		\item $f(x) = 5x² - 10x + 2$
		\item $g(x) = 14x² + x - 3$
		\item $h(x) = 3x² - 1$
	\end{enumerate}
\end{exercice}

\begin{exercice}
	Pour chaque courbe ci-dessous, donner les coordonnées du sommet, les racines si elles existent, et le signe de $a$ :
	\begin{center}
		\begin{tabular}{|*{4}{>{\centering}p{4cm}|}}
			\hline
			A & B                                                                                                     & C & D \tabularnewline \hline
			\begin{tikzpicture}[scale=0.45]
				\tikzRepere{-3.5}{3.5}{-5.5}{5.5}
				\draw[blue,very thick,domain=-2.5:4] plot({\x},{0.5*\x*\x - 1.5*\x - 0.875}) node[above left] {$𝒞_f$};
			\end{tikzpicture}
			  & \begin{tikzpicture}[scale=0.45]
				    \tikzRepere{-3.5}{3.5}{-5.5}{5.5}
				    \draw[blue,very thick,domain=-3.9:4] plot({\x},{-2/9*\x*\x + 2/3*\x}) node[left] {$𝒞_f$};
			    \end{tikzpicture}
			  & \begin{tikzpicture}[scale=0.45]
				    \tikzRepere{-3.5}{3.5}{-5.5}{5.5}
				    \draw[blue,very thick,domain=-3.45:1.45] plot({\x},{\x*\x + 2*\x + 1}) node[left] {$𝒞_f$};
			    \end{tikzpicture}
			  & \begin{tikzpicture}[scale=0.45]
				    \tikzRepere{-3.5}{3.5}{-5.5}{5.5}
				    \draw[blue,very thick,domain=-3.41:-0.58] plot({\x},{2*\x*\x + 8*\x + 10}) node[below right] {$𝒞_f$};
			    \end{tikzpicture}
			\tabularnewline \hline
			$S(\correctionOr{0,5}{\hspace{2em}};\correctionOr{-2}{\hspace{2em}})$
			  & $S(\correctionOr{1,5}{\hspace{2em}};\correctionOr{0,5}{\hspace{2em}})$
			  & $S(\correctionOr{-1}{\hspace{2em}};\correctionOr{0}{\hspace{2em}})$
			  & $S(\correctionOr{-2}{\hspace{2em}};\correctionOr{2}{\hspace{2em}})$
			\tabularnewline \hline
			Racines : \correction{$-0,5$ et $3,5$}
			  & Racines : \correction{$0$ et $3$}
			  & Racines : \correction{$-1$}
			  & Racines : \correction{Aucune}
			\tabularnewline \hline
			Signe de $a$ : $a$ \correctionDots{$>$} $0$
			  & Signe de $a$ : $a$ \correctionDots{$<$} $0$
			  & Signe de $a$ : $a$ \correctionDots{$>$} $0$
			  & Signe de $a$ : $a$ \correctionDots{$>$} $0$
			\tabularnewline \hline
		\end{tabular}
	\end{center}
\end{exercice}

\begin{exercice}
	Soit $f$ une fonction définie par $f(x) = x² + 2x - 3$.
	\begin{enumerate}
		\item Quels sont les coefficients $a$, $b$ et $c$ de cette fonction ?
		\item Comment sont orientés les bras de la fonction ? Justifier.
		\item Quelles sont les coordonnées du sommet de la courbe de $f$ ? Justifier.
		\item Tracer le graphe de la fonction $f$ entre $-5$ et $3$ (prendre deux unités par carreau en ordonnée).
		\item Déterminer graphiquement les racines de $f$. Écrire alors $f$ sous forme factorisée.
	\end{enumerate}
\end{exercice}

\begin{exercice}
	Résoudre les équations suivantes :
	\begin{enumerate}
		\item $(x - 7)(x + 5) = 0$
		\item $4x(2x - 10) = 0$
		\item $(9x + 3)² = 100$
		\item $2x(3x - 13) + 9(3x - 13) = 0$
	\end{enumerate}
\end{exercice}

\begin{exercice}
	Soit $g$ une fonction définie par $g(x) = 2x² + 12x - 54$.
	\begin{enumerate}
		\item Montrer que $g(x) = (2x - 6)(x + 9)$.
		\item Quelles sont les racines de cette fonction ?
		\item Comment sont orientés les bras de la fonction ? Justifier.
		\item Quelles sont les coordonnées du sommet de la courbe de $g$ ? Justifier.
		\item Dresser le tableau de variations de $g$.
	\end{enumerate}
\end{exercice}


\end{document}