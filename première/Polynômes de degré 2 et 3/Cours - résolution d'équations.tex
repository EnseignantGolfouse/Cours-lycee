\documentclass[noheader]{coursclass}

\begin{document}

\newcommand{\Proprietes}{
	\begin{propriete}
		Pour résoudre l'équation $x² = a$ :
		\begin{itemize}
			\item Si $a > 0$, il y a deux solutions : $x = \sqrt{a}$ ou $x = -\sqrt{a}$.
			\item Si $a = 0$, il n'y a qu'une solution : $x = 0$.
			\item Si $a < 0$, il n'y a pas de solution.
		\end{itemize}
	\end{propriete}

	\begin{propriete}
		L'unique solution de l'équation $x³ = a$ est $x = \sqrt[3\hspace{0.5em}]{a}$, appelée la \textbf{racine troisième de $a$}.

		De plus,
		\begin{itemize}
			\item Si $a > 0$, $x > 0$
			\item Si $a = 0$, $x = 0$
			\item Si $a < 0$, $x < 0$
		\end{itemize}
	\end{propriete}
	\vspace{1em}
}

\Proprietes

\Proprietes

\Proprietes

\end{document}