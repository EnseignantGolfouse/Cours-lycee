\documentclass[
	classe=$1^{ere}STI2D$,
	headerTitle=Évaluation\space Chapitre\space 4
]{évaluation}

\usepackage{tikz-repère}
\usepackage{tcolorbox}

\renewcommand{\arraystretch}{1.4}

\date{28 février 2023}

\begin{document}

\title{Évaluation rattrapage (Sujet A) : polynômes de degré 2 \& 3}
\maketitle

\begin{tcolorbox}
	La calculatrice est autorisée.

	La question 1 de l'exercice 2 est à faire sur le sujet, le reste sur une feuille à part.
\end{tcolorbox}

\begin{exercice}
	On donne pour chaque question ci-dessous 3 coefficients $a$, $b$ et $c$.

	Donner l'expression de la fonction associée à $a$, $b$ et $c$, et donner l'abscisse du sommet de la courbe de la fonction.
	\begin{enumerate}
		\item $a = 3$, $b = 4$ et $c = 7$

		      \ifdefined\makeCorrection
			      $f(x) = \correctionOr{3x² + 4x + 7}{................................}$ \hspace{5em} abscisse du sommet : $\correctionOr{-\dfrac{b}{2a} = -\dfrac{4}{2×3} = -2/3}{................................}$
		      \fi
		\item $a = -1$, $b = 1$ et $c = -5$

		      \ifdefined\makeCorrection
			      $f(x) = \correctionOr{-x² + x - 5}{................................}$ \hspace{5em} abscisse du sommet : $\correctionOr{-\dfrac{b}{2a} = -\dfrac{1}{2×(-1)} = 0,5}{................................}$
		      \fi
		\item $a = 9$, $b = 0$ et $c = 15$

		      \ifdefined\makeCorrection
			      $f(x) = \correctionOr{9x² + 15}{................................}$ \hspace{5em} abscisse du sommet : $\correctionOr{-\dfrac{b}{2a} = -\dfrac{0}{2×9} = 0}{................................}$
		      \fi
	\end{enumerate}
\end{exercice}

\begin{exercice}
	Soit $f$ la fonction définie par $f(x) = x² + x - 6$.

	\begin{center}
		\begin{tikzpicture}[scale=0.55]
			\tikzRepere{-3.5}{2.5}{-6.5}{5.5}
			\ifdefined\makeCorrection
				\draw[red,very thick,domain=-4:3] plot({\x},{\x*\x + \x - 6}) node[above] {$𝒞_f$};
			\fi
		\end{tikzpicture}
	\end{center}
	\begin{enumerate}
		\item Tracer le graphe de la fonction $f$ dans le repère ci-dessus.
		\item Lire les racines de $f$ sur le graphe.
		\item En déduire la forme factorisée de $f$.
	\end{enumerate}
\end{exercice}

\begin{exercice}
	Résoudre les équations suivantes :
	\begin{enumerate}
		\item $(x - 3)(x + 9) = 0$
		\item $5x(2x - 10) = 0$
		\item $(6x + 2)² = 100$
		\item $2x(4x - 7) + 6(4x - 7) = 0$
	\end{enumerate}
\end{exercice}

%==============================================
%================ SUJET B =====================
%==============================================
\newpage
\setcounter{exercice}{1}

\title{Évaluation rattrapage (Sujet B) : polynômes de degré 2 \& 3}
\maketitle

\begin{tcolorbox}
	La calculatrice est autorisée.

	La question 1 de l'exercice 2 est à faire sur le sujet, le reste sur une feuille à part.
\end{tcolorbox}

\begin{exercice}
	On donne pour chaque question ci-dessous 3 coefficients $a$, $b$ et $c$.

	Donner l'expression de la fonction associée à $a$, $b$ et $c$, et donner l'abscisse du sommet de la courbe de la fonction.
	\begin{enumerate}
		\item $a = 6$, $b = 4$ et $c = 7$

		      \ifdefined\makeCorrection
			      $f(x) = \correctionOr{6x² + 4x + 7}{................................}$ \hspace{5em} abscisse du sommet : $\correctionOr{-\dfrac{b}{2a} = -\dfrac{4}{2×6} = -1/3}{................................}$
		      \fi
		\item $a = 1$, $b = -1$ et $c = -5$

		      \ifdefined\makeCorrection
			      $f(x) = \correctionOr{-x² + x - 5}{................................}$ \hspace{5em} abscisse du sommet : $\correctionOr{-\dfrac{b}{2a} = -\dfrac{-1}{2×1} = 0,5}{................................}$
		      \fi
		\item $a = 7$, $b = 14$ et $c = 0$

		      \ifdefined\makeCorrection
			      $f(x) = \correctionOr{7x² + 14x}{................................}$ \hspace{5em} abscisse du sommet : $\correctionOr{-\dfrac{b}{2a} = -\dfrac{14}{2×7} = -1}{................................}$
		      \fi
	\end{enumerate}
\end{exercice}

\begin{exercice}
	Soit $f$ la fonction définie par $f(x) = x² - 2x - 3$.

	\begin{center}
		\begin{tikzpicture}[scale=0.55]
			\tikzRepere{-3.5}{2.5}{-6.5}{5.5}
			\ifdefined\makeCorrection
				\draw[red,very thick,domain=-4:3] plot({\x},{\x*\x + 2*\x - 3}) node[above] {$𝒞_f$};
			\fi
		\end{tikzpicture}
	\end{center}
	\begin{enumerate}
		\item Tracer le graphe de la fonction $f$ dans le repère ci-dessus.
		\item Lire les racines de $f$ sur le graphe.
		\item En déduire la forme factorisée de $f$.
	\end{enumerate}
\end{exercice}

\begin{exercice}
	Résoudre les équations suivantes :
	\begin{enumerate}
		\item $(x - 4)(x + 6) = 0$
		\item $7x(3x - 12) = 0$
		\item $(9x + 5)² = 100$
		\item $8x(4x + 2) + 7(4x + 2) = 0$
	\end{enumerate}
\end{exercice}


\end{document}