\documentclass[
	classe=$1^{ere}STI2D$,
	headerTitle=Évaluation\space Chapitre\space 4
]{évaluation}

\usepackage{tikz-repère}
\usepackage{tcolorbox}

\renewcommand{\arraystretch}{1.4}

\date{28 février 2023}

\begin{document}

\title{Évaluation rattrapage (Sujet A) : polynômes de degré 2 \& 3}
\maketitle

\begin{tcolorbox}
	La calculatrice est autorisée.

	La question 1 de l'exercice 2 est à faire sur le sujet, le reste sur une feuille à part.
\end{tcolorbox}

\begin{exercice}
	On donne pour chaque question ci-dessous 3 coefficients $a$, $b$ et $c$.

	Donner l'expression de la fonction associée à $a$, $b$ et $c$, et donner l'abscisse du sommet de la courbe de la fonction.
	\begin{enumerate}
		\item $a = 3$, $b = 4$ et $c = 7$

		      \ifdefined\makeCorrection{\color{red}
				      $f(x) = \correctionOr{3x² + 4x + 7}{................................}$ \hspace{5em} abscisse du sommet : $\correctionOr{-\dfrac{b}{2a} = -\dfrac{4}{2×3} = -2/3}{................................}$
			      }\fi
		\item $a = -1$, $b = 1$ et $c = -5$

		      \ifdefined\makeCorrection{\color{red}
				      $f(x) = \correctionOr{-x² + x - 5}{................................}$ \hspace{5em} abscisse du sommet : $\correctionOr{-\dfrac{b}{2a} = -\dfrac{1}{2×(-1)} = 0,5}{................................}$
			      }\fi
		\item $a = 9$, $b = 0$ et $c = 15$

		      \ifdefined\makeCorrection{\color{red}
				      $f(x) = \correctionOr{9x² + 15}{................................}$ \hspace{5em} abscisse du sommet : $\correctionOr{-\dfrac{b}{2a} = -\dfrac{0}{2×9} = 0}{................................}$
			      }\fi
	\end{enumerate}
\end{exercice}

\begin{exercice}
	Soit $f$ la fonction définie par $f(x) = x² + x - 6$.

	\begin{center}
		\begin{tikzpicture}[scale=0.55]
			\tikzRepere{-3.5}{2.5}{-6.5}{5.5}
			\ifdefined\makeCorrection
				\draw[red,very thick,domain=-4:3] plot({\x},{\x*\x + \x - 6}) node[above] {$𝒞_f$};
			\fi
		\end{tikzpicture}
	\end{center}
	\begin{enumerate}
		\item Tracer le graphe de la fonction $f$ dans le repère ci-dessus.
		\item Lire les racines de $f$ sur le graphe.

		      \ifdefined\makeCorrection{\color{red}
				      Les racines sont $-3$ et $2$.
			      }\fi
		\item En déduire la forme factorisée de $f$.

		      \ifdefined\makeCorrection{\color{red}
				      La forme factorisée de $f$ est donc $f(x) = (x + 3)(x - 2)$.
			      }\fi
	\end{enumerate}
\end{exercice}

\begin{exercice}
	Résoudre les équations suivantes :
	\begin{enumerate}
		\item $(x - 3)(x + 9) = 0$

		      \ifdefined\makeCorrection{\color{red}
				      On a 2 solutions :
				      \begin{itemize}
					      \item Soit $x - 3 = 0$, et alors $x = 3$
					      \item Soit $x + 9 = 0$, et alors $x = -9$
				      \end{itemize}

				      L'ensemble des solutions est donc $\{-9 ; 3\}$.
			      }\fi
		\item $5x(2x - 10) = 0$

		      \ifdefined\makeCorrection{\color{red}
				      On a 2 solutions :
				      \begin{itemize}
					      \item Soit $5x = 0$, et alors $x = 0$
					      \item Soit $2x - 10 = 0$, et alors $x = 5$
				      \end{itemize}

				      L'ensemble des solutions est donc $\{0 ; 5\}$.
			      }\fi
		\item $(6x + 2)² = 100$

		      \ifdefined\makeCorrection{\color{red}
				      On a 2 solutions :
				      \begin{itemize}
					      \item Soit $6x + 2 = \sqrt{100} = 10$, et alors $x = \dfrac{4}{3}$
					      \item Soit $6x + 2 = -\sqrt{100} = -10$, et alors $x = -2$
				      \end{itemize}

				      L'ensemble des solutions est donc $\{-2 ; \dfrac{4}{3}\}$.
			      }\fi
		\item $2x(4x - 7) + 6(4x - 7) = 0$

		      \ifdefined\makeCorrection{\color{red}
				      On commence par factoriser : $2x(4x - 7) + 6(4x - 7) = (2x + 6)(4x - 7)$.

				      On a 2 solutions :
				      \begin{itemize}
					      \item Soit $2x + 6 = 0$, et alors $x = -3$
					      \item Soit $4x - 7 = 0$, et alors $x = \dfrac{7}{4}$
				      \end{itemize}

				      L'ensemble des solutions est donc $\{-3 ; \dfrac{7}{4}\}$.
			      }\fi
	\end{enumerate}
\end{exercice}

%==============================================
%================ SUJET B =====================
%==============================================
\newpage
\setcounter{exercice}{1}

\title{Évaluation rattrapage (Sujet B) : polynômes de degré 2 \& 3}
\maketitle

\begin{tcolorbox}
	La calculatrice est autorisée.

	La question 1 de l'exercice 2 est à faire sur le sujet, le reste sur une feuille à part.
\end{tcolorbox}

\begin{exercice}
	On donne pour chaque question ci-dessous 3 coefficients $a$, $b$ et $c$.

	Donner l'expression de la fonction associée à $a$, $b$ et $c$, et donner l'abscisse du sommet de la courbe de la fonction.
	\begin{enumerate}
		\item $a = 6$, $b = 4$ et $c = 7$

		      \ifdefined\makeCorrection{\color{red}
				      $f(x) = \correctionOr{6x² + 4x + 7}{................................}$ \hspace{5em} abscisse du sommet : $\correctionOr{-\dfrac{b}{2a} = -\dfrac{4}{2×6} = -1/3}{................................}$
			      }\fi
		\item $a = 1$, $b = -1$ et $c = -5$

		      \ifdefined\makeCorrection{\color{red}
				      $f(x) = \correctionOr{x² - x - 5}{................................}$ \hspace{5em} abscisse du sommet : $\correctionOr{-\dfrac{b}{2a} = -\dfrac{-1}{2×1} = 0,5}{................................}$
			      }\fi
		\item $a = 7$, $b = 14$ et $c = 0$

		      \ifdefined\makeCorrection{\color{red}
				      $f(x) = \correctionOr{7x² + 14x}{................................}$ \hspace{5em} abscisse du sommet : $\correctionOr{-\dfrac{b}{2a} = -\dfrac{14}{2×7} = -1}{................................}$
			      }\fi
	\end{enumerate}
\end{exercice}

\begin{exercice}
	Soit $f$ la fonction définie par $f(x) = x² - 2x - 3$.

	\begin{center}
		\begin{tikzpicture}[scale=0.55]
			\tikzRepere{-3.5}{2.5}{-6.5}{5.5}
			\ifdefined\makeCorrection
				\draw[red,very thick,domain=-2.15:3] plot({\x},{\x*\x - 2*\x - 3}) node[above] {$𝒞_f$};
			\fi
		\end{tikzpicture}
	\end{center}
	\begin{enumerate}
		\item Tracer le graphe de la fonction $f$ dans le repère ci-dessus.
		\item Lire les racines de $f$ sur le graphe.

		      \ifdefined\makeCorrection{\color{red}
				      Les racines sont $-1$ et $3$.
			      }\fi
		\item En déduire la forme factorisée de $f$.

		      \ifdefined\makeCorrection{\color{red}
				      La forme factorisée de $f$ est donc $f(x) = (x + 1)(x - 3)$.
			      }\fi
	\end{enumerate}
\end{exercice}

\begin{exercice}
	Résoudre les équations suivantes :
	\begin{enumerate}
		\item $(x - 4)(x + 6) = 0$

		      \ifdefined\makeCorrection{\color{red}
				      On a 2 solutions :
				      \begin{itemize}
					      \item Soit $x - 4 = 0$, et alors $x = 4$
					      \item Soit $x + 6 = 0$, et alors $x = -6$
				      \end{itemize}

				      L'ensemble des solutions est donc $\{-6 ; 4\}$.
			      }\fi
		\item $7x(3x - 12) = 0$

		      \ifdefined\makeCorrection{\color{red}
				      On a 2 solutions :
				      \begin{itemize}
					      \item Soit $7x = 0$, et alors $x = 0$
					      \item Soit $3x - 12 = 0$, et alors $x = 4$
				      \end{itemize}

				      L'ensemble des solutions est donc $\{0 ; 4\}$.
			      }\fi
		\item $(9x + 5)² = 100$

		      \ifdefined\makeCorrection{\color{red}
				      On a 2 solutions :
				      \begin{itemize}
					      \item Soit $9x + 5 = \sqrt{100} = 10$, et alors $x = \dfrac{5}{9}$
					      \item Soit $9x + 5 = -\sqrt{100} = -10$, et alors $x = -\dfrac{5}{3}$
				      \end{itemize}

				      L'ensemble des solutions est donc $\{-\dfrac{5}{3} ; \dfrac{5}{9}\}$.
			      }\fi
		\item $8x(4x + 2) + 7(4x + 2) = 0$

		      \ifdefined\makeCorrection{\color{red}
				      On commence par factoriser : $8x(4x + 2) + 7(4x + 2) = (8x + 7)(4x + 2)$.

				      On a 2 solutions :
				      \begin{itemize}
					      \item Soit $8x + 7 = 0$, et alors $x = -\dfrac{7}{8}$
					      \item Soit $4x + 2 = 0$, et alors $x = -\dfrac{1}{2}$
				      \end{itemize}

				      L'ensemble des solutions est donc $\{-\dfrac{7}{8} ; -\dfrac{1}{2}\}$.
			      }\fi
	\end{enumerate}
\end{exercice}


\end{document}