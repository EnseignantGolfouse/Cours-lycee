\documentclass[
	classe=$1^{ere}STI2D$,
]{coursclass}

\title{Chapitre 4 : Polynômes de degré 2 et 3}
\date{}

\begin{document}

\maketitle

\begin{definition}[Polynôme de degré 2]
	Une \textbf{fonction polynôme de degré 2} est une fonction pouvant s'écrire sous la forme
	$$ f(x) = ax² + bx + c $$
	Où $a$, $b$ et $c$ sont des nombres constants.
\end{definition}

\begin{definition}[Racines]
	Une fonction de degré 2 peut \textit{parfois} (mais pas tout le temps) s'écrire sous la forme
	$$ f(x) = (x - r₁) × (x - r₂) $$
	Dans ce cas, on dit que $r₁$ et $r₂$ sont les \textbf{racines} de $f$.
\end{definition}

\begin{exemple}
	Si on développe l'expression $(x + 2)(x - 1)$, on obtient $............$.

	On dit que la fonction $f(x) = x² + x - 2$ s'écrit aussi $f(x) = (x + 2)(x - 1)$, et a pour racines

	$-2$ et $1$.
\end{exemple}

\begin{propriete}
	Si $r₁$ et $r₂$ sont les racines d'une fonction de degré 2 $f$, on a $f(r₁) = f(r₂) = 0$.
\end{propriete}

\end{document}