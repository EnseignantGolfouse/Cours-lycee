\documentclass{beamer}

\usepackage{préambule}
\usepackage{listings}

\begin{document}

\lstset{
	language={Python},
	showstringspaces=false,
	keywordstyle=\color{purple},
	stringstyle=\color{green}
}

\begin{frame}
	Écrire (sur feuille) un programme python effectuant les actions suivantes :
	\begin{itemize}
		\item Affiche l'entier 5.
		\item Affiche le texte "Bonjour tout le monde" !
		\item Affiche les entiers multiples de 3 entre 1 et 200.
		\item Définit une fonction qui double le nombre qui lui est donné.
	\end{itemize}
\end{frame}

\begin{frame}[containsverbatim]
	\begin{lstlisting}
print(5)

print("Bonjour tout le monde")

entier = 3
while entier <= 200:
    print(entier)
    entier += 3

def double(x):
    return 2 * x
	\end{lstlisting}
\end{frame}

\end{document}