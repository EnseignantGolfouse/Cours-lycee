\documentclass[
	classe=$2^{de}$
]{informatique}

\usepackage{tcolorbox}


\begin{document}

\title{Activité : le paradoxe de Saint-Petersbourg}
\maketitle

On considère le jeu de « Pile ou face » suivant :
\begin{tcolorbox}
	Le joueur gagne le double de sa mise en cas de victoire (« Face »), et perd sa mise en cas de défaite (« Pile »).

	Le joueur peut rejouer tant que sa réserve d'argent le lui permet.
\end{tcolorbox}

\begin{enumerate}
	\item Compléter la fonction ci-dessous, qui simule une partie :
	      \begin{lstlisting}
from random import randint
def partie(mise):
	tirage = randint(0, 1)
	gain = 0
	if tirage == 0: # "pile"
		gain = ........
	else:           # "face"
		gain = ........
	return gain
\end{lstlisting}
	\item On note $X$ la variable aléatoire correspondant au gain du joueur.

	      Calculer et interpréter l'espérance de $X$. \correction{$𝔼(X) = 0$}
\end{enumerate}
\begin{tcolorbox}
	Pierre, qui dispose d'une réserve de $1000$€, décide de jouer de la manière suivante :
	\begin{itemize}
		\item Il mise $1$€ au départ du jeu ;
		\item Tant qu'il perd, il rejoue une partie en \textbf{doublant} sa mise ;
		\item Il arrête si il gagne, ou si il n'a plus assez d'argent.
	\end{itemize}
\end{tcolorbox}
\begin{enumerate}
	\setcounter{enumi}{2}
	\item Que se passe-t'il si Pierre joue et fait Pile-Pile-Pile-Face ? \correction{$+1$€}
	\item Afin de simuler une partie, on propose la fonction (incomplète) ci-dessous :
	      \begin{lstlisting}
from random import randint
def jeu():
	victoire = False
	mise = 1
	reserve = 1000
	while reserve > ....... and victoire == .......:
		gain = partie(mise)
		reserve = ................
		if gain > 0:
			victoire = True
		else:
			mise = ................
	return reserve
\end{lstlisting}
	      \begin{enumerate}
		      \item Retrouver, dans l'énoncé, la condition à vérifier dans la boucle \texttt{while} de la fonction.
		      \item Compléter les autre pointillés.
	      \end{enumerate}
	\item On note $Y$ la variable aléatoire correspondant au gain de Pierre à la fin du jeu.
	      \begin{enumerate}
		      \item Quelle est la valeur de $Y$ si Pierre fait Pile-Pile-Face ? \correction{$1$}
		      \item Quelle est la valeur de $Y$ si Pierre fait Pile-Face ? \correction{$1$}
		      \item Au bout de combien de « Pile » successives Pierre perd-il de l'argent ? Quelle est alors la valeur de $Y$ ? \correction{Au bout de 9 « Pile » successives. $Y$ vaut alors $-511$€.}
	      \end{enumerate}
	\item Donner alors la loi de probabilités de $Y$ sous forme d'un tableau.
	\item Calculer l'espérance de $Y$.
\end{enumerate}

\end{document}