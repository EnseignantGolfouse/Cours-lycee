\documentclass[
	classe=$1^{ere}STI2D$
]{automatisme}

\renewcommand{\arraystretch}{1.4}

\title{Exercice : loi de Bernoulli et loi binomiale}
\date{}

\begin{document}

\begin{frame}
	\maketitle

	\vspace*{-5em}
	On suppose qu'un sportif a $\dfrac{1}{2}$ chances de gagner chacun de ses matchs.

	\begin{enumerate}
		\item Faire un arbre de probabilités qui décrit les résultats possibles de $3$ matchs d'affilés.
		\item On note $X$ la variable aléatoire qui correspond au nombre de matchs gagnés.

		      Remplir alors le tableau suivant :
		      \begin{center}
			      \begin{tabular}{|c|c|c|c|c|}
				      \hline
				      $a_i$        & $0$                         & $1$                         & $2$                         & $3$                         \\ \hline
				      $P(X = a_i)$ & $\correction{\dfrac{1}{8}}$ & $\correction{\dfrac{3}{8}}$ & $\correction{\dfrac{3}{8}}$ & $\correction{\dfrac{1}{8}}$ \\ \hline
			      \end{tabular}
		      \end{center}
		\item Quelle est alors l'espérance de $X$ ? \correction{$3/2$}
		\item Reprendre les questions $1$ à $3$ avec un sportif qui a $\dfrac{4}{7}$ chances de gagner chacun de ses matchs.
	\end{enumerate}
\end{frame}

\end{document}