\documentclass[
	classe=$1^{ere}STI2D$,
	landscape,
	twocolumn
]{exercice}

\usepackage{tcolorbox}

\setlength{\columnsep}{1cm}

\title{Exercice : Lancés de pièces truquées}

\begin{document}

\newcommand{\Exercice}{
	\maketitle

	\begin{tcolorbox}
		On tire une pièce à pile ou face, mais cette dernière est truquée : la probabilité de faire face est de $0,8$.
	\end{tcolorbox}

	\begin{enumerate}
		\item Représenter deux tirages de la pièce par un arbre de probabilités \textit{(Espacer les branches de l'arbre, car on va l'agrandir dans les questions suivantes)} :

		      \vfill

		      Quelle est la probabilité de faire exactement un pile et une face ?
		\item On fait maintenant 3 tirages : agrandir l'arbre afin de représenter ce nouveau tirage.

		      Quelle est maintenant la probabilité de faire exactement une face et deux piles ?
		\item Si on fait $n$ tirages (pour $n$ un nombre entier $> 2$), quel est le nombre de d'issues dans l'arbre de probabilités correspondant ?
	
	 Quel est le nombre de chemins menant à l'évènement « On fait exactement une face » ?
		\item Quelle est alors la probabilité de faire exactement une face si on tire $10$ pièces truquées ? Et si on en tire $20$ ?
	\end{enumerate}
}

\Exercice

\newpage

\Exercice

\end{document}