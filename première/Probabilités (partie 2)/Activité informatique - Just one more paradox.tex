\documentclass[
	classe=$1^{ere}STI2D$
]{informatique}

\usepackage{tcolorbox}

\title{Activité : paradoxe “Just one more”}

\begin{document}

\maketitle

\begin{tcolorbox}
	On propose la situation suivante :

	On démarre avec $100$€, et on tire une pièce parfaitement équilibrée à pile ou face.
	\begin{itemize}
		\item Si on fait face, on multiplie notre argent par $1,8$.
		\item Sinon, on le divise par $2$.
	\end{itemize}
\end{tcolorbox}

\begin{enumerate}
	\item Si on alterne parfaitement face et pile, quelle est notre argent au bout de $2$ lancés ? Et de $4$ ?

	      \correctionDots{au bout de $2$ lancés, on a $90$€. Au bout de $4$, on a $81$€.}

	      Est-ce qu'il semble profitable de jouer à ce jeu ? \correctionDots{a priori non}
	\item Recopier et compléter la fonction suivante dans l'éditeur de Spyder. Cette fonction calcule le gain d'un individu au bout de $n$ lancés :

	      \begin{lstlisting}
def gain_n_lances(n):
	argent = 100
	for i in range(........):
		piece = randint(0,1)
		if piece == 0:
			argent = argent * 1.8
		else:
			argent = ........
	return argent
\end{lstlisting}
	\item À l'aide d'une boucle, calculer la moyenne obtenue par $10\ 000$ joueurs sur $20$ lancés.

	      On pourra utiliser le code ci-dessous en l'écrivant à la suite du code précédent, et en le complétant :
	      \begin{lstlisting}
def calcule_moyenne(nombre_joueurs, nombre_lances):
	total = 0
	for i in range(........):
		total = total + ........
	moyenne = total / ........
	return moyenne

calcule_moyenne(10000, 20)
\end{lstlisting}

	      Quelle moyenne trouve-t-on ? \correctionDots{$~5000$€ de moyenne}
	\item Dessiner un arbre de probabilités qui représente le lancé de deux pièces, et noter l'argent gagné pour chaque issue :

	      \begin{center}
		      \begin{tikzpicture}[scale=0.7]
			      \coordinate (Start) at (0,0);
			      \node (P) at (3,2) {\correction{$P$}};
			      \node (F) at (3,-2) {\correction{$F$}};
			      \node (PP) at (6,3.5) {\correction{$P$}};
			      \node (PF) at (6,1) {\correction{$F$}};
			      \node (FP) at (6,-1) {\correction{$P$}};
			      \node (FF) at (6,-3.5) {\correction{$F$}};

			      \ifdefined\makeCorrection
				      \draw (Start) -- (P)
				      (Start) -- (F)
				      (P) -- (PP) node[right] {$→ 25$€}
				      (P) -- (PF) node[right] {$→ 90$€}
				      (F) -- (FP) node[right] {$→ 90$€}
				      (F) -- (FF) node[right] {$→ 324$€};
			      \fi
		      \end{tikzpicture}
	      \end{center}
	\item Quel est l'argent moyen gagné sur deux lancés ? \correctionDots{$132,25$€}
	\item Bonus: À chaque lancé, l'argent moyen est multiplié par un nombre : lequel ? \correctionDots{On × par $1,5$}
	\item Dans Python, calculer la \textit{médiane} des gains obtenus par $10\ 000$ joueurs sur $20$ lancés.
\end{enumerate}

\end{document}