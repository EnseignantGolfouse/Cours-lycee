\documentclass[
	classe=$1^{ere}STI2D$,
	landscape,
	twocolumn,
]{exercice}

\usepackage{tcolorbox}
\usepackage{makecell}

\setlength{\columnsep}{1cm}
\renewcommand{\arraystretch}{1.4}

\title{Exercice : tickets à gratter}

\begin{document}

\newcommand{\Exercice}{
	\maketitle

	\begin{tcolorbox}
		Sur $3$ $000$ $000$ de tickets de loterie, on trouve :
		\begin{multicols}{2}
			\begin{itemize}
				\item $324$ $000$ lots de $1$€
				\item $294$ $000$ lots de $2$€
				\item $60$ $000$ lots de $4$€
				\item $78$ $000$ lots de $10$€
				\item $210$ lots de $100$€
				\item $60$ lots de $400$€
				\item $3$ lots de $4000$€
			\end{itemize}
		\end{multicols}
	\end{tcolorbox}

	On note $X$ la variable aléatoire correspondant au gain du joueur.
	\begin{enumerate}
		\item Exprimer l'évènement $\{X = 100\}$ en français.
		\item Remplir le tableau suivant :
		      \begin{center}
			      \hspace*{-1cm}\begin{tabular}{|*{8}{c|}}
				      \hline
				      \makecell{gain $a$                                                                                                                                                      \\ (en €)} & $1$                  & $2$                  & $4$                 & $10$                 & $100$               & $400$               & $4000$              \\ \hline
				      $P(X = a)$ & \correction{$0,108$} & \correction{$0,098$} & \correction{$0,02$} & \correction{$0,026$} & \correction{$7E-5$} & \correction{$2E-5$} & \correction{$1E-6$} \\ \hline
			      \end{tabular}
		      \end{center}
		\item Donner alors l'espérance de $X$ :

		      $$ E(X) = \correction{0,663€} $$
	\end{enumerate}
}

\Exercice

\newpage\Exercice

\end{document}