\documentclass[
	classe=$1^{ere}STI2D$
]{exercice}

\title{Activité : tirage de cartes}

\begin{document}

\maketitle

On considère un jeu de cartes de 40 cartes :
\begin{itemize}
	\item Quatres couleurs : cœur, pique, trèfle et carreau.
	\item $10$ cartes par couleur, numérotées de $1$ à $10$.
\end{itemize}

\begin{enumerate}
	\item Donner la probabilité des évènements suivants :
	      \begin{itemize}
		      \item «La carte tirée est un cœur» : \correctionDots{$1/4$}
		      \item «Le numéro de la carte tirée est 4» : \correctionDots{$1/10$}
		      \item «Le numéro de la carte tirée est inférieur ou égal à 6» : \correctionDots{$6/10$}
	      \end{itemize}
	\item On considère maintenant qu'on tire une \textit{première carte}, on la remet dans le jeu, puis on tire une \textit{deuxième carte}.

	      À chaque tirage, on regarde si la carte est un cœur, un pique, ou autre.
	      \begin{enumerate}
		      \item Combien d'issues possible y-a-t'il pour cette expérience ? \correction{$3×3 = 9$}
		      \item On appelle $A$ l'évènement «La carte tirée est un cœur», $B$ l'évènement «La carte tirée est un pique», et $C$ l'évènement «La carte tirée est un trèfle ou un carreau».

		            Représenter alors la situation par un arbre de probabilités ci-dessous :

		            \begin{center}
			            \begin{tikzpicture}
				            \ifdefined\makeCorrection
					            \coordinate (A) at (0,0);
					            \node (B1) at (3,4) {$A$};
					            \node (B2) at (3,0) {$B$};
					            \node (B3) at (3,-4) {$C$};
					            \node (C11) at (6,5) {$A$};
					            \node (C12) at (6,4) {$B$};
					            \node (C13) at (6,3) {$C$};
					            \node (C21) at (6,1) {$A$};
					            \node (C22) at (6,0) {$B$};
					            \node (C23) at (6,-1) {$C$};
					            \node (C31) at (6,-3) {$A$};
					            \node (C32) at (6,-4) {$B$};
					            \node (C33) at (6,-5) {$C$};
					            \draw (A) -- node[above left] {$1/4$} (B1)
					            (A) -- node[above] {$1/4$} (B2)
					            (A) -- node[below left] {$1/2$} (B3);
					            \draw (B1) -- node[above] {$1/4$} (C11)
					            (B1) -- node[above right] {$1/4$} (C12)
					            (B1) -- node[below] {$1/2$} (C13);
					            \draw (B2) -- node[above] {$1/4$} (C21)
					            (B2) -- node[above right] {$1/4$} (C22)
					            (B2) -- node[below] {$1/2$} (C23);
					            \draw (B3) -- node[above] {$1/4$} (C31)
					            (B3) -- node[above right] {$1/4$} (C32)
					            (B3) -- node[below] {$1/2$} (C33);
				            \else
					            \node (A) at (0,0) { };
					            \node (C11) at (6,5) { };
					            \node (C33) at (6,-5) { };
				            \fi
			            \end{tikzpicture}
		            \end{center}
		      \item Quelle est alors la probabilité que la première carte soit un cœur, et la deuxième un trèfle ou un carreau ? \correction{$1/8$}
		      \item Quelle est la probabilité de tirer exactement une carte de cœur ? \correction{$1/16 + 1/8 + 1/16 + 1/8 = 3/8$}
		      \item On dit qu'il y a \textbf{équiprobabilité} si toutes les issues ont la même probabilité.

		            Y a-t-il équiprobabilité dans la situation de la question $2$ ? Justifier. \correction{Non, car $1/4$ et $1/2$.}
	      \end{enumerate}

	\item Toujours dans la situation où on tire deux cartes, on considère maintenant les évènements suivants :
	      \begin{itemize}
		      \item $D$: «La \textit{première} carte est un cœur»
		      \item $E$: «La \textit{deuxième} carte est inférieure ou égale à $3$»
	      \end{itemize}
	      \begin{enumerate}
		      \item Représenter la situation par un arbre de probabilités.
		      \item Quel évènement à la plus haute probabilité :
		            \begin{itemize}
			            \item La première carte est un cœur, la deuxième est strictement supérieure à $3$.
			            \item La première carte n'est pas un cœur, la deuxième est inférieure ou égale à $3$.
		            \end{itemize}
		            ?
	      \end{enumerate}
\end{enumerate}

\end{document}