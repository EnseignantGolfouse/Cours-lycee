\documentclass[10pt]{beamer}

\usepackage{préambule-cours}

\begin{document}

\begin{frame}
	\begin{definition}[Épreuve de Bernoulli]
		\vspace*{-0.5em}Une \textbf{épreuve de Bernoulli} est une expérience aléatoire qui n'a que deux issue : le succès, ou l'échec.
	\end{definition}

	\begin{definition}[Loi de Bernoulli]
		\vspace*{-0.5em}Une \textbf{loi de Bernoulli} la variable aléatoire qui vaut $1$ en cas de succès, et $0$ en cas d'échec.

		La probabilité $p$ du succès est appelée le \textbf{paramètre} de cette loi. La loi de Bernoulli est alors donnée par le tableau suivant :
		\begin{center}
			\begin{tabular}{|c|c|c|}
				\hline
				$a_i$        & $0$     & $1$ \\ \hline
				$P(X = a_i)$ & $1 - p$ & $p$ \\ \hline
			\end{tabular}
		\end{center}
	\end{definition}

	\begin{exemple}
		Si on regarde la probabilité d'un sportif à gagner un de ses matchs :
		\vspace*{-1.2em}\begin{itemize}
			\item[-] Il s'agit d'une épreuve de Bernoulli, car le sportif ne peut que perdre ou gagner.
			\item[-] Le nombre de victoires qu'il obtient (zéro ou une) est alors une loi de Bernoulli.
		\end{itemize}
	\end{exemple}
\end{frame}

\end{document}