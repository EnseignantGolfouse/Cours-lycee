\documentclass[
	classe=$1^{ere}STI2D$,
	landscape,
	twocolumn,
]{évaluation}

\usepackage{tikz-repère}

\setlength{\columnsep}{1cm}

\begin{document}

\date{3 mars 2023}

\title{Interrogation : dérivées (sujet A)}
\maketitle

\begin{luacode}
	A = -0.25
	B = 1
	C = 3

	function print_f()
	tex.print(A)
	tex.print("*\\x*\\x + ")
	tex.print(B)
	tex.print("*\\x + ")
	tex.print(C)
	end

	function tangente_en(x)
	local f_x = A * x * x + B * x + C
	local f_prime_x = 2 * A * x + B
	tex.print(f_prime_x)
	tex.print("*\\x + ")
	tex.print(f_x - x * f_prime_x)
	end
\end{luacode}

\begin{exercice}
	\begin{center}
		\begin{tikzpicture}[scale=0.7]
			\tikzRepere{-4.5}{4.5}{-3.5}{4.5}

			\clip (-5,-4) rectangle (5,5);
			\draw[domain=-5:5,purple,thick] plot({\x},{\directlua{print_f()}}) node[right] {$𝒞_f$};

			\draw[domain=-5:5,green,ultra thick] plot({\x},{\directlua{tangente_en(2)}});
			\draw[domain=-5:5,green,ultra thick] plot({\x},{\directlua{tangente_en(0)}});
		\end{tikzpicture}
	\end{center}

	Sur le graphe ci-dessus, lire les valeurs de $f'(0) \correctionOr{{\color{red}= 1}}{}$ et $f'(2) \correctionOr{{\color{red}= 0}}{}$.
\end{exercice}

\begin{exercice}\
	Soit $f$ la fonction telle que $f(x) = 2x² - 3$.

	\begin{enumerate}
		\item Calculer $f'(1)$, en détaillant les calculs.

		      \correctionOr{{\color{red}\begin{align*}
						      f'(1) & = \lim_{h→0}\dfrac{f(1+h) - f(1)}{h}              \\
						            & = \lim_{h→0}\dfrac{2(1 + h)² - 3 - (2×1² - 3)}{h} \\
						            & = \lim_{h→0}\dfrac{2(1 + 2h + h²) - 3 + 1}{h}     \\
						            & = \lim_{h→0}\dfrac{2 + 4h + 2h² - 2}{h}           \\
						            & = \lim_{h→0}\dfrac{2h² + 4h}{h}                   \\
						            & = \lim_{h→0}2h + 4                                \\
						            & = 4                                               \\
					      \end{align*}}}{}
		\item Quelle est alors l'expression de la tangente à la courbe de $f$ au point d'abscisse $1$ ?

		      \correctionOr{{\color{red}
					      Ainsi l'expression de la tangente au point d'abscisse $1$ est

					      $$ y = f'(1)x + f(1) - 1 × f'(1) $$

					      Soit

					      $$ y = 4x - 5 $$
				      }}{}
	\end{enumerate}
\end{exercice}

%=================================================
%==================== SUJET B ====================
%=================================================
\newpage
\setcounter{exercice}{1}

\title{Interrogation : dérivées (sujet B)}
\maketitle

\begin{luacode}
	A = -0.25
	B = 0
	C = 3
\end{luacode}

\begin{exercice}
	\begin{center}
		\begin{tikzpicture}[scale=0.7]
			\tikzRepere{-4.5}{4.5}{-3.5}{4.5}

			\clip (-5,-4) rectangle (5,5);
			\draw[domain=-5:5,purple,thick] plot({\x},{\directlua{print_f()}}) node[right] {$𝒞_f$};

			\draw[domain=-5:5,green,ultra thick] plot({\x},{\directlua{tangente_en(2)}});
			\draw[domain=-5:5,green,ultra thick] plot({\x},{\directlua{tangente_en(0)}});
		\end{tikzpicture}
	\end{center}

	Sur le graphe ci-dessus, lire les valeurs de $f'(0) \correctionOr{{\color{red}= 0}}{}$ et $f'(2) \correctionOr{{\color{red}= -1}}{}$.
\end{exercice}

\begin{exercice}\
	Soit $f$ la fonction telle que $f(x) = 3x² - 2$.

	\begin{enumerate}
		\item Calculer $f'(1)$, en détaillant les calculs.
		      \correctionOr{{\color{red}\begin{align*}
						      f'(1) & = \lim_{h→0}\dfrac{f(1+h) - f(1)}{h}              \\
						            & = \lim_{h→0}\dfrac{3(1 + h)² - 2 - (3×1² - 2)}{h} \\
						            & = \lim_{h→0}\dfrac{3(1 + 2h + h²) - 2 - 1}{h}     \\
						            & = \lim_{h→0}\dfrac{3 + 6h + 3h² - 3}{h}           \\
						            & = \lim_{h→0}\dfrac{3h² + 6h}{h}                   \\
						            & = \lim_{h→0}3h + 6                                \\
						            & = 6                                               \\
					      \end{align*}}}{}
		\item Quelle est alors l'expression de la tangente à la courbe de $f$ au point d'abscisse $1$ ?

		      \correctionOr{{\color{red}
					      Ainsi l'expression de la tangente au point d'abscisse $1$ est

					      $$ y = f'(1)x + f(1) - 1 × f'(1) $$

					      Soit

					      $$ y = 6x - 5 $$
				      }}{}
	\end{enumerate}
\end{exercice}

\end{document}