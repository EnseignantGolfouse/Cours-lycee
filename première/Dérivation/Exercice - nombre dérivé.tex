\documentclass[
	classe=$1^{ere}STI2D$,
	headerTitle=Exercices,
	landscape,
	twocolumn
]{exercice}

\usepackage{tikz-repère}

\setlength{\columnsep}{1cm}

\title{Exercices : nombre dérivé}

\begin{document}

\newcommand{\NombreDerive}{
\maketitle

\begin{exercice}\ 

	\begin{center}
		\begin{tikzpicture}
			\tikzRepere{-5}{4}{-3}{4}

			\draw[red,thick,domain=-3.5:4] plot({\x},{\x}) node[right] {$f$};
			\draw[blue,thick,domain=-5:4] plot({\x},{-0.25*\x*\x - 0.5*\x + 3}) node[right] {$g$};
		\end{tikzpicture}\vspace{1em}

		Déterminer graphiquement les nombres dérivés suivants :

		\begin{multicols}{3}
			\begin{itemize}
				\setlength{\itemsep}{1em}
				\item $f'(-1) = \correctionOr{1}{.....}$
				\item $g'(-2) = \correctionOr{0,5}{.....}$
				\item $f'(0) = \correctionOr{1}{.....}$
				\item $g'(0) = \correctionOr{-0,5}{.....}$
				\item $f'(3) = \correctionOr{1}{.....}$
				\item $g'(2) = \correctionOr{-1,5}{.....}$
			\end{itemize}
		\end{multicols}
	\end{center}
\end{exercice}

\begin{exercice}
	Soit $f$ la fonction $f(x) = x² + 3$.

	\begin{enumerate}
		\item Montrer que $f$ est dérivable en $2$.
		\item Calculer $f'(2)$.
		\item Montrer que pour n'importe quel nombre réel $x$, $f$ est dérivable en $x$.
	\end{enumerate}
\end{exercice}
}

\NombreDerive

\newpage\setcounter{exercice}{0}

\NombreDerive

\end{document}