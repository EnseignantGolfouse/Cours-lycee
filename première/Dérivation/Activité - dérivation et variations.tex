\documentclass[
	classe=$1^{ere}STI2D$,
	landscape,
	twocolumn
]{exercice}

\usepackage{tcolorbox}

\setlength{\columnsep}{1cm}

\title{Activité : Dérivation et variations}

\begin{document}

\newcommand{\Activité}{
\maketitle

\begin{tcolorbox}
	Une artisane fabrique des pendentifs en bronze. Elle peut en fabriquer jusqu'à $100$ par mois.

	Le coût de production en euros pour $x$ pendentifs est donné par la fonction
	$$ C(x) = 0,01x³ - 0,165x² + 38,72x + 172 $$
\end{tcolorbox}

\begin{enumerate}
	\item \textbf{Étude du coût}

	      \begin{enumerate}
		      \item Combien coûte la fabrication de $30$ pendentifs ?
		      \item Quel est le coût fixe pour l'artisane ?
		      \item Donner l'ensemble de définition de la fonction $C$.
	      \end{enumerate}
	\item \textbf{Étude de la recette}

	      Chaque pendentif est vendu $80$€.
	      \begin{enumerate}
		      \item Quelle est la recette obtenue pour la vente de $30$ pendentifs ?
		      \item Donner l'expression de la recette $R(x)$ en fonction du nombre $x$ de pendentifs vendus.
	      \end{enumerate}
	\item \textbf{Étude du bénéfice}

	      \begin{enumerate}
		      \item Quel est le bénéfice obtenu pour la vente de $30$ pendentifs ?
		      \item Donner l'expression du bénéfice $B(x)$ en fonction du nombre $x$ de pendentifs produits et vendus.
		      \item Montrer que $B'(x) = -0,03(x - 43)(x + 32)$.
		      \item En déduire le tableau de variations de $B$.
		      \item Quel est alors le bénéfice maximal possible ? Pour combien de pendentifs est-il atteint ?
	      \end{enumerate}
\end{enumerate}
}

\Activité

\newpage

\Activité

\end{document}