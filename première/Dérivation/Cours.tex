\documentclass[
	classe=$1^{ere}STI2D$,
	headerTitle=Cours\space Chapitre\space 5
]{coursclass}

\usepackage{tikz-repère}

\title{Chapitre 5 : Dérivation}
\author{}
\date{}

\begin{document}

\maketitle

\begin{definition}[Taux d'accroissement]
	Soit $f$ une fonction définie sur un intervalle $I$, $a$ un nombre dans $I$, et $h ≠ 0$, tel que $a + h ∈ I$.

	Alors, le rapport
	$$ τₐ(h) = \dfrac{f(a+h)-f(a)}{h} $$
	est le \textbf{taux d'accroissement} de $f$ entre $a$ et $a + h$.
\end{definition}

\begin{exemple}
	\begin{center}
		\begin{tikzpicture}[scale=0.9]
			\tikzRepere{0}{5}{-1}{5}

			\draw[variable=\x,blue,thick,domain=0:5] plot({\x},{0.55*\x*\x - 3.55*\x + 5}) node[right] {$f$};
		\end{tikzpicture}
	\end{center}

	Sur le graphique ci-dessus :
	\begin{itemize}
		\item $τ₀(1) = \dfrac{f(0 + 1) - f(0)}{1} = \dfrac{2 - 5}{1} = -3$
		\item $τ₁(4) = \dfrac{f(1 + 4) - f(1)}{4} = \dfrac{1 - 2}{4} = -0,25$
	\end{itemize}
\end{exemple}
\begin{exemple}
	Si $f$ est la fonction telle que $f(x) = x²$, on peut calculer l'expression de $τₐ(h)$ :

	\begin{align*}
		τₐ(h) & = \dfrac{f(a + h) - f(a)}{h}    \\
		      & = \dfrac{(a + h)² - a²}{h}      \\
		      & = \dfrac{a² + 2ah + h² - a²}{h} \\
		      & = \dfrac{2ah + h²}{h}           \\
		      & = 2a + h
	\end{align*}
\end{exemple}

\begin{remarque}
	Le taux d'accroissement de $f$ entre $a$ et $b$ correspond à la \textbf{pente} de la droite passant par $(a ; f(a))$ et $(b ; f(b))$.
\end{remarque}

\begin{definition}[Nombre dérivé]
	Si le taux d'accroissement $τₐ(h)$ admet une limite lorsque $h$ tend vers $0$, on dit que $f$ est \textbf{dérivable} en $a$.

	La limite est alors appelée le nombre dérivée de $f$ en $a$ : on note
	$$ \lim_{h→0} \dfrac{f(a+h)-f(a)}{h} = f'(a) $$
\end{definition}

\begin{remarque}
	Le nombre dérivé de $f$ en $a$ correspond à la \textbf{pente} de la droite \textit{tangente} à la courbe de $f$ au point $a$.
\end{remarque}

\begin{exemple}
	Soit $f$ la fonction telle que $f(x) = 0,1x³ - 0,3x² - 0,4x + 2$.

	\begin{itemize}
		\item \uline{Graphiquement :}

		      \begin{center}
			      \newcommand{\funcFCoefficientA}{0.1}
			      \newcommand{\funcFCoefficientB}{-0.3}
			      \newcommand{\funcFCoefficientC}{-0.4}
			      \newcommand{\funcFCoefficientD}{2.0}
			      \newcommand{\funcF}[1]{(\funcFCoefficientA)*#1*#1*#1 + (\funcFCoefficientB)*#1*#1 + (\funcFCoefficientC)*#1 + (\funcFCoefficientD)}
			      \newcommand{\funcPrimeF}[1]{3*(\funcFCoefficientA)*#1*#1 + 2*(\funcFCoefficientB)*#1 + (\funcFCoefficientC)}
			      \newcommand{\tangenteEnA}[2]{(\funcPrimeF{#1})*#2 + (\funcF{#1}) - #1*(\funcPrimeF{#1})}
			      \begin{tikzpicture}
				      \tikzRepere{-3}{5}{-2}{5}

				      \draw[red,thick,domain=-3:5] plot({\x},{\funcF{\x}}) node[right] {$f$};

				      \foreach \x/\a/\b in {-2/-3/-1,1/0/2,4/3/5} {
						      \pgfmathsetmacro{\tA}{\tangenteEnA{\x}{\a}}
						      \pgfmathsetmacro{\tB}{\tangenteEnA{\x}{\b}}
						      \pgfmathsetmacro{\fX}{\funcF{\x}}
						      \draw[blue,very thick] (\a,\tA) -- (\b,\tB);
						      \draw[blue,dashed] (\x,0) -- (\x,\fX);
						      \node[below] at (\x,-0.2) {$\x$};
					      }
			      \end{tikzpicture}
		      \end{center}

		      On peut déterminer graphiquement la pente de la tangente, et obtenir ainsi le nombre dérivé :
		      \begin{multicols}{3}
			      \begin{itemize}
				      \item $f'(-2) = 2$
				      \item $f'(1) = 0,7$
				      \item $f'(4) = 2$
			      \end{itemize}
		      \end{multicols}
		\item \uline{Par le calcul :}

		      On admet que pour tout $h ≠ 0$, on a

		      $$ τ₂(h) = -0,4 + 0,3h + 0,1h² $$

		      Alors $f$ est dérivable en $2$, car $τ₂(h)$ admet une limite lorsque $h$ tend vers $0$,

		      Et $f'(2) = \lim_{h→0}(-0,4 + 0,3h + 0,1h²) = -0,4$.
	\end{itemize}
\end{exemple}

\begin{propriete}[Équation de la tangente]
	La tangente à la courbe de $f$ au point $(a ; f(a))$ a pour équation
	$$ y = f'(a)x + f(a) - af'(a) $$
\end{propriete}

\begin{exemple}
	La courbe ci-dessous est la courbe de la fonction $f(x) = \dfrac{x²}{3}$.

	\begin{center}

		\begin{tikzpicture}
			\tikzRepere{-2.5}{3.5}{-3.5}{5.5}
			\draw[red,thick,domain=-3:4] plot({\x},{1/3*\x*\x}) node[right] {$𝒞_f$};
			\draw[blue,thick,domain=-0.5:4] plot({\x},{2*\x - 3}) node[midway,xshift=2cm,yshift=-1.2cm] {tangente};
		\end{tikzpicture}
	\end{center}

	On a alors :

	\begin{itemize}
		\item $f'(3) = 2$
		\item $f(3) - 3 × f'(3) = -3$
	\end{itemize}

	Ainsi l'équation de la tangente à la courbe en $3$ est

	$$ y = 2x - 3 $$
\end{exemple}

\end{document}