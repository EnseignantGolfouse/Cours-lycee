\documentclass[
	classe=$1^{ere}STI2D$,
	headerTitle=Cours\space Chapitre\space 5
]{coursclass}

\usepackage{tikz-repère}

\title{Chapitre 5 : Dérivation}
\author{}
\date{}

\begin{document}

\maketitle

\begin{definition}[Taux d'accroissement]
	Soit $f$ une fonction définie sur un intervalle $I$, $a$ un nombre dans $I$, et $h ≠ 0$, tel que $a + h ∈ I$.

	Alors, le rapport
	$$ τₐ(h) = \dfrac{f(a+h)-f(a)}{h} $$
	est le \textbf{taux d'accroissement} de $f$ entre $a$ et $a + h$.
\end{definition}

\begin{exemple}
	\begin{center}
		\begin{tikzpicture}[scale=0.9]
			\tikzRepere{0}{5}{-1}{5}

			\draw[variable=\x,blue,thick,domain=0:5] plot({\x},{0.55*\x*\x - 3.55*\x + 5}) node[right] {$f$};
		\end{tikzpicture}
	\end{center}

	Sur le graphique ci-dessus :
	\begin{itemize}
		\item $τ₀(1) = \dfrac{f(0 + 1) - f(0)}{1} = \dfrac{2 - 5}{1} = -3$
		\item $τ₁(4) = \dfrac{f(1 + 4) - f(1)}{4} = \dfrac{1 - 2}{4} = -0,25$
	\end{itemize}
\end{exemple}
\begin{exemple}
	Si $f$ est la fonction telle que $f(x) = x²$, on peut calculer l'expression de $τₐ(h)$ :

	\begin{align*}
		τₐ(h) & = \dfrac{f(a + h) - f(a)}{h}    \\
		      & = \dfrac{(a + h)² - a²}{h}      \\
		      & = \dfrac{a² + 2ah + h² - a²}{h} \\
		      & = \dfrac{2ah + h²}{h}           \\
		      & = 2a + h
	\end{align*}
\end{exemple}

\begin{remarque}
	Le taux d'accroissement de $f$ entre $a$ et $b$ correspond à la \textbf{pente} de la droite passant par $(a ; f(a))$ et $(b ; f(b))$.
\end{remarque}

\begin{definition}[Nombre dérivé]
	Si le taux d'accroissement $τₐ(h)$ admet une limite lorsque $h$ tend vers $0$, on dit que $f$ est \textbf{dérivable} en $a$.

	La limite est alors appelée le nombre dérivée de $f$ en $a$ : on note
	$$ \lim_{h→0} \dfrac{f(a+h)-f(a)}{h} = f'(a) $$
\end{definition}

\begin{remarque}
	Le nombre dérivé de $f$ en $a$ correspond à la \textbf{pente} de la droite
\end{remarque}

\end{document}