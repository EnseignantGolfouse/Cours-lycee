\documentclass[
	classe=$1^{ere}STI2D$,
	headerTitle=Activité
]{exercice}

\title{Activité : formule de la tangente}

\definecolor{mygreen}{RGB}{0, 170, 40}

\newcommand{\funcFCoefficientA}{0.0}
\newcommand{\funcFCoefficientB}{-0.3}
\newcommand{\funcFCoefficientC}{2.3}
\newcommand{\funcFCoefficientD}{0}
\newcommand{\funcF}[1]{(\funcFCoefficientA)*#1*#1*#1 + (\funcFCoefficientB)*#1*#1 + (\funcFCoefficientC)*#1 + (\funcFCoefficientD)}
\newcommand{\funcPrimeF}[1]{3*(\funcFCoefficientA)*#1*#1 + 2*(\funcFCoefficientB)*#1 + (\funcFCoefficientC)}

\newcommand{\tangenteEnA}[2]{(\funcPrimeF{#1})*#2 + (\funcF{#1}) - #1*(\funcPrimeF{#1})}

\newcommand{\PointTangent}{4}
\newcommand{\XDebut}{0}
\newcommand{\XFin}{8}

\begin{document}

\maketitle

\begin{center}
	\begin{tikzpicture}
		\draw[mygreen,thick] (\XDebut,-1) -- ++(0,8);
		\draw[mygreen,thick] (\XFin,-1) -- ++(0,8);
		\draw[mygreen,thick,domain=\XDebut:\XFin] plot({\x},{\funcF{\x}});
		\pgfmathsetmacro{\tangenteEnZero}{\tangenteEnA{\PointTangent}{\XDebut}}
		\pgfmathsetmacro{\tangenteEnCinq}{\tangenteEnA{\PointTangent}{\XFin}}
		\draw[red,thick] (\XDebut,\tangenteEnZero) -- node[above] {passerelle} (\XFin,\tangenteEnCinq);
		\draw[->] (\XDebut-1,-1.8) -- (\XFin + 1,-1.8);
		\draw (\XDebut,-1.8) -- ++(0,-0.2) node[below] {$0$};
		\draw (\XFin,-1.8) -- ++(0,-0.2) node[below] {$80$ m};
		\node[left] at (\XDebut,6) {paroi 1};
		\node[right] at (\XFin,6) {paroi 2};
	\end{tikzpicture}
\end{center}

Dans un ravin, on cherche à relier les deux parois par une passerelle. Au milieu du ravin, il y a une colline : on souhaite s'en servir comme support pour le pont. On va donc déterminer à quelle hauteurs se trouvent les points de départ et d'arrivée du pont. \bigskip

On sait que la surface de la colline est définie par la fonction $f(x) = -3x² + 5x + 2$.

\begin{enumerate}
	\item On admet que le pont touche la colline au point d'abscisse $40$.

	      Quelle est alors la pente du pont ?
	\item On va maintenant décrire le pont comme une nouvelle fonction, $g$.

	      Cette fonction est-elle constante ? Affine ? Du second degré ? Trigonométrique ?

	      Écrire alors l'expression de cette fonction (on pourra utiliser des lettres $a$, $b$, ... pour les paramètre inconnus).
	\item Donner la valeur de $g(40)$.
	\item Trouver alors l'expression complète de $g$.

	      En déduire la hauteur du pont sur chaque paroi.
	\item Qu'en est-il si le pont touche la colline au point d'abscisse $50$ ?
\end{enumerate}

\end{document}