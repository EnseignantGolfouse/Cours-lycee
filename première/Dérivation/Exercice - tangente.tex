\documentclass{automatisme}

% \usepackage{tikz-repère}

\begin{document}


\begin{frame}
	\uline{Prendre l'énoncé ci-dessous :} \medskip

	Soit $f$ la fonction telle que $f(x) = -0,5x² + 2,6x$.

	\begin{enumerate}
		\item Mettre dans un tableau à deux lignes les valeurs de $f(x)$ pour $x ∈ \{-1\ ;\ 0\ ;\ 1\ ;\ 2\ ;\ 3\ ;\ 4\ ;\ 5\}$.
		      %   \ifdefined\makeCorrection
		      %       \begin{center}
		      % 	      \begin{tabular}{|l|c|c|c|c|c|c|c|}
		      % 		      \hline
		      % 		      $x$    & $-1$   & $0$ & $1$   & $2$   & $3$   & $4$   & $5$   \\ \hline
		      % 		      $f(x)$ & $-3,1$ & $0$ & $2,1$ & $3,2$ & $3,3$ & $2,4$ & $0,5$ \\ \hline
		      % 	      \end{tabular}
		      %       \end{center}
		      %   \fi
		\item Utiliser ces valeurs pour tracer le graphe de $f$ entre $-1$ et $5$.
		      %   \ifdefined\makeCorrection
		      %       \begin{center}
		      % 	      \begin{tikzpicture}
		      % 		      \tikzRepere{-1}{5}{-3}{4}
		      % 		      \draw[domain=-1:5,thick,blue] plot({\x},{-0.5*\x*\x + 2.6*\x}) node[above right] {$𝒞_f$};
		      % 	      \end{tikzpicture}
		      %       \end{center}
		      %   \fi
		\item Quelle semble être la pente de la tangente à $f$ au point d'abscisse $3$ ?
		\item Calculer $f'(3)$.
		\item Quelle est alors l'expression complète de la tangente à la courbe de $f$ au point d'abscisse $3$ ?
	\end{enumerate}
\end{frame}

\end{document}