\documentclass[
	classe=$1^{ere}STI2D$,
	headerTitle=Activité,
	landscape,
	twocolumn,
]{exercice}

\usepackage{tikz-repère}

\setlength{\columnsep}{1cm}

\title{Activité : introduction du nombre dérivé}

\begin{document}

\newcommand{\Activite}{
	\maketitle

	On observe la distance parcourue par une voiture en accélération pendant les premières secondes après un démarrage. Celle-ci suit la fonction $d(t) = t²$ reproduite sur le graphique ci-dessous.

	\begin{center}
		\begin{tikzpicture}[scale=0.6]
			\tikzRepere{0}{10}{-0.4}{10}[$1$][$10$][temps (en secondes)][distance (en mètres)]
			\draw[red,thick,domain=0:10,variable=\x] plot({\x},{0.1*\x*\x}) node[right] {$d(t)$};
		\end{tikzpicture}
	\end{center}

	\begin{enumerate}
		\item Le trajet dure \correctionDots{$10$} secondes.
		\item La distance parcourue est de \correctionDots{$100$} mètres.
		\item Sur l'ensemble du trajet, la vitesse moyenne est de

		      $\dfrac{\correctionDots{100}}{\correctionDots{10\ }} = \correctionDots{10 \text{ m/s}}$
		\item La vitesse moyenne entre les secondes $0$ et $5$ est : $\dfrac{\correctionDots{25}}{\correctionDots{5\ }} = \correctionDots{5 \text{ m/s}}$
		\item La vitesse moyenne entre les secondes $5$ et $10$ est : $\dfrac{\correctionDots{75}}{\correctionDots{5\ }} = \correctionDots{15 \text{ m/s}}$
		\item La vitesse moyenne entre les secondes $1$ et $3$ est : $\dfrac{\correctionDots{8\ }}{\correctionDots{2\ }} = \correctionDots{4 \text{ m/s}}$
	\end{enumerate}
}

\Activite

\newpage

\Activite

\end{document}