\documentclass[
	noheader
]{coursclass}

\usepackage{tikz-repère}
\usepackage{tcolorbox}

\begin{document}

\newcommand{\Fonctions}{
	\begin{greybox}[frametitle={Fonctions de référence}]
		On admet la dérivée des fonctions suivantes :

		\renewcommand{\arraystretch}{1.6}
		\begin{center}
			\begin{tabular}{|c|c|}
				\hline
				Fonction $f$                       & Dérivée $f'$             \\ \hline
				$f(x) = c$ avec $c$ un nombre réel & $f'(x) = \phantom{aaaa}$ \\ \hline
				$f(x) = x$                         & $f'(x) = \phantom{aaaa}$ \\ \hline
				$f(x) = x²$                        & $f'(x) = \phantom{aaaa}$ \\ \hline
				$f(x) = x³$                        & $f'(x) = \phantom{aaaa}$ \\ \hline
			\end{tabular}
		\end{center}

		Ces dérivées sont à connaître !
	\end{greybox}
}

\Fonctions

\Fonctions

\Fonctions

\Fonctions

\end{document}