\documentclass[
	classe=$1^{ere} STI2D$,
	headerTitle=Interrogation,
]{évaluation}

\usepackage{diagbox}
\renewcommand{\arraystretch}{1.4}

\title{Interrogation : tableaux d'effectifs et de fréquences}
\date{18 novembre 2022}

\begin{document}

\maketitle

On fait une étude sur le tourisme en France dans trois grandes villes : Paris, Lyon et Marseille.

On a effectué une enquête parmi $600$ touristes de cette année pour savoir si ils étaient satisfaits ou déçus de leur visites. On a alors obtenu les données suivantes :

\begin{itemize}
	\item $60 \%$ des touristes sont allés à Paris.
	\item La moitié des touristes restants sont allés à Lyon.
	\item $95$ touristes qui sont allés à Lyon sont satisfaits.
	\item $120$ touristes qui sont allés à Paris sont déçus.
	\item Dans l'ensemble, $70$\% des touristes sont satisfaits de leur voyage.
\end{itemize}

\begin{enumerate}
	\item Remplir le tableau \textbf{d'effectifs} suivant :

	      \begin{center}
		      \begin{tabular}{|l|*{4}{>{\centering}p{2.5cm}|}}
			      \hline
			      \diagbox{$X =$ satisfaction}{$Y =$ ville} & $y₁ =$ Paris     & $y₂ =$ Lyon      & $y₃ =$ Marseille & TOTAL \tabularnewline \hline
			      $x₁ =$ satisfaits                         & \correction{240} & \correction{95}  & \correction{85}  & \correction{420}  \tabularnewline \hline
			      $x₂ =$ non satisfaits                     & \correction{120} & \correction{25}  & \correction{35}  & \correction{180}   \tabularnewline \hline
			      TOTAL                                     & \correction{360} & \correction{120} & \correction{120} & \correction{600}   \tabularnewline \hline
		      \end{tabular}
	      \end{center}
	\item Établir le tableau des fréquences par rapport à l'effectif global :

	      \ifdefined\makeCorrection
		      \begin{center}
			      \begin{tabular}{|l|*{4}{>{\centering}p{2.5cm}|}}
				      \hline
				      \diagbox{$X =$ satisfaction}{$Y =$ ville} & $y₁ =$ Paris  & $y₂ =$ Lyon   & $y₃ =$ Marseille & TOTAL \tabularnewline \hline
				      satisfaits                                & \correction{} & \correction{} & \correction{}    & \correction{}  \tabularnewline \hline
				      non satisfaits                            & \correction{} & \correction{} & \correction{}    & \correction{}   \tabularnewline \hline
				      TOTAL                                     & \correction{} & \correction{} & \correction{}    & \correction{}   \tabularnewline \hline
			      \end{tabular}
		      \end{center}
	      \else
		      \vspace{16em}
	      \fi
	\item Établir le tableau des fréquences conditionnelles par rapport à $x₁$ :
	      \ifdefined\makeCorrection
		      \begin{center}
			      \begin{tabular}{|l|*{4}{>{\centering}p{2cm}|}}
				      \hline
				      $Y =$ ville & Paris         & Lyon          & Marseille     & TOTAL \tabularnewline \hline
				      satisfaits  & \correction{} & \correction{} & \correction{} & \correction{}  \tabularnewline \hline
			      \end{tabular}
		      \end{center}
	      \else
		      \vspace{10em}
	      \fi
	\item Quel est la ville avec le plus haut taux de satisfaction ?
\end{enumerate}

\newpage

\maketitle

On fait une étude sur le tourisme en France dans trois grandes villes : Paris, Lyon et Marseille.

On a effectué une enquête parmi $800$ touristes de cette année pour savoir si ils étaient satisfaits ou déçus de leur visites. On a alors obtenu les données suivantes :

\begin{itemize}
	\item $60 \%$ des touristes sont allés à Paris.
	\item La moitié des touristes restants sont allés à Lyon.
	\item $125$ touristes qui sont allés à Lyon sont satisfaits.
	\item $160$ touristes qui sont allés à Paris sont déçus.
	\item Dans l'ensemble, $70$\% des touristes sont satisfaits de leur voyage.
\end{itemize}

\begin{enumerate}
	\item Remplir le tableau \textbf{d'effectifs} suivant :

	      \begin{center}
		      \begin{tabular}{|l|*{4}{>{\centering}p{2.5cm}|}}
			      \hline
			      \diagbox{$X =$ satisfaction}{$Y =$ ville} & $y₁ =$ Paris     & $y₂ =$ Lyon      & $y₃ =$ Marseille & TOTAL \tabularnewline \hline
			      $x₁ =$ satisfaits                         & \correction{240} & \correction{95}  & \correction{85}  & \correction{420}  \tabularnewline \hline
			      $x₂ =$ non satisfaits                     & \correction{120} & \correction{25}  & \correction{35}  & \correction{180}   \tabularnewline \hline
			      TOTAL                                     & \correction{360} & \correction{120} & \correction{120} & \correction{600}   \tabularnewline \hline
		      \end{tabular}
	      \end{center}
	\item Établir le tableau des fréquences par rapport à l'effectif global :

	      \ifdefined\makeCorrection
		      \begin{center}
			      \begin{tabular}{|l|*{4}{>{\centering}p{2.5cm}|}}
				      \hline
				      \diagbox{$X =$ satisfaction}{$Y =$ ville} & $y₁ =$ Paris  & $y₂ =$ Lyon   & $y₃ =$ Marseille & TOTAL \tabularnewline \hline
				      satisfaits                                & \correction{} & \correction{} & \correction{}    & \correction{}  \tabularnewline \hline
				      non satisfaits                            & \correction{} & \correction{} & \correction{}    & \correction{}   \tabularnewline \hline
				      TOTAL                                     & \correction{} & \correction{} & \correction{}    & \correction{}   \tabularnewline \hline
			      \end{tabular}
		      \end{center}
	      \else
		      \vspace{16em}
	      \fi
	\item Établir le tableau des fréquences conditionnelles par rapport à $x₁$ :
	      \ifdefined\makeCorrection
		      \begin{center}
			      \begin{tabular}{|l|*{4}{>{\centering}p{2cm}|}}
				      \hline
				      $Y =$ ville & Paris         & Lyon          & Marseille     & TOTAL \tabularnewline \hline
				      satisfaits  & \correction{} & \correction{} & \correction{} & \correction{}  \tabularnewline \hline
			      \end{tabular}
		      \end{center}
	      \else
		      \vspace{10em}
	      \fi
	\item Quel est la ville avec le plus haut taux de satisfaction ?
\end{enumerate}

\end{document}