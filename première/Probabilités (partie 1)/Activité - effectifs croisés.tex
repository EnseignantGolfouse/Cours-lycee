\documentclass[table]{beamer}

\usepackage{préambule}

\renewcommand{\arraystretch}{1.2}

\setlength{\leftmargini}{0.5cm}
\setlength{\leftmarginii}{0.5cm}

\title{Activité - effectifs croisés}

\begin{document}
\footnotesize

\begin{frame}
	Une usine produit des pièces aréonautiques. Elle peuvent être soit en plastique, soit en aluminium.

	Des études statistiques menées sur un lot de 1000 pièces ont données les renseignements suivants :
	\begin{itemize}
		\item $45$ \% de ces pièces proviennent de l'usine A ;
		\item $270$ pièces qui proviennent de l'usine A sont en aluminium ;
		\item l'usine B fabrique autant de pièces en plastique que de pièces en aluminium.
	\end{itemize}

	\begin{enumerate}
		\item Combien de pièces proviennent de l'usine B ? % 550

		      Combien de pièces en plastique proviennent de l'usine A ? % 195
		\item Organiser toutes ces données dans un tableau à double entrée.
		\item On prélève «au hasard» une pièce de ce lot.

		      On note :

		      - A l'évènement «La pièce provient de l'usine A» ;

		      - B l'évènement «La pièce provient de l'usine B» ;

		      - P l'évènement «La pièce est en plastique».
		      \begin{enumerate}[a.]
			      \item Déterminer la probabilité de l'évènement A.
			      \item Définir par une phrase l'évènement $A ∩ B$. Calculer sa probabilité.
			      \item On choisit au hasard une pièce provenant de l'usine A.

			            Quelle est la probabilité qu'elle soit en aluminium ?
		      \end{enumerate}
	\end{enumerate}
\end{frame}

\begin{frame}
	\begin{enumerate}
		\color{red}
		\item 550 pièces proviennent de l'usine B.

		      180 pièces en plastique proviennent de l'usine A.
		\item \begin{center}
			      \begin{tabular}{l|c|c|c|}
				      \cline{2-4}
				                                                                        & \cellcolor[HTML]{00D2CB}{\color[HTML]{000000} Usine A} & \cellcolor[HTML]{00D2CB}{\color[HTML]{000000} Usine B} & \cellcolor[HTML]{00D2CB}{\color[HTML]{000000} TOTAL} \\ \hline
				      \multicolumn{1}{|l|}{\cellcolor[HTML]{00D2CB}Pièces en plastique} & 180                                                    & 275                                                    & 455                                                  \\ \hline
				      \multicolumn{1}{|l|}{\cellcolor[HTML]{00D2CB}Pièces en aluminium} & 270                                                    & 275                                                    & 545                                                  \\ \hline
				      \multicolumn{1}{|l|}{\cellcolor[HTML]{00D2CB}TOTAL}               & 450                                                    & 550                                                    & 1000                                                 \\ \hline
			      \end{tabular}
		      \end{center}
		\item \begin{enumerate}[a.]
			      \color{red}
			      \item La probabilité de l'évènement A est $\dfrac{450}{1000} = 0.45$.
			      \item L'évènement $A∩B$ correspond à la probabilité que la pièce provienne de l'usine A \uline{et} de l'usine B. Sa probabilité est donc $0$.
			      \item La probabilité qu'une pièce provenant de l'usine A soit en aluminium est $\dfrac{270}{450} = 0.6$.
		      \end{enumerate}
	\end{enumerate}
\end{frame}

\end{document}