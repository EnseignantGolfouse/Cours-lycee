\documentclass{beamer}

\usepackage{préambule}
\usepackage{diagbox}

\begin{document}

\begin{frame}
	On donne le tableau d'effectifs suivant :
	\begin{center}
		\begin{tabular}{|l|c|c|}
			\hline
			\diagbox{$X$}{$Y$} & $y₁$ & $y₂$ \\ \hline
			$x₁$               & 18   & 54   \\ \hline
			$x₂$               & 25   & 162  \\ \hline
		\end{tabular}
	\end{center}

	\begin{enumerate}
		\item Construire le tableau des fréquences conditionnelles par rapport à $y₂$.
		\item Construire le tableau des fréquences conditionnelles par rapport à $x₁$.
		\item Que remarque-t-on ?
		      \pause
		\item Proposer une situation modélisée par ce tableau.
	\end{enumerate}
\end{frame}

\newcommand{\makeCorrection}{}
\begin{frame}
	{\color{red}Correction}
	\renewcommand{\arraystretch}{1.4}
	\begin{enumerate}
		\item \begin{center}
			      \begin{tabular}{|l|c|}
				      \hline
				      $X$   & $y₂$                                  \\ \hline
				      $x₁$  & \correction{$\frac{54}{216} = 0.25$}  \\ \hline
				      $x₂$  & \correction{$\frac{162}{216} = 0.75$} \\ \hline
				      TOTAL & \correction{$1$}                      \\ \hline
			      \end{tabular}
		      \end{center}
		\item \begin{center}
			      \begin{tabular}{|l|c|c|c|}
				      \hline
				      $Y$  & $y₁$                                & $y₂$                                & TOTAL            \\ \hline
				      $x₁$ & \correction{$\frac{18}{72} = 0.25$} & \correction{$\frac{54}{72} = 0.75$} & \correction{$1$} \\ \hline
			      \end{tabular}
		      \end{center}
	\end{enumerate}
\end{frame}

\end{document}