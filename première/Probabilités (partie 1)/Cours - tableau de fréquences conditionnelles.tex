\documentclass[noheader,landscape,twocolumn]{coursclass}

\usepackage{diagbox}

\newgeometry{margin=0.5cm,top=1cm}

\begin{document}

\renewcommand{\arraystretch}{1.7}
\newcommand{\Cours}{
	\begin{definition}[Tableau de fréquences conditionnelles]
		Si on a un tableau d'effectifs, on peut pour chaque caractère dresser un \textbf{tableau de fréquences conditionnelles} par rapport à ce caractère.

		Dans ce cas, on ne garde que la ligne (ou colonne) liée à ce caractère, et on divise toutes les cases par le total de cette ligne (ou colonne).
	\end{definition}

	\begin{exemple}
		Si on reprend l'exemple des smartphones, on peut se demander :
		\begin{itemize}
			\item Parmi ceux qui ont 64 Go de capacité, quelle est la répartition des couleurs ?

			      On dresse alors le tableau suivant :\vspace{1em}

			      \begin{tabular}{|l|*{1}{>{\centering}p{2.3cm}|}}
				      \hline
				      $X$ = couleur & $y₁ = 64$ Go \tabularnewline
				      \hline
				      $x₁ =$ Noir   & $\frac{36}{80} = $ \tabularnewline
				      \hline
				      $x₂ =$ Blanc  & \tabularnewline
				      \hline
				      $x₃ =$ Rouge  & \tabularnewline
				      \hline
				      Total         & \tabularnewline
				      \hline
			      \end{tabular}
			\item Parmi ceux qui sont rouges, quelle est la répartition des capacités ?

			      On dresse alors le tableau suivant :\vspace{1em}

			      \begin{tabular}{|l|*{3}{>{\centering}p{2.3cm}|c|}}
				      \hline
				      $Y$ = capacité & $y₁ = 64$ Go       & $y₂ = 64$ Go & $y₃ = 64$ Go & Total \tabularnewline\hline
				      $x₃ =$ Rouge   & $\frac{24}{50} = $ &              &              & \tabularnewline\hline
			      \end{tabular}
		\end{itemize}
	\end{exemple}
}

\Cours

\newpage

\Cours

\end{document}