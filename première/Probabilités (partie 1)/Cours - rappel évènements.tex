\documentclass[noheader]{coursclass}

\usepackage{diagbox}
\usetikzlibrary{calc}

\begin{document}

\begin{center}
	\LARGE
	\uline{Rappel : vocabulaire des évènements}
	\vspace{1em}
\end{center}

\begin{definition}[Expérience aléatoire, évènements]
	Une \textbf{expérience aléatoire} est une expérience dont l'\textbf{issue} n'est pas connue à l'avance.

	Un \textbf{évènement} est un regroupement de plusieurs issues.
\end{definition}

\begin{exemple}
	Un jeté de dé est une expérience aléatoire, dont les issues sont $\{1, 2, 3, 4, 5, 6\}$.

	On peut noter $A$ l'évènement « le résultat obtenu est pair ». Cet évènement contient les issues $2$, $4$ et $6$ : on note alors $A = \{2, 4, 6\}$.
\end{exemple}

\begin{definition}[Opérations sur les évènements]
	Si $A$ et $B$ sont des évènements :
	\begin{itemize}
		\item $A ∪ B$ est l'évènement qui regroupe toutes les issues qui sont dans $A$ \textbf{ou} dans $B$ (ou les deux). On le lit « $A$ union $B$ ».
		\item $A ∩ B$ est l'évènement qui regroupe toutes les issues qui sont dans $A$ \textbf{et} dans $B$. On le lit « $A$ inter $B$ ».
		\item $\overline{A}$ est l'évènement qui regroupes toutes les issues qui \textbf{ne sont pas} dans $A$. On le lit « $A$ barre ».
	\end{itemize}

	\begin{center}
		\begin{tikzpicture}
			\coordinate (Rect1) at (0,1);
			\coordinate (Rect2) at (5,1);
			\coordinate (Rect3) at (10,1);

			\draw[rounded corners] (Rect1) rectangle ++(4,3);
			\node at ($(Rect1) + (2,-0.5)$) {$A ∪ B$};
			\draw[fill=gray] (Rect1) ++(1.5,1.5) ellipse (1.2 and 0.8);
			\draw[fill=gray] (Rect1) ++(2.5,1.5) ellipse (1.2 and 0.8);
			\draw (Rect1) ++(1.5,1.5) ellipse (1.2 and 0.8);

			\draw[rounded corners] (Rect2) rectangle ++(4,3);
			\node at ($(Rect2) + (2,-0.5)$) {$A ∩ B$};
			\begin{scope}
				\clip (Rect2) ++(1.5,1.5) ellipse (1.2 and 0.8);
				\draw[fill=gray] (Rect2) ++(2.5,1.5) ellipse (1.2 and 0.8);
			\end{scope}
			\draw (Rect2) ++(1.5,1.5) ellipse (1.2 and 0.8);
			\draw (Rect2) ++(2.5,1.5) ellipse (1.2 and 0.8);

			\draw[rounded corners,fill=gray] (Rect3) rectangle ++(4,3);
			\node at ($(Rect3) + (2,-0.5)$) {$\overline{A}$};
			\draw[fill=white] (Rect3) ++(1.5,1.5) ellipse (1.2 and 0.8);
			\draw (Rect3) ++(2.5,1.5) ellipse (1.2 and 0.8);
		\end{tikzpicture}
	\end{center}
\end{definition}

\begin{definition}[Probabilités]
	Chaque issue et évènement a une \textbf{probabilité} d'être réalisé. On note $P(A)$ la probabilité de l'évènement $A$.
\end{definition}

\begin{remarque}
	On a toujours $P(\overline{A}) = 1 - P(A)$.
\end{remarque}

\begin{propriete}[Probabilités]
	Si chaque issue à la même probabilité d'être réalisée, on dit qu'il y a \textbf{équiprobabilité}.

	\uline{Dans ce cas}, on a
	$$ P(A) = \dfrac{\text{nombre d'issues dans A}}{\text{nombre total d'issues}} $$
\end{propriete}

\end{document}