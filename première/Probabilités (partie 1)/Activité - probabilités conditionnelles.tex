\documentclass[
	classe=$1^eSTI2D$,
	headerTitle={Probabilités},
	landscape,twocolumn,noheader
]{exercice}

\usepackage{setspace}

\newgeometry{a4paper,twocolumn,margin=1cm}
\setlength{\columnsep}{1.2cm}
\setstretch{1.2}
\renewcommand{\arraystretch}{1.3}

\title{Activité : Probabilités}

\begin{document}

\maketitle

% Un jeu, dans lequel nos chances de victoire dépendent du coup précédent.

On joue à un jeu de dé. Pour gagner, il faut :
\begin{itemize}
	\item faire un $6$ au premier et au deuxième lancés ;
	\item OU, faire un résultat inférieur ou égal à trois au premier lancé, puis un résultat pair au deuxième lancé, et enfin faire un $1$ au troisième lancé.
\end{itemize}

\begin{enumerate}
	% Ils font eux-mêmes ! C'est bien !
	\item Donner un nom à chaque évènement qui apparaît dans l'énoncé.

	      \begin{itemize}
		      \item[$G$] : « On a gagné »
		      \item[$A$] : « Le résultat du premier lancé est $6$ » \correction{$P(A) = \frac{1}{6}$}
		      \item[$B$] : \correction{« Le résultat du deuxième lancé est $6$ » $P(B) = \frac{1}{6}$}
		      \item[$C$] : \correction{« Le résultat du premier lancé est inférieur ou égal à $3$ » $P(C) = \frac{1}{2}$}
		      \item[$D$] : \correction{« Le résultat du deuxième lancé est pair » $P(D) = \frac{1}{2}$}
		      \item[$E$] : \correction{« Le résultat du troisième lancé est $1$ » $P(E) = \frac{1}{6}$}
	      \end{itemize}
	\item Donner la probabilité de chaque évènement ci-dessus (sauf $G$).
	\item À quel évènement, ou union/intersection d'évènements la première méthode correspond-elle ? \correctionDots{$D ∩ E$}.

	      % Bien noter: cela ne fonctionne que parce que les évènements sont indépendants. C'est un cas particulier !
	      Quelle est notre probabilité de gagner en utilisant la première méthode ? \correctionDots{$\frac{1}{6} × \frac{1}{6} = \frac{1}{36}$}.
	\item À quel évènement, ou union/intersection d'évènements la deuxième méthode correspond-elle ? \correctionDots{$A ∩ B ∩ C$}.

	      Quelle est notre probabilité de gagner en utilisant la deuxième méthode ? \correctionDots{$\frac{3}{6} × \frac{3}{6} × \frac{1}{6} = \frac{1}{24}$}.
	\item Donner alors la valeur de $P(G) = \correctionDots{\frac{1}{24} + \frac{1}{36} = \frac{3}{72} + \frac{2}{72} = \frac{5}{72}}$
	\item  Si on a obtenu un $6$ au premier lancé, quel évènement nous permettra de gagner au deuxième lancé ? \correctionDots{l'évènement $E$}

	      Quelle est alors \textbf{dans ce cas} notre probabilité de gagner ? \correctionDots{c'est $P(E)$, soit $\frac{1}{6}$}
\end{enumerate}

\newpage

\title{Activité : Probabilités conditionnelles}

\maketitle

Dans une usine, deux machines $A$ et $B$ produisent le même type de pièce. On choisit une pièce au hasard produite par l'usine, et on considère les évènements suivants :

\begin{itemize}
	\item $A$ : « La pièce provient de la machine $A$ » ;
	\item $B$ : « La pièce provient de la machine $B$ » ;
	\item $D$ : « La pièce est défectueuse ».
\end{itemize}

On sait que $P(A) = 0,55$, $P_A(D) = 0,01$, et $P_B(D) = 0,02$.
% \newcommand{\makeCorrection}{}
\begin{enumerate}
	\item Pour chacune de ces probabilités, écrire une phrase expliquant sa signification.

	      $P(A)$ : \correctionDots{probabilité que la pièce provienne de la machine $A$.\hspace{20em}}

	      $P_A(D)$ : \correctionDots{probabilité que la pièce soit défectueuse, sachant qu'elle provient de la machine $A$.\hspace{3.5em}}

	      $P_B(D)$ : \correctionDots{probabilité que la pièce soit défectueuse, sachant qu'elle provient de la machine $B$.\hspace{3.5em}}
	\item Calculer $P(B) = $ \correctionDots{$1 - P(A) = 0,45$}
	\item Calculer $P(A ∩ D) = $ \correctionDots{$1 - P(A) = 0,45$}.

	      Écrire une phrase expliquant la signification de cette probabilité :

	      \correctionDots{probabilité que la pièce soit défectueuse, et qu'elle provient de la machine $A$.\hspace{9.5em}}
	\item À quel union/intersection d'évènements l'évènement « la pièce provient de la machine $B$ et est défectueuse » correspond-il ?

	      \correctionDots{$B ∩ D$}

	      Calculer sa probabilité. \correctionDots{$P(B ∩ D) = $}
\end{enumerate}

\end{document}