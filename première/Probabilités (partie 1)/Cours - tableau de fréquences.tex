\documentclass[noheader,landscape,twocolumn]{coursclass}

\usepackage{diagbox}

\newgeometry{margin=0.5cm,top=1cm}

\begin{document}

\newcommand{\Cours}{
	\begin{definition}[Tableau de fréquences]
		Lorsqu'on a un tableau d'effectifs, on peut dresser en parallèle un \textbf{tableau de fréquences}. \smallskip

		Chaque case contient le \uline{rapport de l'effectif considéré par l'effectif global.} \smallskip

		Chaque fréquence peut être exprimée comme un \uline{nombre décimal}, une \uline{fraction}

		ou un \uline{pourcentage}. \smallskip

		La fréquence d'un effectif marginal est une \textbf{fréquence marginale}. \smallskip

		La fréquence de la ligne $i$ et de la colonne $j$ est appelée $f_{ij}$.
	\end{definition}

	\begin{remarque}
		La fréquence totale est \textbf{toujours} $1$.
	\end{remarque}

	\begin{exemple}
		On reprend l'exemple des smartphones : Chaque effectif doit être divisé par ...... (l'effectif total).
		\bigskip

		\begin{minipage}{0.8\linewidth}
			\resizebox{1.03\textwidth}{!}{
				\renewcommand{\arraystretch}{1.4}
				\begin{tabular}{|l|*{4}{>{\centering}p{1.4cm}|}}
					\hline
					\diagbox{couleur}{mémoire} & $y₁ = 64$ Go            & $y₂ = 128$ Go & $y₃ = 256$ Go & Total\tabularnewline
					\hline
					$x₁ =$ Noir                & \phantom{$\frac{0}{0}$} &               &               & \tabularnewline
					\hline
					$x₂ =$ Blanc               & \phantom{$\frac{0}{0}$} &               &               & \tabularnewline
					\hline
					$x₃ =$ Rouge               & \phantom{$\frac{0}{0}$} &               &               & \tabularnewline
					\hline
					Total                      & \phantom{$\frac{0}{0}$} &               &               & \tabularnewline
					\hline
				\end{tabular}
			}
		\end{minipage}
		\begin{minipage}{0.05\linewidth}
			\tikz{
				\node at (0,1) {\phantom{ }};
				\draw[->] (0.6,0) node[right] {\begin{tabular}{c} Fréquences\\ marginales \end{tabular}} -- (0,0);
				\draw[->] (0.6,0) -- (0,-0.5);
				\draw[->] (0.6,0) -- (0,0.5);
			}
		\end{minipage}

		\tikz{
			\node at (-6.2,0) {\phantom{ }};
			\draw[->] (0,-1) node[below] {Fréquences marginales} -- (0,0);
			\draw[->] (0,-1) -- (1.6,0);
			\draw[->] (0,-1) -- (-1.6,0);
			\draw[->] (3.4,-0.5) node[below] {Fréquence totale ($N$)} -- ++(0,0.5);
		}
	\end{exemple}
}

\Cours

\newpage

\Cours

\end{document}