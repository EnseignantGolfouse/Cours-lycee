\documentclass[
	classe=$1^{ere} STI2D$,
	headerTitle=Cours\space Chapitre\space 3
]{coursclass}

\usepackage{diagbox}
\usetikzlibrary{calc}

\title{Chapitre 3 : Effectifs, fréquences et probabilités}
\date{}
\author{}

\begin{document}

\maketitle

\section{Tableaux}

\begin{definition}[Tableau croisé d'effectifs]
	Lorsqu'on étudie deux \textbf{caractères} d'un objet, on utilise un \textbf{tableau croisé d'effectifs} (aussi appelé \textbf{tableau à double entrée}).

	Le premier caractère est noté $X$, et le deuxième $Y$. \medskip

	Les valeurs prises par le caractère $X$ sont notées $x₁$, $x₂$, $x₃$, ..., et sont notées sur la première colonne.

	Les valeurs prises par le caractère $Y$ sont notées $y₁$, $y₂$, $y₃$, ..., et sont notées sur la première ligne. \medskip

	Dans chaque case, on place \textbf{l'effectif} correspondant au valeurs décrites.

	L'effectif correspondant aux valeurs $(xᵢ ; yⱼ)$ est noté $n_{ij}$.

	Les totaux de chaque colonne et chaque ligne sont appelés les \textbf{effectifs marginaux}. On les place dans la dernière ligne/colonne.

	Dans la case en bas à droite, on place \textbf{l'effectif total} $N$.
\end{definition}

\begin{exemple}
	un constructeur de smartphone vend un modèle disponible en trois couleurs différentes et avec trois capacités de stockage possibles. Un magasin fait le bilan du nombre de smartphones vendus selon ces deux caractères. \bigskip

	\begin{minipage}{0.9\linewidth}
		\renewcommand{\arraystretch}{1.4}
		\begin{tabular}{|l|*{4}{>{\centering}p{2cm}|}}
			\hline
			\diagbox{$X$ = couleur}{$Y$ = mémoire} & $y₁ = 64$ Go     & $y₂ = 128$ Go     & $y₃ = 256$ Go    & Total\tabularnewline
			\hline
			$x₁ =$ Noir                            & 36               & 73                & 16               & {\color{blue}125} \tabularnewline
			\hline
			$x₂ =$ Blanc                           & 20               & 52                & 3                & {\color{blue}75} \tabularnewline
			\hline
			$x₃ =$ Rouge                           & {24}             & 17                & 9                & {\color{blue}50} \tabularnewline
			\hline
			Total                                  & {\color{blue}80} & {\color{blue}142} & {\color{blue}28} & {\color{red}250}\tabularnewline
			\hline
		\end{tabular}
	\end{minipage}
	\begin{minipage}{0.05\linewidth}
		\tikz{
			\node at (0,1) {\phantom{ }};
			\draw[->] (1,0) node[right] {\begin{tabular}{c} Effectifs\\ marginaux \end{tabular}} -- (0,0);
			\draw[->] (1,0) -- (0,-0.8);
			\draw[->] (1,0) -- (0,0.8);
		}
	\end{minipage}

	\tikz{
		\node at (-9,0) {\phantom{ }};
		\draw[->] (0,-1) node[below] {Effectifs marginaux} -- (0,0);
		\draw[->] (0,-1) -- (2,0);
		\draw[->] (0,-1) -- (-2,0);
		\draw[->] (5,-1) node[below] {Effectif total ($N$)} -- (5,0);
	}
\end{exemple}

\begin{definition}[Tableau de fréquences]
	Lorsqu'on a un tableau d'effectifs, on peut dresser en parallèle un \textbf{tableau de fréquences}. \smallskip

	Chaque case contient le \uline{rapport de l'effectif considéré par l'effectif global.} \smallskip

	Chaque fréquence peut être exprimée comme un \uline{nombre décimal}, une \uline{fraction}

	ou un \uline{pourcentage}. \smallskip

	La fréquence d'un effectif marginal est une \textbf{fréquence marginale}. \smallskip

	La fréquence de la ligne $i$ et de la colonne $j$ est appelée $f_{ij}$.
\end{definition}

\begin{remarque}
	La fréquence totale est \textbf{toujours} $1$.
\end{remarque}

\begin{exemple}
	On reprend l'exemple des smartphones : Chaque effectif doit être divisé par $250$ (l'effectif total).
	\bigskip

	\begin{minipage}{0.89\linewidth}
		\resizebox{\textwidth}{!}{
			\renewcommand{\arraystretch}{1.5}
			\begin{tabular}{|l|*{4}{>{\centering}p{2.3cm}|}}
				\hline
				\diagbox{$X$ = couleur}{$Y$ = mémoire} & $y₁ = 64$ Go                  & $y₂ = 128$ Go                  & $y₃ = 256$ Go                  & Total\tabularnewline
				\hline
				$x₁ =$ Noir                            & $\frac{36}{250} = 0,144$      & $\frac{73}{250} = 0,292$       & $\frac{16}{250} = 0,064$       & \noindent{\color{blue}$0,5$} \tabularnewline
				\hline
				$x₂ =$ Blanc                           & $0,08$                        & $0,208$                        & $0,012$                        & \noindent{\color{blue}0,3} \tabularnewline
				\hline
				$x₃ =$ Rouge                           & $0,096$                       & $0,068$                        & $0,036$                        & \noindent{\color{blue}$0,2$} \tabularnewline
				\hline
				Total                                  & \noindent{\color{blue}$0,32$} & \noindent{\color{blue}$0,568$} & \noindent{\color{blue}$0,112$} & \noindent{\color{red}$1$}\tabularnewline
				\hline
			\end{tabular}
		}
	\end{minipage}
	\begin{minipage}{0.05\linewidth}
		\tikz{
			\node at (0,1) {\phantom{ }};
			\draw[->] (1,0) node[right] {\begin{tabular}{c} Fréquences\\ marginales \end{tabular}} -- (0,0);
			\draw[->] (1,0) -- (0,-0.8);
			\draw[->] (1,0) -- (0,0.8);
		}
	\end{minipage}

	\tikz{
		\node at (-8.6,0) {\phantom{ }};
		\draw[->] (0,-1) node[below] {Fréquences marginales} -- (0,0);
		\draw[->] (0,-1) -- (2,0);
		\draw[->] (0,-1) -- (-2,0);
		\draw[->] (4.9,-1) node[below] {Fréquence totale ($N$)} -- ++(0,1);
	}
\end{exemple}

\begin{definition}[Tableau de fréquences conditionnelles]
	Si on a un tableau d'effectifs, on peut pour chaque caractère dresser un \textbf{tableau de fréquences conditionnelles} par rapport à ce caractère.

	Dans ce cas, on ne garde que la ligne (ou colonne) liée à ce caractère, et on divise toutes les cases par le total de cette ligne (ou colonne).
\end{definition}

\begin{exemple}
	Si on reprend l'exemple des smartphones, on peut se demander :
	\begin{itemize}
		\item Parmi ceux qui ont 64 Go de capacité, quelle est la répartition des couleurs ?

		      On dresse alors le tableau suivant :

		      \renewcommand{\arraystretch}{1.5}
		      \begin{tabular}{|l|*{1}{>{\centering}p{2.3cm}|}}
			      \hline
			      $X$ = couleur & $y₁ = 64$ Go \tabularnewline
			      \hline
			      $x₁ =$ Noir   & $\frac{36}{80} = 0,45$ \tabularnewline
			      \hline
			      $x₂ =$ Blanc  & $0,25$ \tabularnewline
			      \hline
			      $x₃ =$ Rouge  & $0,3$ \tabularnewline
			      \hline
			      Total         & \noindent{\color{blue}$1$} \tabularnewline
			      \hline
		      \end{tabular}
		\item Parmi ceux qui sont rouges, quelle est la répartition des capacités ?

		      On dresse alors le tableau suivant :

		      \renewcommand{\arraystretch}{1.4}
		      \begin{tabular}{|l|*{4}{>{\centering}p{2.3cm}|}}
			      \hline
			      $Y$ = capacité & $y₁ = 64$ Go & $y₂ = 64$ Go & $y₃ = 64$ Go & Total \tabularnewline\hline
			      $x₃ =$ Rouge   & $0,48$       & $0,34$       & $0,18$       & \noindent{\color{blue}$1$} \tabularnewline\hline
		      \end{tabular}
	\end{itemize}
\end{exemple}

\newpage
\section{Probabilités conditionnelles}

\subsection*{Rappel : vocabulaire des évènements}

\begin{definition}[Expérience aléatoire, évènements]
	Une \textbf{expérience aléatoire} est une expérience dont l'\textbf{issue} n'est pas connue à l'avance.

	Un \textbf{évènement} est un regroupement de plusieurs issues.
\end{definition}

\begin{exemple}
	Un jeté de dé est une expérience aléatoire, dont les issues sont $\{1, 2, 3, 4, 5, 6\}$.

	On peut noter $A$ l'évènement « le résultat obtenu est pair ». Cet évènement contient les issues $2$, $4$ et $6$ : on note alors $A = \{2, 4, 6\}$.
\end{exemple}

\begin{definition}[Opérations sur les évènements]
	Si $A$ et $B$ sont des évènements :
	\begin{itemize}
		\item $A ∪ B$ est l'évènement qui regroupe toutes les issues qui sont dans $A$ \textbf{ou} dans $B$ (ou les deux). On le lit « $A$ union $B$ ».
		\item $A ∩ B$ est l'évènement qui regroupe toutes les issues qui sont dans $A$ \textbf{et} dans $B$. On le lit « $A$ inter $B$ ».
		\item $\overline{A}$ est l'évènement qui regroupes toutes les issues qui \textbf{ne sont pas} dans $A$. On le lit « $A$ barre ».
	\end{itemize}

	\begin{center}
		\begin{tikzpicture}
			\coordinate (Rect1) at (0,1);
			\coordinate (Rect2) at (5,1);
			\coordinate (Rect3) at (10,1);

			\draw[rounded corners] (Rect1) rectangle ++(4,3);
			\node at ($(Rect1) + (2,-0.5)$) {$A ∪ B$};
			\draw[fill=gray] (Rect1) ++(1.5,1.5) ellipse (1.2 and 0.8);
			\draw[fill=gray] (Rect1) ++(2.5,1.5) ellipse (1.2 and 0.8);
			\draw (Rect1) ++(1.5,1.5) ellipse (1.2 and 0.8);

			\draw[rounded corners] (Rect2) rectangle ++(4,3);
			\node at ($(Rect2) + (2,-0.5)$) {$A ∩ B$};
			\begin{scope}
				\clip (Rect2) ++(1.5,1.5) ellipse (1.2 and 0.8);
				\draw[fill=gray] (Rect2) ++(2.5,1.5) ellipse (1.2 and 0.8);
			\end{scope}
			\draw (Rect2) ++(1.5,1.5) ellipse (1.2 and 0.8);
			\draw (Rect2) ++(2.5,1.5) ellipse (1.2 and 0.8);

			\draw[rounded corners,fill=gray] (Rect3) rectangle ++(4,3);
			\node at ($(Rect3) + (2,-0.5)$) {$\overline{A}$};
			\draw[fill=white] (Rect3) ++(1.5,1.5) ellipse (1.2 and 0.8);
			\draw (Rect3) ++(2.5,1.5) ellipse (1.2 and 0.8);
		\end{tikzpicture}
	\end{center}
\end{definition}

\begin{definition}[Probabilités]
	Chaque issue et évènement a une \textbf{probabilité} d'être réalisé. On note $P(A)$ la probabilité de l'évènement $A$.
\end{definition}

\begin{remarque}
	On a toujours $P(\overline{A}) = 1 - P(A)$.
\end{remarque}

\begin{propriete}[Probabilités]
	Si chaque issue à la même probabilité d'être réalisée, on dit qu'il y a \textbf{équiprobabilité}.

	\uline{Dans ce cas}, on a 
	$$ P(A) = \dfrac{\text{nombre d'issues dans A}}{\text{nombre total d'issues}} $$
\end{propriete}

\subsection*{Probabilités conditionnelles}

\begin{definition}[Cardinal]
	Le \textbf{Cardinal} de l'évènement $A$ est le nombres d'issues dans $A$. On le note $Card(A)$.
\end{definition}

\begin{definition}[Probabilité conditionnelle]
	Si $A$ et $B$ sont des évènements, on appelle \textbf{probabilité de $A$ sachant $B$} le nombre

	$$ P_B(A) = \dfrac{Card(A ∩ B)}{Card(B)} $$
\end{definition}

\begin{exemple}
	Dans l'exemple des smartphones, on peut se demander : 

	Si on a pris un smartphone au hasard, et on a vu qu'il est rouge ; quelle est la probabilité qu'il ai $256$ Go de capacité ? \medskip

	On a alors
	\begin{itemize}
		\item $A$ l'évènement « le smartphone a $256$ Go de capacité ».
		\item $B$ l'évènement « le smartphone est rouge ».
	\end{itemize}

	En reprenant le tableau des fréquences, on voit que $Card(B) = 50$, et $Card(A ∩ B) = 9$.

	Ainsi $P_B(A) = \dfrac{9}{50} = 18\%$.
\end{exemple}

\end{document}