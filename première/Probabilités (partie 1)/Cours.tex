\documentclass[
	classe=$1^{ere} STI2D$,
	headerTitle=Cours\space Chapitre\space 3
]{coursclass}

\usepackage{diagbox}

\title{Chapitre 3 : Effectifs, fréquences et probabilités}
\date{}
\author{}

\begin{document}

\maketitle

\begin{definition}[Tableau croisé d'effectifs]
	Lorsqu'on étudie deux \textbf{caractères} d'un objet, on utilise un \textbf{tableau croisé d'effectifs} (aussi appelé \textbf{tableau à double entrée}).

	Le premier caractère est noté $X$, et le deuxième $Y$. \medskip

	Les valeurs prises par le caractère $X$ sont notées $x₁$, $x₂$, $x₃$, ..., et sont notées sur la première colonne.

	Les valeurs prises par le caractère $Y$ sont notées $y₁$, $y₂$, $y₃$, ..., et sont notées sur la première ligne. \medskip

	Dans chaque case, on place \textbf{l'effectif} correspondant au valeurs décrites.

	L'effectif correspondant aux valeurs $(xᵢ ; yⱼ)$ est noté $n_{ij}$.

	Les totaux de chaque colonne et chaque ligne sont appelés les \textbf{effectifs marginaux}. On les place dans la dernière ligne/colonne.

	Dans la case en bas à droite, on place \textbf{l'effectif total} $N$.
\end{definition}

\begin{exemple}
	un constructeur de smartphone vend un modèle disponible en trois couleurs différentes et avec trois capacités de stockage possibles. Un magasin fait le bilan du nombre de smartphones vendus selon ces deux caractères. \bigskip

	\begin{minipage}{0.9\linewidth}
		\renewcommand{\arraystretch}{1.4}
		\begin{tabular}{|l|*{4}{>{\centering}p{2cm}|}}
			\hline
			\diagbox{$X$ = couleur}{$Y$ = mémoire} & $y₁ = 64$ Go     & $y₂ = 128$ Go     & $y₃ = 256$ Go    & Total\tabularnewline
			\hline
			$x₁ =$ Noir                            & 36               & 73                & 16               & {\color{blue}125} \tabularnewline
			\hline
			$x₂ =$ Blanc                           & 20               & 52                & 3                & {\color{blue}75} \tabularnewline
			\hline
			$x₃ =$ Rouge                           & {24}             & 17                & 9                & {\color{blue}50} \tabularnewline
			\hline
			Total                                  & {\color{blue}80} & {\color{blue}142} & {\color{blue}28} & {\color{red}250}\tabularnewline
			\hline
		\end{tabular}
	\end{minipage}
	\begin{minipage}{0.05\linewidth}
		\tikz{
			\node at (0,1) {\phantom{ }};
			\draw[->] (1,0) node[right] {\begin{tabular}{c} Effectifs\\ marginaux \end{tabular}} -- (0,0);
			\draw[->] (1,0) -- (0,-0.8);
			\draw[->] (1,0) -- (0,0.8);
		}
	\end{minipage}

	\tikz{
		\node at (-9,0) {\phantom{ }};
		\draw[->] (0,-1) node[below] {Effectifs marginaux} -- (0,0);
		\draw[->] (0,-1) -- (2,0);
		\draw[->] (0,-1) -- (-2,0);
		\draw[->] (5,-1) node[below] {Effectif total ($N$)} -- (5,0);
	}
\end{exemple}

\begin{definition}[Tableau de fréquences]
	Lorsqu'on a un tableau d'effectifs, on peut dresser en parallèle un \textbf{tableau de fréquences}. \smallskip

	Chaque case contient le \uline{rapport de l'effectif considéré par l'effectif global.} \smallskip

	Chaque fréquence peut être exprimée comme un \uline{nombre décimal}, une \uline{fraction} 
	
	ou un \uline{pourcentage}. \smallskip

	La fréquence d'un effectif marginal est une \textbf{fréquence marginale}. \smallskip

	La fréquence de la ligne $i$ et de la colonne $j$ est appelée $f_{ij}$.
\end{definition}

\begin{remarque}
	La fréquence totale est \textbf{toujours} $1$.
\end{remarque}

\begin{exemple}
	On reprend l'exemple des smartphones : Chaque effectif doit être divisé par $250$ (l'effectif total).
	\bigskip

	\begin{minipage}{0.89\linewidth}
		\resizebox{\textwidth}{!}{
			\renewcommand{\arraystretch}{1.4}
			\begin{tabular}{|l|*{4}{>{\centering}p{2.3cm}|}}
				\hline
				\diagbox{$X$ = couleur}{$Y$ = mémoire} & $y₁ = 64$ Go             & $y₂ = 128$ Go            & $y₃ = 256$ Go            & Total\tabularnewline
				\hline
				$x₁ =$ Noir                            & $\frac{36}{250} = 0,144$ & $\frac{73}{250} = 0,292$ & $\frac{16}{250} = 0,064$ & {\color{blue}$0,5$} \tabularnewline
				\hline
				$x₂ =$ Blanc                           & $0,08$                   & $0,208$                  & $0,012$                  & {\color{blue}0,3} \tabularnewline
				\hline
				$x₃ =$ Rouge                           & $0,096$                  & $0,068$                  & $0,036$                  & {\color{blue}$0,2$} \tabularnewline
				\hline
				Total                                  & {\color{blue}$0,32$}     & {\color{blue}$0,568$}    & {\color{blue}$0,112$}    & {\color{red}$1$}\tabularnewline
				\hline
			\end{tabular}
		}
	\end{minipage}
	\begin{minipage}{0.05\linewidth}
		\tikz{
			\node at (0,1) {\phantom{ }};
			\draw[->] (1,0) node[right] {\begin{tabular}{c} Fréquences\\ marginales \end{tabular}} -- (0,0);
			\draw[->] (1,0) -- (0,-0.8);
			\draw[->] (1,0) -- (0,0.8);
		}
	\end{minipage}

	\tikz{
		\node at (-8.6,0) {\phantom{ }};
		\draw[->] (0,-1) node[below] {Fréquences marginales} -- (0,0);
		\draw[->] (0,-1) -- (2,0);
		\draw[->] (0,-1) -- (-2,0);
		\draw[->] (4.9,-1) node[below] {Fréquence totale ($N$)} -- ++(0,1);
	}
\end{exemple}

\end{document}