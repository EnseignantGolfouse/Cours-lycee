\documentclass{coursclass}

\usepackage{clipboard}

\title{Cours Chapitre 2}
\author{Généralités sur les fonctions}
\date{}

\begin{document}
\thispagestyle{plain}

\Copy{Definition}{
	\begin{exemple}
		\newcommand{\sqrtThree}{1.7320508075688772}
		\begin{center}
			\begin{tikzpicture}
				\draw[thin,gray] (-5.5,-1.5) grid (5.5,6.5);
				\draw[thick,\myArrow] (-5.5,0) -- (5.5,0) node[below right] {$x$};
				\draw[thick,\myArrow] (0,-1.5) -- (0,6.5) node[above left] {$y$};

				\draw[red,very thick,domain=-5.5:-\sqrtThree,variable=\x] plot({\x},{0.03*\x*(\x - 3)*(\x + 3) + 2.5});
				\draw[blue,very thick,domain=-\sqrtThree:\sqrtThree,variable=\x] plot({\x},{0.03*\x*(\x - 3)*(\x + 3) + 2.5});
				\draw[red,very thick,domain=\sqrtThree:5.5,variable=\x] plot({\x},{0.03*\x*(\x - 3)*(\x + 3) + 2.5}) node[above] {$𝒞_f$};

				\coordinate (A) at (-5.5,0);
				\coordinate (B) at (-\sqrtThree,0);
				\coordinate (C) at (\sqrtThree,0);
				\coordinate (D) at (5.5,0);

				\draw[red,ultra thick] (A) -- node[midway,above] {$I₁$} (B);
				\draw[blue,ultra thick] (B) -- node[midway,above left] {$I₂$} (C);
				\draw[red,ultra thick] (C) -- node[midway,above] {$I₃$} (D);
			\end{tikzpicture}
		\end{center}

		Ici, la fonction $f$ est \correctionDots{croissante} sur $I₁$ et $I₃$, et \correctionDots{décroissante} sur $I₂$.
	\end{exemple}
}

\vspace{3em}

\Paste{Definition}

\end{document}