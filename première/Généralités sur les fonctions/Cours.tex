\documentclass[
	classe=1 STI2D
]{coursclass}

\usepackage{annotate-equations}

\title{Cours Chapitre 2}
\author{Généralités sur les fonctions}
\date{}

\begin{document}

\maketitle

\begin{definition}[Fonction]
	Une \textbf{fonction} $f$ est un objet mathématique qui associe des valeurs d'un ensemble de \textbf{départ} $D$ à celles d'un ensemble \textbf{d'arrivée} $A$.

	On la note
	\begin{align*}
		f :\  & \eqnmark[red]{A}{A\vphantom{A⁵}} → \eqnmark[blue]{B}{B\vphantom{A⁵}} \\
		      & x ↦ f(x)
	\end{align*}
	\annotate[yshift=2em]{above}{A}{Départ}
	\annotate[yshift=1em]{above}{B}{Arrivée}

	\begin{itemize}
		\item $f(x)$ est \textbf{l'image} de $x$ par la fonction $f$.
		\item $x$ est \textbf{\uline{un} antécédent} de $f(x)$ par la fonction $f$.
	\end{itemize}
\end{definition}

\begin{remarque}
	\begin{itemize}
		\item Pour un nombre donné $x$, il n'y a \uline{q'une seule image} $f(x)$.
		\item Pour un nombre donné $y$, il peut y avoir \uline{plusieurs antécédents} $x$ tels que $y = f(x)$.
	\end{itemize}
\end{remarque}

\section{Représentation graphique}

TODO ! :)

\end{document}