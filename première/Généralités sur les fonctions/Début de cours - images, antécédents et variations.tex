\documentclass{beamer}

\usepackage{préambule}

\begin{document}

\begin{frame}
	\begin{minipage}{0.6\linewidth}
		\begin{tikzpicture}[scale=0.7]
			\draw[thin,gray] (-3.5,-3.5) grid (5.5,5.5);
			\draw[thick,\myArrow] (-3.5,0) -- (5.5,0);
			\draw[thick,\myArrow] (0,-3.5) -- (0,5.5);
			\draw[thick] (1,0) -- ++(0,-0.2) node[below] {$1$};
			\draw[thick] (0,1) -- ++(-0.2,0) node[left] {$1$};
			\node at (-0.2,-0.2) {$0$};

			\draw[ultra thick,orange,variable=\x,domain=-3:0] plot({\x},{\x+2});
			\draw[ultra thick,orange,variable=\x,domain=0:4] plot({\x},{-\x+2});
			\draw[ultra thick,orange,variable=\x,domain=4:5] plot({\x},{\x-6}) node[above] {$𝒞_f$};
		\end{tikzpicture}
	\end{minipage}
	\begin{minipage}{0.37\linewidth}
		Soit $f$ la fonction dont la courbe $𝒞_f$ est représentée ci-contre. 
		
		Déterminer :
		\begin{itemize}
			\item L'image de $3$ :
			\item Le/les points de $𝒞_f$ d'abscisse $-2$ :
			\item Le/les points de $𝒞_f$ d'ordonnée $1$ :
			\item Sur quel(s) intervalle(s) $f$ est-elle croissante ?
		\end{itemize}
	\end{minipage}
\end{frame}

\end{document}