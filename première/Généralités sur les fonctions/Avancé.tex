\documentclass[
	classe=$1^{ere}$STI2D,
	noheader
]{exercice}

\usepackage{xfp}
\usepackage{clipboard}

\newcommand{\computef}[1]{
	\fpeval{#1*#1 - 3*#1 + 1}%
}
\newcommand{\computeg}[1]{
	\fpeval{#1*#1*#1 - 3*#1*#1 - #1 + 3.1}%
}
\renewcommand{\arraystretch}{1.3}

\begin{document}

\Copy{Exercices}{
	\setcounter{exercice}{0}
	\begin{exercice}
		Soit $f$ la fonction telle que $f(x) = x² - 3x + 1$.

		Remplir le tableau suivant : \vspace{0.5em}

		\begin{tabular}{|c|c|c|c|c|c|c|c|c|c|c|c|}
			\hline
			$x$    & -1                         & -0.5                         & 0                         & 0.5                         & 1                         & 1.5                         & 2                         & 2.5                         & 3                         & 3.5                         & 4                         \\ \hline
			$f(x)$ & \correction{\computef{-1}} & \correction{\computef{-0.5}} & \correction{\computef{0}} & \correction{\computef{0.5}} & \correction{\computef{1}} & \correction{\computef{1.5}} & \correction{\computef{2}} & \correction{\computef{2.5}} & \correction{\computef{3}} & \correction{\computef{3.5}} & \correction{\computef{4}} \\ \hline
		\end{tabular}
		\vspace{1em}

		Placer ces points dans un repère orthonormé. \vspace{0.5em}

		D'après la courbe obtenue, combien y-a-t'il d'antécédents de $0$ par la fonction $f$ ? ......

		% \begin{tikzpicture}
		% 	\draw[thin,gray] (-1.5,-5) grid (4.5,8);
		% 	\draw[thick,\myArrow] (-1.5,0) -- (4.5,0);

		% 	\foreach \x in {-1,-0.5,...,4} {
		% 			\node[red,thick] at (\x,\computef{\x}) {×};
		% 		}
		% \end{tikzpicture}
	\end{exercice}

	\begin{exercice}
		Soit $g$ la fonction telle que $f(x) = x³ - 3x² - x + 3,1$.

		Remplir le tableau suivant : \vspace{0.5em}

		\begin{tabular}{|c|c|c|c|c|c|c|c|c|c|c|}
			\hline
			$x$    & -1                         & -0.5                         & 0                         & 0.5                         & 1                         & 1.5                         & 2                         & 2.5                         & 3                         & 3.5                         \\ \hline
			$f(x)$ & \correction{\computeg{-1}} & \correction{\computeg{-0.5}} & \correction{\computeg{0}} & \correction{\computeg{0.5}} & \correction{\computeg{1}} & \correction{\computeg{1.5}} & \correction{\computeg{2}} & \correction{\computeg{2.5}} & \correction{\computeg{3}} & \correction{\computeg{3.5}} \\ \hline
		\end{tabular}
		\vspace{1em}

		Placer ces points dans un repère orthonormé. \vspace{0.5em}

		D'après la courbe obtenue, combien y-a-t'il d'antécédents de $0$ par la fonction $g$ ? ......

		% \begin{tikzpicture}
		% 	\draw[thin,gray] (-1.5,-5) grid (3.5,8);

		% 	\foreach \x in {-1,-0.5,...,3.5} {
		% 			\node[red,thick] at (\x,\computeg{\x}) {×};
		% 		}
		% \end{tikzpicture}
	\end{exercice}

	\begin{exercice}\

		\begin{enumerate}
			\item Vérifier que la fonction $h$ telle que $h(x) = 2x + 1$ n'a qu'un seul antécédent de $0$.
			\item Faire une hypothèse sur le nombre d'antécédents de $0$ par une fonction donnée, dépendant de la plus grande puissance de $x$ apparaissant dans la fonction.

			      \correction{Le nombre d'antécédents de $0$ d'une fonction est $n$ si la plus grande puissance de $x$ appariassant dans cette fonction est $n$.}
			\item Trouver une fonction qui contredise cette hypothèse.
		\end{enumerate}
	\end{exercice}
}

\vspace{5em}

\Paste{Exercices}

\end{document}