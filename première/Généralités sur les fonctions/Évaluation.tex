\documentclass[
	classe=1 STI2D,
	gray
]{évaluation}

\title{Évaluation : Généralités sur les fonctions}
\author{}
\date{}

\begin{document}

\maketitle

\begin{exercice}

	On a donné les graphes des fonctions $f$ et $g$ ci-dessous, définies sur l'intervalle $[-5; 5]$ :

	\begin{center}
		\begin{tikzpicture}
			\draw[very thin,gray] (-5.5,-5.5) grid (5.5,5.5);
			\draw[thick,\myArrow] (-5.5,0) -- (5.5,0);
			\draw[thick,\myArrow] (0,-5.5) -- (0,5.5);
			\foreach \x in {-5,-4,-3,-2,-1,1,2,...,5} {
					\draw (\x,0) -- ++(0,-0.2) node[below] {$\x$};
					\draw (0,\x) -- ++(-0.2,0) node[left] {$\x$};
				}
			\node[below left] at (0,0) {$0$};

			\draw[ultra thick,orange,domain=-5:-3,variable=\x] plot({\x},{41 + 19*\x + 2*\x*\x});
			\draw[ultra thick,orange,domain=-3:0,variable=\x] plot({\x},{-2 + 1/6*\x + 3*\x*\x + 5/6*\x*\x*\x});
			\draw[ultra thick,orange,domain=0:4,variable=\x] plot({\x},{-2 - 29/60*\x + 27/40*\x*\x - 11/120*\x*\x*\x});
			\draw[ultra thick,orange,domain=4:5,variable=\x] plot({\x},{7 - 3.5*\x + 0.5*\x*\x}) node[right] {$𝒞_f$};

			\draw[ultra thick,violet,domain=-5:5,variable=\x] plot({\x},{(-4*\x + 1/3*\x*\x*\x) / (65/12)}) node[right] {$𝒞_g$};
		\end{tikzpicture}
	\end{center}

	\begin{enumerate}
		\item Donner l'image par $f$ de $-4$, $-2$, $0$ et $2$.
		\item Donner, si ils existent, les antécédents par $f$ de $3$.
		\item Quelle est le sens de variation de $f$ sur $[1 ; 5]$ ?
		\item Calculer le taux de variation de $f$ entre $-2$ et $-2$.
		\item Donner l'image par $g$ de $-4$, $-2$, $2$ et $5$.
		\item Donner, si ils existent, les antécédents par $g$ de $-1$, $1$ et $5$.
		\item Donner le tableau de variation de $g$.
	\end{enumerate}
\end{exercice}

\end{document}