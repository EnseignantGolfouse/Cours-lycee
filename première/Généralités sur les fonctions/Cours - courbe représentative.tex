\documentclass{coursclass}

\usepackage{clipboard}

\title{Cours Chapitre 2}
\author{Généralités sur les fonctions}
\date{}

\begin{document}
\thispagestyle{plain}

\Copy{Definition}{
\begin{definition}[Courbe représentative]
	La \textbf{courbe représentative} $𝒞_f$ d'une fonction $f$ dans un repère du plan est l'ensemble des points $(x ; y)$ du repère tels que $y = f(x)$.
\end{definition}

\begin{exemple}
	\begin{center}
		\begin{tikzpicture}
			\draw[thin,gray] (-1.5,-1.5) grid (8.5,6.5);
			\draw[thick,\myArrow] (-1.5,0) -- (8.5,0) node[below right] {$x$};
			\draw[thick,\myArrow] (0,-1.5) -- (0,6.5) node[above left] {$y$};

			\draw[orange,very thick,domain=-1.5:8.5,variable=\x] plot({\x},{-0.16*\x*\x + 1.6*\x + 1}) node[above] {$𝒞_f$};
			\coordinate (P) at (2.6,4.0784);
			\coordinate (PX) at (2.6,0);
			\coordinate (PY) at (0,4.0784);
			\draw[dashed,blue] (PX) node[below] {$x$} -- (P) -- (PY) node[left] {$f(x)$};
			\node[yshift=-1,blue] at (P) {∙};
			\node[below right] at (P) {point de coordonnées};
			\node[below right] at (2.6,3.5) {$(x ; f(x))$};
		\end{tikzpicture}
	\end{center}
\end{exemple}
}

\vspace{1em}

\Paste{Definition}

\end{document}