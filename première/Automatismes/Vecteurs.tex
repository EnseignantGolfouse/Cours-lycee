\documentclass{automatisme}

\usepackage{tikz-repère}

\begin{document}

\begin{frame}
	\begin{center}
		\begin{tikzpicture}[scale=0.7]
			\tikzRepere{-3}{3}{-3}{3}

			\coordinate (A) at (1,1);
			\coordinate (B) at (3,-2);
			\coordinate (C) at (-2,1);
			\coordinate (D) at (-1,-1);

			\foreach \p in {A,B,C,D} {
					\node at (\p) {×};
					\node[below left] at (\p) {$\p$};
				}
		\end{tikzpicture}
	\end{center}

	Donner les coordonnées des vecteurs suivants :

	\begin{itemize}
		\setlength{\itemsep}{1em}
		\item $\vec{AB} =$
		\item $\vec{AC} =$
		\item $\vec{DC} + \vec{CB} =$
	\end{itemize}
\end{frame}

\end{document}