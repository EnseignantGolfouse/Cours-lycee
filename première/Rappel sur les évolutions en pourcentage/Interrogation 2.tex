\documentclass{beamer}

\usepackage{préambule}
\usepackage{amsmath,enumerate}

\begin{document}

\small

\begin{frame}
	\setlength{\columnseprule}{0.7pt}
	\begin{multicols}{2}
		\uline{Sujet A}\vspace{1em}

		{\color{blue} 1.} Déterminer le coefficient des évolutions suivantes :\vspace{0.5em}

		{\color{blue} 1.a.} Augmentation de 20\%.

		{\color{blue} 1.b.} Diminution de 30\% suivie d'une augmentation de 50\%.

		{\color{blue} 1.c.} Augmentation de 400\%, suivie d'une diminution de 60\%.

		\vspace{0.5em}{\color{blue} 2.} Pour chaque évolution ci-dessus, déterminer son coefficient réciproque.\vspace{0.5em}

		{\color{blue} 3.} Calculer le coefficient des évolutions suivantes :

		{\color{blue} 3.a} évolution de $50$ à $60$ ;

		{\color{blue} 3.b} évolution de $90$ à $305$ ;

		{\color{blue} 3.c} évolution de $1000$ à $520$ ;

		{\color{blue} 3.d} évolution de $10$ à $9,2$ ;

		\columnbreak

		\uline{Sujet B}\vspace{1em}

		{\color{blue} 1.} Déterminer le coefficient des évolutions suivantes :\vspace{0.5em}

		{\color{blue} 1.a.} Augmentation de 30\%.

		{\color{blue} 1.b.} Diminution de 40\% suivie d'une augmentation de 40\%.

		{\color{blue} 1.c.} Augmentation de 500\%, suivie d'une diminution de 70\%.

		\vspace{0.5em}{\color{blue} 2.} Pour chaque évolution ci-dessus, déterminer son coefficient réciproque.\vspace{0.5em}

		{\color{blue} 3.} Calculer le coefficient des évolutions suivantes :

		{\color{blue} 3.a} évolution de $50$ à $80$ ;

		{\color{blue} 3.b} évolution de $75$ à $325$ ;

		{\color{blue} 3.c} évolution de $1000$ à $480$ ;

		{\color{blue} 3.d} évolution de $10$ à $7,3$ ;
	\end{multicols}
\end{frame}

\end{document}