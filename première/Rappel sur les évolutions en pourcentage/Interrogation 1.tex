\documentclass{beamer}

\usepackage{préambule}

\begin{document}

\footnotesize

\begin{frame}
	Calculer les évolutions suivantes :

	\begin{itemize}
		\item[1.1] $90$ augmenté de $10\%$
		\item[1.2] $200$ augmenté de $35\%$
		\item[1.3] $120$ diminué de $30\%$
		\item[1.4] $600$ diminué de $7\%$
	\end{itemize}

	\vspace{1em}\hrule\vspace{1em}

	Déterminer le coefficient des évolutions suivantes :

	\begin{enumerate}
		\item[2.1] Une augmentation de $45\%$
		\item[2.2] Une diminution de $83\%$
		\item[2.3] Une augmentation de $50\%$  suivie d'une autre augmentation de $50\%$
		\item[2.4] Une augmentation de $20\%$ suivie d'une diminution de $10\%$
	\end{enumerate}

	\vspace{1em}\hrule\vspace{1em}


	Déterminer l'évolution en pourcentage (augmentation ou diminution) correspondant aux coefficients suivants :
	\begin{itemize}
		\item[3.1] $c = 1,9$
		\item[3.2] $c = 0,43$
	\end{itemize}
\end{frame}

\end{document}