\documentclass[a4paper,landscape,twocolumn,classe=1STI2D]{exercice}

\usepackage{clipboard}

\renewcommand{\arraystretch}{1.5}

\newcounter{exercicesCounter}
\setcounter{exercicesCounter}{1}
\newcommand{\makeExercice}{
	\uline{\textbf{Exercice 1.\arabic{exercicesCounter}}}\stepcounter{exercicesCounter}
}

\title{Proportionnalité et pourcentages}
\author{}
\date{}

\begin{document}

\newcommand{\placeDeCorrection}{}


\Copy{Exercices}{
	\maketitle

	\makeExercice : Exercices de référence
	\vspace{1em}

	\renewcommand{\placeDeCorrection}{6em}
	\begin{minipage}{0.23\linewidth}
		Le prix d’un ordinateur est de
		1 100 €. Le nouveau modèle
		coûtera 13\% de plus. Calculer
		le prix du nouveau modèle.

		\correction{1243 €}
		\vspace{\placeDeCorrection}
	\end{minipage}
	\hfill\vline\hfill
	\begin{minipage}{0.23\linewidth}
		Une forêt a une surface de 550
		hectares. Après un feu la surface
		a diminué de 6\%. Calculer la
		surface de la forêt après le feu.

		\correction{517 hectares}
		\vspace{\placeDeCorrection}
	\end{minipage}
	\hfill\vline\hfill
	\begin{minipage}{0.23\linewidth}
		Après une augmentation de $10\%$,
		le prix d’un jeu est de $49,5€$.
		Calculer le prix initial du jeu.

		\correction{45 €}
		\vspace{8em}
	\end{minipage}
	\hfill\vline\hfill
	\begin{minipage}{0.23\linewidth}
		Le nombre de visites d’un site
		web est passé de 1200 à 1308 en
		un jour. \\
		Calculer le taux d’augmentation.

		\correction{1.09}
		\vspace{\placeDeCorrection}
	\end{minipage}

	\makeExercice : Exercices de référence sur successions
	\vspace{1em}

	\renewcommand{\placeDeCorrection}{4em}
	\begin{minipage}{0.23\linewidth}
		Il y a 8 milliards d’humains sur
		Terre. Combien serons-nous dans
		30 ans si l’augmentation est de
		1\% par an ?

		\correction{10 782 791 322}
		\vspace{7em}
	\end{minipage}
	\hfill\vline\hfill
	\begin{minipage}{0.23\linewidth}
		J’emprunte la somme de $50 000
			€$ à un taux annuel de 4\% sur une
		durée de 10 ans. Si je ne
		rembourse rien avant l’échéance,
		combien devrai-je rembourser ?

		\correction{74 012 €}
		\vspace{\placeDeCorrection}
	\end{minipage}
	\hfill\vline\hfill
	\begin{minipage}{0.23\linewidth}
		Un Youtubeur a perdu en
		popularité. Il avait 1 200 000
		abonnés en 2017 puis a perdu
		10\% d’abonnés chaque année.
		Combien a-t-il encore d’abonnés
		en 2022 ?

		\correction{708 588}
		\vspace{\placeDeCorrection}
	\end{minipage}
	\hfill\vline\hfill
	\begin{minipage}{0.23\linewidth}
		Le chiffre d’affaire d’une
		entreprise est passé de $100 k€$ à
		$144 k€$ en deux ans avec le même
		taux annuel d’augmentation.
		Quel est ce taux annuel ?

		\correction{1.12}
		\vspace{\placeDeCorrection}
	\end{minipage}
}

\newpage

\setcounter{exercicesCounter}{1}
\Paste{Exercices}

\end{document}